\newcommand*{\PRINTED}{}	% comment this line for generating online variant


% ░█▀▀█ ▒█▀▀█ ▒█▀▀▀█ ▒█░▒█ ▀▀█▀▀ 
% ▒█▄▄█ ▒█▀▀▄ ▒█░░▒█ ▒█░▒█ ░▒█░░ 
% ▒█░▒█ ▒█▄▄█ ▒█▄▄▄█ ░▀▄▄▀ ░▒█░░

% general document info for simpler reuse
\newcommand{\AuthorName}{Toma\v{z}}
\newcommand{\AuthorNameSurname}{\AuthorName ~Kragelj}
\newcommand{\BookTitle}{ZX Spectrum Next Assembler Developer Guide}
\newcommand{\LatestVersion}{1.0}
\newcommand{\LatestYear}{2021}

% definitions based on whether PDF is generated for book or online.
\newcommand{\email}[3]{\ifdefined\PRINTED{\tt #1@#2.#3}\else{\tt #1 AT #2 DOT #3}\fi}


% ▒█▀▀█ ░█▀▀█ ▒█▀▀█ ▒█▀▀▀   ▒█▀▀▀ ▒█▀▀▀█ ▒█▀▀█ ▒█▀▄▀█ ░█▀▀█ ▀▀█▀▀ ▀▀█▀▀ ▀█▀ ▒█▄░▒█ ▒█▀▀█ 
% ▒█▄▄█ ▒█▄▄█ ▒█░▄▄ ▒█▀▀▀   ▒█▀▀▀ ▒█░░▒█ ▒█▄▄▀ ▒█▒█▒█ ▒█▄▄█ ░▒█░░ ░▒█░░ ▒█░ ▒█▒█▒█ ▒█░▄▄ 
% ▒█░░░ ▒█░▒█ ▒█▄▄█ ▒█▄▄▄   ▒█░░░ ▒█▄▄▄█ ▒█░▒█ ▒█░░▒█ ▒█░▒█ ░▒█░░ ░▒█░░ ▄█▄ ▒█░░▀█ ▒█▄▄█

% redefine the default "plain" pagestyle (used by chapter pages)
\fancypagestyle{plain}{
	\fancyhf{} 
	\fancyhead{}
	\fancyfoot[L]{\thepage}
	\renewcommand{\headrulewidth}{0pt}
	\renewcommand{\footrulewidth}{0pt}
}

% define style used for normal pages
\fancypagestyle{clean}{
	\fancyhf{}
	\fancyhead[RE,LO]{\leftmark}
	\fancyfoot[RE,LO]{\thepage}
}

% we can use this for pages that are intentionally left blank
\newcommand{\IntentionallyEmpty}{
	\mbox{}
	\vfill
	\begin{center}
	This page intentionally left empty
	\end{center}
	\vfill
	\mbox{}
}

% empty pages before chapters should use empty style (aka nothing)
\let\mtcleardoublepage\cleardoublepage
\renewcommand{\cleardoublepage}{\clearpage{\pagestyle{empty}\mtcleardoublepage}}

% define default paragraph styles
\setlength{\parindent}{0em}
\setlength{\parskip}{1.5ex}

% don't stretch content into empty space vertically
\raggedbottom

% use empty style for first couple pages, until defined otherwise
\pagestyle{empty}


% ▀▀█▀▀ ░█▀▀█ ▒█▀▀█ ▒█░░░ ▒█▀▀▀   ▒█▀▀▀ ▒█▀▀▀█ ▒█▀▀█ ▒█▀▄▀█ ░█▀▀█ ▀▀█▀▀ ▀▀█▀▀ ▀█▀ ▒█▄░▒█ ▒█▀▀█ 
% ░▒█░░ ▒█▄▄█ ▒█▀▀▄ ▒█░░░ ▒█▀▀▀   ▒█▀▀▀ ▒█░░▒█ ▒█▄▄▀ ▒█▒█▒█ ▒█▄▄█ ░▒█░░ ░▒█░░ ▒█░ ▒█▒█▒█ ▒█░▄▄ 
% ░▒█░░ ▒█░▒█ ▒█▄▄█ ▒█▄▄█ ▒█▄▄▄   ▒█░░░ ▒█▄▄▄█ ▒█░▒█ ▒█░░▒█ ▒█░▒█ ░▒█░░ ░▒█░░ ▄█▄ ▒█░░▀█ ▒█▄▄█

% same as X and p{size} respectively, but centers text horizontally
\newcolumntype{Y}{>{\centering\arraybackslash}X}
\newcolumntype{C}[1]{>{\centering\arraybackslash}p{#1}}

% top and bottom "struts" for instruction lines (note different height can be passed in via optional parameter)
\newcommand{\instrt}[1][2.5ex]{\rule{0pt}{#1}}
\newcommand{\instrb}[1][-1.4ex]{\rule[#1]{0pt}{0pt}}

% top and bottom "struts" for note lines
\newcommand{\notet}{\rule{0pt}{2.4ex}}
\newcommand{\noteb}{\rule[-1.3ex]{0pt}{0pt}}


% ▒█▀▀▀█ ▒█░▒█ ▒█▀▀▀█ ▒█▀▀█ ▀▀█▀▀ ▒█░▒█ ░█▀▀█ ▒█▄░▒█ ▒█▀▀▄ ▒█▀▀▀█ 
% ░▀▀▀▄▄ ▒█▀▀█ ▒█░░▒█ ▒█▄▄▀ ░▒█░░ ▒█▀▀█ ▒█▄▄█ ▒█▒█▒█ ▒█░▒█ ░▀▀▀▄▄ 
% ▒█▄▄▄█ ▒█░▒█ ▒█▄▄▄█ ▒█░▒█ ░▒█░░ ▒█░▒█ ▒█░▒█ ▒█░░▀█ ▒█▄▄▀ ▒█▄▄▄█

% less tall slash
\newcommand\scslash{\stretchrel*{$/$}{\textsc{e}}}

% couple shorthands that can be used throughout the document
\newcommand{\Deg}{\textsuperscript{o}}
\newcommand{\ddd}{\makebox[1em][c]{.\hfil.\hfil.}}
\newcommand{\See}[1]{\textsuperscript{#1}}
\newcommand{\UNDOC}{\textnormal{\textsuperscript{**}}}
\newcommand{\ZXN}{\textnormal{\textsuperscript{ZX}}}
\newcommand{\ZXNS}{\tiny\textnormal{\textsuperscript{ZX}}}
\newcommand{\High}{\textsubscript{h}}
\newcommand{\Low}{\textsubscript{l}}

% instruction flags definitions - for unified formatting
\newcommand{\FS}{$\updownarrow$} % standard effect
\newcommand{\FN}{-}				% no effect
\newcommand{\FU}{?}				% unknown effect
\newcommand{\FX}{$\bullet$}		% special case
\newcommand{\FPV}{VF}			% PV=Overflow
\newcommand{\FPP}{PF}			% PV=Parity


% ▒█▀▀█ ▒█▀▀▀█ ▒█▀▄▀█ ▒█▀▄▀█ ▒█▀▀▀█ ▒█▄░▒█   ▒█▀▀▄ ▒█▀▀█ ░█▀▀█ ▒█░░▒█ ░█▀▀█ ▒█▀▀█ ▒█░░░ ▒█▀▀▀ ▒█▀▀▀█ 
% ▒█░░░ ▒█░░▒█ ▒█▒█▒█ ▒█▒█▒█ ▒█░░▒█ ▒█▒█▒█   ▒█░▒█ ▒█▄▄▀ ▒█▄▄█ ▒█▒█▒█ ▒█▄▄█ ▒█▀▀▄ ▒█░░░ ▒█▀▀▀ ░▀▀▀▄▄ 
% ▒█▄▄█ ▒█▄▄▄█ ▒█░░▒█ ▒█░░▒█ ▒█▄▄▄█ ▒█░░▀█   ▒█▄▄▀ ▒█░▒█ ▒█░▒█ ▒█▄▀▄█ ▒█░▒█ ▒█▄▄█ ▒█▄▄█ ▒█▄▄▄ ▒█▄▄▄█

% various types of lines
\newcommand{\DottedLine}{\hdashline[1pt/1pt]}

% horizontal arrows of arbitrary size (lehgth specified through parameter)
\newcommand{\RArrow}[1]{\parbox{#1}{\tikz{
	\draw[->,line width=0.5pt](0,0)--(#1,0);
}}}
\newcommand{\LArrow}[1]{\parbox{#1}{\tikz{
	\draw[<-,line width=0.5pt](0,0)--(#1,0);
}}}

% horizontal arrows with vertical line on the arrow side; parameters:
% - mandatory horizontal line width
% - optional any additional horizontal line styles (dotted, dashed etc)
% - optional vertical line height (divided by 2), default 0.1
% - optional arrow scale, default 1
% - optional line width, default 0.5pt
\NewDocumentCommand{\RArrowLine}{ m O{} O{0.1} O{1.6} O{0.5pt} }{\parbox{#1}{\tikz{
	\draw[-{>[scale=#4,length=#4,width=#4]},line width=#5,#2](0,0)--(#1,0);
	\draw[line width=#5](#1,-#3)--(#1,#3);
}}}
\NewDocumentCommand{\LArrowLine}{ m O{} O{0.1} O{1.6} O{0.5pt} }{\parbox{#1}{\tikz{
	\draw[line width=#5](0,-#3)--(0,#3);
	\draw[-{>[scale=#4,length=#4,width=#4]},line width=#5,#2](#1,0)--(0,0);
}}}


% ▒█▀▀█ ▒█░▒█ ▒█▀▀▀█ ▀▀█▀▀ ▒█▀▀▀█ ▒█▀▄▀█ ▀█▀ ▒█▀▀▀█ ░█▀▀█ ▀▀█▀▀ ▀█▀ ▒█▀▀▀█ ▒█▄░▒█ ▒█▀▀▀█ 
% ▒█░░░ ▒█░▒█ ░▀▀▀▄▄ ░▒█░░ ▒█░░▒█ ▒█▒█▒█ ▒█░ ░▄▄▄▀▀ ▒█▄▄█ ░▒█░░ ▒█░ ▒█░░▒█ ▒█▒█▒█ ░▀▀▀▄▄ 
% ▒█▄▄█ ░▀▄▄▀ ▒█▄▄▄█ ░▒█░░ ▒█▄▄▄█ ▒█░░▒█ ▄█▄ ▒█▄▄▄█ ▒█░▒█ ░▒█░░ ▄█▄ ▒█▄▄▄█ ▒█░░▀█ ▒█▄▄▄█

% customize Verbatim environment wiht default settings
\RecustomVerbatimEnvironment{Verbatim}{Verbatim}{
	numbers=left,
	xleftmargin=5mm,
	tabsize=4,
	fontsize=\small
}