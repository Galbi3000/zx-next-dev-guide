\newcommand*{\PRINTED}{}	% comment this line for generating online variant

% ░█▀▀█ ▒█▀▀█ ▒█▀▀▀█ ▒█░▒█ ▀▀█▀▀ 
% ▒█▄▄█ ▒█▀▀▄ ▒█░░▒█ ▒█░▒█ ░▒█░░ 
% ▒█░▒█ ▒█▄▄█ ▒█▄▄▄█ ░▀▄▄▀ ░▒█░░

% general document info for simpler reuse
\newcommand{\AuthorName}{Toma\v{z}}
\newcommand{\AuthorNameSurname}{\AuthorName ~Kragelj}
\newcommand{\BookTitle}{ZX Spectrum Next Assembler Developer Guide}
\newcommand{\LatestYear}{2021}
\newcommand{\LatestReleaseDate}{16 July 2021}

% definitions based on whether PDF is generated for book or online.
\newcommand{\email}[3]{\ifdefined\PRINTED{\tt #1@#2.#3}\else{\tt #1 AT #2 DOT #3}\fi}


% ▒█▀▀█ ▒█▀▀▀█ ▒█░░░ ▒█▀▀▀█ ▒█░▒█ ▒█▀▀█ ▒█▀▀▀█ 
% ▒█░░░ ▒█░░▒█ ▒█░░░ ▒█░░▒█ ▒█░▒█ ▒█▄▄▀ ░▀▀▀▄▄ 
% ▒█▄▄█ ▒█▄▄▄█ ▒█▄▄█ ▒█▄▄▄█ ░▀▄▄▀ ▒█░▒█ ▒█▄▄▄█

\definecolor{PrintableLightGray}{rgb}{0.85, 0.85, 0.85}
\definecolor{PrintableDarkGray}{rgb}{0.5, 0.5, 0.5}


% ▒█▀▀█ ░█▀▀█ ▒█▀▀█ ▒█▀▀▀   ▒█▀▀▀ ▒█▀▀▀█ ▒█▀▀█ ▒█▀▄▀█ ░█▀▀█ ▀▀█▀▀ ▀▀█▀▀ ▀█▀ ▒█▄░▒█ ▒█▀▀█ 
% ▒█▄▄█ ▒█▄▄█ ▒█░▄▄ ▒█▀▀▀   ▒█▀▀▀ ▒█░░▒█ ▒█▄▄▀ ▒█▒█▒█ ▒█▄▄█ ░▒█░░ ░▒█░░ ▒█░ ▒█▒█▒█ ▒█░▄▄ 
% ▒█░░░ ▒█░▒█ ▒█▄▄█ ▒█▄▄▄   ▒█░░░ ▒█▄▄▄█ ▒█░▒█ ▒█░░▒█ ▒█░▒█ ░▒█░░ ░▒█░░ ▄█▄ ▒█░░▀█ ▒█▄▄█

% these are here to avoid compiler warnings
\setlength{\headheight}{14.49998pt}
\addtolength{\topmargin}{-2.49998pt}

% redefine the default "plain" pagestyle (used by chapter pages)
\fancypagestyle{plain}{
	\fancyhf{} 
	\fancyhead{}
	\fancyfoot[L]{\thepage}
	\renewcommand{\headrulewidth}{0pt}
	\renewcommand{\footrulewidth}{0pt}
}

% define style used for normal pages
\fancypagestyle{clean}{
	\fancyhf{}
	\fancyhead[RE,LO]{\leftmark}
	\fancyfoot[RE,LO]{\thepage}
}

% we can use this for pages that are intentionally left blank
\newcommand{\IntentionallyEmpty}{
	\mbox{}
	\vfill
	\begin{center}
	This page intentionally left empty
	\end{center}
	\vfill
	\mbox{}
}

% empty pages before chapters should use empty style (aka nothing)
\let\mtcleardoublepage\cleardoublepage
\renewcommand{\cleardoublepage}{\clearpage{\pagestyle{empty}\mtcleardoublepage}}

% define default paragraph styles
\setlength{\parindent}{0em}
\setlength{\parskip}{1.5ex}

% don't stretch content into empty space vertically
\raggedbottom

% use empty style for first couple pages, until defined otherwise
\pagestyle{empty}


% ▀▀█▀▀ ░█▀▀█ ▒█▀▀█ ▒█░░░ ▒█▀▀▀   ▒█▀▀▀ ▒█▀▀▀█ ▒█▀▀█ ▒█▀▄▀█ ░█▀▀█ ▀▀█▀▀ ▀▀█▀▀ ▀█▀ ▒█▄░▒█ ▒█▀▀█ 
% ░▒█░░ ▒█▄▄█ ▒█▀▀▄ ▒█░░░ ▒█▀▀▀   ▒█▀▀▀ ▒█░░▒█ ▒█▄▄▀ ▒█▒█▒█ ▒█▄▄█ ░▒█░░ ░▒█░░ ▒█░ ▒█▒█▒█ ▒█░▄▄ 
% ░▒█░░ ▒█░▒█ ▒█▄▄█ ▒█▄▄█ ▒█▄▄▄   ▒█░░░ ▒█▄▄▄█ ▒█░▒█ ▒█░░▒█ ▒█░▒█ ░▒█░░ ░▒█░░ ▄█▄ ▒█░░▀█ ▒█▄▄█

% same as X and p{size} respectively, but centers text horizontally
\newcolumntype{Y}{>{\centering\arraybackslash}X}
\newcolumntype{C}[1]{>{\centering\arraybackslash}p{#1}}

% top and bottom "struts" for instruction lines (note different height can be passed in via optional parameter)
\newcommand{\instrt}[1][2.5ex]{\rule{0pt}{#1}}
\newcommand{\instrb}[1][-1.4ex]{\rule[#1]{0pt}{0pt}}

% top and bottom "struts" for note lines
\newcommand{\notet}{\rule{0pt}{2.4ex}}
\newcommand{\noteb}{\rule[-1.3ex]{0pt}{0pt}}


% ▒█▀▀▀█ ▒█░▒█ ▒█▀▀▀█ ▒█▀▀█ ▀▀█▀▀ ▒█░▒█ ░█▀▀█ ▒█▄░▒█ ▒█▀▀▄ ▒█▀▀▀█ 
% ░▀▀▀▄▄ ▒█▀▀█ ▒█░░▒█ ▒█▄▄▀ ░▒█░░ ▒█▀▀█ ▒█▄▄█ ▒█▒█▒█ ▒█░▒█ ░▀▀▀▄▄ 
% ▒█▄▄▄█ ▒█░▒█ ▒█▄▄▄█ ▒█░▒█ ░▒█░░ ▒█░▒█ ▒█░▒█ ▒█░░▀█ ▒█▄▄▀ ▒█▄▄▄█

% less tall slash
\newcommand\scslash{\stretchrel*{$/$}{\textsc{e}}}

% couple shorthands that can be used throughout the document
\newcommand{\Deg}{\textsuperscript{o}}
\newcommand{\ddd}{\makebox[1em][c]{.\hfil.\hfil.}}
\newcommand{\See}[1]{\textsuperscript{#1}}
\newcommand{\UNDOC}{\textnormal{\textsuperscript{**}}}
\newcommand{\ZXN}{\textnormal{\textsuperscript{ZX}}}
\newcommand{\ZXNS}{\tiny\textnormal{\textsuperscript{ZX}}}
\newcommand{\High}{\textsubscript{h}}
\newcommand{\Low}{\textsubscript{l}}

% instruction flags definitions - for unified formatting
\newcommand{\FS}{$\updownarrow$} % standard effect
\newcommand{\FN}{-}				% no effect
\newcommand{\FU}{?}				% unknown effect
\newcommand{\FX}{$\bullet$}		% special case
\newcommand{\FPV}{VF}			% PV=Overflow
\newcommand{\FPP}{PF}			% PV=Parity

% PortLink is short link to port with description and address, suitable for inline use, while PortReference is longer reference with chapter title and section number mostly used for register lists when referring to previously described register
\newcommand{\PortLink}[2]{{\small \textbf{#1} {\tt \$#2}}}
\newcommand{\PortReference}[2]{
	\vspace*{-2ex}
	See description under #2 chapter, section \ref{#1}.
}

% other commonly used definitions
\newcommand{\MemAddr}[1]{{\tt \$#1}}


% ▒█▀▀█ ▒█▀▀▀█ ▒█▀▄▀█ ▒█▀▄▀█ ▒█▀▀▀█ ▒█▄░▒█   ▒█▀▀▄ ▒█▀▀█ ░█▀▀█ ▒█░░▒█ ░█▀▀█ ▒█▀▀█ ▒█░░░ ▒█▀▀▀ ▒█▀▀▀█ 
% ▒█░░░ ▒█░░▒█ ▒█▒█▒█ ▒█▒█▒█ ▒█░░▒█ ▒█▒█▒█   ▒█░▒█ ▒█▄▄▀ ▒█▄▄█ ▒█▒█▒█ ▒█▄▄█ ▒█▀▀▄ ▒█░░░ ▒█▀▀▀ ░▀▀▀▄▄ 
% ▒█▄▄█ ▒█▄▄▄█ ▒█░░▒█ ▒█░░▒█ ▒█▄▄▄█ ▒█░░▀█   ▒█▄▄▀ ▒█░▒█ ▒█░▒█ ▒█▄▀▄█ ▒█░▒█ ▒█▄▄█ ▒█▄▄█ ▒█▄▄▄ ▒█▄▄▄█

% horizontal arrows of arbitrary size (lehgth specified through parameter)
\newcommand{\RArrow}[1]{\parbox{#1}{\tikz{
	\draw[->,line width=0.5pt](0,0)--(#1,0);
}}}
\newcommand{\LArrow}[1]{\parbox{#1}{\tikz{
	\draw[<-,line width=0.5pt](0,0)--(#1,0);
}}}

% horizontal arrows with vertical line on the arrow side; parameters:
% - mandatory horizontal line width
% - optional any additional horizontal line styles (dotted, dashed etc)
% - optional vertical line height (divided by 2), default 0.1
% - optional arrow scale, default 1
% - optional line width, default 0.5pt
\NewDocumentCommand{\RArrowLine}{ m O{} O{0.1} O{1.6} O{0.5pt} }{\parbox{#1}{\tikz{
	\draw[-{>[scale=#4,length=#4,width=#4]},line width=#5,#2](0,0)--(#1,0);
	\draw[line width=#5](#1,-#3)--(#1,#3);
}}}
\NewDocumentCommand{\LArrowLine}{ m O{} O{0.1} O{1.6} O{0.5pt} }{\parbox{#1}{\tikz{
	\draw[line width=#5](0,-#3)--(0,#3);
	\draw[-{>[scale=#4,length=#4,width=#4]},line width=#5,#2](#1,0)--(0,0);
}}}


% ▒█▀▀█ ▒█░▒█ ▒█▀▀▀█ ▀▀█▀▀ ▒█▀▀▀█ ▒█▀▄▀█ ▀█▀ ▒█▀▀▀█ ░█▀▀█ ▀▀█▀▀ ▀█▀ ▒█▀▀▀█ ▒█▄░▒█ ▒█▀▀▀█ 
% ▒█░░░ ▒█░▒█ ░▀▀▀▄▄ ░▒█░░ ▒█░░▒█ ▒█▒█▒█ ▒█░ ░▄▄▄▀▀ ▒█▄▄█ ░▒█░░ ▒█░ ▒█░░▒█ ▒█▒█▒█ ░▀▀▀▄▄ 
% ▒█▄▄█ ░▀▄▄▀ ▒█▄▄▄█ ░▒█░░ ▒█▄▄▄█ ▒█░░▒█ ▄█▄ ▒█▄▄▄█ ▒█░▒█ ░▒█░░ ▄█▄ ▒█▄▄▄█ ▒█░░▀█ ▒█▄▄▄█

% define the style for lstlisting
\lstdefinestyle{CodeStyle}{
	basicstyle=\ttfamily\small,
	commentstyle=\color{PrintableDarkGray},
	columns=flexible,
	tabsize=4,
	numbers=left,
	numberstyle=\ttfamily\tiny,
	numbersep=1.8em,
	morecomment=[l]{;},
	moredelim=[is][\rmfamily\itshape]{|}{|},	% any text within |...| will be roman/italic
	literate={&}{\$}1							% replace `&` with `$` (to avoid syntax higlight issues)
}

% define default settings for all tcolorbox instances
\tcbset{
	arc=4pt,			% radius for rounded corners
	boxrule=0pt,		% no frame around the box
	boxsep=1.3ex,		% add some spacing between frame and content
	left=0ex,			% left side should be flush with content
	right=0ex,			% right side should be flush with content
	top=-1.5ex,			% less spacing between frame and start of content
	bottom=-1.5ex,		% less spacing between end of content and frame
	pad at break=1pt,	% leave some small spacing between content and frame when page break occurs
	colback=PrintableLightGray,
	colframe=PrintableLightGray,
	enhanced,			% only use rounded corners on top and bottom part, not between page breaks
	breakable,			% allow breaking tcolorbox to multiple pages
	listing only,		% only show listing
	listing options={style=CodeStyle},
}
