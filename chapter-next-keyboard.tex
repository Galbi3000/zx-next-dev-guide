\section{Keyboard}
\label{zx_next_keyboard}

% ───────────────────────────────────────────────────────────────────
% ─██████──████████─██████████████─████████──████████─██████████████─
% ─██░░██──██░░░░██─██░░░░░░░░░░██─██░░░░██──██░░░░██─██░░░░░░░░░░██─
% ─██░░██──██░░████─██░░██████████─████░░██──██░░████─██░░██████████─
% ─██░░██──██░░██───██░░██───────────██░░░░██░░░░██───██░░██─────────
% ─██░░██████░░██───██░░██████████───████░░░░░░████───██░░██████████─
% ─██░░░░░░░░░░██───██░░░░░░░░░░██─────████░░████─────██░░░░░░░░░░██─
% ─██░░██████░░██───██░░██████████───────██░░██───────██████████░░██─
% ─██░░██──██░░██───██░░██───────────────██░░██───────────────██░░██─
% ─██░░██──██░░████─██░░██████████───────██░░██───────██████████░░██─
% ─██░░██──██░░░░██─██░░░░░░░░░░██───────██░░██───────██░░░░░░░░░░██─
% ─██████──████████─██████████████───────██████───────██████████████─
% ───────────────────────────────────────────────────────────────────

Next inherits ZX Spectrum keyboard handling, so all legacy programs will work out of the box. Additionally, it allows reading the status of extended keys.


\subsection{Legacy Keyboard Status}

ZX Spectrum uses 8$\times$5 matrix for reading keyboard status. This means 40 distinct keys can be represented. The keyboard is read from \PortLink{ULA Control Port Read}{xxFE} with particular high bytes. There are 8 possible bytes, each will return the status of 5 associated keys. If a key is pressed, the corresponding bit is set to {\tt 0} and vice versa.

Example for checking if {\tt P} or {\tt I} is pressed:
	
\begin{tcblisting}{}
	LD BC, &DFFE     ; We want to read keys..... YUIOP
	IN A, (C)        ; A holds values in bits... 43210
checkP:
	BIT 0, A         ; test bit 0 of A (P key)
	JR NZ checkI     ; if bit0=1, P not pressed
	...              ; P is pressed
checkI:
	BIT 2, A         ; test bit 2 of A (I key)
	JR NZ continue   ; if bit2=1, I not pressed
	...              ; I is pressed
continue:
\end{tcblisting}

As mentioned in Ports chapter, section \ref{zx_next_ports_examples}, we can slightly improve performance if we replace first two lines with:

\begin{tcblisting}{}
	LD A, &DF
	IN (&FE)
\end{tcblisting}

Reading the port in first example requires 22 t-states (10+12) vs. 18 (7+11). The difference is small, but it can add up as typically keyboard is read multiple times per frame.

The first program is more understandable at a glance - the port address is given as a whole 16-bit value, as usually provided in the documentation. The second program splits it into 2 8-bit values, so intent may not be immediately apparent. Of course, one learns the patterns with experience, but it nonetheless demonstrates the compromise between readability and speed.


\subsection{Next Extended Keys}

Next uses larger 8$\times$7 matrix for keyboard, with 10 additional keys. By default, hardware is translating keys from extra two columns into the existing 8$\times$5 set. But you can turn this off with bit {\tt 4} of \PortLink{ULA Control Register}{68}. Extra keys can be read separately via \PortLink{Extended Keys 0 Register}{B0} and \PortLink{Extended Keys 1 Register}{B1}.


\subsection{Keyboard Ports and Registers}
\label{zx_next_keyboard_registers}

\subsubsection{\PortTitle{ULA Control Port Read}{xxFE}}

\vspace*{-1em} % for some reason LaTeX adds some unwanted vertical space between title and text!?
Returns keyboard status when read with certain high byte values:

{
	\tt
	\setlength{\extrarowheight}{0pt}
	\def\arraystretch{0.1}
	
	\begin{tabular}{p{0.7cm}|cp{1cm}p{1cm}p{1cm}p{1.3cm}p{1.5cm}}

		~xx & & 4 & 3 & 2 & 1 & 0 \instrb \\
		\hline
		\MemAddr{7F}\instrt & & B & N & M & Symb & Space \\
		\MemAddr{BF}\instrt & & H & J & K & L & Enter \\
		\MemAddr{DF}\instrt & & Y & U & I & O & P \\
		\MemAddr{EF}\instrt & & 6 & 7 & 8 & 9 & 0 \\
		\MemAddr{F7}\instrt & & 5 & 4 & 3 & 2 & 1 \\
		\MemAddr{FB}\instrt & & T & R & E & W & Q \\
		\MemAddr{FD}\instrt & & G & F & D & S & A \\
		\MemAddr{FE}\instrt\instrb & & V & C & X & Z & Caps \\

	\end{tabular}
}

Bits are reversed: if a key is pressed, the corresponding bit is {\tt 0}, if a key is not pressed, bit is {\tt 1}.

Note: when written to, \PortLink{ULA Control Port}{xxFE} is used to set border colour and audio devices. See description under ULA chapter, section \ref{zx_next_ula_registers} for details.


\subsubsection{\PortTitleLink{ULA Control Register}{68}}

\PortReference{zx_next_tilemap_registers}{Tilemap}


\subsubsection{\PortTitle{Extended Keys 0 Register}{B0}}
\vspace*{-2ex}
\subsubsection{\PortTitle{Extended Keys 1 Register}{B1}}

\newcommand{\PortNextExtKey}[2]{{\tt 0} if key pressed, {\tt 1} otherwise & {\tt #1} & {\tt #2} \\ }
\begin{NextPort}[cp{5.5cm}p{1.5cm}X][Bit & Effect & \MemAddr{B0} & \MemAddr{B1}]
	\PortBits{7}
		\PortNextExtKey{;}{Delete}
	\PortBits{6}
		\PortNextExtKey{"}{Edit}
	\PortBits{5}
		\PortNextExtKey{,}{Break}
	\PortBits{4}
		\PortNextExtKey{.}{Inv Video}
	\PortBits{3}
		\PortNextExtKey{Up}{True Video}
	\PortBits{2}
		\PortNextExtKey{Down}{Graph}
	\PortBits{1}
		\PortNextExtKey{Left}{Caps Lock}
	\PortBits{0}
		\PortNextExtKey{Right}{Extend}
\end{NextPort}

Available since core 3.1.5

\pagebreak