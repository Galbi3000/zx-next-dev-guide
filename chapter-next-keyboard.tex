\section{Keyboard}
\label{zx_next_keyboard}

% ───────────────────────────────────────────────────────────────────
% ─██████──████████─██████████████─████████──████████─██████████████─
% ─██░░██──██░░░░██─██░░░░░░░░░░██─██░░░░██──██░░░░██─██░░░░░░░░░░██─
% ─██░░██──██░░████─██░░██████████─████░░██──██░░████─██░░██████████─
% ─██░░██──██░░██───██░░██───────────██░░░░██░░░░██───██░░██─────────
% ─██░░██████░░██───██░░██████████───████░░░░░░████───██░░██████████─
% ─██░░░░░░░░░░██───██░░░░░░░░░░██─────████░░████─────██░░░░░░░░░░██─
% ─██░░██████░░██───██░░██████████───────██░░██───────██████████░░██─
% ─██░░██──██░░██───██░░██───────────────██░░██───────────────██░░██─
% ─██░░██──██░░████─██░░██████████───────██░░██───────██████████░░██─
% ─██░░██──██░░░░██─██░░░░░░░░░░██───────██░░██───────██░░░░░░░░░░██─
% ─██████──████████─██████████████───────██████───────██████████████─
% ───────────────────────────────────────────────────────────────────

Keyboard is read from port {\tt \$xxFE} where bits in result are as follows:

{
	\tt
	\setlength{\extrarowheight}{0pt}
	\def\arraystretch{0.1}
	
	\begin{tabular}{p{0.7cm}|cp{1cm}p{1cm}p{1cm}p{1.3cm}p{1.5cm}}

		~xx & & 4 & 3 & 2 & 1 & 0 \instrb \\
		\hline
		\$7F\instrt & & B & N & M & Symb & Space \\
		\$BF\instrt & & H & J & K & L & Enter \\
		\$DF\instrt & & Y & U & I & O & P \\
		\$EF\instrt & & 6 & 7 & 8 & 9 & 0 \\
		\$F7\instrt & & 5 & 4 & 3 & 2 & 1 \\
		\$FB\instrt & & T & R & E & W & Q \\
		\$FD\instrt & & G & F & D & S & A \\
		\$FE\instrt\instrb & & V & C & X & Z & Caps \\

	\end{tabular}
}

If key is pressed, corresponding bit is set to {\tt 0} and vice versa. Example for checking if P or I is pressed:
	
\begin{tcblisting}{}
	LD BC, &DFFE   ; We want to read keys:   YUIOP
	IN A, (C)      ; A holds values in bits: 43210
checkP:
	BIT 0, A       ; test bit 0 of A (P key)
	JR NZ checkI   ; if bit0=1, P not pressed
	...            ; P is pressed
checkI:
	BIT 2, A       ; test bit 2 of A (I key)
	JR NZ continue ; if bit2=1, I not pressed
	...            ; I is pressed
continue:
\end{tcblisting}

We can slightly improve performance if we replace first two lines with:

\begin{tcblisting}{}
	LD A, &DF
	IN (&FE)
\end{tcblisting}

Using 16-bit load requires 10 T states for ``{\tt LD BC,nn}'' + 12 for ``{\tt IN A,(C)}'' whereas later variant requires 7 T states for ``{\tt LD A,nn}'' + 11 for ``{\tt IN (nn)}''. So 22 vs 18. The difference is small, but it can add up as typically keyboard is read multiple times per frame.

First program is more understandable at a glance - port address is given as whole 16-bit value, as usually provided in documentation. The second program splits it into 2 8-bit values, so intent may not be immediately apparent. Of course, one learns the patterns with experience, but it nonetheless demonstrates the compromise between readability and speed.



\pagebreak