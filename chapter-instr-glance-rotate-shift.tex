\section{Rotate and Shift}

\begin{minipage}{\textwidth}
	
\begin{instrtable}
    % we need slightly more vertical space to accomodate symbolic operation drawing for RLC; it's negative because we only need a fraction of what new line produces
    \\[-10pt]

    \begin{instruction}{RLC r}
        \Symbol{\SymRLC[0]{r}}
            \FlagsRLCr
            \OpCode{11}{001}{011}
            \Hex{CB}{2}
            \Cycles{2}{8}
            \Comment{
                \multirow{6}{*}{
                    \tt
                    \begin{tabular}{ll}
                        r & \OCT{r} \\
                        \hline
                        B & 000 \\
                        C & 001 \\
                        D & 010 \\
                        E & 011 \\
                        H & 100 \\
                        L & 101 \\
                        A & 111 \\
                    \end{tabular}
                }
            }
        \SkipToOpCode 
            \OpCode{00}{\fbox{000}}{\OCT{r}}
            \Hex{..}{}
    \end{instruction}

    \begin{instruction}{RLC (HL)}
        \Symbol{\SymRLC[0]{(HL)}}
            \FlagsRLCr
            \OpCode{11}{001}{011}
            \Hex{CB}{2}
            \Cycles{4}{15}
        \SkipToOpCode 
            \OpCode{00}{\fbox{000}}{110}
            \Hex{06}{}
    \end{instruction}

    \begin{instruction}{RLC (IX+d)}
        \Symbol{\SymRLC[0]{(IX+d)}}
            \FlagsRLCr
            \OpCode{11}{011}{101}
            \Hex{DD}{4}
            \Cycles{6}{23}
        \SkipToOpCode 
            \OpCode{11}{001}{011}
            \Hex{CB}{}
        \SkipToOpCode 
            \OpRange{d} 
            \Hex{..}{}
        \SkipToOpCode 
            \OpCode{00}{\fbox{000}}{110}
            \Hex{06}{}
    \end{instruction}

    \begin{instruction}{RLC (IY+d)}
        \Symbol{\SymRLC[0]{(IY+d)}}
            \FlagsRLCr
            \OpCode{11}{111}{101}
            \Hex{FD}{4}
            \Cycles{6}{23}
        \SkipToOpCode 
            \OpCode{11}{001}{011}
            \Hex{CB}{}
        \SkipToOpCode 
            \OpRange{d} 
            \Hex{..}{}
        \SkipToOpCode 
            \OpCode{00}{\fbox{000}}{110}
            \Hex{06}{}
    \end{instruction}

    \begin{instruction}{RLC r,(IX+d)}
        \Symbol{\SymRLCu[0]{r}{IX}}
            \FlagsRLCr
            \OpCode{11}{011}{101}
            \Hex{DD}{4}
            \Cycles{6}{23}
        \SkipToSymbol
            \Symbol{\SymRLCu[1]{r}{IX}}
            \FromSymbolToOpCode
            \OpCode{11}{001}{011}
            \Hex{CB}{}
        \SkipToSymbol
            \Symbol{\SymRLCu[2]{r}{IX}}
            \FromSymbolToOpCode
            \OpRange{d} 
            \Hex{..}{}
        \SkipToOpCode 
            \OpCode{00}{\fbox{000}}{\OCT{r}}
            \Hex{..}{}
    \end{instruction}

    \begin{instruction}{RLC r,(IY+d)}
        \Symbol{\SymRLCu[0]{r}{IY}}
            \FlagsRLCr
            \OpCode{11}{111}{101}
            \Hex{FD}{4}
            \Cycles{6}{23}
        \SkipToSymbol
            \Symbol{\SymRLCu[1]{r}{IY}}
            \FromSymbolToOpCode
            \OpCode{11}{001}{011}
            \Hex{CB}{}
        \SkipToSymbol
            \Symbol{\SymRLCu[2]{r}{IY}}
            \FromSymbolToOpCode
            \OpRange{d} 
            \Hex{..}{}
        \SkipToOpCode
            \OpCode{00}{\fbox{000}}{\OCT{r}}
            \Hex{..}{}
    \end{instruction}

    \StartWithOpCode\OpCode{}{$\uparrow$}{} \\

    \begin{instruction}{RRC m\See{1}}
        \Symbol{\SymRRC[0]{m}}
            \FlagsRRCr
            \OpCode{..}{\fbox{001}}{...}
    \end{instruction}

    \begin{instruction}{RL m\See{1}}
        \Symbol{\SymRL[0]{m}}
            \FlagsRLr
            \OpCode{..}{\fbox{010}}{...}
    \end{instruction}

    \begin{instruction}{RR m\See{1}}
        \Symbol{\SymRR[0]{m}}
            \FlagsRRr
            \OpCode{..}{\fbox{011}}{...}
    \end{instruction}

    \begin{instruction}{SLA m\See{1}}
        \Symbol{\vspace*{2pt}\SymSLA[0]{m}}
            \FlagsSLAr
            \OpCode{..}{\fbox{100}}{...}
    \end{instruction}

    \begin{instruction}{SRA m\See{1}}
        \Symbol{\vspace*{3pt}\SymSRA[0]{m}}
            \FlagsSRAr
            \OpCode{..}{\fbox{101}}{...}
    \end{instruction}

    \begin{instruction}{SLI m\See{1,2}}
        \Symbol{\vspace*{2pt}\SymSLI[0]{m}}
            \FlagsSLIr
            \OpCode{..}{\fbox{110}}{...}
    \end{instruction}

    \begin{instruction}{SRL m\See{1}}
        \Symbol{\vspace*{2pt}\SymSRL[0]{m}}
            \FlagsSRLr
            \OpCode{..}{\fbox{111}}{...}
    \end{instruction}

    \begin{instruction}{SLL m\See{3}}
    \end{instruction}

    \Empty{}
	
    \begin{instruction}{RLA}
        \Symbol{\SymRL[0]{A}}
            \FlagsRLA
            \OpCode{00}{010}{111}
            \Hex{17}{1}
            \Cycles{1}{4}
    \end{instruction}
	
    \begin{instruction}{RLCA}
        \Symbol{\SymRLC[0]{A}}
            \FlagsRLCA
            \OpCode{00}{000}{111}
            \Hex{07}{1}
            \Cycles{1}{4}
    \end{instruction}
	
    \begin{instruction}{RRA}
        \Symbol{\SymRR[0]{A}}
            \FlagsRRA
            \OpCode{00}{011}{111}
            \Hex{1F}{1}
            \Cycles{1}{4}
    \end{instruction}
	
    \begin{instruction}{RRCA}
        \Symbol{\SymRRC[0]{A}}
            \FlagsRRCA
            \OpCode{00}{001}{111}
            \Hex{0F}{1}
            \Cycles{1}{4}
    \end{instruction}

    \Empty{}

    \begin{instruction}{RLD}
        \Symbol{\vspace*{-12pt}\hspace*{-0.8cm}\SymRLD}
            \FlagsRLD 
            \OpCode{11}{101}{101}
            \Hex{ED}{2}
            \Cycles{5}{18}
        \SkipToOpCode 
            \OpCode{01}{101}{111}
            \Hex{6F}{}
    \end{instruction}
			
    \begin{lastinstruction}{RRD}
        \Symbol{\vspace*{-12pt}\hspace*{-0.8cm}\SymRRD}
            \FlagsRRD
            \OpCode{11}{101}{101}
            \Hex{ED}{2}
            \Cycles{5}{18}
        \SkipToOpCode 
            \OpCode{01}{100}{111}
            \Hex{67}{}
    \end{lastinstruction}
			
\end{instrtable}

\NoteTableSingleItemSpaceCorrection

\begin{notestable}
    \NoteItem{\See{1}{\tt m} is one of {\tt r}, {\tt (HL)}, {\tt (IX+d)}, {\tt (IY+d)}. To form new opcode replace \fbox{000} of {\tt RLC}s with shown code. Ts also the same\rule{0pt}{3.5ex}}

    \NoteItem{\See{2}Some assemblers may also allow {\tt SL1} to be used instead of {\tt SLI}}

    \NoteItem{\See{3}Shift Left Logical; no associated opcode, there is no difference between logical and arithmetic shift left, use {\tt SLA} for both. Some assemblers will allow {\tt SLL} as equivalent, but unfortunately some will assemble it as {\tt SLI}, so it's best avoiding}
\end{notestable}

\end{minipage}
