\section{Interrupts on ZX Spectrum Next}
\label{zx_next_interrupts}

% ─────────────────────────────────────────────────────────────────────
% ─██████████─██████──────────██████─██████████████─████████████████───
% ─██░░░░░░██─██░░██████████──██░░██─██░░░░░░░░░░██─██░░░░░░░░░░░░██───
% ─████░░████─██░░░░░░░░░░██──██░░██─██████░░██████─██░░████████░░██───
% ───██░░██───██░░██████░░██──██░░██─────██░░██─────██░░██────██░░██───
% ───██░░██───██░░██──██░░██──██░░██─────██░░██─────██░░████████░░██───
% ───██░░██───██░░██──██░░██──██░░██─────██░░██─────██░░░░░░░░░░░░██───
% ───██░░██───██░░██──██░░██──██░░██─────██░░██─────██░░██████░░████───
% ───██░░██───██░░██──██░░██████░░██─────██░░██─────██░░██──██░░██─────
% ─████░░████─██░░██──██░░░░░░░░░░██─────██░░██─────██░░██──██░░██████─
% ─██░░░░░░██─██░░██──██████████░░██─────██░░██─────██░░██──██░░░░░░██─
% ─██████████─██████──────────██████─────██████─────██████──██████████─
% ─────────────────────────────────────────────────────────────────────

Maskable interrupts on ZX Spectrum:

\begin{itemize}
	\item {\textbf Mode 0}: meant for interrupts triggered by external device. Instruction to be executed needs to be placed on data bus ({\tt RST} or {\tt CALL} for example). On ZX Spectrum this is the mode that is enabled by default when device powers up. However ROM soon sets up mode 1.
	
	\item {\textbf Mode 1}: on ZX Spectrum, this interrupt is triggered by vertical blanking if the screen refresh, roughly 50 times per second. When this occurs, current contents of {\tt PC} counter are pushed onto stack {\tt SP}, then address of \MemAddr{0038} is loaded and program stored on that location will start running. On ZX Spectrum Next this interrupt is responsible for updating system variable frame counter and scanning the keyboard.
	
	\item {\textbf Mode 2}: similar to {\tt IM 1} in frequency and handling, but uses vector table to jump to interrupt program instead of executing hard code ROM routine thus allowing user to setup their own interrupt handler.
\end{itemize}

On ZX Spectrum Next interrupt handler can be replaced by either:

\begin{itemize}
	\item Setting Z80 to {\tt IM 2} mode and configuring custom interrupt handler routine
	
	\item Page out ROM (as described in section \ref{zx_next_memorypaging}) and replace it with RAM page with custom interrupt routine at address \MemAddr{0038}
\end{itemize}

You can also adjust timing of the interrupts with Next/TBBlue Feature Control Registers {\tt \$22} and {\tt \$23}.

\pagebreak
Example of setting up custom interrupt vectors with {\tt IM 2}\footnote{Based on \url{http://codersbucket.blogspot.com/2015/04/interrupts-on-zx-spectrum-what-are.html}}:

\begin{lstlisting}
	DI
	
	LD HL, vectorTable  ; HL=address of vector table
	LD DE, IM2Handler   ; DE=address of IM2 handler routine
	LD B, 128           ; 128 vector addresses

	LD A, H
	LD I, A             ; I=high byte of vector table

setupVectorTable:       ; fills in vector table 128x
	LD (HL), E
	INC HL
	LD (HL), D
	INC HL
	DJNZ setupVectorTable 

	IM 2                ; enable mode 2 interrupts
	EI
	RET

IM2Handler:             ; this is called every time IM2 interrupt occurs
	...                 ; implement interrupt handler here
	EI                  ; when complete call EI
	RETI                ; end return from interrupt

	ORG $F000           ; needs to be on 256 byte boundary
vectorTable:
	DEFS 256            ; 128x2
\end{lstlisting}



\pagebreak
