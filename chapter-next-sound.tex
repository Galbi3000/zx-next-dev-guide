\section{Sound}
\label{zx_next_sound}

% ────────────────────────────────────────────────────────────────────────────────────
% ─██████████████─██████████████─██████──██████─██████──────────██████─████████████───
% ─██░░░░░░░░░░██─██░░░░░░░░░░██─██░░██──██░░██─██░░██████████──██░░██─██░░░░░░░░████─
% ─██░░██████████─██░░██████░░██─██░░██──██░░██─██░░░░░░░░░░██──██░░██─██░░████░░░░██─
% ─██░░██─────────██░░██──██░░██─██░░██──██░░██─██░░██████░░██──██░░██─██░░██──██░░██─
% ─██░░██████████─██░░██──██░░██─██░░██──██░░██─██░░██──██░░██──██░░██─██░░██──██░░██─
% ─██░░░░░░░░░░██─██░░██──██░░██─██░░██──██░░██─██░░██──██░░██──██░░██─██░░██──██░░██─
% ─██████████░░██─██░░██──██░░██─██░░██──██░░██─██░░██──██░░██──██░░██─██░░██──██░░██─
% ─────────██░░██─██░░██──██░░██─██░░██──██░░██─██░░██──██░░██████░░██─██░░██──██░░██─
% ─██████████░░██─██░░██████░░██─██░░██████░░██─██░░██──██░░░░░░░░░░██─██░░████░░░░██─
% ─██░░░░░░░░░░██─██░░░░░░░░░░██─██░░░░░░░░░░██─██░░██──██████████░░██─██░░░░░░░░████─
% ─██████████████─██████████████─██████████████─██████──────────██████─████████████───
% ────────────────────────────────────────────────────────────────────────────────────

Next inherits the same 3 AY-3-8912 chips setup as used in 128K Spectrums. This allows us to reuse many of the pre-existing applications and routines to play sound effects and music.

\subsection{AY Chip Registers}

AY chip has 3 sound channels, called A, B and C. Combined with 3 chips, this allows us to produce 9 channel music. Programming wise, each of the 3 chips needs to be selected first via \PortLink{Turbo Sound Next Control}{FFFD} register. Afterwards, we can set various parameters through \PortLink{Peripheral 3 Register}{08} and \PortLink{Peripheral 4 Register}{09}.

AY chip is controlled by 14 internal registers. To program them, we first need to select the register with \PortLink{Turbo Sound Next Control}{FFFD} and then write the value with \PortLink{Sound Chip Register Write}{BFFD}.


\subsection{Editing and Players}

Several applications can produce sounds or music compatible with the AY chip. For sounds, Shiru's AYFX Player\footnote{\url{https://shiru.untergrund.net/software.shtml\#old}} can be used. This program also includes a Z80 native player that can directly load and play sound effects. Alternatively, Remy's AY audio generator website\footnote{\url{https://zx.remysharp.com/audio/}} can produce exactly the same results and is fully compatible with AYFX Player.

A different way of playing sounds is to convert the WAV file into 1, 2 or 4-bit per sample sound with the ChibiWave application. Sounds take a bit more memory this way but are much easier to create. You can find the application, as well as tutorial and playback source code on Chibi Akumas website\footnote{\url{https://www.chibiakumas.com/z80/platform4.php\#LessonP35}}. While there, definitely check other tutorials too - they're all high quality and available as both, written posts and YouTube videos.

For creating music there are also several options. NextDAW\footnote{\url{https://nextdaw.biasillo.com/}} is native composer that runs on ZX Spectrum Next itself. Or if you prefer cross-platform, Arkos Tracker\footnote{\url{https://www.julien-nevo.com/arkostracker/}} or Vortex Tracker\footnote{\url{https://bulba.untergrund.net/vortex_e.htm}} should do the job. All include ``drivers''; Z80 code you can include in your program that can load and play created music.


\pagebreak
\subsection{Examples}

Before we can start playing sounds, we need to enable the sound hardware. While this is usually enabled by default, it's nonetheless a good idea to ensure our program will always run under the same conditions.

\begin{tcblisting}{}
	; Setup Turbo Sound chip
	LD BC, &FFFD            ; Turbo Sound Next Control Register
	LD A, %11111101         ; Enable left+right audio, select AY1
	OUT (C), A
	
	; Setup mapping of chip channels to stereo channels
	NEXTREG &08, %00010010  ; Use ABC, enable internal speaker & turbosound
	NEXTREG &09, %11100000  ; Enable mono for AY1-3
\end{tcblisting}

Programming AY consists of writing various values to its registers. As mentioned, this is a two-step process: first select register number, then write the value. Multiple writes are required for each tone to set period, volume etc. To make it simpler, I created a subroutine. It takes 2 parameters: {\tt A} for register number ({\tt 0-13}) and {\tt D} with value to write.

\begin{tcblisting}{}
WriteDToAYReg:
	; Select desired register
	LD BC, &FFFD
	OUT (C), A
	
	; Write given value
	LD A, D
	LD BC, &BFFD
	OUT (C), A
	
	RET
\end{tcblisting}

Companion code on GitHub, folder {\tt sound} includes expanded code as well as a simple player that plays multiple tones in sequence. For the purposes of this book, I used Remy's AY audio generator website to load one of the example effects, then manually copied raw values into the source code. Laborious process to say the least - this is not how effects should be handled in real life. But I wanted to learn and demonstrate how to program AY chip, not how to use ready-made drivers to play effects or music. Furthermore, my ``player'' blocks the main loop; ideally, sound effects and music would play on the interrupt handler. This could be a nice homework for the reader - example in section \ref{zx_next_interrupts} should give you an idea of how to achieve this - happy coding!


\pagebreak
\subsection{Sound Ports and Registers}
\label{zx_next_sound_registers}

\subsubsection{\PortTitle{Turbo Sound Next Control}{FFFD}}

When bit {\tt 7} is {\tt 1}:

\begin{NextPort}
	\PortBits{7}
		\PortDesc{{\tt 1}}
	\PortBits{6}
		\PortDesc{{\tt 1} to enable left audio}
	\PortBits{5}
		\PortDesc{{\tt 1} to enable right audio}
	\PortBits{4-2}
		\PortDesc{Must be {\tt 1}}
	\PortBits{1-0}
		\PortDesc{Selects active chip:}
		\PortDescOnly{
			\begin{PortBitConfig}
				\PortBitLine{00}{Unused}
				\PortBitLine{01}{AY3}
				\PortBitLine{10}{AY2}
				\PortBitLine{11}{AY1}
			\end{PortBitConfig}
		}
\end{NextPort}

When bit {\tt 7} is {\tt 0}:

\begin{NextPort}
	\PortBits{7}
		\PortDesc{{\tt 0}}
	\PortBits{6-0}
		\PortDesc{Selects given AY register number for read or write from active sound chip} 
\end{NextPort}


\subsubsection{\PortTitle{Sound Chip Register Write}{BFFD}}

\begin{NextPort}
	\PortBits{7-0}
		\PortDesc{Writes given value to currently selected register:}
\end{NextPort}

\begingroup
	\small
	\vspace*{8pt}					% some vertical spacing between port table and this sectino
	\setlength{\leftskip}{1.25cm}	% Setup left margin as if this whole section was within port table (tried adding it there but got syntax errors)

	\newcommand{\AYRegTitle}[1]{
		\subsubsection{#1}
		\vspace*{-1ex}
	}

	\begin{multicols}{2}
		\AYRegTitle{0 - Channel A tone, low byte}
		\begin{BitTableByte}
			\BitStartMulti{8}{A tone} \\
		\end{BitTableByte}
		
		\AYRegTitle{1 - Channel A tone, high 4-bits}
		\begin{BitTableByte}
			\BitMono{0} & \BitMono{0} & \BitMono{0} & \BitMono{0} & \BitMulti{4}{A tone high} \\
		\end{BitTableByte}		
	\end{multicols}

	\begin{multicols}{2}
		\AYRegTitle{2 - Channel B tone, low byte}
		\begin{BitTableByte}
			\BitStartMulti{8}{B tone} \\
		\end{BitTableByte}

		\AYRegTitle{3 - Channel B tone, high 4-bits}
		\begin{BitTableByte}
			\BitMono{0} & \BitMono{0} & \BitMono{0} & \BitMono{0} & \BitMulti{4}{B tone high} \\
		\end{BitTableByte}
	\end{multicols}

	\begin{multicols}{2}
		\AYRegTitle{4 - Channel C tone, low byte}
		\begin{BitTableByte}
			\BitStartMulti{8}{C tone} \\
		\end{BitTableByte}

		\AYRegTitle{5 - Channel C tone, high 4-bits}
		\begin{BitTableByte}
			\BitMono{0} & \BitMono{0} & \BitMono{0} & \BitMono{0} & \BitMulti{4}{C tone high} \\
		\end{BitTableByte}
	\end{multicols}

	\pagebreak
	\begin{multicols}{2}
		\AYRegTitle{6 - Noise period}
		\begin{BitTableByte}
			\BitMono{0} & \BitMono{0} & \BitMono{0} & \BitMulti{5}{Noise Period} \\
		\end{BitTableByte}
		\leavevmode	% for some reason new line \\ doesn't work here with BitTableByte/ElegantTable

		\columnbreak

		\AYRegTitle{7 - Flags}
		\begin{BitTableByte}
			\BitMono{0} & \BitMono{0} & \BitMono{C} & \BitMono{B} & \BitMono{A} & \BitMono{C} & \BitMono{B} & \BitMono{A} \\
			\hline
			\BitMono{0} & \BitMono{0} & \BitMulti{3}{Noise} & \BitMulti{3}{Tone} \\
		\end{BitTableByte}
	\end{multicols}

	\begin{multicols}{2}
		\AYRegTitle{8 - Channel A volume/envelope}
		\begin{BitTableByte}
			\BitMono{0} & \BitMono{0} & \BitMono{0} & \BitMono{0} & \BitMulti{4}{A Volume} \\
		\end{BitTableByte}

		\AYRegTitle{9 - Channel B volume/envelope}
		\begin{BitTableByte}
			\BitMono{0} & \BitMono{0} & \BitMono{0} & \BitMono{0} & \BitMulti{4}{B Volume} \\
		\end{BitTableByte}
	\end{multicols}

	\begin{multicols}{2}
		\AYRegTitle{10 - Channel C volume/envelope}
		\begin{BitTableByte}
			\BitMono{0} & \BitMono{0} & \BitMono{0} & \BitMono{0} & \BitMulti{4}{C Volume} \\
		\end{BitTableByte}

		\columnbreak

		\textbf{Note:} Registers {\tt 8-10} work as volume control if bit {\tt 4} is {\tt 0}, otherwise envelop generator is
		used (see registers {\tt 11-13}). In this case bits {\tt 3-0} are ignored.
	\end{multicols}

	\begin{multicols}{2}
		\AYRegTitle{11 - Envelope period fine}
		\begin{BitTableByte}
			\BitStartMulti{8}{Envelope bits 7-0} \\
		\end{BitTableByte}

		\AYRegTitle{12 - Envelope period coarse}
		\begin{BitTableByte}
			\BitStartMulti{8}{Envelope bits 15-8} \\
		\end{BitTableByte}
	\end{multicols}

	\begin{multicols}{2}
		\AYRegTitle{13 - Envelope shape}
		\begin{BitTableByte}
			\BitMono{0} & \BitMono{0} & \BitMono{0} & \BitMono{0} & \BitMono{$C$} & \BitMono{$A_t$} & \BitMono{$A_l$} & \BitMono{$H$} \\
		\end{BitTableByte}
	\end{multicols}

	\hspace*{1.3cm}
	\begin{tabular}{lllp{18cm}}
		$H$ & \multicolumn{3}{l}{``Hold''} \\
			& {\tt 1} & \multicolumn{2}{p{10cm}}{envelope generator performs 1 cycle then holds the end value} \\
			& {\tt 0} & \multicolumn{2}{p{10cm}}{cycles continuously} \\

		$A_l$ & \multicolumn{3}{l}{``Alternate''} \\
			& \multicolumn{3}{l}{If ``hold'' set} \\
			& & {\tt 1} & the value held is initial value \\
			& & {\tt 0} & the value held is the final value \\
			& \multicolumn{3}{l}{If ``hold'' not set} \\
			& & {\tt 1} & envelope generator alters direction after each cycle \\
			& & {\tt 0} & resets after each cycle \\

		$A_t$ & \multicolumn{3}{l}{``Attack''} \\
			& {\tt 1} & \multicolumn{2}{l}{the generator counts up} \\
			& {\tt 0} & \multicolumn{2}{l}{the generator counts down} \\

		$C$ & \multicolumn{3}{l}{``Continue''} \\
			& {\tt 1} & \multicolumn{2}{l}{``hold'' is followed} \\
			& {\tt 0} & \multicolumn{2}{p{12cm}}{the envelope generator performs one cycle then drops volume to 0 and stays there, overriding ``hold''} \\
	\end{tabular}
\endgroup

\subsubsection{\PortTitle{Peripheral 2 Register}{06}}

\begin{NextPort}
	\PortBits{7}
		\PortDesc{{\tt 1} to enable CPU speed mode key "F8", {\tt 0} to disable ({\tt 1} after soft reset)}
	\PortBits{6}
		\PortDesc{Core 3.1.2+: Divert BEEP-only to internal speaker ({\tt 0} after hard reset)}
		\PortDescOnly{Pre core 3.1.2: DMA mode, {\tt 0} zxnDMA, {\tt 1} Z80 DMA ({\tt 0} after hard reset)}
	\PortBits{5}
		\PortDesc{Core 2.0+: {\tt 1} to enable "F3" key (50/60 Hz switch) ({\tt 1} after soft reset)}
		\PortDescOnly{Pre core 2.0: "Enable Lightpen"}
	\PortBits{4}
		\PortDesc{{\tt 1} to enable DivMMC automap and DivMMC NMI by DRIVE button ({\tt 0} after hard reset)}
	\PortBits{3}
		\PortDesc{{\tt 1} to enable multiface NMI by M1 button ({\tt 0} after hard reset)}
	\PortBits{2}
		\PortDesc{{\tt 1} to set primary device to mouse in PS/2 mode, {\tt 0} to set to keyboard}
	\PortBits{1-0}
		\PortDesc{Audio chip mode:}
		\PortDescOnly{
			\begin{PortBitConfig}
				\PortBitLine{00}{YM}				
				\PortBitLine{01}{AY}
				\PortBitLine{10}{Disabled}
				\PortBitLine{11}{Core 3.0+: Hold all AY in reset}
			\end{PortBitConfig}
		}
\end{NextPort}

\subsubsection{\PortTitle{Peripheral 3 Register}{08}}

\begin{NextPort}
	\PortBits{7}
		\PortDesc{{\tt 1} unlock / {\tt 0} lock port \MemAddr{7FFD} paging}
	\PortBits{6}
		\PortDesc{{\tt 1} to disable RAM and I/O port contention ({\tt 0} after soft reset)}
	\PortBits{5}
		\PortDesc{AY stereo mode ({\tt 0} = ABC, {\tt 1} = ACB) ({\tt 0} after hard reset)}
	\PortBits{4}
		\PortDesc{Enable internal speaker ({\tt 1} after hard reset)}
	\PortBits{3}
		\PortDesc{Enable 8-bit DACs (A,B,C,D) ({\tt 0} after hard reset)}
	\PortBits{2}
		\PortDesc{Enable port \MemAddr{FF} Timex video mode read ({\tt 0} after hard reset)}
	\PortBits{1}
		\PortDesc{Enable Turbosound (currently selected AY is frozen when disabled) ({\tt 0} after hard reset)}
	\PortBits{0}
		\PortDesc{Implement Issue 2 keyboard (port \MemAddr{FE} reads as early ZX boards) ({\tt 0} after hard reset)}
\end{NextPort}

\subsubsection{\PortTitleLink{Peripheral 4 Register}{09}}

\PortReference{zx_next_sprites_registers}{Sprites}


\pagebreak
\IntentionallyEmpty
\pagebreak
