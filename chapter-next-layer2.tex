\section{Layer 2}
\label{zx_next_layer2}

% ───────────────────────────────
% ─██████─────────██████████████─
% ─██░░██─────────██░░░░░░░░░░██─
% ─██░░██─────────██████████░░██─
% ─██░░██─────────────────██░░██─
% ─██░░██─────────██████████░░██─
% ─██░░██─────────██░░░░░░░░░░██─
% ─██░░██─────────██░░██████████─
% ─██░░██─────────██░░██─────────
% ─██░░██████████─██░░██████████─
% ─██░░░░░░░░░░██─██░░░░░░░░░░██─
% ─██████████████─██████████████─
% ───────────────────────────────

As we saw in the previous section, drawing with ULA graphics is much simplified on Next. But it can't eliminate the colour clash. Well, not with ULA mode at least. However, Next brings a couple of brand new graphic modes to the table, hidden behind a somewhat casual name ``Layer 2''. But don't let its name deceive you; Layer 2 raises Next graphics capabilities to a whole new level!

Layer 2 may appear behind or above the ULA layer. It supports different resolutions with every pixel coloured independently and memory organized sequentially, line by line, pixel by pixel. Consequently, Layer 2 requires more memory compared to ULA; each mode needs multiple 16K banks. But of course, Next has far more memory than the original Speccy ever did!

\begin{tabularx}{\textwidth}{cccX}
	\BitHead{Resolution} & \BitHead{Colours} & \BitHead{BPP} & \BitHead{Memory Organization} \\
	256$\times$192 & 256 & 8 & 48K, 3 horizontal banks of 64 lines \\
	320$\times$256 & 256 & 8 & 80K, 5 vertical banks of 64 columns\footnote{Core 3.0.6+ only} \\
	640$\times$256 & 16 & 4 & 80K, 5 vertical banks of 128 columns\footnotemark[\value{footnote}] \\
\end{tabularx}


\subsection{Initialization}

Drawing on Layer 2 is much simpler than using ULA mode. But in contrast with ULA, which is always ``on'', Layer 2 needs to be explicitly enabled. This is done by setting bit {\tt 1} of \PortLink{Layer 2 Access Port}{123B}.

By default, Layer 2 will use 256$\times$192 with 256 colours, supported across all Next core versions. You can select another resolution with \PortLink{Layer 2 Control Register}{70}. 320$\times$256 and 640$\times$256 modes also require setting up clip window correctly with \PortLink{Clip Window Layer 2 Register}{18}.


\subsection{Paging}

After Layer 2 is enabled, we can start writing into memory banks. As mentioned above, Layer 2 requires 3-5 contiguous 16K banks. While Next initializes default configuration during boot, it's nonetheless a good idea to set it up manually to ensure our code will work across all devices. \PortLink{Layer 2 Ram Page Register}{12} selects the bank number where Layer 2 video memory begins. Note it's a good idea to store the original bank values so we can restore them afterwards.

All supported modes can be used for paging, as described in section \ref{zx_next_memorypaging}, by swapping in bank numbers to 16K slot at \MemAddr{C000}. However, the simplest and most versatile is MMU mode; MMU6 and MMU7 registers correspond to 2 8K slots starting at \MemAddr{C000}.

\pagebreak
\subsection{Drawing}

In general, drawing pixels requires the programmer to:

\begin{itemize}[topsep=1pt,itemsep=1pt]
	\item Determine and select bank to write to
	\item Calculate address of the pixel within the bank
	\item Write byte with colour data
\end{itemize}

All Layer 2 modes use the same approach when drawing pixels. Each pixel uses one byte (except 640$\times$320 where each byte contains data for 2 pixels). The value is simply an index into the palette entries list. Similar to other layers, Layer 2 also has two palettes, of which only one can be active at any given time. \PortLink{Enhanced ULA Control Register}{43} is used to select active palette. See Palette chapter \ref{zx_next_palette} for details on how to program palettes.

See specific modes in the following pages for examples of writing pixel data.


\subsection{Effects}

\PortLink{Sprite and Layers System Register}{15} can be used to change Layer 2 priority, effectively moving Layer 2 above or below other layers - see Tilemap chapter, section \ref{zx_next_tilemap_registers} for details.

We can even be more specific and only prioritize specific colours, so only pixels using those colours will appear on top while other pixels below other layers. This way we can achieve a simple depth effect. Per-pixel priority is available when writing a custom palette with \PortLink{Enhanced ULA Palette Extension}{44} (9-bit colours). See description under Palette chapter, section \ref{zx_next_palette} for details on how to program palette.

We can also use both Layer 2 palettes to achieve simple effects. For example, certain colours can be marked with the priority flag on one palette but not on the other. When swapping palettes, pixels drawn with these colours would appear on top or below other layers. Another simple effect using both palettes could be colour animation, though it can't be very smooth with only two states.

\PortLink{Global Transparency Register}{14} can be used to alter the transparent colour of Layer 2. This same register also affects ULA, LoRes and 1-bit (``text mode'') tilemap. 

Scrolling effects can be achieved by writing pixel offsets to \PortLink{Layer 2 X Offset Register}{16}, \PortLink{Layer 2 X Offset MSB Register}{71} and \PortLink{Layer 2 Y Offset Register}{17}.





\pagebreak
\subsection{256$\times$192 256 Colour Mode}

\begin{multicols}{2}
	3 horizontal banks:

	\begin{tabularx}{0.455\textwidth}{c|YY|YY|}
		\multicolumn{1}{l}{} & 
			\multicolumn{1}{l}{0} &
			\multicolumn{2}{c}{\dots} &
			\multicolumn{1}{r}{255} \\
		\cline{2-5}
			0 & 
			\multicolumn{2}{l|}{16K BANK 0} & 
			\multicolumn{2}{l|}{8K BANK 0} \\
		\multirow{2}{*}{\vdots} & & & 
			\multicolumn{2}{l|}{0\dots 31} \\
		\cline{4-5}
			& & & \multicolumn{2}{l|}{8K BANK 1} \\
			63 & & & \multicolumn{2}{l|}{32 \dots 63} \\
		\cline{2-5}
			64 & 
			\multicolumn{2}{l|}{16K BANK 1} & 
			\multicolumn{2}{l|}{8K BANK 2} \\
		\multirow{2}{*}{\vdots} & & & 
			\multicolumn{2}{l|}{64 \dots 95} \\
		\cline{4-5}
			& & & \multicolumn{2}{l|}{8K BANK 3} \\
			127 & & & \multicolumn{2}{l|}{96 \dots 127} \\
		\cline{2-5}
			128 & 
			\multicolumn{2}{l|}{16K BANK 2} & 
			\multicolumn{2}{l|}{8K BANK 4} \\
		\multirow{2}{*}{\vdots} & & & 
			\multicolumn{2}{l|}{128 \dots 159} \\
		\cline{4-5}
			& & & \multicolumn{2}{l|}{8K BANK 5} \\
			191 & & & \multicolumn{2}{l|}{160 \dots 191} \\
		\cline{2-5}
	\end{tabularx}

	\columnbreak
	8BPP:\\

	\begin{BitTableByte}
		\BitSmall{$I_7$} & 
			\BitSmall{$I_6$} & 
			\BitSmall{$I_5$} &
			\BitSmall{$I_4$} &
			\BitSmall{$I_3$} & 
			\BitSmall{$I_2$} &
			\BitSmall{$I_1$} &
			\BitSmall{$I_0$} \\
		\hline
		\BitStartMulti{8}{Colour index} \\
	\end{BitTableByte}

	Banking Setup:

	\begin{tabularx}{0.445\textwidth}{|c|c|c|c|Y|}
		\hline
		\BitHead{15} & 
			\BitHead{14} & 
			\BitHead{13} &
			\BitHead{12-8} &
			\BitHead{7-0} \\
		\hline
		\BitStartMulti{4}{$Y$} & 
			\BitMulti{1}{$X$} \\
		\hline
		\BitStartMulti{2}{16K} &
			\BitMulti{2}{$Y_{5-0}$} &
			\BitMulti{1}{$X$} \\
		\hline
		\BitStartMulti{3}{8K} &
			\BitMulti{1}{$Y_{4-0}$} &
			\BitMulti{1}{$X$} \\
		\hline
	\end{tabularx}
\end{multicols}

This mode is the closest to ULA, resolution wise, so is perhaps the simplest to grasp. It's also supported across all Next core versions. Pixels are laid out from left to right and top to bottom. Each pixel uses one byte that represents an 8-bit index into the palette. 3 16K banks are needed to cover the whole screen, each holding data for 64 lines. Or, if using 8K, 6 banks, 32 lines each. Combined, colour data requires 48K of memory.

Each (x,y) coordinate pair requires 16-bits. If the upper byte is used for Y and lower for the X coordinate, together they will form exact memory location offset from the top of the first bank. But to account for bank swapping; for 16K banks, the most significant 2 bits of Y correspond to bank number and for 8K banks, top 3 bits. The rest of Y + X is memory location within the bank.

Example of filling the screen with a vertical rainbow:

\begin{tcblisting}{}
START_16K_BANK  EQU 9
START_8K_BANK   EQU START_16K_BANK*2

	; Enable Layer 2
	LD BC, &123B
	LD A, 2
	OUT (C), A
	
	; Setup starting Layer2 16K bank
	NEXTREG &12, START_16K_BANK
	
	LD D, 0                   ; D=Y, start at top of the screen
	
nextY:
	; Calculate bank number and swap it in
	LD A, D                   ; Copy current Y to A
	AND %11100000             ; 32100000 (3MSB = bank number)
	RLCA                      ; 21000003
	RLCA                      ; 10000032
	RLCA                      ; 00000321
	ADD A, START_8K_BANK      ; A=bank number to swap in
	NEXTREG &56, A            ; Swap bank
	
	; Convert DE (yx) to screen memory location starting at &C000
	PUSH DE                   ; (DE) will be changed to bank offset
	LD A, D                   ; Copy current Y to A
	AND %00011111             ; Discard bank number
	OR &C0                    ; Screen starts at &C000
	LD D, A                   ; D=high byte for &C000 screen memory

	; Loop X through 0..255; we don't have to deal with bank swapping
	; here because it only occurs when changing Y
	LD E, 0
nextX:
	LD A, E                   ; A=current X
	LD (DE), A                ; Use X as colour index
	INC E                     ; Increment to next X
	JR NZ, nextX              ; Repeat until E rolls over
	
	; Continue with next line or exit
	POP DE                    ; Restore DE to coordinates
	INC D                     ; Increment to next Y
	LD A, D                   ; A=current Y
	CP 192                    ; Did we just complete last line?
	JP C, nextY               ; No, continue with next linee
\end{tcblisting}

Worth noting: MMU page 6 (next register \MemAddr{56}) covers memory \MemAddr{C000} - \MemAddr{DFFF}. As we swap different 8K banks there, we're effectively changing 8K banks that are readable and writable at those memory addresses. That's why we {\tt OR \$C0} in line 24; we need to convert zero based address to \MemAddr{C000} based. See section \ref{zx_next_bank_mmu_mode} for details on MMU paging mode.

We don't have to handle bank swapping on every iteration; once per 32 rows would do for this example. But the code is more versatile this way and could be easily converted into a reusable pixel setting routine.


\pagebreak
\subsection{320$\times$256 256 Colour Mode}

\begin{multicols}{2}
	5 vertical banks:

	\begin{tabularx}{0.455\textwidth}{l|X|X|X|X|X|X|X|X|X|X|}
		\multicolumn{1}{l}{} &
			\multicolumn{1}{l}{0} &
			\multicolumn{7}{X}{} &
			\multicolumn{2}{r}{319} \\
		\cline{2-11}
		\rotatebox[origin=c]{90}{~~~~~~~~~~~~~~0} &
			\multicolumn{2}{X|}{\rotatebox[origin=c]{90}{~16K BANK 0~}} &
			\multicolumn{2}{X|}{\rotatebox[origin=c]{90}{16K BANK 1}} &
			\multicolumn{2}{X|}{\rotatebox[origin=c]{90}{16K BANK 2}} &
			\multicolumn{2}{X|}{\rotatebox[origin=c]{90}{16K BANK 3}} &
			\multicolumn{2}{X|}{\rotatebox[origin=c]{90}{16K BANK 4}} \\
		\cline{2-11}
		\rotatebox[origin=c]{90}{255~~~~~~~~~~~} &
			\rotatebox[origin=c]{90}{~8K BANK 0~} &
			\rotatebox[origin=c]{90}{8K BANK 1} &
			\rotatebox[origin=c]{90}{8K BANK 2} &
			\rotatebox[origin=c]{90}{8K BANK 3} &
			\rotatebox[origin=c]{90}{8K BANK 4} &
			\rotatebox[origin=c]{90}{8K BANK 5} &
			\rotatebox[origin=c]{90}{8K BANK 6} &
			\rotatebox[origin=c]{90}{8K BANK 7} &
			\rotatebox[origin=c]{90}{8K BANK 8} &
			\rotatebox[origin=c]{90}{8K BANK 9} \\
		\cline{2-11}
		\multicolumn{1}{c}{} & \multicolumn{10}{c}{} \\[-5pt]
		\multicolumn{1}{c}{} & 
			\multicolumn{10}{l}{16K bank contains 64 columns} \\
		\multicolumn{1}{c}{} & 
			\multicolumn{10}{l}{8K bank contains 32 columns} \\
	\end{tabularx}

	\columnbreak
	8BPP:\\

	\begin{BitTableByte}
		\BitSmall{$I_7$} & 
			\BitSmall{$I_6$} & 
			\BitSmall{$I_5$} &
			\BitSmall{$I_4$} &
			\BitSmall{$I_3$} & 
			\BitSmall{$I_2$} &
			\BitSmall{$I_1$} &
			\BitSmall{$I_0$} \\
		\hline
		\BitStartMulti{8}{Colour index} \\
	\end{BitTableByte}

	Banking Setup:

	\begin{tabularx}{0.445\textwidth}{|c|c|c|c|c|Y|}
		\hline
		\BitHead{16} &
			\BitHead{15} &
			\BitHead{14} &
			\BitHead{13} &
			\BitHead{12-8} &
			\BitHead{7-0} \\
		\hline
		\BitStartMulti{1}{$X_{8}$} & 
			\BitMulti{4}{$X_{7-0}$} &
			\BitMulti{1}{$Y$} \\
		\hline
		\BitStartMulti{3}{16K} &
			\BitMulti{2}{$X_{5-0}$} &
			\BitMulti{1}{$Y$} \\
		\hline
		\BitStartMulti{4}{8K} &
			\BitMulti{1}{$X_{4-0}$} &
			\BitMulti{1}{$Y$} \\
		\hline
	\end{tabularx}
\end{multicols}

320$\times$256 mode is only available on Next core 3.0.6 or later. Pixels are laid out from top to bottom and left to right. Each pixel uses one byte that represents an 8-bit index into the palette. To cover the whole screen, 5 16K banks of 64 columns or 10 8K banks of 32 columns are needed. Together colour data requires 80K of memory.

In contrast with 256$\times$192, this mode allows drawing to the whole screen, including border. In fact, you can think of it as the regular 256$\times$192 mode with additional 32 pixel border around (32 + 256 + 32 = 320 and 32 + 192 + 32 = 256).

Addressing is more complicated though. As we need 9 bits for X and 8 for Y, we can't address all screen pixels with single 16-bit register pair. But we can use 16-bit register pair to address all pixels within each bank. From this perspective, the setup is similar to 256$\times$192 mode, except that X and Y are reversed: if the upper byte is used for X and lower for Y, then most significant 2 bits of 16-bit register pair represent lower 2 bits of 16K bank number. And for 8K banks, the most significant 3 bits correspond to the lower 3 bits of 8K bank number. In either case, the most significant bit of the bank number arrives from the 9th bit of the X coordinate ($X_8$ in the table above). The rest of the X + Y is memory location within the bank.

To use this mode, we must explicitly select it with \PortLink{Layer 2 Control Register}{70}. We must also not forget to set clip window correctly with \PortLink{Clip Window Layer 2 Register}{18} and \PortLink{Clip/ Window Control Register}{1C}, as demonstrated in example below:

\begin{tcblisting}{}
START_16K_BANK  EQU 9
START_8K_BANK   EQU START_16K_BANK*2

RESOLUTION_X    EQU 320
RESOLUTION_Y    EQU 256

BANK_8K_SIZE    EQU 8192
NUM_BANKS       EQU RESOLUTION_X * RESOLUTION_Y / BANK_8K_SIZE
BANK_X          EQU BANK_8K_SIZE / RESOLUTION_Y

	; Enable Layer 2
	LD BC, &123B
	LD A, 2
	OUT (C), A

	; Setup starting Layer2 16K bank
	NEXTREG &12, START_16K_BANK
	NEXTREG &70, %00010000    ; 320x256 256 colour mode

	; Setup window clip for 320x256 resolution
	NEXTREG &1C, 1            ; Reset Layer 2 clip window reg index
	NEXTREG &18, 0            ; X1; X2 next line
	NEXTREG &18, RESOLUTION_X / 2 - 1
	NEXTREG &18, 0            ; Y1; Y2 next line
	NEXTREG &18, RESOLUTION_Y - 1

	LD B, START_8K_BANK       ; Bank number
	LD H, 0                   ; Colour index
nextBank:
	; Swap to next bank, exit once all 5 are done
	LD A, B                   ; Copy current bank number to A
	NEXTREG &56, A            ; Switch to bank

	; Fill in current bank
	LD DE, &C000              ; Prepare starting address
nextY:
	; Fill in 256 pixels of current line
	LD A, H                   ; Copy colour index to A
	LD (DE), A                ; Write colour index into memory
	INC E                     ; Increment Y
	JR NZ, nextY              ; Continue with next Y until we wrap to next X

	; Prepare for next line until bank is full
	INC H                     ; Increment colour
	INC D                     ; Increment X
	LD A, D                   ; Copy X to A
	AND %00111111             ; Clear &C0 to get pure X coordinate
	CP BANK_X                 ; Did we reach next bank?
	JP NZ, nextY              ; No, continue with next Y

	; Prepare for next bank
	INC B                     ; Increment to next bank
	LD A, B                   ; Copy bank to A
	CP START_8K_BANK+NUM_BANKS; Did we fill last bank?
	JP NZ, nextBank           ; No, proceed with next bank
\end{tcblisting}


\pagebreak
\subsection{640$\times$256 16 Colour Mode}

\begin{multicols}{2}
	5 vertical banks:

	\begin{tabularx}{0.455\textwidth}{l|X|X|X|X|X|X|X|X|X|X|}
		\multicolumn{1}{l}{} &
			\multicolumn{1}{l}{0} &
			\multicolumn{7}{X}{} &
			\multicolumn{2}{r}{639} \\
		\cline{2-11}
		\rotatebox[origin=c]{90}{~~~~~~~~~~~~~~0} &
			\multicolumn{2}{X|}{\rotatebox[origin=c]{90}{~16K BANK 0~}} &
			\multicolumn{2}{X|}{\rotatebox[origin=c]{90}{16K BANK 1}} &
			\multicolumn{2}{X|}{\rotatebox[origin=c]{90}{16K BANK 2}} &
			\multicolumn{2}{X|}{\rotatebox[origin=c]{90}{16K BANK 3}} &
			\multicolumn{2}{X|}{\rotatebox[origin=c]{90}{16K BANK 4}} \\
		\cline{2-11}
		\rotatebox[origin=c]{90}{255~~~~~~~~~~~} &
			\rotatebox[origin=c]{90}{~8K BANK 0~} &
			\rotatebox[origin=c]{90}{8K BANK 1} &
			\rotatebox[origin=c]{90}{8K BANK 2} &
			\rotatebox[origin=c]{90}{8K BANK 3} &
			\rotatebox[origin=c]{90}{8K BANK 4} &
			\rotatebox[origin=c]{90}{8K BANK 5} &
			\rotatebox[origin=c]{90}{8K BANK 6} &
			\rotatebox[origin=c]{90}{8K BANK 7} &
			\rotatebox[origin=c]{90}{8K BANK 8} &
			\rotatebox[origin=c]{90}{8K BANK 9} \\
		\cline{2-11}
		\multicolumn{1}{c}{} & \multicolumn{10}{c}{} \\[-5pt]
		\multicolumn{1}{c}{} & 
			\multicolumn{10}{l}{16K bank contains 128 columns} \\
		\multicolumn{1}{c}{} & 
			\multicolumn{10}{l}{8K bank contains 64 columns} \\
	\end{tabularx}

	\columnbreak
	4BPP:\\

	\begin{BitTableByte}
		\BitSmall{$I_3$} & 
			\BitSmall{$I_2$} & 
			\BitSmall{$I_1$} &
			\BitSmall{$I_0$} &
			\BitSmall{$I_3$} & 
			\BitSmall{$I_2$} &
			\BitSmall{$I_1$} &
			\BitSmall{$I_0$} \\
		\hline
		\BitStartMulti{4}{Colour 1} &
			\BitMulti{4}{Colour 2} \\
	\end{BitTableByte}

	Banking Setup:

	\begin{tabularx}{0.445\textwidth}{|c|c|c|c|c|Y|}
		\hline
		\BitHead{16} &
			\BitHead{15} &
			\BitHead{14} &
			\BitHead{13} &
			\BitHead{12-8} &
			\BitHead{7-0} \\
		\hline
		\BitStartMulti{1}{$X_{8}\times 2$} & 
			\BitMulti{4}{$X_{7-0}\times 2$} &
			\BitMulti{1}{$Y$} \\
		\hline
		\BitStartMulti{3}{16K} &
			\BitMulti{2}{$X_{5-0}\times 2$} &
			\BitMulti{1}{$Y$} \\
		\hline
		\BitStartMulti{4}{8K} &
			\BitMulti{1}{$X_{4-0}\times 2$} &
			\BitMulti{1}{$Y$} \\
		\hline
	\end{tabularx}
\end{multicols}

640$\times$256 mode is very similar to 320$\times$256, except that each byte represents 2 colours instead of 1. It's also available on Next core 3.0.6 or later only. Pixels are laid out from top to bottom and left to right. Each pixel takes 4 bits, so each byte contains data for 2 pixels. To cover the whole screen, 5 16K banks of 128 columns or 10 8K banks of 64 columns are needed. Together colour data requires 80K of memory. Similar to 320$\times$256, this mode also covers the whole screen, including the border.

Addressing wise, this mode is the same as 230$\times$256. Using 16-bit register pair we can't address all pixels on the screen, but we can address all pixels within each bank. Again, assuming upper byte of 16-bit register pair is used for X and lower for Y and using 9th bit of X coordinate (bit $X_8$ in the table above) as the most significant bit of bank number, then most significant 2 bits of 16-bit register pair represent lower 2 bits of 16K bank number. And for 8K banks, the most significant 3 bits correspond to the lower 3 bits of 8K bank number. The rest of the X + Y is memory location within the bank. Don't forget: each colour byte represents 2 screen pixels, so the memory X coordinate (as described above) needs to be multiplied by 2 to convert to screen X coordinate.

To use this mode, we must explicitly select it with \PortLink{Layer 2 Control Register}{70}. We must also not forget to set clip window correctly with \PortLink{Clip Window Layer 2 Register}{18} and \PortLink{Clip/ Window Control Register}{1C}, as demonstrated in example below:

\begin{tcblisting}{}
START_16K_BANK  EQU 9
START_8K_BANK   EQU START_16K_BANK*2

RESOLUTION_X    EQU 640
RESOLUTION_Y    EQU 256

BANK_8K_SIZE    EQU 8192
NUM_BANKS       EQU RESOLUTION_X * RESOLUTION_Y / BANK_8K_SIZE / 2
BANK_X          EQU BANK_8K_SIZE / RESOLUTION_Y

	; Enable Layer 2
	LD BC, &123B
	LD A, 2
	OUT (C), A

	; Setup starting Layer2 16K bank
	NEXTREG &12, START_16K_BANK
	NEXTREG &70, %00100000    ; 640x256 16 colour mode

	NEXTREG &1C, 1            ; Reset Layer 2 clip window reg index
	NEXTREG &18, 0
	NEXTREG &18, RESOLUTION_X / 4 - 1
	NEXTREG &18, 0
	NEXTREG &18, RESOLUTION_Y - 1

	LD B, START_8K_BANK       ; Bank number
	LD H, 0                   ; Colour index for 2 pixels
nextBank:
	; Swap to next bank, exit once all 5 are done
	LD A, B                   ; Copy current bank number to A
	NEXTREG &56, A            ; Switch to bank

	; Fill in current bank
	LD DE, &C000              ; Prepare starting address
nextY:
	; Fill in 256 pixels of current line
	LD A, H                   ; Copy colour indexes for 2 pixels to A
	LD (DE), A                ; Write colour indexes into memory
	INC E                     ; Increment Y
	JR NZ, nextY              ; Continue with next Y until we wrap to next X

	; Prepare for next line until bank is full
	INC H                     ; Increment colour index for both colours
	INC D                     ; Increment X
	LD A, D                   ; Copy X to A
	AND %00111111             ; Clear &C0 to get pure X coordinate
	CP BANK_X                 ; Did we reach next bank?
	JP NZ, nextY              ; No, continue with next Y

	; Prepare for next bank
	INC B                     ; Increment to next bank
	LD A, B                   ; Copy bank to A
	CP START_8K_BANK+NUM_BANKS; Did we fill last bank?
	JP NZ, nextBank           ; No, proceed with next bank
\end{tcblisting}


\subsection{Layer 2 Registers}
\label{zx_next_layer2_registers}

\subsubsection{Layer 2 Access Port \MemAddr{123B}}

\begin{NextPort}
	\PortBits{7-6}
		\PortDesc{Video RAM bank select}
		\PortDescOnly{
			\begin{PortBitConfig}
				\PortBitLine{00}{First 16K of layer 2 in the bottom 16K}
				\PortBitLine{01}{Second 16K of layer 2 in the bottom 16K}
				\PortBitLine{10}{Third 16K of layer 2 in the bottom 16K}
				\PortBitLine{11}{First 48K of layer 2 in the bottom 48K (core 3.0+)}
			\end{PortBitConfig}
		}
	\PortBits{5}
		\PortDesc{Reserved, use {\tt 0}}
	\PortBits{4}
		\PortDesc{{\tt 0} (see below)}
	\PortBits{3}
		\PortDesc{Use Shadow Layer 2 for paging}
		\PortDescOnly{
			\begin{PortBitConfig}
				\PortBitLine{0}{Map \PortLink{Layer 2 RAM Page Register}{12}}
				\PortBitLine{1}{Map \PortLink{Layer 2 RAM Shadow Page}{13}}
			\end{PortBitConfig}
		}
	\PortBits{2}
		\PortDesc{Enable Layer 2 read-only paging}
	\PortBits{1}
		\PortDesc{Layer 2 visible, see \PortLink{Layer 2 RAM Page Register}{12}}
		\PortDescOnly{Since core 3.0 this bit has mirror in \PortLink{Display Control 1 Register}{69}}
	\PortBits{0}
		\PortDesc{Enable Layer 2 write-only paging}
\end{NextPort}

Since core 3.0.7, write with bit {\tt 4} set was also added:

\begin{NextPort}
	\PortBits{7-5}
		\PortDesc{Reserved, use {\tt 0}}
	\PortBits{4}
		\PortDesc{{\tt 1}}
	\PortBits{3}
		\PortDesc{Reserved, use {\tt 0}}
	\PortBits{2-0}
		\PortDesc{16K bank relative offset (+0..+7) applied to Layer 2 memory mapping}
\end{NextPort}


\subsubsection{Layer 2 Ram Page Register \MemAddr{12}}

\begin{NextPort}
	\PortBits{7}
		\PortDesc{Reserved, must be {\tt 0}}
	\PortBits{6-0}
		\PortDesc{Starting 16K bank of Layer 2}
\end{NextPort}

Default 256$\times$192 mode requires 3 16K banks while new, 320$\times$256 and 640$\times$256 modes require 5 16K banks. Banks need to be contiguous in memory, so here we only specify the first one. Valid bank numbers are therefore {\tt 0} - {\tt 45} ({\tt 109} for 2MB RAM models) for standard mode and {\tt 0} - {\tt 43} ({\tt 107} for 2MB RAM models) for new modes.

Note: this register uses 16K bank numbers. If you're using 8K banks, you have to multiply this value by 2. For example, 16K bank 9 corresponds to 8K banks 18 and 19.


\pagebreak
\subsubsection{Layer 2 X Offset Register \MemAddr{16}}

\begin{NextPort}
	\PortBits{7-0}
		\PortDesc{Writes or reads X pixel offset used for drawing Layer 2 graphics on the screen.}
\end{NextPort}

This can be used for creating scrolling effects. For 320$\times$256 and 640$\times$256 modes, 9 bits are required; use \PortLink{Layer 2 X Offset MSB Register}{71} to set it up.


\subsubsection{Layer 2 Y Offset Register \MemAddr{17}}

\begin{NextPort}
	\PortBits{7-0}
		\PortDesc{Writes or reads Y pixel offset used for drawing Layer 2 graphics on the screen.}
\end{NextPort}

Valid range is:

\begin{itemize}[topsep=1pt,itemsep=1pt]
	\item 256$\times$192: {\tt 191}
	\item 320$\times$256: {\tt 255}
	\item 640$\times$256: {\tt 255}
\end{itemize}


\subsubsection{Clip Window Layer 2 Register \MemAddr{18}}

\begin{NextPort}
	\PortBits{7-0}
		\PortDesc{Reads and writes clip-window coordinates for Layer 2}
\end{NextPort}

4 coordinates need to be set: X1, X2, Y1 and Y2. Which coordinate gets set, is determined by index. As each write to this register will also increment index, the usual flow is to reset the index to {\tt 0} in \PortLink{Clip Window Control Register}{1C}, then write all 4 coordinates in succession. Positions are inclusive. Furthermore, X positions are doubled for 320$\times$256 mode, quadrupled for 640$\times$256. Therefore, to view the whole of Layer 2, the values are:

\begin{tabular}{cllll}
	& & 
		\BitHead{256$\times$192} & 
		\BitHead{320$\times$256} & 
		\BitHead{640$\times$256} \\
	0 & X1 position & \BitMono{0}   & \BitMono{0}   & \BitMono{0} \\
	1 & X2 position & \BitMono{255} & \BitMono{159} & \BitMono{159} \\
	2 & Y1 position & \BitMono{0}   & \BitMono{0}   & \BitMono{0} \\
	3 & Y2 position & \BitMono{191} & \BitMono{255} & \BitMono{255} \\
\end{tabular}

\pagebreak
\subsubsection{Clip Window Control Register \MemAddr{1C}}

Write:

\begin{NextPort}
	\PortBits{7-4}
		\PortDesc{Reserved, must be {\tt 0}}
	\PortBits{3}
		\PortDesc{{\tt 1} to reset Tilemap clip-window register index}
	\PortBits{2}
		\PortDesc{{\tt 1} to reset ULA/LoRes clip-window register index}
	\PortBits{1}
		\PortDesc{{\tt 1} to reset Sprite clip-window register index}
	\PortBits{0}
		\PortDesc{{\tt 1} to reset Layer 2 clip-window register index}
\end{NextPort}

Read:

\begin{NextPort}
	\PortBits{7-6}
		\PortDesc{Current Tilemap clip-window register index}
	\PortBits{5-4}
		\PortDesc{Current ULA/LoRes clip-window register index}
	\PortBits{3-2}
		\PortDesc{Current Sprite clip-window register index}
	\PortBits{1-0}
		\PortDesc{Current Layer 2 clip-window register index}
\end{NextPort}


\subsubsection{Palette Index Register \MemAddr{40}}
\vspace*{-2ex}
\subsubsection{Palette Value Register \MemAddr{41}}
\vspace*{-2ex}
\subsubsection{Enhanced ULA Control Register \MemAddr{43}}
\vspace*{-2ex}
\subsubsection{Enhanced ULA Palette Extension \MemAddr{44}}
\PortReference{zx_next_palette_registers}{Palette}


\subsubsection{Layer 2 Control Register \MemAddr{70}}

\begin{NextPort}
	\PortBits{7-6}
		\PortDesc{Reserved, must be {\tt 0}}
	\PortBits{5-4}
		\PortDesc{Layer 2 resolution ({\tt 0} after soft reset)}
		\PortDescOnly{
			\begin{PortBitConfig}
				\PortBitLine{00}{256$\times$192, 8BPP}
				\PortBitLine{01}{320$\times$256, 8BPP}
				\PortBitLine{10}{640$\times$256, 4BPP}
			\end{PortBitConfig}
		}
	\PortBits{3-0}
		\PortDesc{Palette offset ({\tt 0} after soft reset)}
\end{NextPort}


\subsubsection{Layer 2 X Offset MSB Register \MemAddr{71}}

\begin{NextPort}
	\PortBits{7-1}
		\PortDesc{Reserved, must be {\tt 0}}
	\PortBits{0}
		\PortDesc{MSB for X pixel offset}
\end{NextPort}

This is only used for 320$\times$256 and 640$\times$256 modes. Together with \PortLink{Layer 2 X Offset Register}{16} full 319 pixels offsets are available. For 640$\times$256 only 2 pixel offsets are possible.


\pagebreak
\IntentionallyEmpty
\pagebreak
