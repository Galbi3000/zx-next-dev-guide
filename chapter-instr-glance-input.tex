\section{Input}

\begin{minipage}{\textwidth}

\begin{instrtable}

    \begin{instruction}{IN A,(n)\See{1}} 
        \Symbol{\SymIN{A}{n}}
            \FlagsINan
            \OpCode{11}{011}{011}
            \Hex{DB}{2}
            \Cycles{3}{11}
            \Comment{
                \multirow{7}{*}{
                    \tt
                    \begin{tabular}{ll}
                        r & \OCT{r} \\
                        \hline
                        B & 000 \\
                        C & 001 \\
                        D & 010 \\
                        E & 011 \\
                        H & 100 \\
                        L & 101 \\
                        A & 111 \\
                    \end{tabular}
                }
            }
        \SkipToOpCode	
            \OpRange{n} 
            \Hex{..}{}
    \end{instruction}

    \begin{instruction}{IN r,(C)\See{2}} 
        \Symbol{\SymIN{r}{BC}}
            \FlagsINrc
            \OpCode{11}{101}{101}
            \Hex{ED}{2}
            \Cycles{3}{12}
        \SkipToOpCode 
            \OpCode{01}{\OCT{r}}{000}
            \Hex{..}{}
    \end{instruction}

    \begin{instruction}{IN (C)\See{2,3}} 
        \Symbol{(BC)}
            \FlagsINc
            \OpCode{11}{101}{101}
            \Hex{ED}{2}
            \Cycles{3}{12}
        \SkipToOpCode 
            \OpCode{01}{110}{000}
            \Hex{70}{}
    \end{instruction}

    \begin{instruction}{IND} 
        \Symbol{\SymIND[0]}
            \FlagsIND
            \OpCode{11}{101}{101}
            \Hex{ED}{2}
            \Cycles{4}{16}
        \SkipToSymbol
            \Symbol{\SymIND[1]}
            \FromSymbolToOpCode
            \OpCode{10}{101}{010}
            \Hex{AA}{}
        \SkipToSymbol
            \Symbol{\SymIND[2]}
    \end{instruction}
	
    \begin{instruction}{INDR} 
        \Symbol{\SymINDR[0]}
            \FlagsINDR
            \OpCode{11}{101}{101}
            \Hex{ED}{2}
            \Cycles{4}{16}
            \Comment{if {\tt B}=0}
        \SkipToSymbol
            \Symbol{\SymINDR[1]}
            \FromSymbolToOpCode
            \OpCode{10}{111}{010}
            \Hex{BA}{}
            \Cycles{5}{21}
            \Comment{if {\tt B}$\neq$0}
    \end{instruction}

    \begin{instruction}{INI} 
        \Symbol{\SymINI[0]}
            \FlagsINI
            \OpCode{11}{101}{101} 
            \Hex{ED}{2}
            \Cycles{4}{16}
        \SkipToSymbol
            \Symbol{\SymINI[1]}
            \FromSymbolToOpCode
            \OpCode{10}{100}{010}
            \Hex{A2}{}
        \SkipToSymbol
            \Symbol{\SymINI[2]}
    \end{instruction}
	
    \begin{lastinstruction}{INIR} 
        \Symbol{\SymINIR[0]}
            \FlagsINIR
            \OpCode{11}{101}{101}
            \Hex{ED}{2}
            \Cycles{4}{16}
            \Comment{if {\tt B}=0}
        \SkipToSymbol
            \Symbol{\SymINIR[1]}
            \FromSymbolToOpCode
            \OpCode{10}{110}{010}
            \Hex{B2}{}
            \Cycles{5}{21}
            \Comment{if {\tt B}$\neq$0}
    \end{lastinstruction}	
		
\end{instrtable}

\begin{notestable}
    \NoteItem{\See{1}Some assemblers allow {\tt IN (n)} to be used instead of {\tt IN A,(n)}}
    \NoteItem{\See{2}Some assemblers allow instruction to be written with  {\tt (BC)} instead of {\tt (C)}}
    \NoteItem{\See{3}Performs the input without storing the result. Some assemblers allow {\tt IN F,(C)} to be used instead of {\tt IN (C)}}
    \NoteItem{\See{4}Flag is {\tt 1} if {\tt B=0} after execution, otherwise {\tt 0}; similar to {\tt DEC B}}
    \NoteItem{\See{5}On Next this flag is destroyed, for other Z80 computers see section \ref{block_io}}
\end{notestable}
	
\end{minipage}
