\section{Call and Return}

\begin{minipage}{\textwidth}
	
\begin{instrtable}

	\begin{instruction}{CALL nm} 
		\Symbol{\SymCALL[0]{nm}}
			\FlagsCALLnn
			\OpCode{11}{001}{101}
			\Hex{CD}{3}
			\Cycles{5}{17}
		\SkipToSymbol
			\Symbol{\SymCALL[1]{nm}}
			\FromSymbolToOpCode
			\OpRange{m} 
			\Hex{..}{}
		\SkipToSymbol
			\Symbol{\SymCALL[2]{nm}}
			\FromSymbolToOpCode
			\OpRange{n} 
			\Hex{..}{}
		\SkipToSymbol
			\Symbol{\SymCALL[3]{nm}}
	\end{instruction}

	\begin{instruction}{CALL c,nm} 
		\Symbol{\SymCALLc{nm}}
			\FlagsCALLccnn
			\OpCode{11}{\OCT{c}}{100}
			\Hex{..}{3}
			\Cycles{3}{10}
			\Comment{if {\tt c}=false}
		\SkipToOpCode
			\OpRange{m} 
			\Hex{..}{}
			\Cycles{5}{17}
			\Comment{if {\tt c}=true}
		\SkipToOpCode
			\OpRange{n} 
			\Hex{..}{}
	\end{instruction}

	\begin{instruction}{RET} 
		\Symbol{\SymRET[0]}
			\FlagsRET
			\OpCode{11}{001}{001}
			\Hex{C9}{1}
			\Cycles{3}{10}
		\SkipToSymbol
			\Symbol{\SymRET[1]}
		\SkipToSymbol
			\Symbol{\SymRET[2]}
	\end{instruction}

	\begin{instruction}{RET c} 
		\Symbol{\SymRETc{c}}
			\FlagsRETcc
			\OpCode{11}{\OCT{c}}{000}
			\Hex{..}{1}
			\Cycles{1}{5}
			\Comment{if {\tt c}=false}
		\SkipToOpCode
			\OpCode{}{}{}
			\Hex{}{}
			\Cycles{3}{11}
			\Comment{if {\tt c}=true}
	\end{instruction}

	\begin{instruction}{RETI\See{1}} 
		\Symbol{\SymRETI[0]}
			\FlagsRETI
			\OpCode{11}{101}{101}
			\Hex{ED}{2}
			\Cycles{4}{14}
			\Comment{
				\multirow{7}{*}{
					\tt
					\begin{tabular}{ll}
						c & \OCT{c} \\
						\hline
						NZ & 000 \\
						Z & 001 \\
						NC & 010 \\
						C & 011 \\
						PO & 100 \\
						PE & 101 \\
						P & 110 \\
						M & 111 \\
					\end{tabular}
				}
			}
		\SkipToSymbol
			\Symbol{\SymRETI[1]}
			\FromSymbolToOpCode
			\OpCode{01}{001}{101}
			\Hex{4D}{}
		\SkipToSymbol
			\Symbol{\SymRETI[2]}
	\end{instruction}

	\begin{instruction}{RETN\See{2}} 
		\Symbol{\SymRETN[0]}
			\FlagsRETN
			\OpCode{11}{101}{101}
			\Hex{ED}{2}
			\Cycles{4}{14}
		\SkipToSymbol
			\Symbol{\SymRETN[1]}
			\FromSymbolToOpCode
			\OpCode{01}{000}{101}
			\Hex{45}{}
		\SkipToSymbol
			\Symbol{\SymRETN[2]}
		\SkipToSymbol
			\Symbol{\SymRETN[3]}
	\end{instruction}

	\begin{lastinstruction}{RST p} 
		\Symbol{\SymRST[0]{p}}
			\FlagsRSTn
			\OpCode{11}{\OCT{p}}{111}
			\Hex{..}{1}
			\Cycles{3}{11}
			\Comment{
				\multirow{7}{*}{
					\tt
					\begin{tabular}{ll}
						p & \OCT{p} \\
						\hline
						\$0 & 000 \\
						\$8 & 001 \\
						\$10 & 010 \\
						\$18 & 011 \\
						\$20 & 100 \\
						\$28 & 101 \\
						\$30 & 110 \\
						\$38 & 111 \\
					\end{tabular}
				}
			}
		\SkipToSymbol
			\Symbol{\SymRST[1]{p}}
		\SkipToSymbol
			\Symbol{\SymRST[2]{p}}
		\SkipToSymbol
			\Symbol{\SymRST[3]{p}}
	\end{lastinstruction}

	\\[7ex] % note we need to add some manual spacing after this instruction so that tt/p comment table is visible in full (we don't have enough rows for multirow command)

\end{instrtable}

\begin{notestable}
	\NoteItem{\See{1}{\tt RETI} also copies {\tt IFF2} into {\tt IFF1}, like {\tt RETN}}
	\NoteItem{\See{2}This instruction has other undocumented opcodes}
\end{notestable}

\end{minipage}