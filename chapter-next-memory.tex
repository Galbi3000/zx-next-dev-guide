\section{Memory Map and Paging}
\label{zx_next_memorypaging}

% ──────────────────────────────────────────────────────────────
% ─██████──────────██████─██████████████─██████──────────██████─
% ─██░░██████████████░░██─██░░░░░░░░░░██─██░░██████████████░░██─
% ─██░░░░░░░░░░░░░░░░░░██─██░░██████████─██░░░░░░░░░░░░░░░░░░██─
% ─██░░██████░░██████░░██─██░░██─────────██░░██████░░██████░░██─
% ─██░░██──██░░██──██░░██─██░░██████████─██░░██──██░░██──██░░██─
% ─██░░██──██░░██──██░░██─██░░░░░░░░░░██─██░░██──██░░██──██░░██─
% ─██░░██──██████──██░░██─██░░██████████─██░░██──██████──██░░██─
% ─██░░██──────────██░░██─██░░██─────────██░░██──────────██░░██─
% ─██░░██──────────██░░██─██░░██████████─██░░██──────────██░░██─
% ─██░░██──────────██░░██─██░░░░░░░░░░██─██░░██──────────██░░██─
% ─██████──────────██████─██████████████─██████──────────██████─
% ──────────────────────────────────────────────────────────────

\newcommand{\MemEmpty}{\multicolumn{1}{c}{}}
\newcommand{\MemArrow}[1]{\multicolumn{1}{c}{\IfEq{#1}{<}{\LArrowLine{1em}}{\RArrowLine{1em}}}}

ZX Spectrum Next comes with 1024K (expanded version with 2048K) of memory. But it can't see it all at once.


\subsection{Banks and Slots}

Due to its 16-bit address bus, Next can only address 2\textsuperscript{16} = 65.536 bytes or 64K of memory at a time. To get access to all available memory, it's divided into smaller chunks called ``banks''\footnote{You may also see the term ``page'' used instead of ``bank'' (in fact, that's why the process of swapping banks into slots is usually called ``paging''). I also noticed sometimes 64K addressable memory is referred to as ``bank''. In this book, I will keep naming consistent to avoid confusion.}.

Next supports two interchangeable memory management models. One is inherited from the original Spectrums and clones and uses 16K banks. The other is unique to Next and uses 8K banks. Hence, addressable 64K is also divided into 16K or 8K ``slots'' into which banks are swapped in and out. In a way, slots are like windows into available memory.

Banks are selected by their number - the first bank is {\tt 0}, second {\tt 1} and so on. If you ever worked with arrays, banks and their numbers work the same as array data and indexes. Both 16K and 8K banks start with number {\tt 0} at the same address. So if 16K bank $n$ is selected, then the two corresponding 8K bank numbers would be $n \times 2$ and $n \times 2 + 1$.

After startup, addressable 64K space is mapped like this:

\begin{ElegantTableX}{|l|l|l|l|l|X|}[
	\newcommand{\StartRow}[7]{\MemAddr{#1}-\MemAddr{#2} & \multirow[t]{2}{*}{{\tt #3}} & {\tt #4} & \multirow[t]{2}{*}{#5} & #6 & #7 \\}
	\newcommand{\EndRow}[5]{\MemAddr{#1}-\MemAddr{#2} & & {\tt #3} & & #4 & #5 \\}
]

	\ElegantHeader[]{
		& 
		\multicolumn{2}{c|}{\EH{Slots}} & 
		\multicolumn{2}{c|}{\EH{Banks}} &
	}

	\ElegantHeader{
		\multirow{-2}{*}{\EH{Address}} & % in second line and negative value to draw over row colour
		\multicolumn{1}{c}{\EH{16K}} & 
		\multicolumn{1}{c|}{\EH{8K}} & 
		\multicolumn{1}{c}{\EH{16K}} & 
		\multicolumn{1}{c|}{\EH{8K}} & 
		\multirow{-2}{*}{\EH{Description}}
	}

	\StartRow{0000}{1FFF}{0}{0}{ROM}{ROM}{ROM, R/W redirect by L2, IRQ, NMI}
	\EndRow{2000}{3FFF}     {1}     {ROM}{ROM, R/W redirect by Layer 2}
	
	\hline
	
	\StartRow{4000}{5FFF}{1}{2}{5}{10}{Normal/shadow ULA screen, Tilemap}
	\EndRow{6000}{7FFF}     {3}   {11}{ULA extended attribute/graphics, Tilemap}
	
	\hline
	
	\StartRow{8000}{9FFF}{2}{4}{2}{4}{Free RAM}
	\EndRow{A000}{BFFF}     {5}   {5}{Free RAM}
		
	\hline
	
	\StartRow{C000}{DFFF}{3}{6}{0}{0}{Free RAM}
	\EndRow{E000}{FFFF}     {7}   {1}{Free RAM}

\end{ElegantTableX}


\subsection{Default Bank Traits}

First few addressable banks have certain uses and traits:

\NewDocumentEnvironment{BankTraitTable}{ +!b }{
	\begin{ElegantTableX}{|l|l|X|}

		\ElegantHeader[]{ \multicolumn{2}{|c|}{\EH{Banks}} & }
		\ElegantHeader{
			\multicolumn{1}{|c}{\EH{16K}} & 
			\multicolumn{1}{c|}{\EH{8K}} &
			\multirow{-2}{*}{\EH{Description}}
		}

		#1

	\end{ElegantTableX}
}{}

\newcommand{\BankTraitRow}[5][\hline]{{\tt #2} & {\tt #3}-{\tt #4} & #5 \\ #1}
\newcommand{\BankTraitRowMulti}[6][\hline]{{\tt #2}-{\tt #3} & {\tt #4}-{\tt #5} & #6 \\ #1}

\begin{BankTraitTable}

	\BankTraitRow{0}{0}{1}{Standard RAM, maybe used by EsxDOS. Initially mapped to \MemAddr{C000}-\MemAddr{FFFF}}
	\BankTraitRow[]{1}{2}{3}{Standard RAM, contended on 128, may be used by EsxDOS, RAM disk on NextZXOS}

\end{BankTraitTable}

\begin{BankTraitTable}

	\BankTraitRow{2}{4}{5}{Standard RAM. Initially mapped to \MemAddr{8000}-\MemAddr{BFFF}}
	\BankTraitRow{3}{6}{7}{Standard RAM, contended on 128, may be used by EsxDOS, RAM disk on NextZXOS}
	\BankTraitRow{4}{8}{9}{Standard RAM, contended on +2/+3, RAM disk on NextZXOS}
	\BankTraitRow{5}{10}{11}{ULA Screen, contended except on Pentagon, cannot be used by NextBASIC commands. Initially mapped to \MemAddr{4000}-\MemAddr{7FFF}}
	\BankTraitRow{6}{12}{13}{Standard RAM, contended on +2/+3, RAM disk on NextZXOS}
	\BankTraitRow{7}{14}{15}{ULA Shadow Screen, contended except on Pentagon, NextZXOS Workspace, cannot be used by NextBASIC commands}
	\BankTraitRow{8}{16}{17}{Next RAM, Default Layer 2, NextZXOS screen and extra data, cannot be used by NextBASIC commands}
	\BankTraitRowMulti{9}{10}{18}{21}{Next RAM, Rest of default Layer 2}
	\BankTraitRowMulti[]{11}{13}{22}{27}{Next RAM, Default Layer 2 Shadow Screen}
	
\end{BankTraitTable}


\subsection{Memory Map}

As hinted before, not all available memory is addressable by programs. The first 256K is always reserved for ROMs and firmware. Hence bank {\tt 0} starts at absolute address \MemAddr{40000}:

\begin{ElegantTableX}{|l|l|l|l|l|l|X|}[
	\newcommand{\MapROMRowNoPrefix}[4]{- & - & #1 & \MemAddr{#2}-\MemAddr{#3} & #4 \\}
	\newcommand{\MapROMRow}[4]{& & \MapROMRowNoPrefix{#1}{#2}{#3}{#4}}
	\newcommand{\MapRow}[8]{& & {\tt #1}-{\tt #2} & {\tt #3}-{\tt #4} & #5 & \MemAddr{#6}-\MemAddr{#7} & #8 \\}
	\newcommand{\MapDelim}{\cline{3-7}}
]

	\ElegantHeader{\multicolumn{2}{|c|}{} & \EH{16K bank} & \EH{8K bank} & \EH{Size} & \EH{Absolute Address} & \EH{Description}}
	
	\multirow{11}{*}{\rotatebox[origin=c]{90}{\LArrowLine{1.75cm} Expanded Next \RArrowLine{1.75cm}}} &
	\multirow{9}{*}{\rotatebox[origin=c]{90}{\LArrowLine{0.92cm} Unexpanded Next \RArrowLine{0.92cm}}} &
	\MapROMRowNoPrefix{64K}{000000}{00FFFF}{ZX Spectrum ROM}
	\MapDelim
			
	\MapROMRow{16K}{010000}{013FFF}{EsxDOS ROM}
	\MapDelim
			
	\MapROMRow{16K}{014000}{017FFF}{Multiface ROM}
	\MapDelim
			
	\MapROMRow{16K}{018000}{01BFFF}{Multiface Extra ROM}
	\MapDelim
			
	\MapROMRow{16K}{01C000}{01FFFF}{Multiface RAM}
	\MapDelim
			
	\MapROMRow{128K}{020000}{03FFFF}{DivMMC RAM}
	\MapDelim
			
	\MapRow{0}{7}{0}{15}{128K}{040000}{05FFFF}{Standard 128K RAM}
	\MapDelim
			
	\MapRow{8}{15}{16}{31}{128K}{060000}{07FFFF}{Extra RAM}
	\MapDelim
			
	\MapRow{16}{47}{32}{95}{512K}{080000}{0FFFFF}{1st Extra IC RAM}
	\cline{2-7}
			
	\MapRow{48}{79}{96}{159}{512K}{080000}{0FFFFF}{1st Extra IC RAM}
	\MapDelim
			
	\MapRow{80}{111}{160}{223}{512K}{080000}{0FFFFF}{2st Extra IC RAM}
	
\end{ElegantTableX}

So when swapping in, for example:

\begin{itemize}[topsep=0pt,itemsep=0pt]
	\item 16K bank 20 to slot 3 and writing 10 bytes to memory \MemAddr{C000} (start of 16K slot 3), we're effectively writing to absolute memory \MemAddr{90000}-\MemAddr{90009} ({\tt \MemAddr{40000} + 20 $\times$ 16384})
	
	\item 8K bank 30 to slot 5 and writing 10 bytes to memory \MemAddr{A000} (start of 8K slot 5), we're effectively writing to absolute memory \MemAddr{7C000}-\MemAddr{7C009} ({\tt \MemAddr{40000} + 30 $\times$ 8192})
\end{itemize}.


\subsection{Legacy Paging Modes}

As mentioned, Next inherits the memory management models from the Spectrum 128K/+2/+3 models and Pentagon clones. It's unlikely you will use these modes for Next programs, as Next own model is much simpler to use. They are still briefly described here though in case you will encounter them in older programs. All legacy models use 16K slots and banks.


\NewDocumentEnvironment{PagingTableLegacy}{ +!b }{
	% note: we need to use constant column widths so that all tables exactly match their cells...

	% this table uses elegant background and spacing
	\begin{ElegantTable}{|p{1cm}|C{1.55cm}|C{1.55cm}|C{1.55cm}|C{1.55cm}|}[][]
		\ElegantHeader{\EH{Slot}	& {\tt 0}		& {\tt 1}			& {\tt 2}			& {\tt 3}}
	\end{ElegantTable}

	\vspace*{-0.75em}

	% second table touches top one, repeats borders, but without extra spacing that elegant table uses
	\begin{tabular}{|p{1cm}|C{1.55cm}|C{1.55cm}|C{1.55cm}|C{1.55cm}|}
		Start	& \MemAddr{0000}	& \MemAddr{4000}	& \MemAddr{8000}	& \MemAddr{C000} \\
		End		& \MemAddr{3FFF}	& \MemAddr{7FFF}	& \MemAddr{BFFF}	& \MemAddr{FFFF} \\
		\hline
	\end{tabular}

	\vspace*{-0.7em}

	% third table touches second one but without borders and with extra column at the end.
	\begin{tabular}{p{1cm}C{1.55cm}C{1.55cm}C{1.55cm}C{1.55cm}l}
		#1
	\end{tabular}
}{}

\newcommand{\PagingTableLegacyItem}[5]{ #1 & #2 & #3 & #4 & #5 \\ }

\subsubsection{128K Mode}

\begin{PagingTableLegacy}
	\PagingTableLegacyItem{}{$\uparrow$}{}{}{$\uparrow$}
	\PagingTableLegacyItem{}{\tt ROM 0-1}{}{}{\multicolumn{2}{l}{{\tt BANK 0-7} on 128K}}
	\PagingTableLegacyItem{}{}{}{}{\multicolumn{2}{l}{{\tt BANK 0-127} on Next}}
\end{PagingTableLegacy}

Allows selecting:

\begin{itemize}[topsep=0pt,itemsep=0pt]
	\item 16K ROM to be visible in the bottom 16K slot (0) from 2 possible banks
	\item 16K RAM to be visible in the top 16K slot (3) from 8 possible banks (128 banks on Next)
\end{itemize}

Ports involved:

\begin{itemize}[topsep=0pt,itemsep=0pt]
	\item \PortLink{Memory Paging Control}{7FFD} bit 4 selects ROM bank for slot 0
	\item \PortLink{Memory Paging Control}{7FFD} bits 2-0 select one of 8 RAM banks for slot 3
	\item \PortLink{Next Memory Bank Select}{DFFD} bits 3-0 are added as MSB to 2-0 from \MemAddr{7FFD} to form 128 banks for slot 3 (Next specific)
\end{itemize}

If you are using the standard interrupt handler or OS routines, then any time you write to \PortLink{Memory Paging Control}{7FFD} you should also store the value at \MemAddr{5B5C}.


\subsubsection{+3 Normal Mode}

\begin{PagingTableLegacy}
	\PagingTableLegacyItem{}{$\uparrow$}{}{}{$\uparrow$}
	\PagingTableLegacyItem{}{\tt ROM 0-3}{}{}{\multicolumn{2}{l}{{\tt BANK 0-7} on 128K}}
	\PagingTableLegacyItem{}{}{}{}{\multicolumn{2}{l}{{\tt BANK 0-127} on Next}}
\end{PagingTableLegacy}

Allows selecting:

\begin{itemize}[topsep=0pt,itemsep=0pt]
	\item 16K ROM to be visible in the bottom 16K slot (0) from 4 possible banks
	\item 16K RAM to be visible in the top 16K slot (3) from 8 possible banks (128 banks on Next)
\end{itemize}

Ports involved:

\begin{itemize}[topsep=0pt,itemsep=0pt]
	\item \PortLink{Plus 3 Memory Paging Control}{1FFD} bit 2 as LSB for selecting ROM bank for slot 0
	\item \PortLink{Memory Paging Control}{7FFD} bit 4 forms MSB for selecting ROM bank for slot 0
	\item \PortLink{Memory Paging Control}{7FFD} bits 2-0 select one of 8 RAM banks for slot 3
	\item \PortLink{Next Memory Bank Select}{DFFD} bits 3-0 are added as MSB to 2-0 from \MemAddr{7FFD} to form 128 banks for slot 3 (Next specific)
\end{itemize}

If you are using the standard interrupt handler or OS routines, then any time you write to \PortLink{Plus 3 Memory Paging Control}{1FFD} you should also store the same value at \MemAddr{5B67} and every time your write to \PortLink{Memory Paging Control}{7FFD} you should also store the value at \MemAddr{5B5C}.


\subsubsection{+3 All-RAM Mode}

\begin{PagingTableLegacy}
	\PagingTableLegacyItem{}{$\uparrow$}{$\uparrow$}{$\uparrow$}{$\uparrow$}
	\PagingTableLegacyItem{\tt 00 =}{\tt BANK 0}{\tt BANK 1}{\tt BANK 2}{\tt BANK 3}
	\PagingTableLegacyItem{\tt 01 =}{\tt BANK 4}{\tt BANK 5}{\tt BANK 6}{\tt BANK 7}
	\PagingTableLegacyItem{\tt 10 =}{\tt BANK 4}{\tt BANK 5}{\tt BANK 6}{\tt BANK 3}
	\PagingTableLegacyItem{\tt 11 =}{\tt BANK 4}{\tt BANK 7}{\tt BANK 6}{\tt BANK 3}
	\multirow{2}{*}{$\Big\downarrow$}$\downarrow$ & \\
	\multicolumn{5}{l}{~~Lo bit = bit 1 from \MemAddr{1DDF}} \\
	\multicolumn{5}{l}{Hi bit = bit 2 from \MemAddr{1DDF}} \\
\end{PagingTableLegacy}

Also called ``Special Mode'' or ``CP/M Mode''. Allows selecting all 4 slots from limited selection of banks as shown in the table above.

Ports involved:

\begin{itemize}[topsep=0pt,itemsep=0pt]
	\item \PortLink{Plus 3 Memory Paging Control}{1FFD} bit 0 enables All-RAM (if {\tt 1}) or normal mode ({\tt 0})
	\item \PortLink{Plus 3 Memory Paging Control}{1FFD} bits 2-1 select memory configuration
\end{itemize}

If you are using the standard interrupt handler or OS routines, then any time you write to \PortLink{Plus 3 Memory Paging Control}{1FFD} you should also store the same value at \MemAddr{5B67}.


\subsubsection{Pentagon 512K/1024K Mode}

Next also supports paging implementation from Pentagon spectrums. It's unlikely you will ever use it on Next, so just mentioning for completness sake. You can find more information on Next Dev Wiki\footnote{\url{https://wiki.specnext.dev/Next_Memory_Bank_Select}} or internet if interested.


\pagebreak
\subsection{Next MMU Paging Mode}

Next MMU based paging mode is much more flexible in that it allows mapping 8K banks into any 8K slot of memory available to the CPU. This is the only mode that allows paging in all 2048K on extended Next. It's also the simplest to use - a single {\tt NEXTREG} instruction assigning bank number to desired MMU slot register.

In this mode, 64K memory accessible to the CPU is divided into 8 slots called MMU0 through MMU7, as shown in the diagram below. Physical memory is thus divided into 96 (or 224 on expanded Next) 8K banks.

% similar to legacy tables, but less width per column
\begin{ElegantTable}{|p{1.8cm}|C{1.1cm}|C{1.1cm}|C{1.1cm}|C{1.1cm}|C{1.1cm}|C{1.1cm}|C{1.1cm}|C{1.1cm}|}[][]
	\ElegantHeader{
		\EH{16K Slot} &
		\multicolumn{2}{c|}{\tt 0} &
		\multicolumn{2}{c|}{\tt 1} &
		\multicolumn{2}{c|}{\tt 2} &
		\multicolumn{2}{c|}{\tt 3}
	}
	\ElegantHeader{
		\EH{8K Slot} &
		{\tt 0} &
		{\tt 1} &
		{\tt 2} &
		{\tt 3} &
		{\tt 4} &
		{\tt 5} &
		{\tt 6} &
		{\tt 7}
	}
\end{ElegantTable}

\vspace*{-0.75em}

\begin{tabular}{|p{1.8cm}|C{1.1cm}|C{1.1cm}|C{1.1cm}|C{1.1cm}|C{1.1cm}|C{1.1cm}|C{1.1cm}|C{1.1cm}|}
	\hline
	Start	& \MemAddr{0000}	& \MemAddr{2000}	& \MemAddr{4000}	& \MemAddr{6000}	& \MemAddr{8000}	& \MemAddr{A000}	& \MemAddr{C000}	& \MemAddr{E000} \\
	End		& \MemAddr{1FFF}	& \MemAddr{3FFF}	& \MemAddr{5FFF}	& \MemAddr{7FFF}	& \MemAddr{9FFF}	& \MemAddr{BFFF}	& \MemAddr{DFFF}	& \MemAddr{FFFF} \\
	\hline
\end{tabular}

\vspace*{-0.7em}

\begin{tabular}{p{1.8cm}C{1.1cm}C{1.1cm}C{1.1cm}C{1.1cm}C{1.1cm}C{1.1cm}C{1.1cm}C{1.1cm}}
			& $\uparrow$		& $\uparrow$		& $\uparrow$		& $\uparrow$		& $\uparrow$		& $\uparrow$		& $\uparrow$		& $\uparrow$ \\
			& {\tt BANK}		& {\tt BANK}		& {\tt BANK}		& {\tt BANK}		& {\tt BANK}		& {\tt BANK}		& {\tt BANK}		& {\tt BANK} \\
			& {\tt 0-255}		& {\tt 0-255}		& {\tt 0-255}		& {\tt 0-255}		& {\tt 0-255}		& {\tt 0-255}		& {\tt 0-255}		& {\tt 0-255} \\
\end{tabular}

Bank selection is set via Next registers:

\begin{itemize}[topsep=0pt,itemsep=0pt]
	\item \PortLink{Memory Management Slot 0 Bank}{50}
	\item \PortLink{Memory Management Slot 1 Bank}{51}
	\item \PortLink{Memory Management Slot 2 Bank}{52}
	\item \PortLink{Memory Management Slot 3 Bank}{53}
	\item \PortLink{Memory Management Slot 4 Bank}{54}
	\item \PortLink{Memory Management Slot 5 Bank}{55}
	\item \PortLink{Memory Management Slot 6 Bank}{56}
	\item \PortLink{Memory Management Slot 7 Bank}{57}
\end{itemize}

While not absolutely required, it's good practice to store original slot values and then restore before exiting program or returning from subroutines.

Example of writing 10 bytes ({\tt 00 01 02 03 04 05 06 07 08 09}) to 8K bank 30 swapped in to slot 5. As mentioned before, this will effectively write to absolute memory \MemAddr{7C000}-\MemAddr{7C009}:

\begin{tcblisting}{}
	NEXTREG &55, 30     ; swap bank 30 to slot 5

	LD DE, &A000        ; slot 5 starts at &A000
	LD A, 0             ; starting data to write
	LD B, 10            ; number of bytes to write
next:
	LD (DE), A          ; write next byte
	INC A               ; increment source byte
	INC DE              ; increment destination location
	DJNZ next
\end{tcblisting}

Note: \PortLink{Memory Management Slot 0 Bank}{50} and \PortLink{Memory Management Slot 1 Bank}{51} have extra ``functionality'': ROM can be automatically paged in if otherwise nonexistent 8K page \MemAddr{FF} is set. Low or high 8K ROM bank is automatically determined based on which 8K slot is used. This may be useful if temporarily paging RAM into the bottom 16K region and then wanting to restore back to ROM.


\subsection{Interaction Between Paging Modes}

As mentioned, legacy and Next paging modes are interchangeable. Changing banks in one will be reflected in the other. The most recent change always has priority. Again, keep in mind that legacy modes use 16K banks, therefore single bank change will affect 2 8K banks.


\subsubsection{Paging Out ROM}

ROM is usually mapped to the bottom 16K slot, addresses \MemAddr{0000}-\MemAddr{3FFF}. This area can only be remapped using +3 All-RAM or Next MMU-based mode. Beware though that some programs may expect to find ROM routines at fixed addresses between \MemAddr{0000} and \MemAddr{3FFF}. And if default interrupt mode (IM 1) is set, Z80 will expect to find interrupt handler at the address \MemAddr{0038}.


\subsubsection{ULA}

ULA always reads content from 16K bank 5. This is mapped to 16K slot 1 by default, addresses \MemAddr{4000}-\MemAddr{7FFF}. ULA will always use bank 5, regardless of which bank is mapped to slot 1, or which slot bank 5 is mapped to (or if it is mapped into any slot at all).

You can redirect ULA to read from 16K bank 7 instead (the ``shadow screen'', used for double-buffering), using bit 3 of \PortLink{Memory Paging Control}{7FFD}. However, you still need to map bank 7 into one of the slots if you want to read or write to it (that's 8K banks 14 and 15 if using MMU for paging). Read more in ULA chapter, section \ref{zx_next_ula}.


\subsubsection{Layer 2 Paging}

Layer 2 uses specific ports and registers that are can be used to change memory mapping. For example, the bottom 16K slot can be set for write and read access. Layer 2 also supports a double-buffering scheme of its own. Though any other mapping mode discussed here can be used as well. See Layer 2 chapter, section \ref{zx_next_layer2} for more details.



\pagebreak

\subsection{Paging Mode Ports and Registers}
\label{zx_next_memorypaging_registers}

\subsubsection{+3 Memory Paging Control \MemAddr{1FFD}}

\begin{NextPort}
	\PortBits{7-3}
		\PortDesc{Unused, use 0}
	\PortBits{2}
		\PortDesc{In normal mode high bit of ROM selection. With low bit from bit 4 of \MemAddr{7FFD}:}
		\PortDescOnly{
			\begin{PortBitConfig}
				\PortBitLine{00}{ROM0 = 128K editor and menu system}
				\PortBitLine{01}{ROM1 = 128K syntax checker}
				\PortBitLine{10}{ROM2 = +3DOS}
				\PortBitLine{11}{ROM3 = 48K BASIC}
			\end{PortBitConfig}
		}
		\PortDescOnly{In special mode: high bit of memory configuration number}
	\PortBits{1}
		\PortDesc{In special mode: low bit of memory configuration number}
	\PortBits{0}
		\PortDesc{Paging mode: 0 = normal, 1 = special}
\end{NextPort}

\subsubsection{Memory Paging Control \MemAddr{7FFD}}

\begin{NextPort}
	\PortBits{7-6}
		\PortDesc{Extra two bits for 16K RAM bank if in Pentagon 512K/1024K mode (see \PortLink{Next Memory Bank Select}{DFFD})}
	\PortBits{5}
		\PortDesc{{\tt 1} locks pages; cannot be unlocked until next reset on regular ZX128)}
	\PortBits{4}
		\PortDesc{128K: ROM select ({\tt 0} = 128K editor, {\tt 1} = 48K BASIC)}
		\PortDescOnly{+2/+3: low bit of ROM select (see \PortLink{+3 Memory Paging Control}{1FFD} above)}
	\PortBits{3}
		\PortDesc{ULA layer shadow screen toggle (0 = bank 5, 1 = bank 7)}
	\PortBits{2-0}
		\PortDesc{Bank number for slot 4 (\MemAddr{C000})}
\end{NextPort}

\subsubsection{Next Memory Bank Select \MemAddr{DFFD}}

\begin{NextPort}
	\PortBits{7}
		\PortDesc{1 to set Pentagon 512K/1024K mode}
	\PortBits{3-0}
		\PortDesc{Most significant bits of the 16K RAM bank selected in \PortLink{Memory Paging Control}{7FFD}}
\end{NextPort}


\subsubsection{Memory Management Slot 0-7 Bank \MemAddr{50}-\MemAddr{57}}

\begin{NextPort}
	\PortBits{7-0}
		\PortDesc{Selects 8K bank stored in corresponding 8K slot}
\end{NextPort}


\subsubsection{Memory Mapping Register \MemAddr{8E}}

\begin{NextPort}
	\PortBits{7}
		\PortDesc{Access to bit 0 of \PortLink{Next Memory Bank Select}{DFFD}}
	\PortBits{6-4}
		\PortDesc{Access to bits 2-0 of \PortLink{Memory Paging Control}{7FFD}}
	\PortBits{3}
		\PortDesc{Read will always return {\tt 1}}
		\PortDescOnly{Write {\tt 1} to change RAM bank, {\tt 0} for no change to MMU6,7, \MemAddr{7FFD} and \MemAddr{DFFD}}
	\PortBits{2}
		\PortDesc{{\tt 0} for normal paging mode, {\tt 1} for special all-RAM mode}
	\PortBits{1}
		\PortDesc{Access to bit 2 of \PortLink{+3 Memory Paging Control}{1FFD}}
	\PortBits{0}
		\PortDesc{If bit 2 = {\tt 0} (normal mode): bit 4 of \PortLink{Memory Paging Control}{7FFD}}
		\PortDescOnly{If bit 2 = {\tt 1} (special mode): bit 1 of \PortLink{+3 Memory Paging Control}{1FFD}}
\end{NextPort}

Acts as a shortcut for reading and writing \PortLink{+3 Memory Paging Control}{1FFD}, \PortLink{Memory Paging Control}{7FFD} and \PortLink{Next Memory Bank Select}{DFFD} all at once. Mainly to simplify classic Spectrum memory mapping. Though, as mentioned, Next specific programs should prefer MMU based memory mapping.

\pagebreak