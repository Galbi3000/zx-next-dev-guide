\documentclass[oneside,a4paper]{book}

\usepackage{fancyhdr}
\usepackage{a4wide}
\usepackage{url}
\usepackage{arydshln}
\usepackage{multicol}
\usepackage{multirow}
\usepackage{titletoc}

% redefine dash line to be dotted line instead
\setlength{\dashlinedash}{0.1pt}
\setlength{\dashlinegap}{0.8pt}
\setlength{\arrayrulewidth}{0.1pt}

% redefine the default "plain" pagestyle (used by chapter pages)
\fancypagestyle{plain}{
	\fancyhf{} 
	\fancyfoot[C]{\thepage}
	\renewcommand{\headrulewidth}{0pt}
	\renewcommand{\footrulewidth}{0pt}
}

% define style used for normal pages
\fancypagestyle{clean}{
	\fancyhf{} 
	\fancyfoot[C]{\thepage}
	\fancyhead[L]{\leftmark}
}

\pagestyle{clean}




\begin{document}

% TODO
% - Describe BIT A,(HL) undocumented flags from internal MEMPRT/WZ register
% - Add testing code to appendices

\title{The Undocumented Z80 Documented}
\author{Sean Young}
\date{Version 0.9n, \today}
%Maintained by Jan Wilmans

\maketitle

\frontmatter




\chapter*{Copyright Statement}

      Copyright {\copyright} 1997, 1998, 2001, 2003, 2005 Sean Young.\\
	
      Permission is granted to copy, distribute and/or modify this document
      under the terms of the GNU Free Documentation License, Version 1.1
      or any later version published by the Free Software Foundation;
      with no Invariant Sections, with no Front-Cover Texts, and with no 
      Back-Cover Texts. A copy of the license is included in the section 
      entitled ``GNU Free Documentation License''.

\tableofcontents

\mainmatter




\setcounter{page}{5}
\chapter{Introduction}

\section{History}

(Sean)
Ever since I first started working on an MSX emulator, I've been very 
interested in getting the emulation absolutely correct --- including the 
undocumented features. Not just to make sure that all games work, but 
also to make sure that if a program crashes, it crashes exactly the same 
way if running on an emulator as on the real thing. Only then is perfection 
achieved.

I set about collecting information. I found pieces of information on the
Internet, but not everything there is to know. So I tried to fill in the
gaps, the results of which I put on my website. Various people have helped
since then; this is the result of all those efforts and to my knowledge 
this document is the most complete.\\\\

(Jan)
Interested in emulation for a long time, but a few years after Sean
started writing this document, I have also started writing my own MSX emulator 
in 2003 and I've used this document quite a lot. Now (2005) the Z80 emulation 
is nearing perfection, I decided to add what extra I have learned and 
comments various people have sent to Sean, to this document.

I have restyled the document (although very little) to fit my personal needs 
and I have checked a lot of things that were already in here.\\\\

(Tomaz)
Added ZX Spectrum Next specific commands.


\section{Where to get this document}

The latest version is always available in {\LaTeX} and pdf at 
the following location:

{\tt http://www.myquest.nl/z80undocumented/}


\section{Feedback}

I welcome any kind of feedback. I would like to hear about any corrections
or additions you might have. Also note that there are a few flags which 
are still unknown, it would be great if someone found out how they work.
You can reach me at {\ttfamily jw@dds.nl} and my website can be 
found at {\ttfamily http://www.myquest.nl/z80undocumented/}.
Sean's website is at {\ttfamily http://www.msxnet.org/}.


\pagebreak
\section{ChangeLog}
\begin{description}

\item[29th March 2021 (Version 0.9n)]
Added ZX Spectrum Next information and instructions and restructured text for better maintainability and readability.

\item[15th Juni 2005 (version 0.9)] 
Corrected improper notation of JP x,nn mnemonics in opcode list, thanks to
Laurens Holst. Corrected a mistake in the INI, INIR, IND, INDR section 
and documented a mistake in offical Z80 documentation concerning Interrupt 
Mode 2, thanks to Boris Donko. Thanks to Aaldert Dekker for his ideas, for 
verifying many assumptions and writing instruction exercisers for various 
instruction groups.

\item[18th May 2005 (version 0.8)] 
Added an alphabetical list of instructions for easy reference and corrected an 
error in the 16-bit arithmetic section, SBC HL,nn sets the N-flag just like
other subtraction instructions, thanks to Fredrik Olssen for pointing that out.

\item[4th April 2005 (version 0.7)] 
I (Jan $<$jw@dds.nl$>$) will be maintaining this document from this version on.
I restyled the document to fix the page numbering issues, 
corrected an error in the I/O Block Instructions section, added 
graphics for the RLD and RRD instructions and corrected the spelling in several places.

\item[20th November 2003 (version 0.6)] 
Again, thanks to Ramsoft, added PF flag to {\tt OUTI}, {\tt INI} and
friends. Minor fix to {\tt DAA} tables, other minor fixes.

\item[13th November 2003 (version 0.5)]
Thanks to Ramsoft, add the correct tables for the {\tt DAA} instruction
(section \ref{daa}). Minor corrections \& typos, thanks to Jim Battle,
David Sutherland and most of all Fred Limouzin.

\item[September 2001 (version 0.4)]
Previous documents I had written were in plain text and Microsoft
Word, which I now find very embarrassing, so I decided to combine them all 
and use {\LaTeX}. Apart from a full re-write, the only changed information is
``Power on defaults'' (section \ref{pon}) and the algorithm for the CF 
and HF flags for {\tt OTIR} and friends (section \ref{block_io}).

\end{description}




\chapter{Overview}

\section{History of the Z80}

In 1969 Intel was approached by a Japanese company called Busicom to produce 
chips for Busicom's electronic desktop calculator. Intel suggested that the 
calculator should be built around a single-chip generalized computing engine 
and thus was born the first microprocessor --- the 4004. Although it was based 
on ideas from much larger mainframe and mini-computers the 4004 was cut down 
to fit onto a 16-pin chip, the largest that was available at the time, so that 
its data bus and address bus were each only 4-bits wide. 

Intel went on to improve the design and produced the 4040 (an improved 4-bit 
design) the 8008 (the first 8-bit microprocessor) and then in 1974 the 8080. 
This last one turned out to be a very useful and popular design and was used 
in the first home computer, the Altair 8800, and CP/M. 

In 1975 Federico Faggin who had had worked at Intel on the 4004 and its 
successors left the company and joined forces with Masatoshi Shima to from 
Zilog. At their new company Faggin and Shima designed a microprocessor that 
was compatible with Intel's 8080 (it ran all 78 instructions of the 8080 in 
almost the same way that Intel's chip did)\footnote{Thanks to Jim Battle 
$<$frustum@pacbell.net$>$: the 8080 always puts the parity in the PF flag;
VF does not exist and the timing is different. Possibly there are other
differences.} but had many more abilities (an extra 120 instructions, many 
more registers, simplified connection to hardware). Thus was born the 
mighty Z80! and thus was the empire forged.

The original Z80 was first released in July 1976, coincidentally Jan was born 
in the very same month. Since then newer versions have appeared with much of 
the same architecture but running at higher speeds. 
The original Z80 ran with a clock rate of 2.5MHz, the Z80A runs at 4MHz, the 
Z80B at 6MHz and the Z80H at 8Mhz. 

Many companies produced machines based around Zilog's improved chip during the 
1970's and 80's and because the chip could run 8080 code without needing any 
changes to the code the perfect choice of operating system was CP/M. 

Also Zilog has created a Z280, an enhanced version of the Zilog Z80 with a 16 bit
architecture, introduced in July, 1987. It added an MMU to expand addressing 
to 16Mb, features for multitasking, a 256 byte cache, and a huge number of 
new opcodes (giving a total of over 2000!). Its internal clock runs at 2 
or 4 times the external clock (e.g. a 16MHz CPU with a 4MHz bus

The Z380 CPU incorporates advanced architectural while maintaining Z80/
Z180 object code compatibility. The Z380 CPU is an enhanced version 
of the Z80 CPU. The Z80 instruction set has been retained, adding a full 
compliment of 16-bit arithmetic and logical operations, multiply
and divide, a complete set of register-to-register loads and exchanges, 
plus 32-bit load and exchange, and 32-bit arithmetic operations for 
address calculations.

The addressing modes of the Z80 have been enhanced
with Stack pointer relative loads and stores, 16-bit and 24-
bit indexed offsets and more flexible indirect register
addressing. All of the addressing modes allow access to
the entire 32-bit addressing space.


\section{Registers}

The following accessible registers exist in the Z80. \\[1ex]

\begin{tabular}{|c|c|l} \cline{1-2}
\tt A & \tt F & Accumulator and Flags \\ \cline{1-2} \cline{1-2}
\multicolumn{2}{|c|}{\tt BC} & \\ \cline{1-2}
\multicolumn{2}{|c|}{\tt DE} & General purpose registers \\ \cline{1-2}
\multicolumn{2}{|c|}{\tt HL} & \\ \cline{1-2}
\multicolumn{2}{|c|}{\tt IX} & \\ \cline{1-2}
\multicolumn{2}{|c|}{\tt IY} & \raisebox{1.5ex}[0pt]{Index registers} \\ \cline{1-2}
\multicolumn{2}{|c|}{\tt PC} & \\ \cline{1-2}
\multicolumn{2}{|c|}{\tt SP} & Special purpose registers \\ \cline{1-2}
\tt I & \tt R & \\ \cline{1-2}
\multicolumn{2}{|c|}{\tt AF'} & \\ \cline{1-2}
\multicolumn{2}{|c|}{\tt BC'} & \\ \cline{1-2}
\multicolumn{2}{|c|}{\tt DE'} & \raisebox{1.5ex}[0pt]{Alternate general purpose registers}\\ \cline{1-2}
\multicolumn{2}{|c|}{\tt HL'} & \\ \cline{1-2}
\end{tabular}
\\[1ex]

For interrupts, there are two interrupt flop-flops, IFF1 and IFF2, and the 
interrupt mode is retained. See chapter \ref{interrupts} for more about
interrupts. Also there is an internal register which is described in section 
\ref{block_io}.


\section{Flags}
\label{flags}

The conventional way of denoting the flags is with one letter, `C' for
the carry flag for example. It could be confused with the C register, so I've
chosen to use the `CF' notation for flags. Also in previous things I've
written I called the two undocumented flags 5 and 3, but now I've changed
to the same notation used in MAME\footnote{http://www.mame.net/}, which is
YF and XF, respectively. Note that in mnemonics the original way is still
maintained.\\[0.5ex]

\begin{tabular}{|l|c|c|c|c|c|c|c|c|} \hline
bit & 7 & 6 & 5 & 4 & 3 & 2 & 1 & 0 \\ \hline
flag & SF & ZF & YF & HF & XF & PF & NF & CF \\ \hline
\end{tabular}

\begin{description}

\item[SF flag]
  Set if the 2-complement value is negative. It's simply a copy of the most 
  significant bit.

\item[ZF flag]
  Set if the result is zero.

\item[YF flag]
  A copy of bit 5 of the result.

\item[HF flag]
  The half-carry of an addition/subtraction (from bit 3 to 4). Needed for
  BCD correction with {\tt DAA}.

\item[XF flag]
  A copy of bit 3 of the result.

\item[PF flag]
  This flag can either be the parity of the result (PF), or the \\ 2-compliment
  signed overflow (VF): set if 2-compliment value doesn't fit in the register.

\item[NF flag]
  Shows whether the last operation was an addition (0) or an subtraction (1).
  This information is needed for {\tt DAA}.\footnote{Wouldn't it be better to
  have separate instructions for {\tt DAA} after addition and subtraction, 
  like the 80x86 has in stead of sacrificing a bit in the flag register?}

\item[CF flag]
  The carry flag, set if there was a carry after the most significant bit.

\end{description}

Note that the only way to read the XF, YF and NF can only be read using
{\tt PUSH AF}.


\section{Power on defaults}
\label{pon}

Matt\footnote{redflame@xmission.com} has done some excellent research on 
this. He found that AF and SP are always set to FFFFh after a reset, and 
all other registers are undefined (different depending on how long the CPU 
has been powered off, different for different Z80 chips). Of course the PC 
should be set to 0 after a reset, and so should the IFF1 and IFF2 flags 
(otherwise strange things could happen). Also since the Z80 is 8080 
compatible, interrupt mode is probably 0.

Probably the best way to simulate this in an emulator is set PC, IFF1,
IFF2, IM to 0 and set all other registers to FFFFh.


\section{Pin Descriptions \cite{datasheet}}

This section might also relevant even if you don't do anything with 
hardware; it might give so insight into how the Z80 operates. Besides, 
it took me hours to draw this.\\[1ex]

{\tt
\setlength{\unitlength}{1mm}
\begin{picture}(50,80)
\multiput(11.8,2)(0,3.9){20}{\framebox(2,2){}}
\multiput(36.2,2)(0,3.9){20}{\framebox(2,2){}}
\thicklines \footnotesize
\put(14,0){\framebox(22,80){Z80 CPU}}
\put(25,80){\oval(3,3)[b]}
\thinlines 
\newcounter{pinno}
\multiput(15,76)(0,-3.9){20}{\stepcounter{pinno}\makebox(2,2){\arabic{pinno}}}
\multiput(33,2)(0,3.9){20}{\stepcounter{pinno}\makebox(2,2)[r]{\arabic{pinno}}}

\put(9,2){\makebox(2,2)[r]{$\mathtt{\overline{IORQ}}$}}
\put(9,5.9){\makebox(2,2)[r]{$\mathtt{\overline{MREQ}}$}}
\put(9,9.8){\makebox(2,2)[r]{$\mathtt{\overline{HALT}}$}}
\put(9,13.7){\makebox(2,2)[r]{$\mathtt{\overline{NMI}}$}}
\put(9,17.6){\makebox(2,2)[r]{$\mathtt{\overline{INT}}$}}
\put(9,21.5){\makebox(2,2)[r]{$\mathtt{D_1}$}}
\put(9,25.4){\makebox(2,2)[r]{$\mathtt{D_0}$}}
\put(9,29.3){\makebox(2,2)[r]{$\mathtt{D_7}$}}
\put(9,33.2){\makebox(2,2)[r]{$\mathtt{D_2}$}}
\put(9,37.1){\makebox(2,2)[r]{$\mathtt{+ 5V}$}}
\put(9,41){\makebox(2,2)[r]{$\mathtt{D_6}$}}
\put(9,44.9){\makebox(2,2)[r]{$\mathtt{D_5}$}}
\put(9,48.8){\makebox(2,2)[r]{$\mathtt{D_3}$}}
\put(9,52.7){\makebox(2,2)[r]{$\mathtt{D_4}$}}
\put(9,56.6){\makebox(2,2)[r]{$\mathtt{CLK}$}}
\put(9,60.5){\makebox(2,2)[r]{$\mathtt{A_{15}}$}}
\put(9,64.4){\makebox(2,2)[r]{$\mathtt{A_{14}}$}}
\put(9,68.3){\makebox(2,2)[r]{$\mathtt{A_{13}}$}}
\put(9,72.2){\makebox(2,2)[r]{$\mathtt{A_{12}}$}}
\put(9,76.1){\makebox(2,2)[r]{$\mathtt{A_{11}}$}}

\put(39,2){\makebox(2,2)[l]{$\mathtt{\overline{RD}}$}}
\put(39,5.9){\makebox(2,2)[l]{$\mathtt{\overline{WR}}$}}
\put(39,9.8){\makebox(2,2)[l]{$\mathtt{\overline{BUSACK}}$}}
\put(39,13.7){\makebox(2,2)[l]{$\mathtt{\overline{WAIT}}$}}
\put(39,17.6){\makebox(2,2)[l]{$\mathtt{\overline{BUSREQ}}$}}
\put(39,21.5){\makebox(2,2)[l]{$\mathtt{\overline{RESET}}$}}
\put(39,25.4){\makebox(2,2)[l]{$\mathtt{\overline{M1}}$}}
\put(39,29.3){\makebox(2,2)[l]{$\mathtt{\overline{RFSH}}$}}
\put(39,33.2){\makebox(2,2)[l]{$\mathtt{GND}$}}
\put(39,37.1){\makebox(2,2)[l]{$\mathtt{A_0}$}}
\put(39,41.0){\makebox(2,2)[l]{$\mathtt{A_1}$}}
\put(39,44.9){\makebox(2,2)[l]{$\mathtt{A_2}$}}
\put(39,48.8){\makebox(2,2)[l]{$\mathtt{A_3}$}}
\put(39,52.7){\makebox(2,2)[l]{$\mathtt{A_4}$}}
\put(39,56.6){\makebox(2,2)[l]{$\mathtt{A_5}$}}
\put(39,60.5){\makebox(2,2)[l]{$\mathtt{A_{6}}$}}
\put(39,64.4){\makebox(2,2)[l]{$\mathtt{A_{7}}$}}
\put(39,68.3){\makebox(2,2)[l]{$\mathtt{A_{8}}$}}
\put(39,72.2){\makebox(2,2)[l]{$\mathtt{A_{9}}$}}
\put(39,76.1){\makebox(2,2)[l]{$\mathtt{A_{10}}$}}

\end{picture}
}

\begin{description}

\item[$\mathtt{A_{15}-A_0}$] 
  {\em Address bus} (output, active high, 3-state). This bus is used for 
  accessing the memory and for I/O ports. During the refresh cycle the IR 
  register is put on this bus.

\item[$\mathtt{\overline{BUSACK}}$]
  {\em Bus Acknowledge} (output, active low). Bus Acknowledge indicates 
  to the requesting device that the CPU address bus, data bus, and control
  signals $\mathtt{\overline{MREQ}}$, 
  $\mathtt{\overline{IORQ}}$, 
  $\mathtt{\overline{RD}}$ and $\mathtt{\overline{WR}}$ have been entered into
  their high-impedance states. The external device now control these lines.

\item[$\mathtt{\overline{BUSREQ}}$]
  {\em Bus Request} (input, active low). Bus Request has a higher priority than
  $\mathtt{\overline{NMI}}$ and is always recognised at the end of the current
  machine cycle. $\mathtt{\overline{BUSREQ}}$ forces the CPU address bus, data
  bus and control signals $\mathtt{\overline{MREQ}}$,  
  $\mathtt{\overline{IORQ}}$, 
  $\mathtt{\overline{RD}}$ and $\mathtt{\overline{WR}}$ to go to a 
  high-impedance state so that other devices can control these lines. 
  $\mathtt{\overline{BUSREQ}}$ is normally wired-OR and requires an external 
  pullup for these applications. Extended $\mathtt{\overline{BUSREQ}}$ 
  periods due to extensive DMA operations can prevent the CPU from refreshing
  dynamic RAMs.

\item[$\mathtt{D_7-D_0}$]
  {\em Data Bus} (input/output, active low, 3-state). Used for data exchanges
  with memory, I/O and interrupts.

\item[$\mathtt{\overline{HALT}}$]
  {\em Halt State} (output, active low). Indicates that the CPU has executed 
  a {\tt HALT} instruction and is waiting for either a maskable or nonmaskable 
  interrupt (with the mask enabled) before operation can resume. While halted,
  the CPU stops increasing the PC so the instruction is re-executed, to
  maintain memory refresh.

\item[$\mathtt{\overline{INT}}$]
  {\em Interrupt Request} (input, active low). Interrupt Request is generated
  by I/O devices. The CPU honours a request at the end of the current 
  instruction if IFF1 is set. $\mathtt{\overline{INT}}$ is normally wired-OR
  and requires an external pullup for these applications.

\item[$\mathtt{\overline{IORQ}}$]
  {\em Input/Output Request} (output, active low, 3-state). Indicates that
  the address bus holds a vailid I/O address for an I/O read or write 
  operation. $\mathtt{\overline{IORQ}}$ is also generated concurrently with
  $\mathtt{\overline{M1}}$ during an interrupt acknowledge cycle to indicate
  that an interrupt response vector can be placed on the databus.

\item[$\mathtt{\overline{M1}}$]
  {\em Machine Cycle One} (output, active low). $\mathtt{\overline{M1}}$,
  together with $\mathtt{\overline{MREQ}}$, indicates that the current 
  machine cycle is the opcode fetch cycle of an instruction execution.
  $\mathtt{\overline{M1}}$, together with $\mathtt{\overline{IORQ}}$,
  indicates an interrupt acknowledge cycle.

\item[$\mathtt{\overline{MREQ}}$]
  {\em Memory Request} (output, active low, 3-state). Indicates that the
  address holds a valid address for a memory read or write cycle operations.

\item[$\mathtt{\overline{NMI}}$]
  {\em Non-Maskable Interrupt} (input, negative edge-triggered). 
  $\mathtt{\overline{NMI}}$ has a higher priority than 
  $\mathtt{\overline{INT}}$. $\mathtt{\overline{NMI}}$ is always recognised 
  at the end of an instruction, independent of the status of the 
  interrupt flip-flops and automatically forces the CPU to restart at 
  location 0066h.

\item[$\mathtt{\overline{RD}}$]
  {\em Read} (output, active low, 3-state). Indicates that the 
  CPU wants to read data from memory or an I/O device. The addressed I/O device
  or memory should use this signal to place data onto the data bus.

\item[$\mathtt{\overline{RESET}}$]
  {\em Reset} (input, active low). Initializes the CPU as follows: it resets
  the interrupt flip-flops, clears the PC and IR registes, and set the 
  interrupt mode to 0. During reset time, the address bus and data bus go
  to a high-impedance state, and all control output signals go to 
  the inactive state. Note that $\mathtt{\overline{RESET}}$ must be active
  for a minimum of three full clock cycles before the reset operation is
  complete. Note that Matt found that SP and AF are set to FFFFh.
  
\item[$\mathtt{\overline{RFSH}}$]
  {\em Refresh} (output, active low). $\mathtt{\overline{RFSH}}$, together
  with $\mathtt{\overline{MREQ}}$, indicates that the IR registers are on
  the address bus (note that only the lower 7 bits are useful) and can be 
  used for the refresh of dynamic memories.

\item[$\mathtt{\overline{WAIT}}$]
  {\em Wait} (input, active low). Indicates to the CPU that the addressed
  memory or I/O device are not ready for data transfer. The CPU continues
  to enter a wait state as long as this signal is active. Note that during
  this period memory is not refreshed.

\item[$\mathtt{\overline{WR}}$]
  {\em Write} (output, active low, 3-state). Indicates that the 
  CPU wants to write data to memory or an I/O device. The addressed I/O device
  or memory should use this signal to store the data on the data bus.

\end{description}


\chapter{Undocumented Opcodes}

There are quite a few undocumented opcodes/instructions. This section 
should describe every possible opcode so you know what will be executed, 
whatever the value of the opcode is.

The following prefixes exist: CB, ED, DD, FD, DDCB and FDCB. Prefixes change 
the way the following opcodes are interpreted. All instructions without a 
prefix (not a value of one the above) are single byte opcodes\footnote{
Without the operand, that is.}, which are documented in the official 
documentation.


\section{CB Prefix \cite{gerton}}
\label{prefix_cb}

An opcode with a CB prefix is a rotate, shift or bit test/set/reset 
instruction. There are a few instructions missing from the official list, 
which are usually denoted with {\tt SLL} (Shift Logical Left). It works like 
{\tt SLA}, for one exception: it sets bit 0 ({\tt SLA} resets it).

{\tt
  \begin{tabbing}
    {\qquad}\=CB30\qquad\=SLL B \+\\
       CB31\>SLL C\\
       CB32\>SLL D\\
       CB33\>SLL E\\
       CB34\>SLL H\\
       CB35\>SLL L\\
       CB36\>SLL (HL)\\
       CB37\>SLL A
\end{tabbing}
}


\section{DD Prefix \cite{gerton}}

In general, the instruction following the DD prefix is executed as is, but
if the HL register is supposed to be used the IX register is used instead.
Here are the rules:

\begin{itemize}

\item Any usage of HL is treated as an access to IX (except {\tt EX DE,HL} and 
  {\tt EXX} and the ED prefixed instructions that use HL).
\item Any access to (HL) is changed to (IX+d), where `d' is a signed displacement
  byte placed after the main opcode --- except {\tt JP (HL)}, which isn't indirect
  anyway. The mnemonic should be {\tt JP HL}.
\item Any access to H is treated as an access to IXh (the high byte of IX)
  Except if (IX+d) is used as well.
\item Any access to L is treated as an access to IXl (the low byte of IX)
  Except if (IX+d) is used as well.
\item A DD prefix before a CB selects a completely different instruction 
  set, see Section {\ref{prefix_cbdd}}.\\
\end{itemize}

Some examples:

{\tt 
\begin{tabbing}
{\qquad}\={\rm Without DD prefix}\qquad\={\rm With DD prefix}\+ \\
LD H,(HL)\>LD H,(IX+d)\\
LD H,A\>LD IXh,A\\
LD L,H\>LD IXl,IXh\\
JP (HL)\>JP (IX)\\
LD DE,0\>LD DE,0\\
LD HL,0\>LD IX,0\\
\end{tabbing}
}


\section{FD Prefix \cite{gerton}}

This prefix has the same effect as the DD prefix, though IY is used instead 
of IX.  Note {\tt LD IXl,IYh} is not possible: only IX or IY is accessed in 
one instruction, never both.


\section{ED Prefix \cite{gerton}}

There are a number of undocumented EDxx instructions, of which most are
duplicates of documented instructions. Any instruction not listed has
no effect (same behaviour as 2 {\tt NOP} instructions).

The complete list except for the block instructions:

\renewcommand{\thefootnote}{\fnsymbol{footnote}}
{\footnotetext[7]{Undocumented instruction}}
{\tt
\begin{tabbing}
{\qquad}ED40\quad\=IN B,(C)\qquad\qquad\=ED60\quad\=IN H,(C)\\ 
{\qquad}ED41\>OUT (C),B\>ED61\>OUT (C),H\\ 
{\qquad}ED42\>SBC HL,BC\>ED62\>SBC HL,HL\\  
{\qquad}ED43\>LD (nn),BC\>ED63\>LD (nn),HL\\ 
{\qquad}ED44\>NEG\>ED64\>NEG{\footnotemark[7]}\\ 
{\qquad}ED45 \> RETN\> ED65\> RETN{\footnotemark[7]}\\
{\qquad}ED46\>IM 0\>ED66\>IM 0{\footnotemark[7]}\\
{\qquad}ED47\>LD I,A\>ED67\>RRD\\
{\qquad}ED48\>IN C,(C)\>ED68\>IN L,(C)\\
{\qquad}ED49\>OUT (C),C\>ED69\>OUT (C),L\\
{\qquad}ED4A\>ADC HL,BC\>ED6A\>ADC HL,HL\\
{\qquad}ED4B\>LD BC,(nn)\>ED6B\>LD HL,(nn)\\
{\qquad}ED4C\>NEG{\footnotemark[7]}\>ED6C\>NEG{\footnotemark[7]}\\
{\qquad}ED4D\>RETI\>ED6D\>RETN\footnotemark[7]\\
{\qquad}ED4E\>IM 0{\footnotemark[7]}\>ED6E\>IM 0{\footnotemark[7]}\\
{\qquad}ED4F\>LD R,A\>ED6F\>RLD\\
{\qquad}\\
{\qquad}ED50\>IN D,(C)\>ED70\>IN (C) / IN F,(C){\footnotemark[7]}\\
{\qquad}ED51\>OUT (C),D\>ED71\>OUT (C),0{\footnotemark[7]}\\
{\qquad}ED52\>SBC HL,DE\>ED72\>SBC HL,SP\\
{\qquad}ED53\>LD (nn),DE\>ED73\>LD (nn),SP\\
{\qquad}ED54\>NEG{\footnotemark[7]}\>ED74\>NEG{\footnotemark[7]}\\
{\qquad}ED55\>RETN{\footnotemark[7]}\>ED75\>RETN{\footnotemark[7]}\\
{\qquad}ED56\>IM 1\>ED76\>IM 1{\footnotemark[7]}\\
{\qquad}ED57\>LD A,I\>ED77\>NOP{\footnotemark[7]}\\
{\qquad}ED58\>IN E,(C)\>ED78\>IN A,(C)\\
{\qquad}ED59\>OUT (C),E\>ED79\>OUT (C),A\\
{\qquad}ED5A\>ADC HL,DE\>ED7A\>ADC HL,SP\\
{\qquad}ED5B\>LD DE,(nn)\>ED7B\>LD SP,(nn)\\
{\qquad}ED5C\>NEG{\footnotemark[7]}\>ED7C\>NEG{\footnotemark[7]}\\
{\qquad}ED5D\>RETN{\footnotemark[7]}\>ED7D\>RETN{\footnotemark[7]}\\
{\qquad}ED5E\>IM 2\>ED7E\>IM 2{\footnotemark[7]}\\
{\qquad}ED5F\>LD A,R\>ED7F\>NOP{\footnotemark[7]}\\
\end{tabbing}
}
\renewcommand{\thefootnote}{\arabic{footnote}}

The ED70 instruction reads from I/O port C, but does not store the result. 
It just affects the flags like the other {\tt IN x,(C)} instructions. 
ED71 simply outs the value 0 to I/O port C.

The ED63 is a duplicate of the 22 opcode ({\tt LD (nn),HL}) and similarly
ED6B is a duplicate of the 2A opcode. Of course the timings are
different. These instructions are listed in the official documentation.

According to Gerton Lunter\footnote{gerton@math.rug.nl}:

\begin{quote}
  The instructions ED 4E and ED 6E are {\tt IM 0} equivalents: when FF was put
  on the bus (physically) at interrupt time, the Spectrum continued to
  execute normally, whereas when an EF ({\tt RST 28h}) was put on the bus it
  crashed, just as it does in that case when the Z80 is in the official
  interrupt mode 0.  In {\tt IM 1} the Z80 just executes a {\tt RST 38h} 
  (opcode FF) no matter what is on the bus.
\end{quote}

All the {\tt RETI/RETN} instructions are the same, all like the {\tt RETN}
instruction. So they all, including {\tt RETI}, copy IFF2 to IFF1. More 
information on {\tt RETI} and {\tt RETN} and {\tt IM x} is in 
section \ref{flipflop}.


\section{DDCB Prefix}
\label{prefix_cbdd}

The undocumented DDCB instructions store the result (if any) of the 
operation in one of the seven all-purpose registers, which one depends on
the lower 3 bits of the last byte of the opcode (not operand, so not
the offset).

\begin{tabbing}
{\qquad}000\quad\=B\\
{\qquad}001\>C\\
{\qquad}010\>D\\
{\qquad}011\>E\\
{\qquad}100\>H\\
{\qquad}101\>L\\
{\qquad}110\>(none: documented opcode)\\
{\qquad}111\>A
\end{tabbing}

The documented DDCB0106 is {\tt RLC (IX+01h)}. So, clear the lower three bits 
(DDCB0100) and something is done to register B. The result of the RLC
(which is stored in (IX+01h)) is now also stored in register B. Effectively,
it does the following:

{\tt
\begin{tabbing}
{\qquad}\=LD B,(IX+01h)\+\\
RLC B\\
LD (IX+01h),B
\end{tabbing}
}

So you get double value for money. The result is stored in B and (IX+01h). The
most common notation is: {\tt RLC (IX+01h),B}

I've once seen this notation:
 
{\tt
  \begin{tabbing}
    {\qquad}\=RLC (IX+01h)\+\\
    LD B,(IX+01h)
  \end{tabbing}
}

That's not correct: B contains the rotated value, even if (IX+01h) points to 
ROM. The DDCB SET and RES instructions do the same thing as the shift/rotate
instructions:

{\tt 
\begin{tabbing}
{\qquad}\=DDCB10C0{\qquad}\=SET 0,(IX+10h),B\+\\
DDCB10C1	\>SET 0,(IX+10h),C\\
DDCB10C2	\>SET 0,(IX+10h),D\\
DDCB10C3	\>SET 0,(IX+10h),E\\
DDCB10C4	\>SET 0,(IX+10h),H\\
DDCB10C5	\>SET 0,(IX+10h),L\\
DDCB10C6	\>SET 0,(IX+10h) {\rm - documented instruction}\\
DDCB10C7	\>SET 0,(IX+10h),A
\end{tabbing}
}

So for example with the last instruction, the value of (IX+10h) with bit 0 
set is also stored in register A.

The DDCB BIT instructions do not store any value; they merely test a bit.
That's why the undocumented DDCB BIT instructions are no different from
the official ones:

{\tt
  \begin{tabbing}
    {\qquad}\=DDCB d 78{\qquad}\=BIT 7,(IX+d)\+\\
    DDCB d 79\>BIT 7,(IX+d)\\
    DDCB d 7A\>BIT 7,(IX+d)\\
    DDCB d 7B\>BIT 7,(IX+d)\\
    DDCB d 7C\>BIT 7,(IX+d)\\
    DDCB d 7D\>BIT 7,(IX+d)\\
    DDCB d 7E\>BIT 7,(IX+d) {\rm - documented instruction}\\
    DDCB d 7F\>BIT 7,(IX+d)
  \end{tabbing}
}


\section{FDCB Prefixes}

Same as for the DDCB prefix, though IY is used instead of IX.


\section{Combinations of Prefixes}

This part may be of some interest to emulator coders. Here we define
what happens if strange sequences of prefixes appear in the instruction
cycle of the Z80.

If CB or ED is encountered, that byte plus the next make up an instruction.
FD or DD should be seen as prefix setting a flag which says ``use IX or IY in
stead of HL'', and not an instruction. In a large sequence of DD and FD
bytes, it is the last one that counts. Also any other byte (or instruction)
resets this flag. \\[0.5ex]

{\tt {\qquad}FD DD 00 21 00 10{\qquad}NOP NOP NOP LD HL,1000h}




\chapter{Undocumented Effects}

\section{BIT instructions}
\label{bit_flags}

{\tt BIT n,r} behaves much like {\tt AND r,2{\raisebox{1ex}{n}}} with 
the result thrown away, and CF flag unaffected. Compare {\tt BIT 7,A} 
with {\tt AND 80h}: flag YF and XF are reset, SF is set if bit 7 was
actually set; ZF is set if the result was 0 (bit was reset), and PF is
effectively set if ZF is set (the result of the AND leaves either no
bits set (PF set - parity even) or one bit set (PF reset - parity odd). 
So the rules for the flags are:

\begin{description}

\item[SF flag]
  Set if n = 7 and tested bit is set.

\item[ZF flag]
  Set if the tested bit is reset.

\item[YF flag]
  Set if n = 5 and tested bit is set.

\item[HF flag]
  Always set.

\item[XF flag]
  Set if n = 3 and tested bit is set.

\item[PF flag]
  Set just like ZF flag.

\item[NF flag]
  Always reset. 

\item[CF flag]
  Unchanged.

\end{description}

This is where things start to get strange. With the {\tt BIT n,(IX+d)}
instructions, the flags behave just like the 
{\tt BIT n,r} instruction, except for YF and XF. These are not copied 
from the result but from something completely different, namely bit 5 and
3 of the high byte of IX+d (so IX plus the displacement).

Things get more bizarre with the {\tt BIT n,(HL)} instruction. Again,
except for YF and XF the flags are the same. YF and XF are copied from
some sort of internal register. This register is related to 16 bit 
additions. Most instructions do not change this register. Unfortunately, 
I haven't tested all instructions yet, but here is the list so far.

\begin{description}

\item[{\tt ADD HL,xx}]
Use the high byte of HL, ie. H before the addition.

\item[{\tt LD r,(IX+d)}]
Use high byte of the resulting address IX+d.

\item[{\tt JR d}]
Use high byte target address of the jump.

\item[{LD r,r'}]
Doesn't change this register.

\end{description}

Any help here would be most appreciated!


\section{Memory Block Instructions \cite{mrison}}
\label{block_mem}

The {\tt LDI/LDIR/LDD/LDDR} instructions affect the flags in a strange way.
At every iteration, a byte is copied. Take that byte and add the value of 
register A to it. Call that value n. Now, the flags are:

\begin{description}
 
\item[YF flag]
  A copy of bit 1 of n.

\item[HF flag]
  Always reset.

\item[XF flag]
  A copy of bit 3 of n.

\item[PF flag]
  Set if BC not 0.

\item[SF, ZF, CF flags]
  These flags are unchanged.

\end{description}

And now for {\tt CPI/CPIR/CPD/CPDR}. This instruction compares a series of
bytes in memory to register A. Effectively, it can be said it does 
{\tt CP (HL)} at every iteration. The result of that compare sets the
HF flag, which is important for the next step. Take the value of register A,
subtract the value of the memory address, and finally subtract the value
of HF flag, which is set or reset by the hypothetical {\tt CP (HL)}. So,
{\tt n = A - (HL) - HF}.

\begin{description}

\item[SF, ZF, HF flags]
  Set by the hypothetical {\tt CP (HL)}.

\item[YF flag]
  A copy of bit 1 of n.

\item[XF flag]
  A copy of bit 3 of n.

\item[PF flag]
  Set if BC is not 0.

\item[NF flag]
  Always set.

\item[CF flag]
  Unchanged.

\end{description}


\section{I/O Block Instructions}
\label{block_io}

These are the most be bizarre instructions, as far as flags is concerned. 
Ramsoft found all of the flags. The out instructions behave differently 
than the in instructions, which doesn't make the CPU very symmetrical. 

First of all, all instructions affect the following flags:

\begin{description}

\item[SF, ZF, YF, XF flags]
  Affected by decreasing register B, as in {\tt DEC B}.

\item[NF flag]
  A copy of bit 7 of the value read from or written to an I/O port.

\end{description}

And now the for {\tt OUTI/OTIR/OUTD/OTDR} instructions. Take state of the
L after the increment or decrement of HL; add the value written to the I/O 
port to; call that k for now. If k $>$ 255, then the CF and HF flags are set. 
The PF flags is set like the parity of k bitwise and'ed with 7, bitwise 
xor'ed with B.

\begin{description}

\item[HF and CF]
  Both set if ((HL) + L  $>$ 255)

\item[PF]
  The parity of ((((HL) + L) \& 7) xor B)

\end{description}

{\tt INI/INIR/IND/INDR} use the C register in stead of the L register. There 
is a catch though, because not the value of C is used, but C + 1 if it's
{\tt INI/INIR} or C - 1 if it's {\tt IND/INDR}. So, first of all {\tt 
INI/INIR}:

\begin{description}

\item[HF and CF]
  Both set if ((HL) + ((C + 1) \& 255)  $>$ 255)

\item[PF]
  The parity of (((HL) + ((C + 1) \& 255)) \& 7) xor B)

\end{description}

And last {\tt IND/INDR}:

\begin{description}

\item[HF and CF]
  Both set if ((HL) + ((C - 1) \& 255) $>$ 255)

\item[PF]
  The parity of (((HL) + ((C - 1) \& 255)) \& 7) xor B)

\end{description}


\section{16 Bit I/O ports}

Officially the Z80 has an 8 bit I/O port address space. When using the I/O
ports, the 16 address lines are used. And in fact, the high 8 bit do actually
have some value, so you can use 65536 ports after all. {\tt IN r,(C)},
{\tt OUT (C),r}, and the Block I/O instructions  actually place the entire BC
register on the address bus. Similarly {\tt IN A,(n)} and {\tt OUT (n),A} put
A $\times$ 256 + n on the address bus.

The {\tt INI/INIR/IND/INDR} instructions use BC after decrementing B, and
the {\tt OUTI/OTIR/OUTD/OTDR} instructions before.


\section{Block Instructions}

The repeated block instructions simply decrease the PC by two so the 
instruction is simply re-executed. So interrupts can occur during block 
instructions. So, {\tt LDIR} is simply {\tt LDI} + if BC is not 0, decrease 
PC by 2.


\section{16 Bit Additions}

The 16 bit additions are a bit more complicated than 8 bit ones. Since the 
Z80 is an 8-bit CPU, 16 bit additions are done in two stages: first the 
lower bytes are added, then the two higher bytes. The SF, YF, HF, XF flags 
are affected as by the second (high) 8 bit addition. ZF is set if the whole 
16 bit result is 0.


\section{DAA Instruction}
\label{daa}

This instruction is useful when you're using BCD values. After an addition
or subtraction, {\tt DAA} corrects the value back to BCD again.  Note that 
it uses the CF flag, so it cannot be used after {\tt INC} and {\tt DEC}.

Stefano Donati from Ramsoft\footnote{http://www.ramsoft.bbk.org/} has
found the tables which describe the {\tt DAA} operation. The input is
the A register and the CF, NF, HF flags. Result is as follows: \\[2ex]


Depending on the NF flag, the `diff' from this table must be added (NF is
reset) or subtracted (NF is set) to A.\\[2ex]

% ----------------------------------
% | NF | CF |  hi | HF |  lo | add |
% |----|----|-----|----|-----|-----|
% |  0 |  0 | 0-9 |  0 | 0-9 |  00 |
% |  0 |  0 | 0-9 |  1 | 0-9 |  06 |
% |  0 |  0 | 0-8 |  * | a-f |  06 |
% |  0 |  0 | a-f |  0 | 0-9 |  60 |
% |  0 |  1 |  *  |  0 | 0-9 |  60 |
% |  0 |  1 |  *  |  1 | 0-9 |  66 |
% |  0 |  1 |  *  |  * | a-f |  66 |
% |  0 |  0 | 9-f |  * | a-f |  66 |
% |  0 |  0 | a-f |  1 | 0-9 |  66 |
% ----------------------------------
% |  1 |  0 | 0-9 |  0 | 0-9 |  00 |
% |  1 |  0 | 0-9 |  1 | 0-9 |  fa |
% |  1 |  0 | 0-8 |  * | a-f |  fa |
% |  1 |  0 | a-f |  0 | 0-9 |  a0 |
% |  1 |  1 |  *  |  0 | 0-9 |  a0 |
% |  1 |  1 |  *  |  1 | 0-9 |  9a |
% |  1 |  1 |  *  |  * | a-f |  9a |
% |  1 |  0 | 9-f |  * | a-f |  9a |
% |  1 |  0 | a-f |  1 | 0-9 |  9a |
% ----------------------------------
 
\begin{tabular}{c|c|c|c|c}
CF &  high   & HF & low    & diff \\
   &  nibble &    & nibble &      \\ \hline
 0 & 0-9     &  0 & 0-9    &  00  \\
 0 & 0-9     &  1 & 0-9    &  06  \\
 0 & 0-8     &  * & a-f    &  06  \\
 0 & a-f     &  0 & 0-9    &  60  \\
 1 &  *      &  0 & 0-9    &  60  \\
 1 &  *      &  1 & 0-9    &  66  \\
 1 &  *      &  * & a-f    &  66  \\
 0 & 9-f     &  * & a-f    &  66  \\
 0 & a-f     &  1 & 0-9    &  66  \\ \hline
\end{tabular}
\\[2ex]

% ------------------------
% | CF |  hi |  lo | CF' |
% |----|-----|-----|-----|
% |  0 | 0-9 | 0-9 |  0  |
% |  0 | 0-8 | a-f |  0  |
% |  0 | 9-f | a-f |  1  |
% |  0 | a-f | 0-9 |  1  |
% |  1 |  *  |  *  |  1  |
% ------------------------

The CF flag is affected as follows: \\[2ex]

\begin{tabular}{c|c|c|c}
CF & high   & low    & CF' \\
   & nibble & nibble &     \\ \hline
 0 & 0-9    & 0-9    &  0  \\
 0 & 0-8    & a-f    &  0  \\
 0 & 9-f    & a-f    &  1  \\
 0 & a-f    & 0-9    &  1  \\
 1 &  *     &  *     &  1  \\ \hline
\end{tabular}
\\[2ex]

The NF flags is affected as follows: \\[2ex]
% -----------------------
% | NF | HF |  lo | HF' |
% |----|----|-----|-----|
% |  0 |  * | 0-9 |  0  |
% |  0 |  * | a-f |  1  |
% -----------------------
% |  1 |  0 |  *  |  0  |
% |  1 |  1 | 6-f |  0  |
% |  1 |  1 | 0-5 |  1  |
% -----------------------
 
\begin{tabular}{c|c|c|c}
NF & HF & low    & HF' \\
   &    & nibble &     \\ \hline
 0 &  * & 0-9    &  0  \\
 0 &  * & a-f    &  1  \\
 1 &  0 &  *     &  0  \\
 1 &  1 & 6-f    &  0  \\
 1 &  1 & 0-5    &  1  \\ \hline
\end{tabular}
\\[2ex]

SF, YF, XF are copies of bit 7,5,3 of the result respectively; ZF is set 
according to the result and NF is always unchanged.


\chapter{Interrupts}
\label{interrupts}

There are two types of interrupts, maskable and non-maskable. The maskable
type is ignored if IFF1 is reset. Non-maskable interrupts (NMI) will
are always accepted, and they have a higher priority, so if the two are 
requested at the same time the NMI will be accepted first.

For the interrupts, the following things are important: Interrupt Mode
(set with the {\ttfamily IM 0, IM 1, IM 2} instructions), the interrupt 
flip-flops (IFF1 and IFF2), and the I register. When a maskable interrupt
is accepted, a external device can put a value on the databus.

Both types of interrupts increase the R register by one, when accepted.


\section{Non-Maskable Interrupts (NMI)}

When a NMI is accepted, IFF1 is reset. At the end of the routine, IFF1 must
be restored (so the running program is not affected). That's why IFF2 is
there; to keep a copy of IFF1.

An NMI is accepted when the NMI pin on the Z80 is made low (edge-triggered). 
The Z80 responds to the {\em change} of the line from +5 to 0 --- so the 
interrupt line doesn't have a state, it's just a pulse. When this happens, 
a call is done to address 0066h and IFF1 is reset so the routine isn't 
bothered by maskable interrupts. The routine should end with an 
{\ttfamily RETN} (RETurn from Nmi) which is just a usual RET, but also 
copies IFF2 to IFF1, so the IFFs are the same as before the interrupt.

You can check whether interrupts were disabled or not during an NMI by using 
the {\ttfamily LD A,I} or {\ttfamily LD A,R} instruction. These instructions 
copy IFF2 to the PF flag.

Accepting an {\tt NMI} costs 11 t-states.


\section{Maskable Interrupts (INT)}

If the INT line is low and IFF1 is set, a maskable interrupt is accepted --- 
whether or not the last INT routine has finished. That's why you should 
not enable interrupts during such a routine, and make sure that the device 
that generated it has put the INT line up again before ending the routine.
So unlike NMI interrupts, the interrupt line has a state; it's not a pulse.

When an INT is accepted, both IFF1 and IFF2 are cleared, preventing another 
interrupt from occurring which would end up as an infinite loop (and overflowing
the stack). What happens next depends on the Interrupt Mode.

A device can place a value on the databus when the interrupt is accepted. Some
computer systems do not utilize this feature, and this value ends up 
being FFh.

\begin{description}

\item[Interrupt Mode 0]
  This is the 8080 compatibility mode. The instruction on the bus is executed 
  (usually an {\tt RST} instruction, but it can be anything.  The I 
  register is not used. Assuming it a {\tt RST} instruction, accepting
  this takes 13 t-states.

\item[Interrupt Mode 1]
  An RST 38h is executed, no matter what value is put on the
  bus or what value the I register has. Accepting this type costs 13 t-states.

\item[Interrupt Mode 2]
  A call is made to the address read from memory. What address is
  read from is calculated as follows: (I register) $\times$ 256 + 
  (value on bus). Zilog's user manual states (very convincingly) that the 
  least significant bit of the address is always 0, so they calculate the 
  address that is read from as: (I register) $\times$ 256 + (value on bus \& 0xFE). 
  I have tested this and it not correct. Of course a word (two bytes) are read, 
  making the address where the call is made to. In this way, you can have a vector 
  table for interrupts. Accepting this of interrupt type costs 19 t-states.

\end{description}

At the end of a maskable interrupt, the interrupts should be enabled again. 
You can assume that was the state of the IFFs because otherwise the interrupt 
wasn't accepted. So, an INT routine always ends with an {\ttfamily EI} and 
a {\ttfamily RET} ({\ttfamily RETI} according to the official documentation,
more about that later):

{\ttfamily
\begin{tabbing}
INT:\\
{\qquad}\=.\\
\>.\\
\>.\\
\>EI\\
\>RETI {\rmfamily or} RET
\end{tabbing}
}

Note a fact about {\ttfamily EI}: a maskable interrupt isn't accepted directly 
after it, so the next opportunity for an interrupt is after the 
{\ttfamily RETI}. This is very useful; if the INT line is still low, an 
interrupt is accepted again.  If this happens a lot and the interrupt is 
generated before the {\ttfamily RETI}, the stack could overflow (since the 
routine would be called again and again). But this property of {\ttfamily EI} 
prevents this.

{\ttfamily DI} is not necessary at the start of the interrupt routine: the
interrupt flip-flops are cleared when accepting the interrupt.

You can use {\tt RET} instead of {\tt RETI}, depending on the hardware setup. 
{\tt RETI} is only useful if you have something like a Z80 PIO to support 
daisy-chaining: queuing interrupts. The PIO can detect that the routine has 
ended by the opcode of {\tt RETI}, and let another device generate an 
interrupt. That is why I called all the undocumented EDxx RET instructions 
{\tt RETN}: All of them operate alike, the only difference of {\tt RETI} is 
its specific opcode which the Z80 PIO recognises.


\section{Things affecting the Interrupt flip-flops}
\label{flipflop}

All the IFF related things are:

\begin{tabbing}
Accept NMI{\qquad}\=IFF1{\qquad}\=IFF2{\qquad}\= \kill
\>IFF1\>IFF2\>\\
CPU reset\>0\>0\\
{\ttfamily DI}\>0\>0\\
{\ttfamily EI}\>1\>1\\
Accept INT\>0\>0\\
Accept NMI\>0\>-\\
{\ttfamily RETI/N}\>IFF2\>-\>All the EDxx RETI/N instructions\\
{\ttfamily LD A,I/LD A,R}\>-\>-\>Copies IFF2 into PF flag
\end{tabbing}

If you're working with a Z80 system without NMIs (like the MSX), you can 
forget all about the two separate IFFs; since a NMI isn't ever generated, 
the two will always be the same. 

Some documentation says that when an NMI is accepted, IFF1 is first copied
into IFF2 before IFF1 is cleared. If this is true, the state of IFF2
is lost after a nested NMI, which is undesirable. Have tested this in the
following way: make sure the Z80 is in {\tt EI} mode, generate an NMI.
In the NMI routine, wait for another NMI before executing {\tt RETN}. In
the second NMI IFF2 was still set, so IFF1 is {\em not} copied to
IFF2 when accepting an NMI.

Another interesting fact is this. I was trying to figure out whether 
the undocumented ED {\tt RET} instructions were {\tt RETN} or {\tt RETI}.
I tested this by putting the machine in {\tt EI} mode, wait for an
NMI and end with one of the ED {\tt RET} instructions. Then execute
a {\tt HALT} instruction. If IFF1 was not restored, the machine would
hang but this did not happen with any of the instructions, including
the documented {\tt RETI}!

Since every INT routine must end with {\ttfamily EI} followed by 
{\ttfamily RETI} officially, It does not matter that RETI copies IFF2 into 
IFF1; both are set anyway.


\section{HALT instruction}

The HALT instruction halts the Z80; it does not increase the PC so
that the instruction is re-executed, until a maskable or non-maskable 
interrupt is accepted. Only then does the Z80 increase the PC again 
and continues with the next instruction. During the HALT state, the 
HALT line is set. The PC is increased before the interrupt routine is called.


\section{Where interrupts are accepted}

During execution of instructions, interrupts won't be accepted. Only 
{\em between} instructions. This is also true for prefixed instructions.

Directly after an EI or DI instruction, interrupts aren't accepted. They're
accepted again after the instruction after the EI (RET in the following 
example). So for example, look at this MSX2 routine that reads a scanline 
from the keyboard:

{\tt
 \begin{tabbing}
   {\qquad}\=LD{\qquad}\=C,A\+\\
        DI\\
        IN\>A,(0AAh)\\
        AND\>0F0h\\
        ADD\>A,C\\
        OUT\>(0AAh),A\\
        EI\\
        IN\>A,(0A9h)\\
        RET
 \end{tabbing}
}


You can assume that there never is an interrupt after the {\tt EI}, before the
{\tt IN A,(0A9h)} --- which would be a problem because the MSX interrupt routine
reads the keyboard too.

Using this feature of {\tt EI}, it is possible to check whether it is true that
interrupts are never accepted during instructions:

\begin{tabbing}
  {\qquad}\={\tt DI}\+\\
  {\em make sure INT is active}\\
  {\tt EI}\\
  {\em insert instruction to test}\-\\
{\tt INT:}\+\\
  {\em store PC where INT was accepted}\\
  {\tt RET}
\end{tabbing}

And yes, for all instructions, including the prefixed ones, interrupts are
never accepted during an instruction. Only after the tested instruction. 
Remember that block instructions simply re-execute themselves (by decreasing
the PC with 2) so an interrupt is accepted after each iteration.

Another predictable test is this: at the ``insert instruction to test'' insert
a large sequence of {\tt EI} instructions. Of course, during execution of the 
{\tt EI} instructions, no interrupts are accepted. 

But now for the interesting stuff. ED or CB make up instructions,
so interrupts are accepted after them. But DD and FD are 
prefixes, which only slightly affects the next opcode. If you test a large
sequence of DDs or FDs, the same happens as with the {\tt EI} instruction: no
interrupts are accepted during the execution of these sequences.

This makes sense, if you think of DD and FD as a prefix which set the
``use IX instead of HL'' or ``use IY instead of HL'' flag. If an interrupt was
accepted after DD or FD, this flag information would be lost, and:

{\tt
{\qquad}DD 21 00 00{\qquad}LD IX,0
}

could be interpreted as a simple {\tt LD HL,0} if the interrupt was after the 
last DD. Which never happens, so the implementation is correct.
Although I haven't tested this, as I imagine the same holds for NMI interrupts. 




\chapter{Timing and R register}

\section{R register and memory refresh}

During every first machine cycle (beginning of an instruction or part
of it --- prefixes have their own M1 two), the memory refresh cycle is
issued. The whole IR register is put on the address bus, and the RFSH pin is 
lowered.  It unclear whether the Z80 increases the R register before or after 
putting IR on the bus. 

The R register is increased at every first machine cycle (M1). Bit 7 of
the register is never changed by this; only the lower 7 bits are included
in the addition. So bit 7 stays the same, but it can be changed using the
{\ttfamily LD R,A} instruction.

Instructions without a prefix increase R by one. Instructions with an ED, CB,
DD, FD prefix, increase R by two, and so do the DDCBxxxx and FDCBxxxx 
instructions (weird enough). Just a stray DD or FD increases the R by
one. {\tt LD A,R} and {\tt LD R,A} access the R register after it is 
increased (by the instruction itself). 

Remember that the block instructions simply decrease the PC with two, so
the instructions are re-executed. So LDIR increased R by BC times 2 (note that
in the case of BC = 0, R is increased by 10000h times 2, effectively 0).

Accepting an maskable or non-maskable interrupt increases the R by one.

After a hardware reset, or after power on, the R register is reset to 0.

That should cover all there is to say about the R register. It is often
used in programs for a random value, which is good but of course not
truly random.




\chapter{Other Information}

\section{Errors in official documentation}

In some official Zilog documentation, the are some errors. Some don't
have all of these mistakes, so your documentation may not be flawed but 
these are just things to look out for.

\begin{itemize}

\item
  The Flag affection summary table shows that {\ttfamily LDI/LDIR/LDD/LDDR} 
  instructions leave the SF and ZF in an undefined state. This is not correct; 
  the SF and ZF flags are unaffected (like the same documentation says).

\item
  Similarly, the same table shows that {\ttfamily CPI/CPIR/CPD/CPDR} leave the 
  SF and HF flags in an undefined state. Not true, they are affected as defined 
  elsewhere in the documentation.

\item
  Also, the table says about {\ttfamily INI/OUTD/}etc 
  ``Z=0 if B $<>$ 0 otherwise Z=0''; of course the latter should be Z=1.

\item
  The {\tt INI/INIR/IND/INDR/OUTI/OUTD/OTIR/OTDR} instructions do affect\\
  the CF flag (some official documentation says they 
  leave it unaffected, important!) and the NF flag isn't always set but may 
  also be reset (see \ref{block_io} for exact operation).

\item
  When an NMI is accepted, the IFF1 isn't copied to IFF2. Only IFF1
  is reset.

\item
  In the 8-bit Load Group, the last two bits of the second byte of
  the {\ttfamily LD r,(IX + d)} opcode should be 10 and not 01.

\item
  In the 16-bit Arithmetic Group, bit 6 of the second byte of the
  {\ttfamily ADD IX, pp} opcode should be 0, not 1.

\item
  {\ttfamily IN x,(C)} resets the HF flag, it never sets it. Some 
  documentation states it is set according to the result of the operation; 
  this is impossible since no arithmetic is done in this instruction.
\end{itemize}

Note: In zilog's own z80cpu\_um.pdf document, there are a lot of errors,\\
some are very confusing, so I'll mention the ones I have found here:

\begin{itemize}

\item
  page 21, figure 2 says the Alternative Register Set contains 2 B' registers, 
  this should ofcourse be B' and C'.  

\item
  page 26, figure 16 shows very convincingly that the least significant bit of 
  the address to read for Interrupt Mode 2 is always 0. I have tested this and 
  it is not correct, it can also be 1, in my testcase the bus contained 0xFF and
  the address that was read did not end in 0xFE but was 0xFF.
  
\end{itemize}




\begin{thebibliography}{}

 \bibitem{mrison}  Mark Rison Z80 page for !CPC. \\
   {\tt \small http://www.acorn.co.uk/$\sim$mrison/en/cpc/tech.html}

 \bibitem{yaze} YAZE (Yet Another Z80 Emulator). This is a CPM emulator 
   by Frank Cringle. It emulates almost every undocumented flag, very good
   emulator. Also includes a very good instruction exerciser and is released
   under the GPL. \\
   {\tt \small ftp://ftp.ping.de/pub/misc/emulators/yaze-1.10.tar.gz} \\
   Note: the instruction exerciser zexdoc/zexall does not test I/O instructions 
   and not all normal instructions (for instance LD A,(IX+n) is tested, but not 
   with different values of n, just n=1, values above 128 (LD A,(IX-n) are not 
   tested) but it still gives a pretty good idea of how well a simulated Z80 
   works.
   
 \bibitem{thomas} Z80 Family Official Support Page by Thomas Scherrer. Very 
   good -- your one-stop Z80 page. \\
   {\tt \small http://www.geocities.com/SiliconValley/Peaks/3938/z80\_home.htm}

 \bibitem{speccy_faq} Spectrum FAQ technical information. \\
   {\tt \small http://www.worldofspectrum.org/faq/}

 \bibitem{gerton} Gerton Lunter's Spectrum emulator (Z80). In the package 
   there is a file TECHINFO.DOC, which contains a lot of interesting 
   information. Note that the current version can only be unpacked in
   Windows. \\
   {\tt \small ftp://ftp.void.jump.org/pub/sinclair/emulators/pc/dos/z80-400.zip}

 \bibitem{mostek} Mostek Z80 Programming Manual -- a very good reference
   to the Z80.

 \bibitem{datasheet} Z80 Product Specification, from MSX2 Hardware Information. \\
   {\tt \small http://www.hardwareinfo.msx2.com/pdf/Zilog/z80.pdf}

 \bibitem{zx_next} ZX Spectrum Next Extended Instruction Set. \\
   {\tt \small \url{https://wiki.specnext.dev/Extended_Z80_instruction_set/}}

\end{thebibliography}




\chapter{Instruction Tables}


\newcommand{\instrheader}{
	\hline
		
		& 
		Symbolic &
		\multicolumn{8}{c}{Flags} & 
		\multicolumn{3}{c}{Opcode} &
		& & 
		M & 
		T & \\

		Mnemonic & 
		Operation &
		SF & ZF & YF & HF & XF & PF & NF & CF &
		76 & 543 & 210 & 
		Hex & Bytes & 
		Cycles & States & 
		Comments \\

	\hline
}


% Chapters have large gap on top so we'd have to wrap tables. Instead we show table of contents for all sections on chapter title page as filler and start actual tables on next page.
\startcontents[intrstructionsections]
\printcontents[intrstructionsections]{}{1}{}


\section{8-Bit Load Group}

{\tt \scriptsize \setlength{\fboxsep}{0.25mm}
	\setlength{\tabcolsep}{1mm}
	\begin{tabular}{llcccccccccccccccl}
		     
		\instrheader
	
		& & & & & & & & & & & & & & & & &
		\multirow{9}{*}{
			\begin{tabular}{ll}
				r,r' & Reg \\
				\hline
				000 & B \\
				001 & C \\
				010 & D \\
				011 & E \\
				100 & H \\
				101 & L \\
				111 & A \\
			\end{tabular}
		} \\

		LD r,r'	& 
			r$\leftarrow$r' & 
			$\bullet$ & $\bullet$ & $\bullet$ & $\bullet$ & $\bullet$ & $\bullet$ & $\bullet$ & $\bullet$ &
			01 & r & r' &
			.. & 1 & 
			1 & 4 & \\[4pt]
			
		LD p,p' & 
			p$\leftarrow$p' & 
			$\bullet$ & $\bullet$ & $\bullet$ & $\bullet$ & $\bullet$ & $\bullet$ & $\bullet$ & $\bullet$ & 
			11 & 011 & 101 & 
			DD & 2 & 
			2 & 8 & \\
		\multicolumn{10}{c}{} & 01 & p & p' & .. & & & \\[4pt]

		LD q,q' & 
			q$\leftarrow$q' & 
			$\bullet$ & $\bullet$ & $\bullet$ & $\bullet$ & $\bullet$ & $\bullet$ & $\bullet$ & $\bullet$ & 
			11 & 111 & 101 & 
			FD & 2 & 
			2 & 8 & \\
		\multicolumn{10}{c}{} & 01 & q & q' & .. & & & \\[4pt]

		LD r,n	& 
			r$\leftarrow$n & 
			$\bullet$ & $\bullet$ & $\bullet$ & $\bullet$ & $\bullet$ & $\bullet$ & $\bullet$ & $\bullet$ & 
			00 & r & 110 & 
			.. & 2 & 
			2 & 7 & \\
		\multicolumn{10}{c}{} & \multicolumn{3}{c}{$\longleftarrow $ n $\longrightarrow$} & .. & & & \\[4pt]

		LD p,n	& 
			p$\leftarrow$n & 
			$\bullet$ & $\bullet$ & $\bullet$ & $\bullet$ & $\bullet$ & $\bullet$ & $\bullet$ & $\bullet$ & 
			11 & 011 & 101 & 
			DD & 3 & 
			3 & 11 & \\
		\multicolumn{10}{c}{} & 00 & p & 110 & .. & & & \\
		\multicolumn{10}{c}{} & \multicolumn{3}{c}{$\longleftarrow$ n $\longrightarrow$} & .. & & & \\[4pt]
		
		LD q,n	&
			q$\leftarrow$n & 
			$\bullet$ & $\bullet$ & $\bullet$ & $\bullet$ & $\bullet$ & $\bullet$ & $\bullet$ & $\bullet$ & 
			11 & 111 & 101 &
			FD & 3 & 
			3 & 11 & 
			\multirow{8}{*}{
				\begin{tabular}{ll}
					p,p' & Reg \\
					\hline
					000 & B \\
					001 & C \\
					010 & D \\
					011 & E \\
					100 & IXh \\
					101 & IXl \\
					111 & A \\
				\end{tabular}
		}\\
		\multicolumn{10}{c}{} & 00 & q & 110 & .. & & & \\
		\multicolumn{10}{c}{} & \multicolumn{3}{c}{$\longleftarrow$ n $\longrightarrow$} & .. & & & & \\[4pt]

		LD r,(HL) & r$\leftarrow$(HL) & 
			$\bullet$ & $\bullet$ & $\bullet$ & $\bullet$ & $\bullet$ & $\bullet$ & $\bullet$ & $\bullet$ & 01 & r & 110 &
			.. & 1 & 
			2 & 7 & \\[4pt]

		LD r,(IX+d) & r$\leftarrow$(IX+d) & 
			$\bullet$ & $\bullet$ & $\bullet$ & $\bullet$ & $\bullet$ & $\bullet$ & $\bullet$ & $\bullet$ & 
			11 & 011 & 101 & 
			DD & 3 & 
			5 & 19 & \\
		\multicolumn{10}{c}{} & 01 & r & 110 & .. & & & \\
		\multicolumn{10}{c}{} & \multicolumn{3}{c}{$\longleftarrow$ d $\longrightarrow$} & .. & & & \\[4pt]

		LD r,(IY+d) & r$\leftarrow$(IY+d) & 
			$\bullet$ & $\bullet$ & $\bullet$ & $\bullet$ & $\bullet$ & $\bullet$ & $\bullet$ & $\bullet$ & 11 & 111 & 101 & 
			FD & 3 & 
			5 & 19 & \\
		\multicolumn{10}{c}{} & 01 & r & 110 & .. & & & \\
		\multicolumn{10}{c}{} & \multicolumn{3}{c}{$\longleftarrow$ d $\longrightarrow$} & .. & & & \\[4pt]

		LD (HL),r & (HL)$\leftarrow$r & 
			$\bullet$ & $\bullet$ & $\bullet$ & $\bullet$ & $\bullet$ & $\bullet$ & $\bullet$ & $\bullet$ &
			01 & 110 & r & 
			.. & 1 & 
			2 & 7 & \\[4pt]

		LD (IX+d),r & (IX+d)$\leftarrow$r & 
			$\bullet$ & $\bullet$ & $\bullet$ & $\bullet$ & $\bullet$ & $\bullet$ & $\bullet$ & $\bullet$ &
			11 & 011 & 101 & 
			DD & 3 & 
			5 & 19 & 
			\multirow{8}{*}{
				\begin{tabular}{ll}
					q,q' & Reg \\
					\hline
					000 & B \\
					001 & C \\
					010 & D \\
					011 & E \\
					100 & IYh \\
					101 & IYl \\
					111 & A \\
				\end{tabular}
			}\\
		\multicolumn{10}{c}{} & 01 & 110 & r & .. & & & \\
		\multicolumn{10}{c}{} & \multicolumn{3}{c}{$\longleftarrow$ d $\longrightarrow$} & .. & & & \\[4pt]

		LD (IY+d),r & (IY+d)$\leftarrow$r & 
			$\bullet$ & $\bullet$ & $\bullet$ & $\bullet$ & $\bullet$ & $\bullet$ & $\bullet$ & $\bullet$ &
			11 & 111 & 101 & 
			FD & 3 & 
			5 & 19 & \\
		\multicolumn{10}{c}{} & 01 & 110 & r & .. & & & \\
		\multicolumn{10}{c}{} & \multicolumn{3}{c}{$\longleftarrow$ d $\longrightarrow$} & .. & & & \\[4pt]

		LD (HL),n & (HL)$\leftarrow$n & 
			$\bullet$ & $\bullet$ & $\bullet$ & $\bullet$ & $\bullet$ & $\bullet$ & $\bullet$ & $\bullet$ & 
			00 & 110 & 110 & 
			36 & 2 & 
			3 & 10 & \\
		\multicolumn{10}{c}{} & \multicolumn{3}{c}{$\longleftarrow$ n $\longrightarrow$} & .. & & & \\[4pt]

		LD (IX+d),n & (IX+d)$\leftarrow$n & 
			$\bullet$ & $\bullet$ & $\bullet$ & $\bullet$ & $\bullet$ & $\bullet$ & $\bullet$ & $\bullet$ & 
			11 & 011 & 101 & 
			DD & 4 & 
			5 & 19 & \\
		\multicolumn{10}{c}{} & 00 & 110 & 110 &  36 & & & & \\
		\multicolumn{10}{c}{} & \multicolumn{3}{c}{$\longleftarrow$ d $\longrightarrow$} & .. & & & \\
		\multicolumn{10}{c}{} & \multicolumn{3}{c}{$\longleftarrow$ n $\longrightarrow$} & .. & & & \\[4pt]

		LD (IY+d),n & (IY+d)$\leftarrow$n & 
			$\bullet$ & $\bullet$ & $\bullet$ & $\bullet$ & $\bullet$ & $\bullet$ & $\bullet$ & $\bullet$ & 
			11 & 111 & 101 & 
			FD & 4 & 
			5 & 19 & \\
		\multicolumn{10}{c}{} & 00 & 110 & 110 & 36 & .. & & & \\
		\multicolumn{10}{c}{} & \multicolumn{3}{c}{$\longleftarrow$ d $\longrightarrow$} & .. & & & \\
		\multicolumn{10}{c}{} & \multicolumn{3}{c}{$\longleftarrow$ n $\longrightarrow$} & .. & & & \\[4pt]

		LD A,(BC) & A$\leftarrow$(BC) & 
			$\bullet$ & $\bullet$ & $\bullet$ & $\bullet$ & $\bullet$ & $\bullet$ & $\bullet$ & $\bullet$ & 
			00 & 001 & 010 & 
			0A & 1 & 
			2 & 7 & \\[4pt]

		LD A,(DE) & A$\leftarrow$(DE) & 
			$\bullet$ & $\bullet$ & $\bullet$ & $\bullet$ & $\bullet$ & $\bullet$ & $\bullet$ & $\bullet$ & 
			00 & 011 & 01 & 
			1A & 1 & 
			2 & 7 & \\[4pt]

		LD A,(nm) & A$\leftarrow$(nm) & 
			$\bullet$ & $\bullet$ & $\bullet$ & $\bullet$ & $\bullet$ & $\bullet$ & $\bullet$ & $\bullet$ & 
			00 & 111 & 010 & 
			3A & 3 & 
			4 & 13 & \\
		\multicolumn{10}{c}{} & \multicolumn{3}{c}{$\longleftarrow$ m $\longrightarrow$} & .. & & & \\
		\multicolumn{10}{c}{} & \multicolumn{3}{c}{$\longleftarrow$ n $\longrightarrow$} & .. & & & \\[4pt]
		
		LD (BC),A & (BC)$\leftarrow$A & 
			$\bullet$ & $\bullet$ & $\bullet$ & $\bullet$ & $\bullet$ & $\bullet$ & $\bullet$ & $\bullet$ & 
			00 & 000 & 010 & 
			02 & 1 & 
			2 & 7 & \\[4pt]

		LD (DE),A & (DE)$\leftarrow$A & 
			$\bullet$ & $\bullet$ & $\bullet$ & $\bullet$ & $\bullet$ & $\bullet$ & $\bullet$ & $\bullet$ & 
			00 & 010 & 010 & 
			12 & 1 & 
			2 & 7 & \\[4pt]

		LD (nn),A & (nn)$\leftarrow$A & 
			$\bullet$ & $\bullet$ & $\bullet$ & $\bullet$ & $\bullet$ & $\bullet$ & $\bullet$ & $\bullet$ & 
			00 & 110 & 010 & 
			32 & 3 & 
			4 & 13 & \\
		\multicolumn{10}{c}{} & \multicolumn{3}{c}{$\longleftarrow$ n $\longrightarrow$} & .. & & & \\
		\multicolumn{10}{c}{} & \multicolumn{3}{c}{$\longleftarrow$ n $\longrightarrow$} & .. & & & \\[4pt]

		LD A,I & A$\leftarrow$I & 
			$\updownarrow$ & $\updownarrow$ & $\updownarrow$ & 0 & $\updownarrow$ & {\tiny FF2} & 0 & $\bullet$ & 
			11 & 101 & 101 & 
			ED & 2 & 
			2 & 9 & \\ 
		\multicolumn{10}{c}{} & 01 & 010 & 111 & 57 & & & & \\[4pt]

		LD A,R & A$\leftarrow$R & 
			$\updownarrow$ & $\updownarrow$ & $\updownarrow$ & 0 & $\updownarrow$ & {\tiny IFF2} & 0 & $\bullet$ & 
			11 & 101 & 101 & 
			ED & 2 & 
			2 & 9 & \\ 
		\multicolumn{10}{c}{} & 01 & 011 & 111 & 5F & & & & \\[4pt]

		LD I,A & I$\leftarrow$A & 
			$\bullet$ & $\bullet$ & $\bullet$ & $\bullet$ & $\bullet$ & $\bullet$ & $\bullet$ & $\bullet$ & 
			11 & 101 & 101 & 
			ED & 2 & 
			2 & 9 & \\ 
		\multicolumn{10}{c}{} & 01 & 000 & 111 & 47 & & & & \\[4pt]

		LD R,A & R$\leftarrow$A & 
			$\bullet$ & $\bullet$ & $\bullet$ & $\bullet$ & $\bullet$ & $\bullet$ & $\bullet$ & $\bullet$ & 
			11 & 101 & 101 & 
			ED & 2 & 
			2 & 9 & \\ 
		\multicolumn{10}{c}{} & 01 & 001 & 111 & 4F & & & & \\[4pt]

	\hline

	\end{tabular}
}


\section{16-Bit Load Group}

{\tt \scriptsize \setlength{\fboxsep}{0.25mm}
	\setlength{\tabcolsep}{1mm}
	\begin{tabular}{llcccccccccccccccl}
		     
		\instrheader
	
		& & & & & & & & & & & & & & & & &
		\multirow{7}{*}{
			\begin{tabular}{ll}
				dd & Reg \\
				\hline
				00 & BC \\
				01 & DE \\
				10 & HL \\
				11 & SP \\
			\end{tabular}
		} \\

		LD dd,nm & dd$\leftarrow$nm & 
			$\bullet$ & $\bullet$ & $\bullet$ & $\bullet$ & $\bullet$ & $\bullet$ & $\bullet$ & $\bullet$ & 
			00 & dd0 & 001 & 
			.. & 3 & 
			3 & 10 & \\
		\multicolumn{10}{c}{} & \multicolumn{3}{c}{$\longleftarrow$ m $\longrightarrow$} & .. & & & \\
		\multicolumn{10}{c}{} & \multicolumn{3}{c}{$\longleftarrow$ n $\longrightarrow$} & .. & & & \\[4pt]

		LD IX,nm & IX$\leftarrow$nm & 
			$\bullet$ & $\bullet$ & $\bullet$ & $\bullet$ & $\bullet$ & $\bullet$ & $\bullet$ & $\bullet$ &
			11 & 011 & 101 & 
			DD & 4 & 
			4 & 14 & \\
		\multicolumn{10}{c}{} & 00 & 100 & 001 & 21 & & & & \\
		\multicolumn{10}{c}{} & \multicolumn{3}{c}{$\longleftarrow$ m $\longrightarrow$} & .. & & & \\
		\multicolumn{10}{c}{} & \multicolumn{3}{c}{$\longleftarrow$ n $\longrightarrow$} & .. & & & \\[4pt]

		LD IY,nm & IX$\leftarrow$nm & 
			$\bullet$ & $\bullet$ & $\bullet$ & $\bullet$ & $\bullet$ & $\bullet$ & $\bullet$ & $\bullet$ & 11 & 111 & 101 
			& FD & 4 & 
			4 & 14 & \\
		\multicolumn{10}{c}{} & 00 & 100 & 001 & 21 & & & \\
		\multicolumn{10}{c}{} & \multicolumn{3}{c}{$\longleftarrow$ m $\longrightarrow$} & .. & & & \\
		\multicolumn{10}{c}{} & \multicolumn{3}{c}{$\longleftarrow$ n $\longrightarrow$} & .. & & & \\[4pt]

		LD HL,(nm) & H$\leftarrow$(nm+1) &
			$\bullet$ & $\bullet$ & $\bullet$ & $\bullet$ & $\bullet$ & $\bullet$ & $\bullet$ & $\bullet$ & 
			00 & 101 & 010 & 
			2A & 3 & 
			5 & 16 & \\ 
		& L$\leftarrow$(nm) & \multicolumn{8}{c}{} & \multicolumn{3}{c}{$\longleftarrow$ m $\longrightarrow$} & .. & & & \\
		\multicolumn{10}{c}{} & \multicolumn{3}{c}{$\longleftarrow$ n $\longrightarrow$} & .. & & & \\[4pt]

		LD dd,(nm) & ddh$\leftarrow$(nm+1) &
			$\bullet$ & $\bullet$ & $\bullet$ & $\bullet$ & $\bullet$ & $\bullet$ & $\bullet$ & $\bullet$ & 
			11 & 101 & 101 & 
			ED & 4 & 
			6 & 20 & \\ 
		& ddl$\leftarrow$(nm) & \multicolumn{8}{c}{} & 01 & dd1 & 011 & .. & & & \\
		\multicolumn{10}{c}{} & \multicolumn{3}{c}{$\longleftarrow$ m $\longrightarrow$} & .. & & & \\
		\multicolumn{10}{c}{} & \multicolumn{3}{c}{$\longleftarrow$ n $\longrightarrow$} & .. & & & \\[4pt]

		LD IX,(nm) & IXh$\leftarrow$(nm+1) &
			$\bullet$ & $\bullet$ & $\bullet$ & $\bullet$ & $\bullet$ & $\bullet$ & $\bullet$ & $\bullet$ & 
			11 & 011 & 101 & 
			DD & 4 & 
			6 & 20 & \\ 
		& IXl$\leftarrow$(nm) & \multicolumn{8}{c}{} & 00 & 101 & 010 & 2A & & & \\
		\multicolumn{10}{c}{} & \multicolumn{3}{c}{$\longleftarrow$ m $\longrightarrow$} & .. & & & \\
		\multicolumn{10}{c}{} & \multicolumn{3}{c}{$\longleftarrow$ n $\longrightarrow$} & .. & & & \\[4pt]

		LD IY,(nm) & IYh$\leftarrow$(nm+1) &
			$\bullet$ & $\bullet$ & $\bullet$ & $\bullet$ & $\bullet$ & $\bullet$ & $\bullet$ & $\bullet$ & 
			11 & 111 & 101 & 
			FD & 4 & 
			6 & 20 & \\ 
		& IYl$\leftarrow$(nn) & \multicolumn{8}{c}{} & 00 & 101 & 010 & 2A & & & \\
		\multicolumn{10}{c}{} & \multicolumn{3}{c}{$\longleftarrow$ n $\longrightarrow$} & .. & & & \\
		\multicolumn{10}{c}{} & \multicolumn{3}{c}{$\longleftarrow$ n $\longrightarrow$} & .. & & & \\[4pt]

		LD (nm),HL & (nn+1)$\leftarrow$H &
			$\bullet$ & $\bullet$ & $\bullet$ & $\bullet$ & $\bullet$ & $\bullet$ & $\bullet$ & $\bullet$ & 
			00 & 100 & 010 & 
			22 & 3 & 
			5 & 16 & \\ 
		& (nm)$\leftarrow$L & \multicolumn{8}{c}{} & \multicolumn{3}{c}{$\longleftarrow$ m $\longrightarrow$} & .. & & & \\
		\multicolumn{10}{c}{} & \multicolumn{3}{c}{$\longleftarrow$ n $\longrightarrow$} & .. & & & \\[4pt]

		LD (nm),dd & (nm+1)$\leftarrow$ddh &
			$\bullet$ & $\bullet$ & $\bullet$ & $\bullet$ & $\bullet$ & $\bullet$ & $\bullet$ & $\bullet$ & 
			11 & 101 & 101 & 
			ED & 4 & 
			6 & 20 & \\ 
		& (nm)$\leftarrow$ddl & \multicolumn{8}{c}{} & 01 & dd0 & 011 & .. & & & \\
		\multicolumn{10}{c}{} & \multicolumn{3}{c}{$\longleftarrow$ m $\longrightarrow$} & .. & & & \\
		\multicolumn{10}{c}{} & \multicolumn{3}{c}{$\longleftarrow$ n $\longrightarrow$} & .. & & & \\[4pt]

		LD (nm),IX & (nm+1)$\leftarrow$IXh &
			$\bullet$ & $\bullet$ & $\bullet$ & $\bullet$ & $\bullet$ & $\bullet$ & $\bullet$ & $\bullet$ & 
			11 & 011 & 101 & 
			DD & 4 & 
			6 & 20 & \\ 
		& (nm)$\leftarrow$IXl & \multicolumn{8}{c}{} & 00 & 100 & 010 & 22 & & & \\
		\multicolumn{10}{c}{} & \multicolumn{3}{c}{$\longleftarrow$ m $\longrightarrow$} & .. & & & \\
		\multicolumn{10}{c}{} & \multicolumn{3}{c}{$\longleftarrow$ n $\longrightarrow$} & .. & & & \\[4pt]

		LD (nm),IY & (nm+1)$\leftarrow$IYh &
			$\bullet$ & $\bullet$ & $\bullet$ & $\bullet$ & $\bullet$ & $\bullet$ & $\bullet$ & $\bullet$ & 
			11 & 111 & 101 & 
			FD & 4 & 
			6 & 20 & \\ 
		& (nm)$\leftarrow$IYl & \multicolumn{8}{c}{} & 00 & 100 & 010 & 22 & & & \\
		\multicolumn{10}{c}{} & \multicolumn{3}{c}{$\longleftarrow$ m $\longrightarrow$} & .. & & & \\
		\multicolumn{10}{c}{} & \multicolumn{3}{c}{$\longleftarrow$ n $\longrightarrow$} & .. & & & \\[4pt]

		LD SP,HL & SP$\leftarrow$HL & 
			$\bullet$ & $\bullet$ & $\bullet$ & $\bullet$ & $\bullet$ & $\bullet$ & $\bullet$ & $\bullet$ & 
			11 & 111 & 001 & 
			F9 & 1 & 
			1 & 6 & \\[4pt]

		LD SP,IX & SP$\leftarrow$IX & 
			$\bullet$ & $\bullet$ & $\bullet$ & $\bullet$ & $\bullet$ & $\bullet$ & $\bullet$ & $\bullet$ & 
			11 & 011 & 101 & 
			DD & 2 & 
			2 & 10 & \\ 
		\multicolumn{10}{c}{} & 11 & 111 & 001 & F9 & & & \\[4pt]

		LD SP,IY & SP$\leftarrow$IY & 
			$\bullet$ & $\bullet$ & $\bullet$ & $\bullet$ & $\bullet$ & $\bullet$ & $\bullet$ & $\bullet$ & 
			11 & 111 & 101 & 
			FD & 2 & 
			2 & 10 & \\ 
		\multicolumn{10}{c}{} & 11 & 111 & 001 & F9 & & & \\[4pt]

		PUSH qq & (SP-2)$\leftarrow$qql &
			$\bullet$ & $\bullet$ & $\bullet$ & $\bullet$ & $\bullet$ & $\bullet$ & $\bullet$ & $\bullet$ & 
			11 & qq0 & 101 & 
			.. & 1 & 
			3 & 11 &
			\multirow{5}{*}{
				\begin{tabular}{ll}
					qq & Reg \\
					\hline
					00 & BC \\
					01 & DE \\
					10 & HL \\
					11 & AF \\
				\end{tabular}
			}\\
		& (SP-1)$\leftarrow$qqh & \\
		& SP$\leftarrow$SP-2 & \\[4pt]

		PUSH IX & (SP-2)$\leftarrow$IXl &
			$\bullet$ & $\bullet$ & $\bullet$ & $\bullet$ & $\bullet$ & $\bullet$ & $\bullet$ & $\bullet$ & 
			11 & 011 & 101 & 
			DD & 2 & 
			4 & 15 & \\ 
		& (SP-1)$\leftarrow$IXh & \multicolumn{8}{c}{} & 11 & 100 & 101 & E5 & & & \\
		& SP$\leftarrow$SP-2 & \\[4pt]

		PUSH IY & (SP-2)$\leftarrow$IYl &
			$\bullet$ & $\bullet$ & $\bullet$ & $\bullet$ & $\bullet$ & $\bullet$ & $\bullet$ & $\bullet$ &
			11 & 111 & 101 & 
			FD & 2 & 
			4 & 15 & \\ 
		& (SP-1)$\leftarrow$IYh & \multicolumn{8}{c}{} & 11 & 100 & 101 & E5 & & & \\
		& SP$\leftarrow$SP-2 & \\[4pt]

		POP qq & qqh$\leftarrow$(SP+1) &
			$\bullet$ & $\bullet$ & $\bullet$ & $\bullet$ & $\bullet$ & $\bullet$ & $\bullet$ & $\bullet$ & 
			11 & qq0 & 001 & 
			.. & 1 & 
			3 & 10 & \\
		& qql$\leftarrow$(SP) & \\
		& SP$\leftarrow$SP+2 & \\[4pt]

		POP IX & IXh$\leftarrow$(SP+1) &
			$\bullet$ & $\bullet$ & $\bullet$ & $\bullet$ & $\bullet$ & $\bullet$ & $\bullet$ & $\bullet$ & 
			11 & 011 & 101 & 
			DD & 2 & 
			4 & 14 & \\ 
		& IXl$\leftarrow$(SP) & \multicolumn{8}{c}{} & 11 & 100 & 001 & E1 & & & \\
		& SP$\leftarrow$SP+2 & \\[4pt]

		POP IY & IYh$\leftarrow$(SP+1) &
			$\bullet$ & $\bullet$ & $\bullet$ & $\bullet$ & $\bullet$ & $\bullet$ & $\bullet$ & $\bullet$ &
			11 & 111 & 101 & 
			FD & 2 & 
			4 & 14 & \\ 
		& IYl$\leftarrow$(SP) & \multicolumn{8}{c}{} & 11 & 100 & 001 & E1 & & & \\
		& SP$\leftarrow$SP+2 & \\[4pt]

		\hline

	\end{tabular}
}


\section{Exchange, Block Transfer, Search Group}

{\tt \scriptsize \setlength{\fboxsep}{0.25mm}
	\setlength{\tabcolsep}{1mm}
	\begin{tabular}{llcccccccccccccccl}
		     
		\instrheader

		& & & & & & & & & & & & & & & & & \\

		EX DE,HL & DE$\leftrightarrow$HL & 
			$\bullet$ & $\bullet$ & $\bullet$ & $\bullet$ & $\bullet$ & $\bullet$ & $\bullet$ & $\bullet$ & 
			11 & 101 & 011 & 
			EB & 1 & 
			1 & 4 & \\[4pt]

		EX AF,AF' & AF$\leftrightarrow$AF' & 
			$\bullet$ & $\bullet$ & $\bullet$ & $\bullet$ & $\bullet$ & $\bullet$ & $\bullet$ & $\bullet$ & 00 & 001 & 000
			& 08 & 1 & 1 & 4 & \\[4pt]

		EXX & BC$\leftrightarrow$BC' & 
			$\bullet$ & $\bullet$ & $\bullet$ & $\bullet$ & $\bullet$ & $\bullet$ & $\bullet$ & $\bullet$ & 
			11 & 011 & 001 & 
			D9 & 1 & 1 & 4 & \\ 
		& DE$\leftrightarrow$DE' & \\ 
		& HL$\leftrightarrow$HL' & \\[4pt]

		EX (SP),HL & H$\leftrightarrow$(SP+1) & 
			$\bullet$ & $\bullet$ & $\bullet$ & $\bullet$ & $\bullet$ & $\bullet$ & $\bullet$ & $\bullet$ & 
			11 & 100 & 011 & 
			E3 & 1 & 
			5 & 19 & \\ 
		& L$\leftrightarrow$(SP) & \\[4pt]

		EX (SP),IX & IXh$\leftrightarrow$(SP+1) & 
			$\bullet$ & $\bullet$ & $\bullet$ & $\bullet$ & $\bullet$ & $\bullet$ & $\bullet$ & $\bullet$ & 
			11 & 011 & 101 & 
			DD & 2 & 
			6 & 23 & \\ 
		& IXl$\leftrightarrow$(SP) & \multicolumn{8}{c}{} & 11 & 100 & 011 & E3 & & \\[4pt]

		EX (SP),IY & IYh$\leftrightarrow$(SP+1) & 
			$\bullet$ & $\bullet$ & $\bullet$ & $\bullet$ & $\bullet$ & $\bullet$ & $\bullet$ & $\bullet$ & 
			11 & 111 & 101 & 
			FD & 2 & 
			6 & 23 & \\
		& IYl$\leftrightarrow$(SP) & \multicolumn{8}{c}{} & 11 & 100 & 011 & E3 & & \\[4pt]

		LDI & (DE)$\leftarrow$(HL) & 
			$\bullet$ & $\bullet$ & 
				$\updownarrow$\footnotemark[4] & 0 & 
				$\updownarrow$\footnotemark[4] & 
				$\updownarrow$\footnotemark[1] & 0 & $\bullet$ & 
			11 & 101 & 101 & 
			ED & 2 & 
			4 & 16 \\ 
		& DE$\leftarrow$DE+1 & \multicolumn{8}{c}{} & 10 & 100 & 000 & A0 & & & \\ 
		& HL$\leftarrow$HL+1 & \\
		& BC$\leftarrow$BC-1 & \\ [4pt]

		LDIR & (DE)$\leftarrow$(HL) & 
			$\bullet$ & $\bullet$ &
				$\updownarrow$\footnotemark[4] & 0 & 
				$\updownarrow$\footnotemark[4] & 
				0\footnotemark[2] & 0 & $\bullet$ & 
			11 & 101 & 101 & 
			ED & 2 & 
			5 & 21 & {if BC$\not=$0} \\
		& DE$\leftarrow$DE+1 & \multicolumn{8}{c}{} & 10 & 110 & 000 & B0 & & 4 & 16 & {if BC=0} \\
		& HL$\leftarrow$HL+1 & \\
		& BC$\leftarrow$BC-1 & \\ 
		& Repeat until & \\ 
		& BC=0 & \\[4pt]

		LDD & (DE)$\leftarrow$(HL) & 
			$\bullet$ & $\bullet$ & 
				$\updownarrow$\footnotemark[4] & 0 & 
				$\updownarrow$\footnotemark[4] & 
				$\updownarrow$\footnotemark[1] & 0 & $\bullet$ & 
			11 & 101 & 101 & 
			ED & 2 & 
			4 & 16 & \\ 
		& DE$\leftarrow$DE-1 & \multicolumn{8}{c}{} & 10 & 101 & 000 & A8 & & \\ 
		& HL$\leftarrow$HL-1 & \\
		& BC$\leftarrow$BC-1 & \\[4pt]

		LDDR & (DE)$\leftarrow$(HL) & 
			$\bullet$ & $\bullet$ & 
				$\updownarrow$\footnotemark[4] & 0 & 
				$\updownarrow$\footnotemark[4] & 
				0\footnotemark[2] & 0 & $\bullet$ & 
			11 & 101 & 101 & 
			ED & 2 & 
			5 & 21 & {if BC$\not=$0} \\ 
		& DE$\leftarrow$DE-1 & \multicolumn{8}{c}{} & 10 & 111 & 000 & B8 & & 4 & 16 & {if BC=0} \\
		& HL$\leftarrow$HL-1 & \\
		& BC$\leftarrow$BC-1 & \\ 
		& Repeat until & \\ 
		& BC=0 & \\[4pt]

		CPI & A-(HL) &
			$\updownarrow$\footnotemark[4] & $\updownarrow$\footnotemark[3] & $\updownarrow$\footnotemark[4] & $\updownarrow$\footnotemark[4] 
			& $\updownarrow$\footnotemark[4] & $\updownarrow$\footnotemark[1] & 1 & $\bullet$ &
			11 & 101 & 101 & 
			ED & 2 & 
			4 & 16 & \\ 
		& HL$\leftarrow$HL+1 & \multicolumn{8}{c}{} & 10 & 100 & 001 & A1 & \\ 
		& BC$\leftarrow$BC-1 & \\[4pt]

		CPIR & A-(HL) &
			$\updownarrow$\footnotemark[4] & 
			$\updownarrow$\footnotemark[3] & 
			$\updownarrow$\footnotemark[4] & 
			$\updownarrow$\footnotemark[4] & 
			$\updownarrow$\footnotemark[4] & 
			$\updownarrow$\footnotemark[1] & 1 & $\bullet$ &
			11 & 101 & 101 & 
			ED & 2 & 
			5 & 21 & {if BC$\not=$0} \\
		& HL$\leftarrow$HL+1 & \multicolumn{8}{c}{} & 10 & 110 & 001 & B1 & & & & {and A$\not=$(HL)} \\
		& BC$\leftarrow$BC-1 & \multicolumn{12}{c}{} & & 4 & 16 & {if BC=0} \\
		& Repeat until & \multicolumn{15}{c}{} & {or A=(HL)} \\ 
		& A=(HL) or & \\ 
		& BC=0 & \\[4pt]

		CPD & A-(HL) &
			$\updownarrow$\footnotemark[4] & 
			$\updownarrow$\footnotemark[3] & 
			$\updownarrow$\footnotemark[4] & 
			$\updownarrow$\footnotemark[4] & 
			$\updownarrow$\footnotemark[4] & 
			$\updownarrow$\footnotemark[1] & 1 & $\bullet$ &
			11 & 101 & 101 & 
			ED & 2 & 
			4 & 16 & \\ 
			& HL$\leftarrow$HL-1 & \multicolumn{8}{c}{} & 10 & 101 & 001 & A9 & & & \\ 
			& BC$\leftarrow$BC-1 & \\[4pt]

		CPDR & A-(HL) &
			$\updownarrow$\footnotemark[4] & 
			$\updownarrow$\footnotemark[3] & 
			$\updownarrow$\footnotemark[4] & 
			$\updownarrow$\footnotemark[4] & 
			$\updownarrow$\footnotemark[4] & 
			$\updownarrow$\footnotemark[1] & 1 & $\bullet$ &
			11 & 101 & 101 & 
			ED & 2 & 
			5 & 21 & {if BC$\not=$0} \\
		& HL$\leftarrow$HL-1 & \multicolumn{8}{c}{} & 10 & 111 & 001 & B9 & & & & {and A$\not=$(HL)} \\
		& BC$\leftarrow$BC-1 & \multicolumn{12}{c}{} & & 4 & 16 & {if BC=0} \\
		& Repeat until & \multicolumn{15}{c}{} & {or A=(HL)} \\ 
		& A=(HL) or & \\ 
		& BC=0 & \\[4pt]

		\hline

		Note: & \multicolumn{17}{l}{\parbox{12cm}{\footnotemark[1]PF is 0 the result of BC-1=0, otherwise PF is set.}} \\
			& \multicolumn{17}{l}{\parbox{12cm}{\footnotemark[2]PF is 0 only at completion of the instruction.}} \\
			& \multicolumn{17}{l}{\parbox{12cm}{\footnotemark[3]ZF is set if A=(HL), otherwise ZF is reset.}} \\
			& \multicolumn{17}{l}{\parbox{12cm}{\footnotemark[4]See section \ref{block_mem} for a description.}} \\
	  
	\hline

	\end{tabular}
}


\section{8-Bit Arithmetic and Logical Group}

\setlength{\textwidth}{18cm}

{\tt \scriptsize \setlength{\fboxsep}{0.25mm}
	\setlength{\tabcolsep}{1mm}
	\begin{tabular}{llcccccccccccccccl}
		     
		\instrheader

		& & & & & & & & & & & & & & & & &
		\multirow{9}{*}{
			\begin{tabular}{ll}
				r & Reg \\
				\hline
				000 & B \\
				001 & C \\
				010 & D \\
				011 & E \\
				100 & H \\
				101 & L \\
				111 & A \\
			\end{tabular}
		}\\

		ADD A,r & A$\leftarrow$A+r & 
			$\updownarrow$ & $\updownarrow$ & $\updownarrow$ & $\updownarrow$ & $\updownarrow$ & VF & 0 & $\updownarrow$ & 
			10 & \fbox{000} & r & 
			.. & 1 & 
			1 & 4 & \\[4pt]

		ADD A,p & A$\leftarrow$A+p & 
			$\updownarrow$ & $\updownarrow$ & $\updownarrow$ & $\updownarrow$ & $\updownarrow$ & VF & 0 & $\updownarrow$ & 
			11 & 011 & 101 &
			DD & 2 & 
			2 & 8 \\
		\multicolumn{10}{c}{} & 10 & \fbox{000} & p & .. & \\[4pt]

		ADD A,q & A$\leftarrow$A+q & 
			$\updownarrow$ & $\updownarrow$ & $\updownarrow$ & $\updownarrow$ & $\updownarrow$ & VF & 0 & $\updownarrow$ & 
			11 & 111 & 101 &
			FD & 2 & 
			2 & 8 & \\
		\multicolumn{10}{c}{} & 10 & \fbox{000} & q & .. & \\[4pt]

		ADD A,n & A$\leftarrow$A+n & 
			$\updownarrow$ & $\updownarrow$ & $\updownarrow$ & $\updownarrow$ & $\updownarrow$ & VF & 0 & $\updownarrow$ & 
			11 & \fbox{000} & 110 &
			C6 & 2 & 
			2 & 7 & \\
		\multicolumn{10}{c}{} & \multicolumn{3}{c}{$\longleftarrow$ n $\longrightarrow$} & .. & \\[4pt]

		ADD A,(HL) & A$\leftarrow$A+(HL) & 
			$\updownarrow$ & $\updownarrow$ & $\updownarrow$ & $\updownarrow$ & $\updownarrow$ & VF & 0 & $\updownarrow$ &
			10 & \fbox{000} & 110 &
			86 & 1 & 
			2 & 7 & 
			\multirow{8}{*}{
				\begin{tabular}{ll}
					p & Reg \\
					\hline
					000 & B \\
					001 & C \\
					010 & D \\
					011 & E \\
					100 & IXh \\
					101 & IXl \\
					111 & A \\
				\end{tabular}
			}\\[4pt]

		ADD A,(IX+d) & A$\leftarrow$A+(IX+d) & 
			$\updownarrow$ & $\updownarrow$ & $\updownarrow$ & $\updownarrow$ & $\updownarrow$ & VF & 0 & $\updownarrow$ & 
			11 & 011 & 101 & 
			DD & 3 & 
			5 & 19 & \\
		\multicolumn{10}{c}{} & 10 & \fbox{000} & 110 & 68 & \\
		\multicolumn{10}{c}{} & \multicolumn{3}{c}{$\longleftarrow$ d $\longrightarrow$} & .. & \\[4pt]

		ADD A,(IY+d) & A$\leftarrow$A+(IY+d) & 
			$\updownarrow$ & $\updownarrow$ & $\updownarrow$ & $\updownarrow$ & $\updownarrow$ & VF & 0 & $\updownarrow$ & 
			11 & 111 & 101 & 
			FD & 3 & 
			5 & 19 & \\
		\multicolumn{10}{c}{} & 10 & \fbox{000} & 110 & 68 \\
		\multicolumn{10}{c}{} & \multicolumn{3}{c}{$\longleftarrow$ d $\longrightarrow$} & .. & \\[4pt]

	& & & & & & & & & & & $\uparrow$ & & & & & & \\[4pt]

		ADC A,s & A$\leftarrow$A+s+CF &
			$\updownarrow$ & $\updownarrow$ & $\updownarrow$ & $\updownarrow$ & $\updownarrow$ & VF & 0 & $\updownarrow$ &
			.. & \fbox{001} & ... & 
			& & & &
			\multirow{9}{*}{
				\begin{tabular}{ll}
					q & Reg \\
					\hline
					000 & B \\
					001 & C \\
					010 & D \\
					011 & E \\
					100 & IYh \\
					101 & IYl \\
					111 & A \\
				\end{tabular}
			}\\[4pt]

		SUB s & A$\leftarrow$A-s &
			$\updownarrow$ & $\updownarrow$ & $\updownarrow$ & $\updownarrow$ & $\updownarrow$ & VF & 1 & $\updownarrow$ &
			.. & \fbox{010} & ... & \\[4pt]

		SBC A,s & A$\leftarrow$A-s-CF &
			$\updownarrow$ & $\updownarrow$ & $\updownarrow$ & $\updownarrow$ & $\updownarrow$ & VF & 1 & $\updownarrow$ &
			.. & \fbox{011} & ... & \\[4pt]

		AND s & A$\leftarrow$A$\wedge$s &
			$\updownarrow$ & $\updownarrow$ & $\updownarrow$ & 1 & $\updownarrow$ & PF & 0 & 0 &
			.. & \fbox{100} & ... & \\[4pt]

		OR s & A$\leftarrow$A$\vee$s &
			$\updownarrow$ & $\updownarrow$ & $\updownarrow$ & 0 & $\updownarrow$ & PF & 0 & 0 &
			.. & \fbox{110} & ... & \\[4pt]

		XOR s & A$\leftarrow$A$\bullet$s &
			$\updownarrow$ & $\updownarrow$ & $\updownarrow$ & 0 & $\updownarrow$ & PF & 0 & 0 &
			.. & \fbox{101} & ... & \\[4pt]

		CP s & A-s &
			$\updownarrow$ & 
			$\updownarrow$ & 
			$\updownarrow$\footnotemark[1] & 
			$\updownarrow$ & 
			$\updownarrow$\footnotemark[1] & VF & 1 & 
			$\updownarrow$ &
			.. & \fbox{111} & ... & \\[4pt]

	& & & & & & & & & & & & & & & & & \\
	& & & & & & & & & & & & & & & & & \\

		INC r & r$\leftarrow$r+1 & 
			$\updownarrow$ & $\updownarrow$ & $\updownarrow$ & $\updownarrow$ & $\updownarrow$ & VF & 0 & $\bullet$ & 00 & r & \fbox{100} & 
			.. & 1 & 
			1 & 4 & \\[4pt]

		INC p & p$\leftarrow$p+1 & 
			$\updownarrow$ & $\updownarrow$ & $\updownarrow$ & $\updownarrow$ & $\updownarrow$ & VF & 0 & $\bullet$ & 11 & 011 & 101 & 
			DD & 2 & 
			2 & 8 & \\
		\multicolumn{10}{c}{} & 00 & p & \fbox{100} & .. & \\[4pt]

		INC q & q$\leftarrow$q+1 & 
			$\updownarrow$ & $\updownarrow$ & $\updownarrow$ & $\updownarrow$ & $\updownarrow$ & VF & 0 & $\bullet$ & 11 & 111 & 101 & 
			FD & 2 & 
			2 & 8 & \\ 
		\multicolumn{10}{c}{} & 00 & q & \fbox{100} & .. & \\[4pt]

		INC (HL) & (HL)$\leftarrow$(HL)+1 & 
			$\updownarrow$ & $\updownarrow$ & $\updownarrow$ & $\updownarrow$ & $\updownarrow$ & VF & 0 & $\bullet$ & 00 & 110 & \fbox{100} & 
			34 & 1 & 
			3 & 11 & \\[4pt]

		INC (IX+d) & (IX+d)$\leftarrow$(IX+d)+1 & 
			$\updownarrow$ & $\updownarrow$ & $\updownarrow$ & $\updownarrow$ & $\updownarrow$ & VF & 0 & $\bullet$ & 11 & 011 & 101 & 
			DD & 3 & 
			6 & 23 & \\ 
		\multicolumn{10}{c}{} & 00 & 110 & \fbox{100} & 34 & \\
		\multicolumn{10}{c}{} & \multicolumn{3}{c}{$\longleftarrow$ d $\longrightarrow$} & \\[4pt]

		INC (IY+d) & (IY+d)$\leftarrow$(IY+d)+1 & 
			$\updownarrow$ & $\updownarrow$ & $\updownarrow$ & $\updownarrow$ & $\updownarrow$ & VF & 0 & $\bullet$ & 11 & 111 & 101 & 
			FD & 3 & 
			6 & 23 & \\ 
		\multicolumn{10}{c}{} & 00 & 110 & \fbox{100} & 34 & \\
		\multicolumn{10}{c}{} & \multicolumn{3}{c}{$\longleftarrow$ d $\longrightarrow$} &  \\[4pt]

	& & & & & & & & & & & & $\uparrow$ & & & & & \\[4pt]

		DEC m & m$\leftarrow$m-1 & 
			$\updownarrow$ & $\updownarrow$ & $\updownarrow$ & $\updownarrow$ & $\updownarrow$ & VF & 1 & $\bullet$ &
			.. & ... & \fbox{101}
			& & 
			& & & \\[4pt]

	\hline

		Note: & \multicolumn{17}{l}{
			\parbox{12cm}{\footnotemark[1]YF and XF flags are copied from the operand s, not the result A-s}
		} \\[2pt]
		
		& \multicolumn{17}{l}{
			\parbox{12cm}{s is any of r, p, q, n, (HL), (IX+d), (IY+d) as shown for ADD. The indicated bits replace the \fbox{000} in the ADD set above}
		} \\[5pt]
		
		& \multicolumn{17}{l}{
			\parbox{12cm}{m is any of r, p, q, (HL), (IX+d), (IY+d) as shown for INC. Replace \fbox{100} with \fbox{101} in opcode}
		} \\
	
	\hline

	\end{tabular}
}


\section{16-Bit Arithmetic Group}

{\tt \scriptsize \setlength{\fboxsep}{0.25mm}
	\setlength{\tabcolsep}{1mm}
	\begin{tabular}{llcccccccccccccccl}
		     
		\instrheader

		& & & & & & & & & & & & & & & & &
		\multirow{7}{*}{
			\begin{tabular}{ll}
				ss & Reg \\
				\hline
				00 & BC \\
				01 & DE \\
				10 & HL \\
				11 & SP \\
			\end{tabular}
		}\\

		ADD HL,ss & HL$\leftarrow$HL+ss &
			$\bullet$ & $\bullet$ & $\updownarrow$\footnotemark[2] & $\updownarrow$\footnotemark[2] & $\updownarrow$\footnotemark[2] & $\bullet$ & 0 & $\updownarrow$\footnotemark[1] & 
			00 & ss1 & 001 & 
			.. & 1 & 
			3 & 11 & \\[4pt]

		ADC HL,ss & HL$\leftarrow$HL+ss+CF &
			$\updownarrow$\footnotemark[1] & $\updownarrow$\footnotemark[1] & $\updownarrow$\footnotemark[2] & $\updownarrow$\footnotemark[2] & $\updownarrow$\footnotemark[2] & VF\footnotemark[1] & 0 & $\updownarrow$\footnotemark[1] & 
			11 & 101 & 101 & 
			ED & 2 & 
			4 & 15 & \\
		\multicolumn{10}{c}{} & 01 & ss1 & 010 & .. & & \\[4pt]

		SBC HL,ss & HL$\leftarrow$HL-ss-CF &
			$\updownarrow$\footnotemark[1] & $\updownarrow$\footnotemark[1] & $\updownarrow$\footnotemark[2] & $\updownarrow$\footnotemark[2] & $\updownarrow$\footnotemark[2] & VF\footnotemark[1] & 1 & $\updownarrow$\footnotemark[1] & 
			11 & 101 & 101 & 
			ED & 2 & 
			4 & 15 & \\
		\multicolumn{10}{c}{} & 01 & ss0 & 010 & .. & \\[4pt]

		ADD IX,pp & IX$\leftarrow$IX+pp &
			$\bullet$ & $\bullet$ & $\updownarrow$\footnotemark[2] & $\updownarrow$\footnotemark[2] & $\updownarrow$\footnotemark[2] & $\bullet$ & 0 & $\updownarrow$\footnotemark[1] & 
			11 & 011 & 110 & 
			DD & 2 & 
			4 & 15 & 
			\multirow{5}{*}{
				\begin{tabular}{ll}
					pp & Reg \\
					\hline
					00 & BC \\
					01 & DE \\
					10 & IX \\
					11 & SP \\
				\end{tabular}
			}\\
		\multicolumn{10}{c}{} & 00 & pp1 & 001 & .. & \\[4pt]

		ADD IY,qq & IY$\leftarrow$IY+qq &
			$\bullet$ & $\bullet$ & $\updownarrow$\footnotemark[2] & $\updownarrow$\footnotemark[2] & $\updownarrow$\footnotemark[2] & $\bullet$ & 0 & $\updownarrow$\footnotemark[1] & 
			11 & 111 & 110 & 
			FD & 2 & 
			4 & 15 & \\
		\multicolumn{10}{c}{} & 00 & qq1 & 001 & .. & \\[4pt]

		INC ss & ss$\leftarrow$ss+1 &
			$\bullet$ & $\bullet$ & $\bullet$ & $\bullet$ & $\bullet$ & $\bullet$ & $\bullet$ & $\bullet$ &
			00 & ss0 & 011 & 
			.. & 1 & 
			1 & 6 & \\[4pt]

		INC IX & IX$\leftarrow$IX+1 &
			$\bullet$ & $\bullet$ & $\bullet$ & $\bullet$ & $\bullet$ & $\bullet$ & $\bullet$ & $\bullet$ &
			11 & 011 & 101 & 
			DD & 2 & 
			2 & 10 &
			\multirow{5}{*}{
				\begin{tabular}{ll}
					qq & Reg \\
					\hline
					00 & BC \\
					01 & DE \\
					10 & IY \\
					11 & SP \\
				\end{tabular}
			}\\
		\multicolumn{10}{c}{} & 00 & 100 & 011 & 23 & \\[4pt]

		INC IY & IY$\leftarrow$IY+1 &
			$\bullet$ & $\bullet$ & $\bullet$ & $\bullet$ & $\bullet$ & $\bullet$ & $\bullet$ & $\bullet$ &
			11 & 111 & 101 & 
			FD & 2 & 
			2 & 10 & \\
		\multicolumn{10}{c}{} & 00 & 100 & 011 & 23 & \\[4pt]

		DEC ss & ss$\leftarrow$ss-1 &
			$\bullet$ & $\bullet$ & $\bullet$ & $\bullet$ & $\bullet$ & $\bullet$ & $\bullet$ & $\bullet$ & 
			00 & ss1 & 011 & 
			.. & 1 & 
			1 & 6 & \\[4pt]

		DEC IX & IX$\leftarrow$IX-1 &
			$\bullet$ & $\bullet$ & $\bullet$ & $\bullet$ & $\bullet$ & $\bullet$ & $\bullet$ & $\bullet$ &
			11 & 011 & 101 & 
			DD & 2 & 
			2 & 10 & \\
		\multicolumn{10}{c}{} & 00 & 101 & 011 & 2B & \\[4pt]

		DEC IY & IY$\leftarrow$IY-1 &
			$\bullet$ & $\bullet$ & $\bullet$ & $\bullet$ & $\bullet$ & $\bullet$ & $\bullet$ & $\bullet$ &
			11 & 111 & 101 & 
			FD & 2 & 
			2 & 10 & \\
		\multicolumn{10}{c}{} & 00 & 101 & 011 & 2B & \\[4pt]
		
		\hline

		Note: & \multicolumn{17}{l}{\parbox{12cm}{\footnotemark[1]Flag is affected by the 16 bit result.}} \\
			& \multicolumn{17}{l}{\parbox{12cm}{\footnotemark[2]Flag is affected by the high-byte addition.}} \\
			
		\hline

	\end{tabular}
}


\section{General-Purpose Arithmetic and CPU Control Group}

{\tt \scriptsize \setlength{\fboxsep}{0.25mm}
	\setlength{\tabcolsep}{1mm}
	\begin{tabular}{llcccccccccccccccl}
		     
		\instrheader

	& & & & & & & & & & & & & & & & & \\

		DAA & & 
			$\updownarrow$ & $\updownarrow$ & $\updownarrow$ & $\updownarrow$ & $\updownarrow$ & PF & $\bullet$ & $\updownarrow$ &
			00 & 100 & 111 & 
			27 & 1 & 
			1 & 4 & \\[4pt]

		CPL & A$\leftarrow\mathtt{\overline{A}}$ & 
			$\bullet$ & $\bullet$ & $\updownarrow$ & 1 & $\updownarrow$ & $\bullet$ & 1 & $\bullet$ & 
			00 & 101 & 111 & 
			2F & 1 & 
			1 & 4 & \\[4pt]

		NEG & A$\leftarrow$0-A & 
			$\updownarrow$ & $\updownarrow$ & $\updownarrow$ & $\updownarrow$ & $\updownarrow$ & VF & 1 & $\updownarrow$ &
			11 & 101 & 101 & 
			ED & 2 & 
			2 & 8 & \\
		\multicolumn{10}{c}{} & 01 & 000 & 100 & 44 & \\[4pt]

		CCF & CF$\leftarrow\mathtt{\overline{CF}}$ & 
			$\bullet$ & $\bullet$ & 
				$\updownarrow$\footnotemark[1] & 
				$\updownarrow$\footnotemark[2] & 
				$\updownarrow$\footnotemark[1] & 
			$\bullet$ & 0 & $\updownarrow$ & 
			00 & 111 & 111 & 
			3F & 1 & 
			1 & 4 & \\[4pt]

		SCF & CF$\leftarrow$1 & 
			$\bullet$ & $\bullet$ & $\updownarrow$\footnotemark[1] & 0 & $\updownarrow$\footnotemark[1] & $\bullet$ & 0 & 1 & 
			00 & 110 & 111 & 
			37 & 1 & 
			1 & 4 & \\[4pt]

		NOP & & 
			$\bullet$ & $\bullet$ & $\bullet$ & $\bullet$ & $\bullet$ & $\bullet$ & $\bullet$ & $\bullet$ &
			00 & 000 & 000 & 
			00 & 1 & 
			1 & 4 & \\[4pt]

		HALT & & 
			$\bullet$ & $\bullet$ & $\bullet$ & $\bullet$ & $\bullet$ & $\bullet$ & $\bullet$ & $\bullet$ &
			01 & 110 & 110 & 
			76 & 1 & 
			1 & 4 & \\[4pt]

		DI\footnotemark[3] & IFF1,2$\leftarrow$0 & 
			$\bullet$ & $\bullet$ & $\bullet$ & $\bullet$ & $\bullet$ & $\bullet$ & $\bullet$ & $\bullet$ &
			11 & 110 & 011 & 
			F3 & 1 & 
			1 & 4 & \\[4pt]

		EI\footnotemark[3] & IFF1,2$\leftarrow$1 & 
			$\bullet$ & $\bullet$ & $\bullet$ & $\bullet$ & $\bullet$ & $\bullet$ & $\bullet$ & $\bullet$ &
			11 & 111 & 011 & 
			FB & 1 & 
			1 & 4 & \\[4pt]

		IM 0\footnotemark[4] & & 
			$\bullet$ & $\bullet$ & $\bullet$ & $\bullet$ & $\bullet$ & $\bullet$ & $\bullet$ & $\bullet$ &
			11 & 101 & 101 & 
			ED & 2 & 
			2 & 8 & \\
		\multicolumn{10}{c}{} & 01 & 000 & 110 & 46 & \\[4pt]

		IM 1\footnotemark[4] & & 
			$\bullet$ & $\bullet$ & $\bullet$ & $\bullet$ & $\bullet$ & $\bullet$ & $\bullet$ & $\bullet$ &
			11 & 101 & 101 & 
			ED & 2 & 
			2 & 8 & \\
		\multicolumn{10}{c}{} & 01 & 010 & 110 & 56 & \\[4pt]

		IM 2\footnotemark[4] & & 
			$\bullet$ & $\bullet$ & $\bullet$ & $\bullet$ & $\bullet$ & $\bullet$ & $\bullet$ & $\bullet$ &
			11 & 101 & 101 & 
			ED & 2 & 
			2 & 8 & \\
		\multicolumn{10}{c}{} & 01 & 011 & 110 & 5E & \\[4pt]
		
		\hline

		Note: & \multicolumn{17}{l}{\parbox{12cm}{\footnotemark[1]YF and XF are copied from register A.}} \\
			& \multicolumn{17}{l}{\parbox{12cm}{\footnotemark[2]HF is like CF before the instruction.}} \\
			& \multicolumn{17}{l}{\parbox{12cm}{\footnotemark[3]No interrupts are accepted directly after EI or DI.}} \\
			& \multicolumn{17}{l}{\parbox{12cm}{\footnotemark[4]This instruction has other undocumented opcodes.}} \\ 
	  
	\hline

	\end{tabular}
}


\section{Rotate and Shift Group}

{\tt \scriptsize \setlength{\fboxsep}{0.25mm}
\setlength{\tabcolsep}{1mm}
\begin{tabular}{llc@{ }c@{ }c@{ }c@{ }c@{ }c@{ }c@{ }cc@{ }c@{ }cccccll}\hline
 & \multicolumn{1}{c}{Symbolic} & \multicolumn{8}{c}{Flags} & \multicolumn{3}{c}{Opcode} & & & M & T & \\
Mnemonic & \multicolumn{1}{c}{Operation} & SF & ZF & YF & HF & XF & PF & NF & CF & 76 & 543 & 210 
	& Hex & Bytes & Cycles & States & \multicolumn{2}{l}{Comments} \\ \hline

RLCA & 
\setlength{\unitlength}{0.9mm}
\begin{picture}(17,5)
\put(0,0){\framebox(4,3){CF}}
\put(8,0){\framebox(7,3){7$\leftarrow$0}}
\put(8,1.5){\vector(-1,0){4}}
\put(6,1.5){\line(0,1){2.5}}
\put(6,4){\line(1,0){11}}
\put(17,4){\line(0,-1){2.5}}
\put(17,1.5){\vector(-1,0){2}}
\end{picture} &
	$\bullet$ & $\bullet$ & $\updownarrow$ & 0 & $\updownarrow$ & $\bullet$ & 0 & $\updownarrow$ & 00 & 000 & 111 
	& 07 & 1 & 1 & 4 & & \\ 

RLA & 
\setlength{\unitlength}{0.9mm}
\begin{picture}(17,4)
\put(2,0){\framebox(4,3){CF}}
\put(9,0){\framebox(7,3){7$\leftarrow$0}}

\put(9,1.5){\vector(-1,0){3}}
\put(2,1.5){\line(-1,0){2}}

\put(0,1.5){\line(0,1){2.5}}
\put(0,4){\line(1,0){17.5}}
\put(17.5,4){\line(0,-1){2.5}}
\put(17.5,1.5){\vector(-1,0){1.5}}
\end{picture} &
	$\bullet$ & $\bullet$ & $\updownarrow$ & 0 & $\updownarrow$ & $\bullet$ & 0 & $\updownarrow$ & 00 & 010 & 111 
	& 17 & 1 & 1 & 4 & & \\ 

RRCA & 
\setlength{\unitlength}{0.9mm}
\begin{picture}(17,4)
\put(13,0){\framebox(4,3){CF}}
\put(2,0){\framebox(7,3){7$\leftarrow$0}}

\put(0,1.5){\vector(1,0){2}}
\put(9,1.5){\vector(1,0){4}}

\put(0,1.5){\line(0,1){2.5}}
\put(0,4){\line(1,0){11}}
\put(11,4){\line(0,-1){2.5}}
\end{picture} &
	$\bullet$ & $\bullet$ & $\updownarrow$ & 0 & $\updownarrow$ & $\bullet$ & 0 & $\updownarrow$ & 00 & 001 & 111 
	& 0F & 1 & 1 & 4 & & \\ 

RRA & 
\setlength{\unitlength}{0.9mm}
\begin{picture}(17,4)
\put(12,0){\framebox(4,3){CF}}
\put(2,0){\framebox(7,3){7$\leftarrow$0}}

\put(0,1.5){\vector(1,0){2}}
\put(9,1.5){\vector(1,0){3}}

\put(0,1.5){\line(0,1){2.5}}
\put(0,4){\line(1,0){17.5}}
\put(17.5,4){\line(0,-1){2.5}}
\put(16,1.5){\line(1,0){1.5}}
\end{picture} &
	$\bullet$ & $\bullet$ & $\updownarrow$ & 0 & $\updownarrow$ & $\bullet$ & 0 & $\updownarrow$ & 00 & 011 & 111 
	& 1F & 1 & 1 & 4 & & \\ 


RLC r & 
\setlength{\unitlength}{0.9mm}
\begin{picture}(17,5)
\put(0,0){\framebox(4,3){CF}}
\put(8,0){\framebox(7,3){7$\leftarrow$0}}
\put(8,1.5){\vector(-1,0){4}}
\put(6,1.5){\line(0,1){2.5}}
\put(6,4){\line(1,0){11}}
\put(17,4){\line(0,-1){2.5}}
\put(17,1.5){\vector(-1,0){2}}
\end{picture} &
	$\updownarrow$ & $\updownarrow$ & $\updownarrow$ & 0 & $\updownarrow$ & PF & 0 & $\updownarrow$ & 11 & 001 & 011 
	& CB & 2 & 2 & 8 & r & Reg \\ \cline{18-19}
\multicolumn{10}{c}{} & 00 & \fbox{000} & r & \multicolumn{4}{c}{} & 000 & B \\

RLC (HL) & 
\setlength{\unitlength}{0.9mm}
\begin{picture}(17,5)
\put(0,0){\framebox(4,3){CF}}
\put(8,0){\framebox(7,3){7$\leftarrow$0}}
\put(8,1.5){\vector(-1,0){4}}
\put(6,1.5){\line(0,1){2.5}}
\put(6,4){\line(1,0){11}}
\put(17,4){\line(0,-1){2.5}}
\put(17,1.5){\vector(-1,0){2}}
\end{picture} &
	$\updownarrow$ & $\updownarrow$ & $\updownarrow$ & 0 & $\updownarrow$ & PF & 0 & $\updownarrow$ & 11 & 001 & 011 
	& CB & 2 & 4 & 15 & 001 & C \\ 
\multicolumn{10}{c}{} & 00 & \fbox{000} & 110 & \multicolumn{4}{c}{} & 010 & D \\

RLC (IX+d) & 
\setlength{\unitlength}{0.9mm}
\begin{picture}(17,5)
\put(0,0){\framebox(4,3){CF}}
\put(8,0){\framebox(7,3){7$\leftarrow$0}}
\put(8,1.5){\vector(-1,0){4}}
\put(6,1.5){\line(0,1){2.5}}
\put(6,4){\line(1,0){11}}
\put(17,4){\line(0,-1){2.5}}
\put(17,1.5){\vector(-1,0){2}}
\end{picture} &
	$\updownarrow$ & $\updownarrow$ & $\updownarrow$ & 0 & $\updownarrow$ & PF & 0 & $\updownarrow$ & 11 & 011 & 101 
	& DD & 4 & 6 & 23 & 011 & E \\ 
\multicolumn{10}{c}{} & 11 & 001 & 011 & CB & \multicolumn{3}{c}{} & 100 & H \\
\multicolumn{10}{c}{} & \multicolumn{3}{c}{$\leftarrow$ d $\rightarrow$} & \multicolumn{4}{c}{} & 101 & L \\
\multicolumn{10}{c}{} & 00 & \fbox{000} & 110 & & \multicolumn{3}{c}{} & 111 & A \\

RLC (IY+d) & 
\setlength{\unitlength}{0.9mm}
\begin{picture}(17,5)
\put(0,0){\framebox(4,3){CF}}
\put(8,0){\framebox(7,3){7$\leftarrow$0}}
\put(8,1.5){\vector(-1,0){4}}
\put(6,1.5){\line(0,1){2.5}}
\put(6,4){\line(1,0){11}}
\put(17,4){\line(0,-1){2.5}}
\put(17,1.5){\vector(-1,0){2}}
\end{picture} &
	$\updownarrow$ & $\updownarrow$ & $\updownarrow$ & 0 & $\updownarrow$ & PF & 0 & $\updownarrow$ & 11 & 111 & 101 
	& FD & 4 & 6 & 23 & & \\ 
\multicolumn{10}{c}{} & 11 & 001 & 011 & CB & \multicolumn{3}{c}{} & & \\
\multicolumn{10}{c}{} & \multicolumn{3}{c}{$\leftarrow$ d $\rightarrow$} & \multicolumn{4}{c}{} & & \\
	& & \multicolumn{8}{c}{} & 00 & \fbox{000} & 110  & & \multicolumn{3}{c}{} & & \\

RLC (IX+d),r & r$\leftarrow$(IX+d) &
	$\updownarrow$ & $\updownarrow$ & $\updownarrow$ & 0 & $\updownarrow$ & PF & 0 & $\updownarrow$ & 11 & 011 & 101 
	& DD & 4 & 6 & 23 & & \\ 
	& RLC r & \multicolumn{8}{c}{} & 11 & 001 & 011 & CB & \multicolumn{3}{c}{} & & \\
	& (IX+d)$\leftarrow$r & \multicolumn{8}{c}{} & \multicolumn{3}{c}{$\leftarrow$ d $\rightarrow$} & \multicolumn{4}{c}{} & & \\
	& & \multicolumn{8}{c}{} & 00 & \fbox{000} & r & & \multicolumn{3}{c}{} & & \\

RLC (IY+d),r & r$\leftarrow$(IY+d) &
	$\updownarrow$ & $\updownarrow$ & $\updownarrow$ & 0 & $\updownarrow$ & PF & 0 & $\updownarrow$ & 11 & 111 & 101 
	& FD & 4 & 6 & 23 & & \\ 
	& RLC r & \multicolumn{8}{c}{} & 11 & 001 & 011 & CB & \multicolumn{3}{c}{} & & \\
	& (IY+d)$\leftarrow$r & \multicolumn{8}{c}{} & \multicolumn{3}{c}{$\leftarrow$ d $\rightarrow$} & \multicolumn{4}{c}{} & & \\
	& & \multicolumn{8}{c}{} & 00 & \fbox{000} & r & & \multicolumn{3}{c}{} & & \\

RL m & 
\setlength{\unitlength}{0.9mm}
\begin{picture}(17,4)
\put(2,0){\framebox(4,3){CF}}
\put(9,0){\framebox(7,3){7$\leftarrow$0}}

\put(9,1.5){\vector(-1,0){3}}
\put(2,1.5){\line(-1,0){2}}

\put(0,1.5){\line(0,1){2.5}}
\put(0,4){\line(1,0){17.5}}
\put(17.5,4){\line(0,-1){2.5}}
\put(17.5,1.5){\vector(-1,0){1.5}}
\end{picture} &
	$\updownarrow$ & $\updownarrow$ & $\updownarrow$ & 0 & $\updownarrow$ & PF & 0 & $\updownarrow$ & &\fbox{010} & \\

RRC m & 
\setlength{\unitlength}{0.9mm}
\begin{picture}(17,4)
\put(13,0){\framebox(4,3){CF}}
\put(2,0){\framebox(7,3){7$\leftarrow$0}}

\put(0,1.5){\vector(1,0){2}}
\put(9,1.5){\vector(1,0){4}}

\put(0,1.5){\line(0,1){2.5}}
\put(0,4){\line(1,0){11}}
\put(11,4){\line(0,-1){2.5}}
\end{picture} &
	$\updownarrow$ & $\updownarrow$ & $\updownarrow$ & 0 & $\updownarrow$ & PF & 0 & $\updownarrow$ & &\fbox{001} & \\

RR m & 
\setlength{\unitlength}{0.9mm}
\begin{picture}(17,4)
\put(12,0){\framebox(4,3){CF}}
\put(2,0){\framebox(7,3){7$\leftarrow$0}}

\put(0,1.5){\vector(1,0){2}}
\put(9,1.5){\vector(1,0){3}}

\put(0,1.5){\line(0,1){2.5}}
\put(0,4){\line(1,0){17.5}}
\put(17.5,4){\line(0,-1){2.5}}
\put(16,1.5){\line(1,0){1.5}}
\end{picture} &
	$\updownarrow$ & $\updownarrow$ & $\updownarrow$ & 0 & $\updownarrow$ & PF & 0 & $\updownarrow$ & &\fbox{011} & \\

SLA m & 
\setlength{\unitlength}{0.9mm}
\begin{picture}(17,4)
\put(0,0){\framebox(4,3){CF}}
\put(7,0){\framebox(7,3){7$\leftarrow$0}}

\put(7,1.5){\vector(-1,0){3}}
\put(17,1.5){\vector(-1,0){3}}

\put(16,0){\makebox(4,3){0}}
\end{picture} &
	$\updownarrow$ & $\updownarrow$ & $\updownarrow$ & 0 & $\updownarrow$ & PF & 0 & $\updownarrow$ & &\fbox{100} & \\

SLL m & 
\setlength{\unitlength}{0.9mm}
\begin{picture}(17,4)
\put(0,0){\framebox(4,3){CF}}
\put(7,0){\framebox(7,3){7$\leftarrow$0}}

\put(7,1.5){\vector(-1,0){3}}
\put(17,1.5){\vector(-1,0){3}}

\put(16,0){\makebox(4,3){1}}
\end{picture} &
	$\updownarrow$ & $\updownarrow$ & $\updownarrow$ & 0 & $\updownarrow$ & PF & 0 & $\updownarrow$ & &\fbox{110} & \\

SRA m & 
\setlength{\unitlength}{0.9mm}
\begin{picture}(17,4)
\put(13,0){\framebox(4,3){CF}}
\put(3,0){\framebox(7,3){7$\rightarrow$0}}

\put(10,1.5){\vector(1,0){3}}
\put(1,1.5){\vector(1,0){2}}

\put(1,-1.5){\line(0,1){3}}
\put(1,-1.5){\line(1,0){2.5}}
\put(3.5,-1.5){\line(0,1){1.5}}
\end{picture} &
	$\updownarrow$ & $\updownarrow$ & $\updownarrow$ & 0 & $\updownarrow$ & PF & 0 & $\updownarrow$ & &\fbox{101} & \\

SRL m & 
\setlength{\unitlength}{0.9mm}
\begin{picture}(17,4)
\put(-0.5,0){\makebox(1,3){0}}
\put(13,0){\framebox(4,3){CF}}
\put(3,0){\framebox(7,3){7$\rightarrow$0}}

\put(10,1.5){\vector(1,0){3}}
\put(1,1.5){\vector(1,0){2}}
\end{picture} &
	$\updownarrow$ & $\updownarrow$ & $\updownarrow$ & 0 & $\updownarrow$ & PF & 0 & $\updownarrow$ & &\fbox{111} & \\

RLD &
\setlength{\unitlength}{0.9mm}

\put(-7,0){\makebox(4,3){A}}
\put(20,0){\makebox(4,3){(HL)}}

\begin{picture}(25,5.5)

\put(-2,0){\framebox(4.5,3){7-4}}
\put(2.65,0){\framebox(4.5,3){3-0}}

\put(9,0){\framebox(4.5,3){7-4}}
\put(13.65,0){\framebox(4.5,3){3-0}}

\put(5,4.45){\line(0,-1){1.45}}
\put(5,4.45){\line(1,0){11.5}}
\put(16.5,4.45){\vector(0,-1){1.45}}

\put(12,-1.45){\vector(0,1){1.45}}
\put(12,-1.45){\line(1,0){3.75}}
\put(15.75,-1.45){\line(0,1){1.45}}

\put(5,-1.45){\vector(0,1){1.45}}
\put(10.5,-1.45){\line(-1,0){5.5}}
\put(10.5,-1.45){\line(0,1){1.45}}

\end{picture} 
        &
        $\updownarrow$ & $\updownarrow$ & $\updownarrow$ & 0 & $\updownarrow$
& PF & 0 & $\bullet$ & 11 & 101 & 101
        & ED & 2 & 5 & 18 & & \\
\multicolumn{10}{c}{} & 01 & 101 & 111 & 6F & \multicolumn{3}{c}{} & & \\ 
	
RRD &
\setlength{\unitlength}{0.9mm}

\put(-7,0){\makebox(4,3){A}}
\put(20,0){\makebox(4,3){(HL)}}

\begin{picture}(25,5.5)

\put(-2,0){\framebox(4.5,3){7-4}}
\put(2.65,0){\framebox(4.5,3){3-0}}

\put(9,0){\framebox(4.5,3){7-4}}
\put(13.65,0){\framebox(4.5,3){3-0}}

\put(16.5,-1.45){\line(0,1){1.45}}
\put(5,-1.45){\line(1,0){11.5}}
\put(5,-1.45){\vector(0,1){1.45}}

\put(15.75,4.45){\vector(0,-1){1.45}}
\put(12,4.45){\line(1,0){3.75}}
\put(12,4.45){\line(0,-1){1.45}}

\put(10.5,4.45){\vector(0,-1){1.45}}
\put(10.5,4.45){\line(-1,0){5.5}}
\put(5,4.45){\line(0,-1){1.45}}

\end{picture} 
	&
	$\updownarrow$ & $\updownarrow$ & $\updownarrow$ & 0 & $\updownarrow$ & PF & 0 & $\bullet$ & 11 & 101 & 101
	& ED & 2 & 5 & 18 & & \\
\multicolumn{10}{c}{} & 01 & 100 & 111 & 67 & \multicolumn{3}{c}{} & & \\ \hline

Note: & \multicolumn{18}{l}{\parbox{12cm}{m is one of r,(HL),(IX+d),(IY+d). To form new opcode replace \fbox{000} of RLCs 
	with shown code.}} \\ \hline

\end{tabular}
}

\section{Bit Set, Reset and Test Group}

{\tt \scriptsize \setlength{\fboxsep}{0.25mm}
\setlength{\tabcolsep}{1mm}
\begin{tabular}{llc@{ }c@{ }c@{ }c@{ }c@{ }c@{ }c@{ }cc@{ }c@{ }cccccll}\hline
 & \multicolumn{1}{c}{Symbolic} & \multicolumn{8}{c}{Flags} & \multicolumn{3}{c}{Opcode} & & & M & T & \\
Mnemonic & \multicolumn{1}{c}{Operation} & SF & ZF & YF & HF & XF & PF & NF & CF & 76 & 543 & 210 
	& Hex & Bytes & Cycles & States & \multicolumn{2}{l}{Comments} \\ \hline

BIT b,r & ZF$\leftarrow\mathtt{\overline{r_b}}$ &
	$\updownarrow$\footnotemark[1] & $\updownarrow$ & $\updownarrow$\footnotemark[1] & 1 & $\updownarrow$\footnotemark[1] & 
	$\updownarrow$\footnotemark[1] & 0 & $\bullet$ & 11 & 001 & 011 
	& CB & 2 & 2 & 8 & r & Reg \\ \cline{18-19}
\multicolumn{10}{c}{} & 01 & b & r & & \multicolumn{3}{c}{} & 000 & B \\

BIT b,(HL) & \mbox{ZF$\leftarrow\mathtt{\overline{(HL)_b}}$} &
	$\updownarrow$\footnotemark[1] & $\updownarrow$ & $\updownarrow$\footnotemark[1] & 1 & $\updownarrow$\footnotemark[1] & 
	$\updownarrow$\footnotemark[1] & 0 & $\bullet$ & 11 & 001 & 011 
	& CB & 2 & 3 & 12 & 001 & C \\ 
\multicolumn{10}{c}{} & 01 & b & 110 & & \multicolumn{3}{c}{} & 010 & D \\

BIT b,(IX+d)\footnotemark[2] & ZF$\leftarrow\mathtt{\overline{(IX+d)_b}}$ &
	$\updownarrow$\footnotemark[1] & $\updownarrow$ & $\updownarrow$\footnotemark[1] & 1 & $\updownarrow$\footnotemark[1] & 
	$\updownarrow$\footnotemark[1] & 0 & $\bullet$ & 11 & 011 & 101 
	& DD & 4 & 5 & 20 & 011 & E \\ 
\multicolumn{10}{c}{} & 11 & 001 & 011 & CB & \multicolumn{3}{c}{} & 100 & H \\
\multicolumn{10}{c}{} & \multicolumn{3}{c}{$\leftarrow$ d $\rightarrow$} & \multicolumn{4}{c}{} & 101 & L \\
\multicolumn{10}{c}{} & 01 & b  & 110 & & \multicolumn{3}{c}{} & 111 & A \\

BIT b,(IY+d)\footnotemark[2] & ZF$\leftarrow\mathtt{\overline{(IY+d)_b}}$ &
	$\updownarrow$\footnotemark[1] & $\updownarrow$ & $\updownarrow$\footnotemark[1] & 1 & $\updownarrow$\footnotemark[1] & 
	$\updownarrow$\footnotemark[1] & 0 & $\bullet$ & 11 & 111 & 101 
	& FD & 4 & 5 & 20 & & \\ 
\multicolumn{10}{c}{} & 11 & 001 & 011 & CB & \multicolumn{3}{c}{} & & \\
\multicolumn{10}{c}{} & \multicolumn{3}{c}{$\leftarrow$ d $\rightarrow$} & \multicolumn{4}{c}{} & &  \\
\multicolumn{10}{c}{} & 01 & b  & 110 & & \multicolumn{3}{c}{} & & \\

SET b,r & $\mathtt{r_b}\leftarrow$1 &
	$\bullet$ & $\bullet$ & $\bullet$ & $\bullet$ & $\bullet$ & $\bullet$ & $\bullet$ & $\bullet$ & 11 & 001 & 011
	& CB & 2 & 2 & 8 & b & Bit \\ \cline{18-19}
\multicolumn{10}{c}{} & \fbox{11} & b & r & & \multicolumn{3}{c}{} & 000 & 0 \\

SET b,(HL) & $\mathtt{(HL)_b}\leftarrow$1 &
	$\bullet$ & $\bullet$ & $\bullet$ & $\bullet$ & $\bullet$ & $\bullet$ & $\bullet$ & $\bullet$ & 11 & 001 & 011
	& CB & 2 & 4 & 15 & 001 & 1\\ 
\multicolumn{10}{c}{} & \fbox{11} & b & 110 & & \multicolumn{3}{c}{} & 010 & 2 \\

SET b,(IX+d) & $\mathtt{(IX+d)_b}\leftarrow$1 &
	$\bullet$ & $\bullet$ & $\bullet$ & $\bullet$ & $\bullet$ & $\bullet$ & $\bullet$ & $\bullet$ & 11 & 011 & 101
	& DD & 4 & 6 & 23 & 011 & 3 \\ 
\multicolumn{10}{c}{} & 11 & 001 & 011 & CB & \multicolumn{3}{c}{} & 100 & 4 \\
\multicolumn{10}{c}{} & \multicolumn{3}{c}{$\leftarrow$ d $\rightarrow$} & \multicolumn{4}{c}{} & 101 &  5 \\
\multicolumn{10}{c}{} & \fbox{11} & b & 110 & & \multicolumn{3}{c}{} & 110 & 6 \\

SET b,(IY+d) & $\mathtt{(IY+d)_b}\leftarrow$1 &
	$\bullet$ & $\bullet$ & $\bullet$ & $\bullet$ & $\bullet$ & $\bullet$ & $\bullet$ & $\bullet$ & 11 & 111 & 101
	& FD & 4 & 6 & 23 & 111 & 7 \\ 
\multicolumn{10}{c}{} & 11 & 001 & 011 & CB & \multicolumn{3}{c}{} & & \\
\multicolumn{10}{c}{} & \multicolumn{3}{c}{$\leftarrow$ d $\rightarrow$} & \multicolumn{4}{c}{} & & \\
\multicolumn{10}{c}{} & \fbox{11} & b & 110 & & \multicolumn{3}{c}{} & & \\

SET b,(IX+d),r & r$\leftarrow$(IX+d) &
	$\bullet$ & $\bullet$ & $\bullet$ & $\bullet$ & $\bullet$ & $\bullet$ & $\bullet$ & $\bullet$ & 11 & 011 & 101
	& DD & 4 & 6 & 23 & & \\ 
 & $\mathtt{r_b}\leftarrow$1 & \multicolumn{8}{c}{} & 11 & 001 & 011 & CB & \multicolumn{3}{c}{} & & \\
 & (IX+d)$\leftarrow$r & \multicolumn{8}{c}{} & \multicolumn{3}{c}{$\leftarrow$ d $\rightarrow$} & \multicolumn{4}{c}{} & &  \\
\multicolumn{10}{c}{} & \fbox{11} & b & r & & \multicolumn{3}{c}{} & & \\

SET b,(IY+d),r & r$\leftarrow$(IY+d) &
	$\bullet$ & $\bullet$ & $\bullet$ & $\bullet$ & $\bullet$ & $\bullet$ & $\bullet$ & $\bullet$ & 11 & 111 & 101
	& FD & 4 & 6 & 23 & & \\ 
 & $\mathtt{r_b}\leftarrow$1 & \multicolumn{8}{c}{} & 11 & 001 & 011 & CB & \multicolumn{3}{c}{} & & \\
 & (IY+d)$\leftarrow$r & \multicolumn{8}{c}{} & \multicolumn{3}{c}{$\leftarrow$ d $\rightarrow$} & \multicolumn{4}{c}{} & &  \\
\multicolumn{10}{c}{} & \fbox{11} & b & r & & \multicolumn{3}{c}{} & & \\

RES b,m & $\mathtt{m_b}\leftarrow$0 &
	$\bullet$ & $\bullet$ & $\bullet$ & $\bullet$ & $\bullet$ & $\bullet$ & $\bullet$ & $\bullet$ & \fbox{10} & &
	\multicolumn{4}{c}{} & & \\ \hline

Note: & \multicolumn{18}{l}{\parbox{12cm}{\footnotemark[1]See section \ref{bit_flags} for a complete description.}} \\
      & \multicolumn{18}{l}{\parbox{12cm}{\footnotemark[2]Instruction has other undocumented opcodes.}} \\
      & \multicolumn{18}{l}{\parbox{12cm}{m is one of r, (HL), (IX+d), (IY+d). To form RES instruction, replace \fbox{11}
  					  with \fbox{10}.}} \\ \hline

\end{tabular}
}

\section{Jump Group}

{\tt \scriptsize \setlength{\fboxsep}{0.25mm}
\setlength{\tabcolsep}{1mm}
\begin{tabular}{llc@{ }c@{ }c@{ }c@{ }c@{ }c@{ }c@{ }cc@{ }c@{ }cccccll}\hline
 & \multicolumn{1}{c}{Symbolic} & \multicolumn{8}{c}{Flags} & \multicolumn{3}{c}{Opcode} & & & M & T & \\
Mnemonic & \multicolumn{1}{c}{Operation} & SF & ZF & YF & HF & XF & PF & NF & CF & 76 & 543 & 210 
	& Hex & Bytes & Cycles & States & \multicolumn{2}{l}{Comments} \\ \hline

JP nn & PC$\leftarrow$nn &
	$\bullet$ & $\bullet$ & $\bullet$ & $\bullet$ & $\bullet$ & $\bullet$ & $\bullet$ & $\bullet$ & 11 & 000 & 011
	& C3 & 3 & 3 & 10 & cc & Condition \\ \cline{18-19}
\multicolumn{10}{c}{} & \multicolumn{3}{c}{$\leftarrow$ n $\rightarrow$} & \multicolumn{4}{c}{} & 000 & NZ \\
\multicolumn{10}{c}{} & \multicolumn{3}{c}{$\leftarrow$ n $\rightarrow$} & \multicolumn{4}{c}{} & 001 & Z \\

JP cc,nn & if cc &
	$\bullet$ & $\bullet$ & $\bullet$ & $\bullet$ & $\bullet$ & $\bullet$ & $\bullet$ & $\bullet$ & 11 & cc & 010
	& & 3 & 3 & 10 & 010 & NC \\ 
& PC$\leftarrow$nn & \multicolumn{8}{c}{} & \multicolumn{3}{c}{$\leftarrow$ n $\rightarrow$} & \multicolumn{4}{c}{} & 011 & C \\
\multicolumn{10}{c}{} & \multicolumn{3}{c}{$\leftarrow$ n $\rightarrow$} & \multicolumn{4}{c}{} & 100 & PO \\

JR e & PC$\leftarrow$PC+e &
	$\bullet$ & $\bullet$ & $\bullet$ & $\bullet$ & $\bullet$ & $\bullet$ & $\bullet$ & $\bullet$ & 00 & 011 & 000
	& 18 & 2 & 3 & 12 & 101 & PE \\
\multicolumn{10}{c}{} & \multicolumn{3}{c}{$\leftarrow$ e-2 $\rightarrow$} & \multicolumn{4}{c}{} & 110 & P \\
\multicolumn{17}{c}{} & 111 & M \\
\multicolumn{17}{c}{} & & \\

JR ss,e & if ss &
	$\bullet$ & $\bullet$ & $\bullet$ & $\bullet$ & $\bullet$ & $\bullet$ & $\bullet$ & $\bullet$ & 00 & 1ss & 000
	& & 2 & 3 & 12 & \multicolumn{2}{l}{if ss is true} \\
 & PC$\leftarrow$PC+e & \multicolumn{8}{c}{} & \multicolumn{3}{c}{$\leftarrow$ e-2 $\rightarrow$} & & 2 & 2 & 7 & 
	\multicolumn{2}{l}{if ss is false} \\

JP (HL) & PC$\leftarrow$HL &
	$\bullet$ & $\bullet$ & $\bullet$ & $\bullet$ & $\bullet$ & $\bullet$ & $\bullet$ & $\bullet$ & 11 & 101 & 001 
	& E9 & 1 & 1 & 4 & & \\

JP (IX) & PC$\leftarrow$IX &
	$\bullet$ & $\bullet$ & $\bullet$ & $\bullet$ & $\bullet$ & $\bullet$ & $\bullet$ & $\bullet$ & 11 & 011 & 101 
	& DD & 2 & 2 & 8 & ss & Condition \\ \cline{18-19}
\multicolumn{10}{c}{} & 11 & 101 & 001 & E9 & \multicolumn{3}{c}{} & 11 & C \\

JP (IY) & PC$\leftarrow$IY &
	$\bullet$ & $\bullet$ & $\bullet$ & $\bullet$ & $\bullet$ & $\bullet$ & $\bullet$ & $\bullet$ & 11 & 111 & 101 
	& FD & 2 & 2 & 8 & 10 & NC \\
\multicolumn{10}{c}{} & 11 & 101 & 001 & E9 & \multicolumn{3}{c}{} & 01 & Z \\
\multicolumn{17}{c}{} & 00 & NZ \\
\multicolumn{17}{c}{} & & \\

DJNZ e & B$\leftarrow$B-1 &
	$\bullet$ & $\bullet$ & $\bullet$ & $\bullet$ & $\bullet$ & $\bullet$ & $\bullet$ & $\bullet$ & 00 & 010 & 000
	& 10 & 2 & 2 & 8 & \multicolumn{2}{l}{if B=0} \\
 & if B$\not=$0 & \multicolumn{8}{c}{} & \multicolumn{3}{c}{$\leftarrow$ e-2 $\rightarrow$} & & & & &  & \\
 & PC$\leftarrow$PC+e & \multicolumn{8}{c}{} & \multicolumn{3}{c}{} & & 2 & 3 & 13 & 
	\multicolumn{2}{l}{if B$\not=$0} \\ \hline

Note: & \multicolumn{18}{l}{\parbox{12cm}{e is a signed two-compliment in the range -127, 129.}} \\
      & \multicolumn{18}{l}{\parbox{12cm}{e-2 in the opcode provides an effective number of PC+e as PC is incremented 
	by two prior to the addition of e.}} \\[1ex] \hline
\end{tabular}
}

\section{Call and Return Group}

{\tt \scriptsize \setlength{\fboxsep}{0.25mm}
\setlength{\tabcolsep}{1mm}
\begin{tabular}{llc@{ }c@{ }c@{ }c@{ }c@{ }c@{ }c@{ }cc@{ }c@{ }cccccll}\hline
 & \multicolumn{1}{c}{Symbolic} & \multicolumn{8}{c}{Flags} & \multicolumn{3}{c}{Opcode} & & & M & T & \\
Mnemonic & \multicolumn{1}{c}{Operation} & SF & ZF & YF & HF & XF & PF & NF & CF & 76 & 543 & 210 
	& Hex & Bytes & Cycles & States & \multicolumn{2}{l}{Comments} \\ \hline

CALL nn & (SP-1)$\leftarrow$PCh &
	$\bullet$ & $\bullet$ & $\bullet$ & $\bullet$ & $\bullet$ & $\bullet$ & $\bullet$ & $\bullet$ & 11 & 001 & 101 
	& CD & 3 & 5 & 17 &  & \\
	& (SP-2)$\leftarrow$PCl & \multicolumn{8}{c}{} & \multicolumn{3}{c}{$\leftarrow$ n $\rightarrow$} & & & &  &  & \\
	& SP$\leftarrow$SP-2 & \multicolumn{8}{c}{} & \multicolumn{3}{c}{$\leftarrow$ n $\rightarrow$} & & & &  &  & \\
	& PC$\leftarrow$nn & \multicolumn{8}{c}{} & \multicolumn{3}{c}{} & & & &  &  & \\

CALL cc,nn & if cc is true &
	$\bullet$ & $\bullet$ & $\bullet$ & $\bullet$ & $\bullet$ & $\bullet$ & $\bullet$ & $\bullet$ & 11 & cc & 100 
	& & 3 & 3 & 10 & \multicolumn{2}{l}{if cc is false} \\
	& (SP-1)$\leftarrow$PCh & \multicolumn{8}{c}{} & \multicolumn{3}{c}{$\leftarrow$ n $\rightarrow$} & & 3 & 5 & 17 & 
		\multicolumn{2}{l}{if cc is true} \\
	& (SP-2)$\leftarrow$PCl & \multicolumn{8}{c}{} & \multicolumn{3}{c}{$\leftarrow$ n $\rightarrow$} & & & &  &  & \\
	& SP$\leftarrow$SP-2 & \multicolumn{8}{c}{} & \multicolumn{3}{c}{} & & & &  &  & \\
	& PC$\leftarrow$nn & \multicolumn{8}{c}{} & \multicolumn{3}{c}{} & & & &  &  & \\

RET & PCl$\leftarrow$(SP) &
	$\bullet$ & $\bullet$ & $\bullet$ & $\bullet$ & $\bullet$ & $\bullet$ & $\bullet$ & $\bullet$ & 11 & 001 & 001
	& C9 & 1 & 3 & 10 &  & \\
	& PCh$\leftarrow$(SP+1) & \multicolumn{8}{c}{} & \multicolumn{3}{c}{} & & & &  &  & \\
	& SP$\leftarrow$SP+2 & \multicolumn{8}{c}{} & \multicolumn{3}{c}{} & & & &  &  & \\

RET cc & if cc is true &
	$\bullet$ & $\bullet$ & $\bullet$ & $\bullet$ & $\bullet$ & $\bullet$ & $\bullet$ & $\bullet$ & 11 & cc & 000
	&  & 1 & 1 & 5 & \multicolumn{2}{l}{if cc is false} \\
	& PCl$\leftarrow$(SP) & \multicolumn{8}{c}{} & \multicolumn{3}{c}{} & & 1 & 3 & 11 & \multicolumn{2}{l}{if cc is true} \\ 
	& PCh$\leftarrow$(SP+1) & \multicolumn{8}{c}{} & \multicolumn{3}{c}{} & & & &  &  & \\
	& SP$\leftarrow$SP+2 & \multicolumn{8}{c}{} & \multicolumn{3}{c}{} & & & &  &  & \\

RETI\footnotemark[1] & PCl$\leftarrow$(SP) &
	$\bullet$ & $\bullet$ & $\bullet$ & $\bullet$ & $\bullet$ & $\bullet$ & $\bullet$ & $\bullet$ & 11 & 101 & 101 
	& ED & 2 & 4 & 14 & cc & Condition \\ \cline{18-19}
	& PCh$\leftarrow$(SP+1) & \multicolumn{8}{c}{} & 01 & 001 & 101 & 4D & & & & 000 & NZ \\
	& SP$\leftarrow$SP+2 & \multicolumn{8}{c}{} & \multicolumn{3}{c}{} & & & &  & 001 & Z \\
	& IFF1$\leftarrow$IFF2 & \multicolumn{8}{c}{} & \multicolumn{3}{c}{} & & & &  & 010 & NC \\

RETN\footnotemark[2] & PCl$\leftarrow$(SP) &
	$\bullet$ & $\bullet$ & $\bullet$ & $\bullet$ & $\bullet$ & $\bullet$ & $\bullet$ & $\bullet$ & 11 & 101 & 101 
	& ED & 2 & 4 & 14 & 011 & C \\
	& PCh$\leftarrow$(SP+1) & \multicolumn{8}{c}{} & 01 & 000 & 101 & 45 & & & & 100 & PO \\
	& SP$\leftarrow$SP+2 & \multicolumn{8}{c}{} & \multicolumn{3}{c}{} & & & &  & 101 & PE \\
	& IFF1$\leftarrow$IFF2 & \multicolumn{8}{c}{} & \multicolumn{3}{c}{} & & & &  & 110 & P \\
	& & \multicolumn{8}{c}{} & \multicolumn{3}{c}{} & & & &  & 111 & M \\
	& & \multicolumn{8}{c}{} & \multicolumn{3}{c}{} & & & &  & & \\
	& & \multicolumn{8}{c}{} & \multicolumn{3}{c}{} & & & &  & & \\

RST p & (SP-1)$\leftarrow$PCh &
	$\bullet$ & $\bullet$ & $\bullet$ & $\bullet$ & $\bullet$ & $\bullet$ & $\bullet$ & $\bullet$ & 11 & t & 111 
	&  & 1 & 3 & 11 & t & p \\ \cline{18-19}
	& (SP-2)$\leftarrow$PCl & \multicolumn{8}{c}{} & \multicolumn{3}{c}{} & & & &  & 000 & 0h\\
	& SP$\leftarrow$SP-2 & \multicolumn{8}{c}{} & \multicolumn{3}{c}{} & & & &  & 001 & 8h \\
	& PC$\leftarrow$p & \multicolumn{8}{c}{} & \multicolumn{3}{c}{} & & & &  & 010 & 10h \\
	& & \multicolumn{8}{c}{} & \multicolumn{3}{c}{} & & & &  & 011 & 18h \\
	& & \multicolumn{8}{c}{} & \multicolumn{3}{c}{} & & & &  & 100 & 20h \\
	& & \multicolumn{8}{c}{} & \multicolumn{3}{c}{} & & & &  & 101 & 28h \\
	& & \multicolumn{8}{c}{} & \multicolumn{3}{c}{} & & & &  & 110 & 30h \\
	& & \multicolumn{8}{c}{} & \multicolumn{3}{c}{} & & & &  & 111 & 38h \\ \hline

Note: & \multicolumn{18}{l}{\parbox{12cm}{\footnotemark[1] RETI also copies IFF2 into IFF1, like RETN.}} \\
      & \multicolumn{18}{l}{\parbox{12cm}{\footnotemark[2] This instruction has other undocumented opcodes.}} \\ \hline

\end{tabular}
}

\section{Input and Output Group}

{\tt \scriptsize \setlength{\fboxsep}{0.25mm}
\setlength{\tabcolsep}{1mm}
\begin{tabular}{llc@{ }c@{ }c@{ }c@{ }c@{ }c@{ }c@{ }cc@{ }c@{ }cccccll}\hline
 & \multicolumn{1}{c}{Symbolic} & \multicolumn{8}{c}{Flags} & \multicolumn{3}{c}{Opcode} & & & M & T & \\
Mnemonic & \multicolumn{1}{c}{Operation} & SF & ZF & YF & HF & XF & PF & NF & CF & 76 & 543 & 210 
	& Hex & Bytes & Cycles & States & \multicolumn{2}{l}{Comments} \\ \hline

IN A,(n) & A$\leftarrow$(n) & 
	$\bullet$ & $\bullet$ & $\bullet$ & $\bullet$ & $\bullet$ & $\bullet$ & $\bullet$ & $\bullet$ & 11 & 011 & 011 
	& DB & 2 & 3 & 11 & r & Reg \\ \cline{18-19}
	& &  \multicolumn{8}{c}{} & \multicolumn{3}{c}{$\leftarrow$ n $\rightarrow$} & & & &  & 000 & B \\

IN r,(C) & r$\leftarrow$(C) & 
	$\updownarrow$ & $\updownarrow$ & $\updownarrow$ & 0 & $\updownarrow$ & PF & 0 & $\bullet$ & 11 & 101 & 101 
	& ED & 2 & 3 & 12 & 001 & C \\
	& & \multicolumn{8}{c}{} & 01 & r & 000  & & & &  & 010 & D \\

IN F,(n) & $\leftarrow$(C) & 
	$\updownarrow$ & $\updownarrow$ & $\updownarrow$ & 0 & $\updownarrow$ & PF & 0 & $\bullet$ & 11 & 101 & 101 
	& ED & 2 & 3 & 12 & 011 & E \\
	& & \multicolumn{8}{c}{} & 01 & 110 & 000 & 70 & & &  & 100 & H \\

INI & (HL)$\leftarrow$(C) & 
	$\updownarrow$\footnotemark[1] & $\updownarrow$\footnotemark[1] & $\updownarrow$\footnotemark[1] & $\updownarrow$\footnotemark[3] & 
	$\updownarrow$\footnotemark[1] & $\updownarrow$\footnotemark[3] & $\updownarrow$\footnotemark[2] & $\updownarrow$\footnotemark[3] & 11 & 101 & 101 
	& ED & 2 & 4 & 16 & 101 & L \\
	& HL$\leftarrow$HL+1 & \multicolumn{8}{c}{} & 10 & 100 & 010 & A2 & & &  & 111 & A \\
	& B$\leftarrow$B-1 \\

INIR & (HL)$\leftarrow$(C) & 
	0 & 1 & 0 & $\updownarrow$\footnotemark[3] & 0 & $\updownarrow$\footnotemark[3] & $\updownarrow$\footnotemark[2] & $\updownarrow$\footnotemark[3] & 11 & 101 & 101 
	& ED & 2 & 5 & 21 & \multicolumn{2}{l}{if B$\not=$0} \\
	& HL$\leftarrow$HL+1 & \multicolumn{8}{c}{} & 10 & 110 & 010 & B2 & 2 & 4 & 16 & \multicolumn{2}{l}{if B=0} \\
	& B$\leftarrow$B-1 \\
	& Repeat until \\
	& B=0 \\

IND & (HL)$\leftarrow$(C) & 
	$\updownarrow$\footnotemark[1] & $\updownarrow$\footnotemark[1] & $\updownarrow$\footnotemark[1] & $\updownarrow$\footnotemark[3] & 
	$\updownarrow$\footnotemark[1] & $\updownarrow$\footnotemark[3] & $\updownarrow$\footnotemark[2] & $\updownarrow$\footnotemark[4] & 11 & 101 & 101 
	& ED & 2 & 4 & 16 & & \\
	& HL$\leftarrow$HL-1 & \multicolumn{8}{c}{} & 10 & 101 & 010 & AA & & &  & & \\
	& B$\leftarrow$B-1 \\

INDR & (HL)$\leftarrow$(C) & 
	0 & 1 & 0 & $\updownarrow$\footnotemark[3] & 0 & $\updownarrow$\footnotemark[3] & $\updownarrow$\footnotemark[2] & $\updownarrow$\footnotemark[3] & 11 & 101 & 101 
	& ED & 2 & 5 & 21 & \multicolumn{2}{l}{if B$\not=$0} \\
	& HL$\leftarrow$HL-1 & \multicolumn{8}{c}{} & 10 & 111 & 010 & BA & 2 & 4 & 16 & \multicolumn{2}{l}{if B=0} \\
	& B$\leftarrow$B-1 \\
	& Repeat until \\
	& B=0 \\

OUT (n),A & (n)$\leftarrow$A &
	$\bullet$ & $\bullet$ & $\bullet$ & $\bullet$ & $\bullet$ & $\bullet$ & $\bullet$ & $\bullet$ & 11 & 010 & 011 
	& D3 & 2 & 3 & 11 & & \\
	& &  \multicolumn{8}{c}{} & \multicolumn{3}{c}{$\leftarrow$ n $\rightarrow$} & & & &  & & \\

OUT (C),r & (C)$\leftarrow$r &
	$\bullet$ & $\bullet$ & $\bullet$ & $\bullet$ & $\bullet$ & $\bullet$ & $\bullet$ & $\bullet$ & 11 & 101 & 101
	& ED & 2 & 3 & 12 & & \\
	& &  \multicolumn{8}{c}{} & 01 & r & 001  & & & &  & & \\

OUT (C),0 & (C)$\leftarrow$0 &
	$\bullet$ & $\bullet$ & $\bullet$ & $\bullet$ & $\bullet$ & $\bullet$ & $\bullet$ & $\bullet$ & 11 & 101 & 101
	& ED & 2 & 3 & 12 & & \\
	& &  \multicolumn{8}{c}{} & 01 & 110 & 001 & 71 & & &  & & \\

OUTI & (C)$\leftarrow$(HL) & 
	$\updownarrow$\footnotemark[1] & $\updownarrow$\footnotemark[1] & $\updownarrow$\footnotemark[1] & $\updownarrow$\footnotemark[3] & 
	$\updownarrow$\footnotemark[1] & $\updownarrow$\footnotemark[3] & $\updownarrow$\footnotemark[2] & $\updownarrow$\footnotemark[3] & 11 & 101 & 101 
	& ED & 2 & 4 & 16 & & \\
	& HL$\leftarrow$HL+1 & \multicolumn{8}{c}{} & 10 & 100 & 011 & A3 & & &  & & \\
	& B$\leftarrow$B-1 \\

OTIR & (C)$\leftarrow$(HL) & 
	0 & 1 & 0 & $\updownarrow$\footnotemark[3] & 0 & $\updownarrow$ & $\updownarrow$\footnotemark[2] & $\updownarrow$\footnotemark[3] & 11 & 101 & 101 
	& ED & 2 & 5 & 21 & \multicolumn{2}{l}{if B$\not=$0} \\
	& HL$\leftarrow$HL+1 & \multicolumn{8}{c}{} & 10 & 110 & 011 & B3 & 2 & 4 & 16 & \multicolumn{2}{l}{if B=0} \\
	& B$\leftarrow$B-1 \\
	& Repeat until \\
	& B=0 \\

OUTD & (C)$\leftarrow$(HL) & 
	$\updownarrow$\footnotemark[1] & $\updownarrow$\footnotemark[1] & $\updownarrow$\footnotemark[1] & $\updownarrow$\footnotemark[3] & 
	$\updownarrow$\footnotemark[1] & $\updownarrow$\footnotemark[3] & $\updownarrow$\footnotemark[2] & $\updownarrow$\footnotemark[3] & 11 & 101 & 101 
	& ED & 2 & 4 & 16 & & \\
	& HL$\leftarrow$HL-1 & \multicolumn{8}{c}{} & 10 & 101 & 011 & AB & & &  & & \\
	& B$\leftarrow$B-1 \\

OTDR & (C)$\leftarrow$(HL) & 
	0 & 1 & 0 & $\updownarrow$\footnotemark[3] & 0 & $\updownarrow$\footnotemark[3] & $\updownarrow$\footnotemark[2] & $\updownarrow$\footnotemark[5] & 11 & 101 & 101 
	& ED & 2 & 5 & 21 & \multicolumn{2}{l}{if B$\not=$0} \\
	& HL$\leftarrow$HL-1 & \multicolumn{8}{c}{} & 10 & 111 & 011 & BB & 2 & 4 & 16 & \multicolumn{2}{l}{if B=0} \\
	& B$\leftarrow$B-1 \\
	& Repeat until \\
	& B=0 \\ \hline

Note: & \multicolumn{18}{l}{\parbox{12cm}{\footnotemark[1] flag is affected by the result of B$\leftarrow$B-1 as in DEC B.}} \\
      & \multicolumn{18}{l}{\parbox{12cm}{\footnotemark[2] NF is a copy of bit 7 of the transferred byte.}} \\
      & \multicolumn{18}{l}{\parbox{12cm}{\footnotemark[3] This flag is bizarre, see section \ref{block_io}.}} \\ \hline

\end{tabular}
}

% we only want to list sections within this chapter; without stop, it would continue to include all subsequent ones as well
\stopcontents[intrstructionsections]



\chapter{Instructions Sorted by Opcode}

Any instruction marked with * is undocumented.
% TODO: and any instruction not listed has no effect? (same behaviour as an equal amount of NOP instructions)
\begin{multicols}{3}

\scriptsize
{\tt
\begin{tabbing}
00{\qquad}{\qquad}{\quad}\=NOP\\
01 n n\>LD BC,nn\\
02\>LD (BC),A\\
03\>INC BC\\
04\>INC B\\
05\>DEC B\\
06 n\>LD B,n\\
07\>RLCA\\
08\>EX AF,AF'\\
09\>ADD HL,BC\\
0A\>LD A,(BC)\\
0B\>DEC BC\\
0C\>INC C\\
0D\>DEC C\\
0E n\>LD C,n\\
0F\>RRCA\\
10 e\>DJNZ (PC+e)\\
11 n n\>LD DE,nn\\
12\>LD (DE),A\\
13\>INC DE\\
14\>INC D\\
15\>DEC D\\
16 n\>LD D,n\\
17\>RLA\\
18 e\>JR e\\
19\>ADD HL,DE\\
1A\>LD A,(DE)\\
1B\>DEC DE\\
1C\>INC E\\
1D\>DEC E\\
1E n\>LD E,n\\
1F\>RRA\\
20 e\>JR NZ,e\\
21 n n\>LD HL,nn\\
22 n n\>LD (nn),HL\\
23\>INC HL\\
24\>INC H\\
25\>DEC H\\
26 n\>LD H,n\\
27\>DAA\\
28 e\>JR Z,e\\
29\>ADD HL,HL\\
2A n n\>LD HL,(nn)\\
2B\>DEC HL\\
2C\>INC L\\
2D\>DEC L\\
2E n\>LD L,n\\
2F\>CPL\\
30 e\>JR NC,e\\
31 n n\>LD SP,nn\\
32 n n\>LD (nn),A\\
33\>INC SP\\
34\>INC (HL)\\
35\>DEC (HL)\\
36 n\>LD (HL),n\\
37\>SCF\\
38 e\>JR C,e\\
39\>ADD HL,SP\\
3A n n\>LD A,(nn)\\
3B\>DEC SP\\
3C\>INC A\\
3D\>DEC A\\
3E n\>LD A,n\\
3F\>CCF\\
40\>LD B,B\\
41\>LD B,C\\
42\>LD B,D\\
43\>LD B,E\\
44\>LD B,H\\
45\>LD B,L\\
46\>LD B,(HL)\\
47\>LD B,A\\
48\>LD C,B\\
49\>LD C,C\\
4A\>LD C,D\\
4B\>LD C,E\\
4C\>LD C,H\\
4D\>LD C,L\\
4E\>LD C,(HL)\\
4F\>LD C,A\\
50\>LD D,B\\
51\>LD D,C\\
52\>LD D,D\\
53\>LD D,E\\
54\>LD D,H\\
55\>LD D,L\\
56\>LD D,(HL)\\
57\>LD D,A\\
58\>LD E,B\\
59\>LD E,C\\
5A\>LD E,D\\
5B\>LD E,E\\
5C\>LD E,H\\
5D\>LD E,L\\
5E\>LD E,(HL)\\
5F\>LD E,A\\
60\>LD H,B\\
61\>LD H,C\\
62\>LD H,D\\
63\>LD H,E\\
64\>LD H,H\\
65\>LD H,L\\
66\>LD H,(HL)\\
67\>LD H,A\\
68\>LD L,B\\
69\>LD L,C\\
6A\>LD L,D\\
6B\>LD L,E\\
6C\>LD L,H\\
6D\>LD L,L\\
6E\>LD L,(HL)\\
6F\>LD L,A\\
70\>LD (HL),B\\
71\>LD (HL),C\\
72\>LD (HL),D\\
73\>LD (HL),E\\
74\>LD (HL),H\\
75\>LD (HL),L\\
76\>HALT\\
77\>LD (HL),A\\
78\>LD A,B\\
79\>LD A,C\\
7A\>LD A,D\\
7B\>LD A,E\\
7C\>LD A,H\\
7D\>LD A,L\\
7E\>LD A,(HL)\\
7F\>LD A,A\\
80\>ADD A,B\\
81\>ADD A,C\\
82\>ADD A,D\\
83\>ADD A,E\\
84\>ADD A,H\\
85\>ADD A,L\\
86\>ADD A,(HL)\\
87\>ADD A,A\\
88\>ADC A,B\\
89\>ADC A,C\\
8A\>ADC A,D\\
8B\>ADC A,E\\
8C\>ADC A,H\\
8D\>ADC A,L\\
8E\>ADC A,(HL)\\
8F\>ADC A,A\\
90\>SUB B\\
91\>SUB C\\
92\>SUB D\\
93\>SUB E\\
94\>SUB H\\
95\>SUB L\\
96\>SUB (HL)\\
97\>SUB A\\
98\>SBC A,B\\
99\>SBC A,C\\
9A\>SBC A,D\\
9B\>SBC A,E\\
9C\>SBC A,H\\
9D\>SBC A,L\\
9E\>SBC A,(HL)\\
9F\>SBC A,A\\
A0\>AND B\\
A1\>AND C\\
A2\>AND D\\
A3\>AND E\\
A4\>AND H\\
A5\>AND L\\
A6\>AND (HL)\\
A7\>AND A\\
A8\>XOR B\\
A9\>XOR C\\
AA\>XOR D\\
AB\>XOR E\\
AC\>XOR H\\
AD\>XOR L\\
AE\>XOR (HL)\\
AF\>XOR A\\
B0\>OR B\\
B1\>OR C\\
B2\>OR D\\
B3\>OR E\\
B4\>OR H\\
B5\>OR L\\
B6\>OR (HL)\\
B7\>OR A\\
B8\>CP B\\
B9\>CP C\\
BA\>CP D\\
BB\>CP E\\
BC\>CP H\\
BD\>CP L\\
BE\>CP (HL)\\
BF\>CP A\\
C0\>RET NZ\\
C1\>POP BC\\
C2 n n\>JP NZ,nn\\
C3 n n\>JP nn\\
C4 n n\>CALL NZ,nn\\
C5\>PUSH BC\\
C6 n\>ADD A,n\\
C7\>RST 0H\\
C8\>RET Z\\
C9\>RET\\
CA n n\>JP Z,nn\\
CB00\>RLC B\\
CB01\>RLC C\\
CB02\>RLC D\\
CB03\>RLC E\\
CB04\>RLC H\\
CB05\>RLC L\\
CB06\>RLC (HL)\\
CB07\>RLC A\\
CB08\>RRC B\\
CB09\>RRC C\\
CB0A\>RRC D\\
CB0B\>RRC E\\
CB0C\>RRC H\\
CB0D\>RRC L\\
CB0E\>RRC (HL)\\
CB0F\>RRC A\\
CB10\>RL B\\
CB11\>RL C\\
CB12\>RL D\\
CB13\>RL E\\
CB14\>RL H\\
CB15\>RL L\\
CB16\>RL (HL)\\
CB17\>RL A\\
CB18\>RR B\\
CB19\>RR C\\
CB1A\>RR D\\
CB1B\>RR E\\
CB1C\>RR H\\
CB1D\>RR L\\
CB1E\>RR (HL)\\
CB1F\>RR A\\
CB20\>SLA B\\
CB21\>SLA C\\
CB22\>SLA D\\
CB23\>SLA E\\
CB24\>SLA H\\
CB25\>SLA L\\
CB26\>SLA (HL)\\
CB27\>SLA A\\
CB28\>SRA B\\
CB29\>SRA C\\
CB2A\>SRA D\\
CB2B\>SRA E\\
CB2C\>SRA H\\
CB2D\>SRA L\\
CB2E\>SRA (HL)\\
CB2F\>SRA A\\
CB30\>SLL B*\\
CB31\>SLL C*\\
CB32\>SLL D*\\
CB33\>SLL E*\\
CB34\>SLL H*\\
CB35\>SLL L*\\
CB36\>SLL (HL)*\\
CB37\>SLL A*\\
CB38\>SRL B\\
CB39\>SRL C\\
CB3A\>SRL D\\
CB3B\>SRL E\\
CB3C\>SRL H\\
CB3D\>SRL L\\
CB3E\>SRL (HL)\\
CB3F\>SRL A\\
CB40\>BIT 0,B\\
CB41\>BIT 0,C\\
CB42\>BIT 0,D\\
CB43\>BIT 0,E\\
CB44\> 	BIT 0,H\\
CB45\> 	BIT 0,L\\
CB46\> 	BIT 0,(HL)\\
CB47\> 	BIT 0,A\\
CB48\>BIT 1,B\\
CB49\> 	BIT 1,C\\
CB4A\> 	BIT 1,D\\
CB4B\> 	BIT 1,E\\
CB4C\> 	BIT 1,H\\
CB4D\> 	BIT 1,L\\
CB4E\> 	BIT 1,(HL)\\
CB4F\> 	BIT 1,A\\
CB50\>BIT 2,B\\
CB51\> 	BIT 2,C\\
CB52\> 	BIT 2,D\\
CB53\> 	BIT 2,E\\
CB54\> 	BIT 2,H\\
CB55\> 	BIT 2,L\\
CB56\> 	BIT 2,(HL)\\
CB57\> 	BIT 2,A\\
CB58\>BIT 3,B\\
CB59\> 	BIT 3,C\\
CB5A\> 	BIT 3,D\\
CB5B\> 	BIT 3,E\\
CB5C\> 	BIT 3,H\\
CB5D\> 	BIT 3,L\\
CB5E\> 	BIT 3,(HL)\\
CB5F\> 	BIT 3,A\\
CB60\>BIT 4,B\\
CB61\> 	BIT 4,C\\
CB62\> 	BIT 4,D\\
CB63\> 	BIT 4,E\\
CB64\> 	BIT 4,H\\
CB65\> 	BIT 4,L\\
CB66\> 	BIT 4,(HL)\\
CB67\> 	BIT 4,A\\
CB68\>BIT 5,B\\
CB69\> 	BIT 5,C\\
CB6A\> 	BIT 5,D\\
CB6B\> 	BIT 5,E\\
CB6C\> 	BIT 5,H\\
CB6D\> 	BIT 5,L\\
CB6E\> 	BIT 5,(HL)\\
CB6F\> 	BIT 5,A\\
CB70\>BIT 6,B\\
CB71\> 	BIT 6,C\\
CB72\> 	BIT 6,D\\
CB73\> 	BIT 6,E\\
CB74\> 	BIT 6,H\\
CB75\> 	BIT 6,L\\
CB76\> 	BIT 6,(HL)\\
CB77\> 	BIT 6,A\\
CB78\>BIT 7,B\\
CB79\> 	BIT 7,C\\
CB7A\> 	BIT 7,D\\
CB7B\> 	BIT 7,E\\
CB7C\> 	BIT 7,H\\
CB7D\> 	BIT 7,L\\
CB7E\> 	BIT 7,(HL)\\
CB7F\> 	BIT 7,A\\
CB80\>RES 0,B\\
CB81\>RES 0,C\\
CB82\>RES 0,D\\
CB83\>RES 0,E\\
CB84\> 	RES 0,H\\
CB85\> 	RES 0,L\\
CB86\> 	RES 0,(HL)\\
CB87\> 	RES 0,A\\
CB88\>RES 1,B\\
CB89\> 	RES 1,C\\
CB8A\> 	RES 1,D\\
CB8B\> 	RES 1,E\\
CB8C\> 	RES 1,H\\
CB8D\> 	RES 1,L\\
CB8E\> 	RES 1,(HL)\\
CB8F\> 	RES 1,A\\
CB90\>RES 2,B\\
CB91\> 	RES 2,C\\
CB92\> 	RES 2,D\\
CB93\> 	RES 2,E\\
CB94\> 	RES 2,H\\
CB95\> 	RES 2,L\\
CB96\> 	RES 2,(HL)\\
CB97\> 	RES 2,A\\
CB98\>RES 3,B\\
CB99\> 	RES 3,C\\
CB9A\> 	RES 3,D\\
CB9B\> 	RES 3,E\\
CB9C\> 	RES 3,H\\
CB9D\> 	RES 3,L\\
CB9E\> 	RES 3,(HL)\\
CB9F\> 	RES 3,A\\
CBA0\>RES 4,B\\
CBA1\> 	RES 4,C\\
CBA2\> 	RES 4,D\\
CBA3\> 	RES 4,E\\
CBA4\> 	RES 4,H\\
CBA5\> 	RES 4,L\\
CBA6\> 	RES 4,(HL)\\
CBA7\> 	RES 4,A\\
CBA8\>RES 5,B\\
CBA9\> 	RES 5,C\\
CBAA\> 	RES 5,D\\
CBAB\> 	RES 5,E\\
CBAC\> 	RES 5,H\\
CBAD\> 	RES 5,L\\
CBAE\> 	RES 5,(HL)\\
CBAF\> 	RES 5,A\\
CBB0\>RES 6,B\\
CBB1\> 	RES 6,C\\
CBB2\> 	RES 6,D\\
CBB3\> 	RES 6,E\\
CBB4\> 	RES 6,H\\
CBB5\> 	RES 6,L\\
CBB6\> 	RES 6,(HL)\\
CBB7\> 	RES 6,A\\
CBB8\>RES 7,B\\
CBB9\> 	RES 7,C\\
CBBA\> 	RES 7,D\\
CBBB\> 	RES 7,E\\
CBBC\> 	RES 7,H\\
CBBD\> 	RES 7,L\\
CBBE\> 	RES 7,(HL)\\
CBBF\> 	RES 7,A\\
CBC0\>SET 0,B\\
CBC1\>SET 0,C\\
CBC2\>SET 0,D\\
CBC3\>SET 0,E\\
CBC4\> 	SET 0,H\\
CBC5\> 	SET 0,L\\
CBC6\> 	SET 0,(HL)\\
CBC7\> 	SET 0,A\\
CBC8\>SET 1,B\\
CBC9\> 	SET 1,C\\
CBCA\> 	SET 1,D\\
CBCB\> 	SET 1,E\\
CBCC\> 	SET 1,H\\
CBCD\> 	SET 1,L\\
CBCE\> 	SET 1,(HL)\\
CBCF\> 	SET 1,A\\
CBD0\>SET 2,B\\
CBD1\> 	SET 2,C\\
CBD2\> 	SET 2,D\\
CBD3\> 	SET 2,E\\
CBD4\> 	SET 2,H\\
CBD5\> 	SET 2,L\\
CBD6\> 	SET 2,(HL)\\
CBD7\> 	SET 2,A\\
CBD8\>SET 3,B\\
CBD9\> 	SET 3,C\\
CBDA\> 	SET 3,D\\
CBDB\> 	SET 3,E\\
CBDC\> 	SET 3,H\\
CBDD\> 	SET 3,L\\
CBDE\> 	SET 3,(HL)\\
CBDF\> 	SET 3,A\\
CBE0\>SET 4,B\\
CBE1\> 	SET 4,C\\
CBE2\> 	SET 4,D\\
CBE3\> 	SET 4,E\\
CBE4\> 	SET 4,H\\
CBE5\> 	SET 4,L\\
CBE6\> 	SET 4,(HL)\\
CBE7\> 	SET 4,A\\
CBE8\>SET 5,B\\
CBE9\> 	SET 5,C\\
CBEA\> 	SET 5,D\\
CBEB\> 	SET 5,E\\
CBEC\> 	SET 5,H\\
CBED\> 	SET 5,L\\
CBEE\> 	SET 5,(HL)\\
CBEF\> 	SET 5,A\\
CBF0\>SET 6,B\\
CBF1\> 	SET 6,C\\
CBF2\> 	SET 6,D\\
CBF3\> 	SET 6,E\\
CBF4\> 	SET 6,H\\
CBF5\> 	SET 6,L\\
CBF6\> 	SET 6,(HL)\\
CBF7\> 	SET 6,A\\
CBF8\>SET 7,B\\
CBF9\> 	SET 7,C\\
CBFA\> 	SET 7,D\\
CBFB\> 	SET 7,E\\
CBFC\> 	SET 7,H\\
CBFD\> 	SET 7,L\\
CBFE\> 	SET 7,(HL)\\
CBFF\> 	SET 7,A\\
CC n n\>CALL Z,nn\\
CD n n\>CALL nn\\
CE n\>ADC A,n\\
CF\>RST 8H\\
D0\>RET NC\\
D1\>POP DE\\
D2 n n\>JP NC,nn\\
D3 n\>OUT (n),A\\
D4 n n\>CALL NC,nn\\
D5\>PUSH DE\\
D6 n\>SUB n\\
D7\>RST 10H\\
D8\>RET C\\
D9\>EXX\\
DA n n\>JP C,nn\\
DB n\>IN A,(n)\\
DC n n\>CALL C,nn\\
DD09\>ADD IX,BC\\
DD19\>ADD IX,DE\\
DD21 n n\>LD IX,nn\\
DD22 n n\>LD (nn),IX\\
DD23\>INC IX\\
DD24\>INC IXh*\\
DD25\>DEC IXh*\\
DD26 n \>LD IXh,n*\\
DD29\>ADD IX,IX\\
DD2A n n\>LD IX,(nn)\\
DD2B\>DEC IX\\
DD2C\>INC IXl*\\
DD2D\>DEC IXl*\\
DD2E n\>LD IXl,n*\\
DD34 d\>INC (IX+d)\\
DD35 d\>DEC (IX+d)\\
DD36 d n\>LD (IX+d),n\\
DD39\>ADD IX,SP\\
DD44\>LD B,IXh*\\
DD45\>LD B,IXl*\\
DD46 d\>LD B,(IX+d)\\
DD4C\>LD C,IXh*\\
DD4D\>LD C,IXl*\\
DD4E d\>LD C,(IX+d)\\
DD54\>LD D,IXh*\\
DD55\>LD D,IXl*\\
DD56 d\>LD D,(IX+d)\\
DD5C\>LD E,IXh*\\
DD5D\>LD E,IXl*\\
DD5E d\>LD E,(IX+d)\\
DD60\>LD IXh,B*\\
DD61\>LD IXh,C*\\
DD62\>LD IXh,D*\\
DD63\>LD IXh,E*\\
DD64\>LD IXh,IXh*\\
DD65\>LD IXh,IXl*\\
DD66 d\>LD H,(IX+d)\\
DD67\>LD IXh,A*\\
DD68\>LD IXl,B*\\
DD69\>LD IXl,C*\\
DD6A\>LD IXl,D*\\
DD6B\>LD IXl,E*\\
DD6C\>LD IXl,IXh*\\
DD6D\>LD IXl,IXl*\\
DD6E d\>LD L,(IX+d)\\
DD6F\>LD IXl,A*\\
DD70 d\>LD (IX+d),B\\
DD71 d\>LD (IX+d),C\\
DD72 d\>LD (IX+d),D\\
DD73 d\>LD (IX+d),E\\
DD74 d\>LD (IX+d),H\\
DD75 d\>LD (IX+d),L\\
DD77 d\>LD (IX+d),A\\
DD7C\>LD A,IXh*\\
DD7D\>LD A,IXl*\\
DD7E d\>LD A,(IX+d)\\
DD84\>ADD A,IXh*\\
DD85\>ADD A,IXl*\\
DD86 d\>ADD A,(IX+d)\\
DD8C\>ADC A,IXh*\\
DD8D\>ADC A,IXl*\\
DD8E d\>ADC A,(IX+d)\\
DD94\>SUB IXh*\\
DD95\>SUB IXl*\\
DD96 d\>SUB (IX+d)\\
DD9C\>SBC A,IXh*\\
DD9D\>SBC A,IXl*\\
DD9E d\>SBC A,(IX+d)\\
DDA4\>AND IXh*\\
DDA5\>AND IXl*\\
DDA6 d\>AND (IX+d)\\
DDAC\>XOR IXh*\\
DDAD\>XOR IXl*\\
DDAE d\>XOR (IX+d)\\
DDB4\>OR IXh*\\
DDB5\>OR IXl*\\
DDB6 d\>OR (IX+d)\\
DDBC\>CP IXh*\\
DDBD\>CP IXl*\\
DDBE d\>CP (IX+d)\\
DDCB d 00\>RLC (IX+d),B*\\
DDCB d 01\>RLC (IX+d),C*\\
DDCB d 02\>RLC (IX+d),D*\\
DDCB d 03\>RLC (IX+d),E*\\
DDCB d 04\>RLC (IX+d),H*\\
DDCB d 05\>RLC (IX+d),L*\\
DDCB d 06\>RLC (IX+d)\\
DDCB d 07\>RLC (IX+d),A*\\
DDCB d 08\>RRC (IX+d),B*\\
DDCB d 09\>RRC (IX+d),C*\\
DDCB d 0A\>RRC (IX+d),D*\\
DDCB d 0B\>RRC (IX+d),E*\\
DDCB d 0C\>RRC (IX+d),H*\\
DDCB d 0D\>RRC (IX+d),L*\\
DDCB d 0E\>RRC (IX+d)\\
DDCB d 0F\>RRC (IX+d),A*\\
DDCB d 10\>RL (IX+d),B*\\
DDCB d 11\>RL (IX+d),C*\\
DDCB d 12\>RL (IX+d),D*\\
DDCB d 13\>RL (IX+d),E*\\
DDCB d 14\>RL (IX+d),H*\\
DDCB d 15\>RL (IX+d),L*\\
DDCB d 16\>RL (IX+d)\\
DDCB d 17\>RL (IX+d),A*\\
DDCB d 18\>RR (IX+d),B*\\
DDCB d 19\>RR (IX+d),C*\\
DDCB d 1A\>RR (IX+d),D*\\
DDCB d 1B\>RR (IX+d),E*\\
DDCB d 1C\>RR (IX+d),H*\\
DDCB d 1D\>RR (IX+d),L*\\
DDCB d 1E\>RR (IX+d)\\
DDCB d 1F\>RR (IX+d),A*\\
DDCB d 20\>SLA (IX+d),B*\\
DDCB d 21\>SLA (IX+d),C*\\
DDCB d 22\>SLA (IX+d),D*\\
DDCB d 23\>SLA (IX+d),E*\\
DDCB d 24\>SLA (IX+d),H*\\
DDCB d 25\>SLA (IX+d),L*\\
DDCB d 26\>SLA (IX+d)\\
DDCB d 27\>SLA (IX+d),A*\\
DDCB d 28\>SRA (IX+d),B*\\
DDCB d 29\>SRA (IX+d),C*\\
DDCB d 2A\>SRA (IX+d),D*\\
DDCB d 2B\>SRA (IX+d),E*\\
DDCB d 2C\>SRA (IX+d),H*\\
DDCB d 2D\>SRA (IX+d),L*\\
DDCB d 2E\>SRA (IX+d)\\
DDCB d 2F\>SRA (IX+d),A*\\
DDCB d 30\>SLL (IX+d),B*\\
DDCB d 31\>SLL (IX+d),C*\\
DDCB d 32\>SLL (IX+d),D*\\
DDCB d 33\>SLL (IX+d),E*\\
DDCB d 34\>SLL (IX+d),H*\\
DDCB d 35\>SLL (IX+d),L*\\
DDCB d 36\>SLL (IX+d)*\\
DDCB d 37\>SLL (IX+d),A*\\
DDCB d 38\>SRL (IX+d),B*\\
DDCB d 39\>SRL (IX+d),C*\\
DDCB d 3A\>SRL (IX+d),D*\\
DDCB d 3B\>SRL (IX+d),E*\\
DDCB d 3C\>SRL (IX+d),H*\\
DDCB d 3D\>SRL (IX+d),L*\\
DDCB d 3E\>SRL (IX+d)\\
DDCB d 3F\>SRL (IX+d),A*\\
DDCB d 40\>BIT 0,(IX+d)*\\
DDCB d 41\>BIT 0,(IX+d)*\\
DDCB d 42\>BIT 0,(IX+d)*\\
DDCB d 43\>BIT 0,(IX+d)*\\
DDCB d 44\>BIT 0,(IX+d)*\\
DDCB d 45\>BIT 0,(IX+d)*\\
DDCB d 46\>BIT 0,(IX+d)\\
DDCB d 47\>BIT 0,(IX+d)*\\
DDCB d 48\>BIT 1,(IX+d)*\\
DDCB d 49\>BIT 1,(IX+d)*\\
DDCB d 4A\>BIT 1,(IX+d)*\\
DDCB d 4B\>BIT 1,(IX+d)*\\
DDCB d 4C\>BIT 1,(IX+d)*\\
DDCB d 4D\>BIT 1,(IX+d)*\\
DDCB d 4E\>BIT 1,(IX+d)\\
DDCB d 4F\>BIT 1,(IX+d)*\\
DDCB d 50\>BIT 2,(IX+d)*\\
DDCB d 51\>BIT 2,(IX+d)*\\
DDCB d 52\>BIT 2,(IX+d)*\\
DDCB d 53\>BIT 2,(IX+d)*\\
DDCB d 54\>BIT 2,(IX+d)*\\
DDCB d 55\>BIT 2,(IX+d)*\\
DDCB d 56\>BIT 2,(IX+d)\\
DDCB d 57\>BIT 2,(IX+d)*\\
DDCB d 58\>BIT 3,(IX+d)*\\
DDCB d 59\>BIT 3,(IX+d)*\\
DDCB d 5A\>BIT 3,(IX+d)*\\
DDCB d 5B\>BIT 3,(IX+d)*\\
DDCB d 5C\>BIT 3,(IX+d)*\\
DDCB d 5D\>BIT 3,(IX+d)*\\
DDCB d 5E\>BIT 3,(IX+d)\\
DDCB d 5F\>BIT 3,(IX+d)*\\
DDCB d 60\>BIT 4,(IX+d)*\\
DDCB d 61\>BIT 4,(IX+d)*\\
DDCB d 62\>BIT 4,(IX+d)*\\
DDCB d 63\>BIT 4,(IX+d)*\\
DDCB d 64\>BIT 4,(IX+d)*\\
DDCB d 65\>BIT 4,(IX+d)*\\
DDCB d 66\>BIT 4,(IX+d)\\
DDCB d 67\>BIT 4,(IX+d)*\\
DDCB d 68\>BIT 5,(IX+d)*\\
DDCB d 69\>BIT 5,(IX+d)*\\
DDCB d 6A\>BIT 5,(IX+d)*\\
DDCB d 6B\>BIT 5,(IX+d)*\\
DDCB d 6C\>BIT 5,(IX+d)*\\
DDCB d 6D\>BIT 5,(IX+d)*\\
DDCB d 6E\>BIT 5,(IX+d)\\
DDCB d 6F\>BIT 5,(IX+d)*\\
DDCB d 70\>BIT 6,(IX+d)*\\
DDCB d 71\>BIT 6,(IX+d)*\\
DDCB d 72\>BIT 6,(IX+d)*\\
DDCB d 73\>BIT 6,(IX+d)*\\
DDCB d 74\>BIT 6,(IX+d)*\\
DDCB d 75\>BIT 6,(IX+d)*\\
DDCB d 76\>BIT 6,(IX+d)\\
DDCB d 77\>BIT 6,(IX+d)*\\
DDCB d 78\>BIT 7,(IX+d)*\\
DDCB d 79\>BIT 7,(IX+d)*\\
DDCB d 7A\>BIT 7,(IX+d)*\\
DDCB d 7B\>BIT 7,(IX+d)*\\
DDCB d 7C\>BIT 7,(IX+d)*\\
DDCB d 7D\>BIT 7,(IX+d)*\\
DDCB d 7E\>BIT 7,(IX+d)\\
DDCB d 7F\>BIT 7,(IX+d)*\\
DDCB d 80\>RES 0,(IX+d),B*\\
DDCB d 81\>RES 0,(IX+d),C*\\
DDCB d 82\>RES 0,(IX+d),D*\\
DDCB d 83\>RES 0,(IX+d),E*\\
DDCB d 84\>RES 0,(IX+d),H*\\
DDCB d 85\>RES 0,(IX+d),L*\\
DDCB d 86\>RES 0,(IX+d)\\
DDCB d 87\>RES 0,(IX+d),A*\\
DDCB d 88\>RES 1,(IX+d),B*\\
DDCB d 89\>RES 1,(IX+d),C*\\
DDCB d 8A\>RES 1,(IX+d),D*\\
DDCB d 8B\>RES 1,(IX+d),E*\\
DDCB d 8C\>RES 1,(IX+d),H*\\
DDCB d 8D\>RES 1,(IX+d),L*\\
DDCB d 8E\>RES 1,(IX+d)\\
DDCB d 8F\>RES 1,(IX+d),A*\\
DDCB d 90\>RES 2,(IX+d),B*\\
DDCB d 91\>RES 2,(IX+d),C*\\
DDCB d 92\>RES 2,(IX+d),D*\\
DDCB d 93\>RES 2,(IX+d),E*\\
DDCB d 94\>RES 2,(IX+d),H*\\
DDCB d 95\>RES 2,(IX+d),L*\\
DDCB d 96\>RES 2,(IX+d)\\
DDCB d 97\>RES 2,(IX+d),A*\\
DDCB d 98\>RES 3,(IX+d),B*\\
DDCB d 99\>RES 3,(IX+d),C*\\
DDCB d 9A\>RES 3,(IX+d),D*\\
DDCB d 9B\>RES 3,(IX+d),E*\\
DDCB d 9C\>RES 3,(IX+d),H*\\
DDCB d 9D\>RES 3,(IX+d),L*\\
DDCB d 9E\>RES 3,(IX+d)\\
DDCB d 9F\>RES 3,(IX+d),A*\\
DDCB d A0\>RES 4,(IX+d),B*\\
DDCB d A1\>RES 4,(IX+d),C*\\
DDCB d A2\>RES 4,(IX+d),D*\\
DDCB d A3\>RES 4,(IX+d),E*\\
DDCB d A4\>RES 4,(IX+d),H*\\
DDCB d A5\>RES 4,(IX+d),L*\\
DDCB d A6\>RES 4,(IX+d)\\
DDCB d A7\>RES 4,(IX+d),A*\\
DDCB d A8\>RES 5,(IX+d),B*\\
DDCB d A9\>RES 5,(IX+d),C*\\
DDCB d AA\>RES 5,(IX+d),D*\\
DDCB d AB\>RES 5,(IX+d),E*\\
DDCB d AC\>RES 5,(IX+d),H*\\
DDCB d AD\>RES 5,(IX+d),L*\\
DDCB d AE\>RES 5,(IX+d)\\
DDCB d AF\>RES 5,(IX+d),A*\\
DDCB d B0\>RES 6,(IX+d),B*\\
DDCB d B1\>RES 6,(IX+d),C*\\
DDCB d B2\>RES 6,(IX+d),D*\\
DDCB d B3\>RES 6,(IX+d),E*\\
DDCB d B4\>RES 6,(IX+d),H*\\
DDCB d B5\>RES 6,(IX+d),L*\\
DDCB d B6\>RES 6,(IX+d)\\
DDCB d B7\>RES 6,(IX+d),A*\\
DDCB d B8\>RES 7,(IX+d),B*\\
DDCB d B9\>RES 7,(IX+d),C*\\
DDCB d BA\>RES 7,(IX+d),D*\\
DDCB d BB\>RES 7,(IX+d),E*\\
DDCB d BC\>RES 7,(IX+d),H*\\
DDCB d BD\>RES 7,(IX+d),L*\\
DDCB d BE\>RES 7,(IX+d)\\
DDCB d BF\>RES 7,(IX+d),A*\\
DDCB d C0\>SET 0,(IX+d),B*\\
DDCB d C1\>SET 0,(IX+d),C*\\
DDCB d C2\>SET 0,(IX+d),D*\\
DDCB d C3\>SET 0,(IX+d),E*\\
DDCB d C4\>SET 0,(IX+d),H*\\
DDCB d C5\>SET 0,(IX+d),L*\\
DDCB d C6\>SET 0,(IX+d)\\
DDCB d C7\>SET 0,(IX+d),A*\\
DDCB d C8\>SET 1,(IX+d),B*\\
DDCB d C9\>SET 1,(IX+d),C*\\
DDCB d CA\>SET 1,(IX+d),D*\\
DDCB d CB\>SET 1,(IX+d),E*\\
DDCB d CC\>SET 1,(IX+d),H*\\
DDCB d CD\>SET 1,(IX+d),L*\\
DDCB d CE\>SET 1,(IX+d)\\
DDCB d CF\>SET 1,(IX+d),A*\\
DDCB d D0\>SET 2,(IX+d),B*\\
DDCB d D1\>SET 2,(IX+d),C*\\
DDCB d D2\>SET 2,(IX+d),D*\\
DDCB d D3\>SET 2,(IX+d),E*\\
DDCB d D4\>SET 2,(IX+d),H*\\
DDCB d D5\>SET 2,(IX+d),L*\\
DDCB d D6\>SET 2,(IX+d)\\
DDCB d D7\>SET 2,(IX+d),A*\\
DDCB d D8\>SET 3,(IX+d),B*\\
DDCB d D9\>SET 3,(IX+d),C*\\
DDCB d DA\>SET 3,(IX+d),D*\\
DDCB d DB\>SET 3,(IX+d),E*\\
DDCB d DC\>SET 3,(IX+d),H*\\
DDCB d DD\>SET 3,(IX+d),L*\\
DDCB d DE\>SET 3,(IX+d)\\
DDCB d DF\>SET 3,(IX+d),A*\\
DDCB d E0\>SET 4,(IX+d),B*\\
DDCB d E1\>SET 4,(IX+d),C*\\
DDCB d E2\>SET 4,(IX+d),D*\\
DDCB d E3\>SET 4,(IX+d),E*\\
DDCB d E4\>SET 4,(IX+d),H*\\
DDCB d E5\>SET 4,(IX+d),L*\\
DDCB d E6\>SET 4,(IX+d)\\
DDCB d E7\>SET 4,(IX+d),A*\\
DDCB d E8\>SET 5,(IX+d),B*\\
DDCB d E9\>SET 5,(IX+d),C*\\
DDCB d EA\>SET 5,(IX+d),D*\\
DDCB d EB\>SET 5,(IX+d),E*\\
DDCB d EC\>SET 5,(IX+d),H*\\
DDCB d ED\>SET 5,(IX+d),L*\\
DDCB d EE\>SET 5,(IX+d)\\
DDCB d EF\>SET 5,(IX+d),A*\\
DDCB d F0\>SET 6,(IX+d),B*\\
DDCB d F1\>SET 6,(IX+d),C*\\
DDCB d F2\>SET 6,(IX+d),D*\\
DDCB d F3\>SET 6,(IX+d),E*\\
DDCB d F4\>SET 6,(IX+d),H*\\
DDCB d F5\>SET 6,(IX+d),L*\\
DDCB d F6\>SET 6,(IX+d)\\
DDCB d F7\>SET 6,(IX+d),A*\\
DDCB d F8\>SET 7,(IX+d),B*\\
DDCB d F9\>SET 7,(IX+d),C*\\
DDCB d FA\>SET 7,(IX+d),D*\\
DDCB d FB\>SET 7,(IX+d),E*\\
DDCB d FC\>SET 7,(IX+d),H*\\
DDCB d FD\>SET 7,(IX+d),L*\\
DDCB d FE\>SET 7,(IX+d)\\
DDCB d FF\>SET 7,(IX+d),A*\\
DDE1\>POP IX\\
DDE3\>EX (SP),IX\\
DDE5\>PUSH IX\\
DDE9\>JP (IX)\\
DDF9\>LD SP,IX\\
DE n\>SBC A,n\\
DF\>RST 18H\\
E0\>RET PO\\
E1\>POP HL\\
E2 n n\>JP PO,nn\\
E3\>EX (SP),HL\\
E4 n n\>CALL PO,nn\\
E5\>PUSH HL\\
E6 n\>AND n\\
E7\>RST 20H\\
E8\>RET PE\\
E9\>JP (HL)\\
EA n n\>JP PE,nn\\
EB\>EX DE,HL\\
EC n n\>CALL PE,nn\\
ED40\>IN B,(C)\\
ED41\>OUT (C),B\\
ED42\>SBC HL,BC\\
ED43 n n\>LD (nn),BC\\
ED44\>NEG\\
ED45\>RETN\\
ED46\>IM 0\\
ED47\>LD I,A\\
ED48\>IN C,(C)\\
ED49\>OUT (C),C\\
ED4A\>ADC HL,BC\\
ED4B n n\>LD BC,(nn)\\
ED4C\>NEG*\\
ED4D\>RETI\\
ED4E\>IM 0*\\
ED4F\>LD R,A\\
ED50\>IN D,(C)\\
ED51\>OUT (C),D\\
ED52\>SBC HL,DE\\
ED53 n n\>LD (nn),DE\\
ED54\>NEG*\\
ED55\>RETN*\\
ED56\>IM 1\\
ED57\>LD A,I\\
ED58\>IN E,(C)\\
ED59\>OUT (C),E\\
ED5A\>ADC HL,DE\\
ED5B n n\>LD DE,(nn)\\
ED5C\>NEG*\\
ED5D\>RETN*\\
ED5E\>IM 2\\
ED5F\>LD A,R\\
ED60\>IN H,(C)\\
ED61\>OUT (C),H\\
ED62\>SBC HL,HL\\
ED63 n n\>LD (nn),HL\\
ED64\>NEG*\\
ED65\>RETN*\\
ED66\>IM 0*\\
ED67\>RRD\\
ED68\>IN L,(C)\\
ED69\>OUT (C),L\\
ED6A\>ADC HL,HL\\
ED6B n n\>LD HL,(nn)\\
ED6C\>NEG*\\
ED6D\>RETN*\\
ED6E\>IM 0*\\
ED6F\>RLD\\
ED70\>IN F,(C)* / IN (C)*\\
ED71\>OUT (C),0*\\
ED72\>SBC HL,SP\\
ED73 n n\>LD (nn),SP\\
ED74\>NEG*\\
ED75\>RETN*\\
ED76\>IM 1*\\
ED78\>IN A,(C)\\
ED79\>OUT (C),A\\
ED7A\>ADC HL,SP\\
ED7B n n\>LD SP,(nn)\\
ED7C\>NEG*\\
ED7D\>RETN*\\
ED7E\>IM 2*\\
EDA0\>LDI\\
EDA1\>CPI\\
EDA2\>INI\\
EDA3\>OUTI\\
EDA8\>LDD\\
EDA9\>CPD\\
EDAA\>IND\\
EDAB\>OUTD\\
EDB0\>LDIR\\
EDB1\>CPIR\\
EDB2\>INIR\\
EDB3\>OTIR\\
EDB8\>LDDR\\
EDB9\>CPDR\\
EDBA\>INDR\\
EDBB\>OTDR\\
EE n\>XOR n\\
EF\>RST 28H\\
F0\>RET P\\
F1\>POP AF\\
F2 n n\>JP P,nn\\
F3\>DI\\
F4 n n\>CALL P,nn\\
F5\>PUSH AF\\
F6 n\>OR n\\
F7\>RST 30H\\
F8\>RET M\\
F9\>LD SP,HL\\
FA n n\>JP M,nn\\
FB\>EI\\
FC n n\>CALL M,nn\\
FD09\>ADD IY,BC\\
FD19\>ADD IY,DE\\
FD21 n n\>LD IY,nn\\
FD22 n n\>LD (nn),IY\\
FD23\>INC IY\\
FD24\>INC IYh*\\
FD25\>DEC IYh*\\
FD26 n\>LD IYh,n*\\
FD29\>ADD IY,IY\\
FD2A n n\>LD IY,(nn)\\
FD2B\>DEC IY\\
FD2C\>INC IYl*\\
FD2D\>DEC IYl*\\
FD2E n\>LD IYl,n*\\
FD34 d\>INC (IY+d)\\
FD35 d\>DEC (IY+d)\\
FD36 d n\>LD (IY+d),n\\
FD39\>ADD IY,SP\\
FD44\>LD B,IYh*\\
FD45\>LD B,IYl*\\
FD46 d\>LD B,(IY+d)\\
FD4C\>LD C,IYh*\\
FD4D\>LD C,IYl*\\
FD4E d\>LD C,(IY+d)\\
FD54\>LD D,IYh*\\
FD55\>LD D,IYl*\\
FD56 d\>LD D,(IY+d)\\
FD5C\>LD E,IYh*\\
FD5D\>LD E,IYl*\\
FD5E d\>LD E,(IY+d)\\
FD60\>LD IYh,B*\\
FD61\>LD IYh,C*\\
FD62\>LD IYh,D*\\
FD63\>LD IYh,E*\\
FD64\>LD IYh,IYh*\\
FD65\>LD IYh,IYl*\\
FD66 d\>LD H,(IY+d)\\
FD67\>LD IYh,A*\\
FD68\>LD IYl,B*\\
FD69\>LD IYl,C*\\
FD6A\>LD IYl,D*\\
FD6B\>LD IYl,E*\\
FD6C\>LD IYl,IYh*\\
FD6D\>LD IYl,IYl*\\
FD6E d\>LD L,(IY+d)\\
FD6F\>LD IYl,A*\\
FD70 d\>LD (IY+d),B\\
FD71 d\>LD (IY+d),C\\
FD72 d\>LD (IY+d),D\\
FD73 d\>LD (IY+d),E\\
FD74 d\>LD (IY+d),H\\
FD75 d\>LD (IY+d),L\\
FD77 d\>LD (IY+d),A\\
FD7C\>LD A,IYh*\\
FD7D\>LD A,IYl*\\
FD7E d\>LD A,(IY+d)\\
FD84\>ADD A,IYh*\\
FD85\>ADD A,IYl*\\
FD86 d\>ADD A,(IY+d)\\
FD8C\>ADC A,IYh*\\
FD8D\>ADC A,IYl*\\
FD8E d\>ADC A,(IY+d)\\
FD94\>SUB IYh*\\
FD95\>SUB IYl*\\
FD96 d\>SUB (IY+d)\\
FD9C\>SBC A,IYh*\\
FD9D\>SBC A,IYl*\\
FD9E d\>SBC A,(IY+d)\\
FDA4\>AND IYh*\\
FDA5\>AND IYl*\\
FDA6 d\>AND (IY+d)\\
FDAC\>XOR IYh*\\
FDAD\>XOR IYl*\\
FDAE d\>XOR (IY+d)\\
FDB4\>OR IYh*\\
FDB5\>OR IYl*\\
FDB6 d\>OR (IY+d)\\
FDBC\>CP IYh*\\
FDBD\>CP IYl*\\
FDBE d\>CP (IY+d)\\
FDCB d 00\>RLC (IY+d),B*\\
FDCB d 01\>RLC (IY+d),C*\\
FDCB d 02\>RLC (IY+d),D*\\
FDCB d 03\>RLC (IY+d),E*\\
FDCB d 04\>RLC (IY+d),H*\\
FDCB d 05\>RLC (IY+d),L*\\
FDCB d 06\>RLC (IY+d)\\
FDCB d 07\>RLC (IY+d),A*\\
FDCB d 08\>RRC (IY+d),B*\\
FDCB d 09\>RRC (IY+d),C*\\
FDCB d 0A\>RRC (IY+d),D*\\
FDCB d 0B\>RRC (IY+d),E*\\
FDCB d 0C\>RRC (IY+d),H*\\
FDCB d 0D\>RRC (IY+d),L*\\
FDCB d 0E\>RRC (IY+d)\\
FDCB d 0F\>RRC (IY+d),A*\\
FDCB d 10\>RL (IY+d),B*\\
FDCB d 11\>RL (IY+d),C*\\
FDCB d 12\>RL (IY+d),D*\\
FDCB d 13\>RL (IY+d),E*\\
FDCB d 14\>RL (IY+d),H*\\
FDCB d 15\>RL (IY+d),L*\\
FDCB d 16\>RL (IY+d)\\
FDCB d 17\>RL (IY+d),A*\\
FDCB d 18\>RR (IY+d),B*\\
FDCB d 19\>RR (IY+d),C*\\
FDCB d 1A\>RR (IY+d),D*\\
FDCB d 1B\>RR (IY+d),E*\\
FDCB d 1C\>RR (IY+d),H*\\
FDCB d 1D\>RR (IY+d),L*\\
FDCB d 1E\>RR (IY+d)\\
FDCB d 1F\>RR (IY+d),A*\\
FDCB d 20\>SLA (IY+d),B*\\
FDCB d 21\>SLA (IY+d),C*\\
FDCB d 22\>SLA (IY+d),D*\\
FDCB d 23\>SLA (IY+d),E*\\
FDCB d 24\>SLA (IY+d),H*\\
FDCB d 25\>SLA (IY+d),L*\\
FDCB d 26\>SLA (IY+d)\\
FDCB d 27\>SLA (IY+d),A*\\
FDCB d 28\>SRA (IY+d),B*\\
FDCB d 29\>SRA (IY+d),C*\\
FDCB d 2A\>SRA (IY+d),D*\\
FDCB d 2B\>SRA (IY+d),E*\\
FDCB d 2C\>SRA (IY+d),H*\\
FDCB d 2D\>SRA (IY+d),L*\\
FDCB d 2E\>SRA (IY+d)\\
FDCB d 2F\>SRA (IY+d),A*\\
FDCB d 30\>SLL (IY+d),B*\\
FDCB d 31\>SLL (IY+d),C*\\
FDCB d 32\>SLL (IY+d),D*\\
FDCB d 33\>SLL (IY+d),E*\\
FDCB d 34\>SLL (IY+d),H*\\
FDCB d 35\>SLL (IY+d),L*\\
FDCB d 36\>SLL (IY+d)*\\
FDCB d 37\>SLL (IY+d),A*\\
FDCB d 38\>SRL (IY+d),B*\\
FDCB d 39\>SRL (IY+d),C*\\
FDCB d 3A\>SRL (IY+d),D*\\
FDCB d 3B\>SRL (IY+d),E*\\
FDCB d 3C\>SRL (IY+d),H*\\
FDCB d 3D\>SRL (IY+d),L*\\
FDCB d 3E\>SRL (IY+d)\\
FDCB d 3F\>SRL (IY+d),A*\\
FDCB d 40\>BIT 0,(IY+d)*\\
FDCB d 41\>BIT 0,(IY+d)*\\
FDCB d 42\>BIT 0,(IY+d)*\\
FDCB d 43\>BIT 0,(IY+d)*\\
FDCB d 44\>BIT 0,(IY+d)*\\
FDCB d 45\>BIT 0,(IY+d)*\\
FDCB d 46\>BIT 0,(IY+d)\\
FDCB d 47\>BIT 0,(IY+d)*\\
FDCB d 48\>BIT 1,(IY+d)*\\
FDCB d 49\>BIT 1,(IY+d)*\\
FDCB d 4A\>BIT 1,(IY+d)*\\
FDCB d 4B\>BIT 1,(IY+d)*\\
FDCB d 4C\>BIT 1,(IY+d)*\\
FDCB d 4D\>BIT 1,(IY+d)*\\
FDCB d 4E\>BIT 1,(IY+d)\\
FDCB d 4F\>BIT 1,(IY+d)*\\
FDCB d 50\>BIT 2,(IY+d)*\\
FDCB d 51\>BIT 2,(IY+d)*\\
FDCB d 52\>BIT 2,(IY+d)*\\
FDCB d 53\>BIT 2,(IY+d)*\\
FDCB d 54\>BIT 2,(IY+d)*\\
FDCB d 55\>BIT 2,(IY+d)*\\
FDCB d 56\>BIT 2,(IY+d)\\
FDCB d 57\>BIT 2,(IY+d)*\\
FDCB d 58\>BIT 3,(IY+d)*\\
FDCB d 59\>BIT 3,(IY+d)*\\
FDCB d 5A\>BIT 3,(IY+d)*\\
FDCB d 5B\>BIT 3,(IY+d)*\\
FDCB d 5C\>BIT 3,(IY+d)*\\
FDCB d 5D\>BIT 3,(IY+d)*\\
FDCB d 5E\>BIT 3,(IY+d)\\
FDCB d 5F\>BIT 3,(IY+d)*\\
FDCB d 60\>BIT 4,(IY+d)*\\
FDCB d 61\>BIT 4,(IY+d)*\\
FDCB d 62\>BIT 4,(IY+d)*\\
FDCB d 63\>BIT 4,(IY+d)*\\
FDCB d 64\>BIT 4,(IY+d)*\\
FDCB d 65\>BIT 4,(IY+d)*\\
FDCB d 66\>BIT 4,(IY+d)\\
FDCB d 67\>BIT 4,(IY+d)*\\
FDCB d 68\>BIT 5,(IY+d)*\\
FDCB d 69\>BIT 5,(IY+d)*\\
FDCB d 6A\>BIT 5,(IY+d)*\\
FDCB d 6B\>BIT 5,(IY+d)*\\
FDCB d 6C\>BIT 5,(IY+d)*\\
FDCB d 6D\>BIT 5,(IY+d)*\\
FDCB d 6E\>BIT 5,(IY+d)\\
FDCB d 6F\>BIT 5,(IY+d)*\\
FDCB d 70\>BIT 6,(IY+d)*\\
FDCB d 71\>BIT 6,(IY+d)*\\
FDCB d 72\>BIT 6,(IY+d)*\\
FDCB d 73\>BIT 6,(IY+d)*\\
FDCB d 74\>BIT 6,(IY+d)*\\
FDCB d 75\>BIT 6,(IY+d)*\\
FDCB d 76\>BIT 6,(IY+d)\\
FDCB d 77\>BIT 6,(IY+d)*\\
FDCB d 78\>BIT 7,(IY+d)*\\
FDCB d 79\>BIT 7,(IY+d)*\\
FDCB d 7A\>BIT 7,(IY+d)*\\
FDCB d 7B\>BIT 7,(IY+d)*\\
FDCB d 7C\>BIT 7,(IY+d)*\\
FDCB d 7D\>BIT 7,(IY+d)*\\
FDCB d 7E\>BIT 7,(IY+d)\\
FDCB d 7F\>BIT 7,(IY+d)*\\
FDCB d 80\>RES 0,(IY+d),B*\\
FDCB d 81\>RES 0,(IY+d),C*\\
FDCB d 82\>RES 0,(IY+d),D*\\
FDCB d 83\>RES 0,(IY+d),E*\\
FDCB d 84\>RES 0,(IY+d),H*\\
FDCB d 85\>RES 0,(IY+d),L*\\
FDCB d 86\>RES 0,(IY+d)\\
FDCB d 87\>RES 0,(IY+d),A*\\
FDCB d 88\>RES 1,(IY+d),B*\\
FDCB d 89\>RES 1,(IY+d),C*\\
FDCB d 8A\>RES 1,(IY+d),D*\\
FDCB d 8B\>RES 1,(IY+d),E*\\
FDCB d 8C\>RES 1,(IY+d),H*\\
FDCB d 8D\>RES 1,(IY+d),L*\\
FDCB d 8E\>RES 1,(IY+d)\\
FDCB d 8F\>RES 1,(IY+d),A*\\
FDCB d 90\>RES 2,(IY+d),B*\\
FDCB d 91\>RES 2,(IY+d),C*\\
FDCB d 92\>RES 2,(IY+d),D*\\
FDCB d 93\>RES 2,(IY+d),E*\\
FDCB d 94\>RES 2,(IY+d),H*\\
FDCB d 95\>RES 2,(IY+d),L*\\
FDCB d 96\>RES 2,(IY+d)\\
FDCB d 97\>RES 2,(IY+d),A*\\
FDCB d 98\>RES 3,(IY+d),B*\\
FDCB d 99\>RES 3,(IY+d),C*\\
FDCB d 9A\>RES 3,(IY+d),D*\\
FDCB d 9B\>RES 3,(IY+d),E*\\
FDCB d 9C\>RES 3,(IY+d),H*\\
FDCB d 9D\>RES 3,(IY+d),L*\\
FDCB d 9E\>RES 3,(IY+d)\\
FDCB d 9F\>RES 3,(IY+d),A*\\
FDCB d A0\>RES 4,(IY+d),B*\\
FDCB d A1\>RES 4,(IY+d),C*\\
FDCB d A2\>RES 4,(IY+d),D*\\
FDCB d A3\>RES 4,(IY+d),E*\\
FDCB d A4\>RES 4,(IY+d),H*\\
FDCB d A5\>RES 4,(IY+d),L*\\
FDCB d A6\>RES 4,(IY+d)\\
FDCB d A7\>RES 4,(IY+d),A*\\
FDCB d A8\>RES 5,(IY+d),B*\\
FDCB d A9\>RES 5,(IY+d),C*\\
FDCB d AA\>RES 5,(IY+d),D*\\
FDCB d AB\>RES 5,(IY+d),E*\\
FDCB d AC\>RES 5,(IY+d),H*\\
FDCB d AD\>RES 5,(IY+d),L*\\
FDCB d AE\>RES 5,(IY+d)\\
FDCB d AF\>RES 5,(IY+d),A*\\
FDCB d B0\>RES 6,(IY+d),B*\\
FDCB d B1\>RES 6,(IY+d),C*\\
FDCB d B2\>RES 6,(IY+d),D*\\
FDCB d B3\>RES 6,(IY+d),E*\\
FDCB d B4\>RES 6,(IY+d),H*\\
FDCB d B5\>RES 6,(IY+d),L*\\
FDCB d B6\>RES 6,(IY+d)\\
FDCB d B7\>RES 6,(IY+d),A*\\
FDCB d B8\>RES 7,(IY+d),B*\\
FDCB d B9\>RES 7,(IY+d),C*\\
FDCB d BA\>RES 7,(IY+d),D*\\
FDCB d BB\>RES 7,(IY+d),E*\\
FDCB d BC\>RES 7,(IY+d),H*\\
FDCB d BD\>RES 7,(IY+d),L*\\
FDCB d BE\>RES 7,(IY+d)\\
FDCB d BF\>RES 7,(IY+d),A*\\
FDCB d C0\>SET 0,(IY+d),B*\\
FDCB d C1\>SET 0,(IY+d),C*\\
FDCB d C2\>SET 0,(IY+d),D*\\
FDCB d C3\>SET 0,(IY+d),E*\\
FDCB d C4\>SET 0,(IY+d),H*\\
FDCB d C5\>SET 0,(IY+d),L*\\
FDCB d C6\>SET 0,(IY+d)\\
FDCB d C7\>SET 0,(IY+d),A*\\
FDCB d C8\>SET 1,(IY+d),B*\\
FDCB d C9\>SET 1,(IY+d),C*\\
FDCB d CA\>SET 1,(IY+d),D*\\
FDCB d CB\>SET 1,(IY+d),E*\\
FDCB d CC\>SET 1,(IY+d),H*\\
FDCB d CD\>SET 1,(IY+d),L*\\
FDCB d CE\>SET 1,(IY+d)\\
FDCB d CF\>SET 1,(IY+d),A*\\
FDCB d D0\>SET 2,(IY+d),B*\\
FDCB d D1\>SET 2,(IY+d),C*\\
FDCB d D2\>SET 2,(IY+d),D*\\
FDCB d D3\>SET 2,(IY+d),E*\\
FDCB d D4\>SET 2,(IY+d),H*\\
FDCB d D5\>SET 2,(IY+d),L*\\
FDCB d D6\>SET 2,(IY+d)\\
FDCB d D7\>SET 2,(IY+d),A*\\
FDCB d D8\>SET 3,(IY+d),B*\\
FDCB d D9\>SET 3,(IY+d),C*\\
FDCB d DA\>SET 3,(IY+d),D*\\
FDCB d DB\>SET 3,(IY+d),E*\\
FDCB d DC\>SET 3,(IY+d),H*\\
FDCB d DD\>SET 3,(IY+d),L*\\
FDCB d DE\>SET 3,(IY+d)\\
FDCB d DF\>SET 3,(IY+d),A*\\
FDCB d E0\>SET 4,(IY+d),B*\\
FDCB d E1\>SET 4,(IY+d),C*\\
FDCB d E2\>SET 4,(IY+d),D*\\
FDCB d E3\>SET 4,(IY+d),E*\\
FDCB d E4\>SET 4,(IY+d),H*\\
FDCB d E5\>SET 4,(IY+d),L*\\
FDCB d E6\>SET 4,(IY+d)\\
FDCB d E7\>SET 4,(IY+d),A*\\
FDCB d E8\>SET 5,(IY+d),B*\\
FDCB d E9\>SET 5,(IY+d),C*\\
FDCB d EA\>SET 5,(IY+d),D*\\
FDCB d EB\>SET 5,(IY+d),E*\\
FDCB d EC\>SET 5,(IY+d),H*\\
FDCB d ED\>SET 5,(IY+d),L*\\
FDCB d EE\>SET 5,(IY+d)\\
FDCB d EF\>SET 5,(IY+d),A*\\
FDCB d F0\>SET 6,(IY+d),B*\\
FDCB d F1\>SET 6,(IY+d),C*\\
FDCB d F2\>SET 6,(IY+d),D*\\
FDCB d F3\>SET 6,(IY+d),E*\\
FDCB d F4\>SET 6,(IY+d),H*\\
FDCB d F5\>SET 6,(IY+d),L*\\
FDCB d F6\>SET 6,(IY+d)\\
FDCB d F7\>SET 6,(IY+d),A*\\
FDCB d F8\>SET 7,(IY+d),B*\\
FDCB d F9\>SET 7,(IY+d),C*\\
FDCB d FA\>SET 7,(IY+d),D*\\
FDCB d FB\>SET 7,(IY+d),E*\\
FDCB d FC\>SET 7,(IY+d),H*\\
FDCB d FD\>SET 7,(IY+d),L*\\
FDCB d FE\>SET 7,(IY+d)\\
FDCB d FF\>SET 7,(IY+d),A*\\
FDE1\>POP IY\\
FDE3\>EX (SP),IY\\
FDE5\>PUSH IY\\
FDE9\>JP (IY)\\
FDF9\>LD SP,IY\\
FE n\>CP n\\
FF\>RST 38H
\end{tabbing}
}
\end{multicols}
\normalsize

\chapter{Instructions Sorted by MNemonic}

Any instruction marked with * is undocumented.
% TODO: and any instruction not listed has no effect? (same behaviour as an equal amount of NOP instructions)
\begin{multicols}{3}

\scriptsize
{\tt
\begin{tabbing}
ADC A,(HL){\qquad}{\qquad}{\quad}\=8E\\
ADC A,(IX+d)\>DD8E d\\
ADC A,(IY+d)\>FD8E d\\
ADC A,A\>8F\\
ADC A,B\>88\\
ADC A,C\>89\\
ADC A,D\>8A\\
ADC A,E\>8B\\
ADC A,H\>8C\\
ADC A,IXh*\>DD8C\\
ADC A,IXl*\>DD8D\\
ADC A,IYh*\>FD8C\\
ADC A,IYl*\>FD8D\\
ADC A,L\>8D\\
ADC A,n\>CE n\\
ADC HL,BC\>ED4A\\
ADC HL,DE\>ED5A\\
ADC HL,HL\>ED6A\\
ADC HL,SP\>ED7A\\
ADD A,(HL)\>86\\
ADD A,(IX+d)\>DD86 d\\
ADD A,(IY+d)\>FD86 d\\
ADD A,A\>87\\
ADD A,B\>80\\
ADD A,C\>81\\
ADD A,D\>82\\
ADD A,E\>83\\
ADD A,H\>84\\
ADD A,IXh*\>DD84\\
ADD A,IXl*\>DD85\\
ADD A,IYh*\>FD84\\
ADD A,IYl*\>FD85\\
ADD A,L\>85\\
ADD A,n\>C6 n\\
ADD HL,BC\>09\\
ADD HL,DE\>19\\
ADD HL,HL\>29\\
ADD HL,SP\>39\\
ADD IX,BC\>DD09\\
ADD IX,DE\>DD19\\
ADD IX,IX\>DD29\\
ADD IX,SP\>DD39\\
ADD IY,BC\>FD09\\
ADD IY,DE\>FD19\\
ADD IY,IY\>FD29\\
ADD IY,SP\>FD39\\
AND (HL)\>A6\\
AND (IX+d)\>DDA6 d\\
AND (IY+d)\>FDA6 d\\
AND A\>A7\\
AND B\>A0\\
AND C\>A1\\
AND D\>A2\\
AND E\>A3\\
AND H\>A4\\
AND IXh*\>DDA4\\
AND IXl*\>DDA5\\
AND IYh*\>FDA4\\
AND IYl*\>FDA5\\
AND L\>A5\\
AND n\>E6 n\\
BIT 0,(HL)\>CB46\\
BIT 0,(IX+d)*\>DDCB d 40\\
BIT 0,(IX+d)*\>DDCB d 41\\
BIT 0,(IX+d)*\>DDCB d 42\\
BIT 0,(IX+d)*\>DDCB d 43\\
BIT 0,(IX+d)*\>DDCB d 44\\
BIT 0,(IX+d)*\>DDCB d 45\\
BIT 0,(IX+d)*\>DDCB d 47\\
BIT 0,(IX+d)\>DDCB d 46\\
BIT 0,(IY+d)*\>FDCB d 40\\
BIT 0,(IY+d)*\>FDCB d 41\\
BIT 0,(IY+d)*\>FDCB d 42\\
BIT 0,(IY+d)*\>FDCB d 43\\
BIT 0,(IY+d)*\>FDCB d 44\\
BIT 0,(IY+d)*\>FDCB d 45\\
BIT 0,(IY+d)*\>FDCB d 47\\
BIT 0,(IY+d)\>FDCB d 46\\
BIT 0,A\>CB47\\
BIT 0,B\>CB40\\
BIT 0,C\>CB41\\
BIT 0,D\>CB42\\
BIT 0,E\>CB43\\
BIT 0,H\>CB44\\
BIT 0,L\>CB45\\
BIT 1,(HL)\>CB4E\\
BIT 1,(IX+d)*\>DDCB d 48\\
BIT 1,(IX+d)*\>DDCB d 49\\
BIT 1,(IX+d)*\>DDCB d 4A\\
BIT 1,(IX+d)*\>DDCB d 4B\\
BIT 1,(IX+d)*\>DDCB d 4C\\
BIT 1,(IX+d)*\>DDCB d 4D\\
BIT 1,(IX+d)*\>DDCB d 4F\\
BIT 1,(IX+d)\>DDCB d 4E\\
BIT 1,(IY+d)*\>FDCB d 48\\
BIT 1,(IY+d)*\>FDCB d 49\\
BIT 1,(IY+d)*\>FDCB d 4A\\
BIT 1,(IY+d)*\>FDCB d 4B\\
BIT 1,(IY+d)*\>FDCB d 4C\\
BIT 1,(IY+d)*\>FDCB d 4D\\
BIT 1,(IY+d)*\>FDCB d 4F\\
BIT 1,(IY+d)\>FDCB d 4E\\
BIT 1,A\>CB4F\\
BIT 1,B\>CB48\\
BIT 1,C\>CB49\\
BIT 1,D\>CB4A\\
BIT 1,E\>CB4B\\
BIT 1,H\>CB4C\\
BIT 1,L\>CB4D\\
BIT 2,(HL)\>CB56\\
BIT 2,(IX+d)*\>DDCB d 50\\
BIT 2,(IX+d)*\>DDCB d 51\\
BIT 2,(IX+d)*\>DDCB d 52\\
BIT 2,(IX+d)*\>DDCB d 53\\
BIT 2,(IX+d)*\>DDCB d 54\\
BIT 2,(IX+d)*\>DDCB d 55\\
BIT 2,(IX+d)*\>DDCB d 57\\
BIT 2,(IX+d)\>DDCB d 56\\
BIT 2,(IY+d)*\>FDCB d 50\\
BIT 2,(IY+d)*\>FDCB d 51\\
BIT 2,(IY+d)*\>FDCB d 52\\
BIT 2,(IY+d)*\>FDCB d 53\\
BIT 2,(IY+d)*\>FDCB d 54\\
BIT 2,(IY+d)*\>FDCB d 55\\
BIT 2,(IY+d)*\>FDCB d 57\\
BIT 2,(IY+d)\>FDCB d 56\\
BIT 2,A\>CB57\\
BIT 2,B\>CB50\\
BIT 2,C\>CB51\\
BIT 2,D\>CB52\\
BIT 2,E\>CB53\\
BIT 2,H\>CB54\\
BIT 2,L\>CB55\\
BIT 3,(HL)\>CB5E\\
BIT 3,(IX+d)*\>DDCB d 58\\
BIT 3,(IX+d)*\>DDCB d 59\\
BIT 3,(IX+d)*\>DDCB d 5A\\
BIT 3,(IX+d)*\>DDCB d 5B\\
BIT 3,(IX+d)*\>DDCB d 5C\\
BIT 3,(IX+d)*\>DDCB d 5D\\
BIT 3,(IX+d)*\>DDCB d 5F\\
BIT 3,(IX+d)\>DDCB d 5E\\
BIT 3,(IY+d)*\>FDCB d 58\\
BIT 3,(IY+d)*\>FDCB d 59\\
BIT 3,(IY+d)*\>FDCB d 5A\\
BIT 3,(IY+d)*\>FDCB d 5B\\
BIT 3,(IY+d)*\>FDCB d 5C\\
BIT 3,(IY+d)*\>FDCB d 5D\\
BIT 3,(IY+d)*\>FDCB d 5F\\
BIT 3,(IY+d)\>FDCB d 5E\\
BIT 3,A\>CB5F\\
BIT 3,B\>CB58\\
BIT 3,C\>CB59\\
BIT 3,D\>CB5A\\
BIT 3,E\>CB5B\\
BIT 3,H\>CB5C\\
BIT 3,L\>CB5D\\
BIT 4,(HL)\>CB66\\
BIT 4,(IX+d)*\>DDCB d 60\\
BIT 4,(IX+d)*\>DDCB d 61\\
BIT 4,(IX+d)*\>DDCB d 62\\
BIT 4,(IX+d)*\>DDCB d 63\\
BIT 4,(IX+d)*\>DDCB d 64\\
BIT 4,(IX+d)*\>DDCB d 65\\
BIT 4,(IX+d)*\>DDCB d 67\\
BIT 4,(IX+d)\>DDCB d 66\\
BIT 4,(IY+d)*\>FDCB d 60\\
BIT 4,(IY+d)*\>FDCB d 61\\
BIT 4,(IY+d)*\>FDCB d 62\\
BIT 4,(IY+d)*\>FDCB d 63\\
BIT 4,(IY+d)*\>FDCB d 64\\
BIT 4,(IY+d)*\>FDCB d 65\\
BIT 4,(IY+d)*\>FDCB d 67\\
BIT 4,(IY+d)\>FDCB d 66\\
BIT 4,A\>CB67\\
BIT 4,B\>CB60\\
BIT 4,C\>CB61\\
BIT 4,D\>CB62\\
BIT 4,E\>CB63\\
BIT 4,H\>CB64\\
BIT 4,L\>CB65\\
BIT 5,(HL)\>CB6E\\
BIT 5,(IX+d)*\>DDCB d 68\\
BIT 5,(IX+d)*\>DDCB d 69\\
BIT 5,(IX+d)*\>DDCB d 6A\\
BIT 5,(IX+d)*\>DDCB d 6B\\
BIT 5,(IX+d)*\>DDCB d 6C\\
BIT 5,(IX+d)*\>DDCB d 6D\\
BIT 5,(IX+d)*\>DDCB d 6F\\
BIT 5,(IX+d)\>DDCB d 6E\\
BIT 5,(IY+d)*\>FDCB d 68\\
BIT 5,(IY+d)*\>FDCB d 69\\
BIT 5,(IY+d)*\>FDCB d 6A\\
BIT 5,(IY+d)*\>FDCB d 6B\\
BIT 5,(IY+d)*\>FDCB d 6C\\
BIT 5,(IY+d)*\>FDCB d 6D\\
BIT 5,(IY+d)*\>FDCB d 6F\\
BIT 5,(IY+d)\>FDCB d 6E\\
BIT 5,A\>CB6F\\
BIT 5,B\>CB68\\
BIT 5,C\>CB69\\
BIT 5,D\>CB6A\\
BIT 5,E\>CB6B\\
BIT 5,H\>CB6C\\
BIT 5,L\>CB6D\\
BIT 6,(HL)\>CB76\\
BIT 6,(IX+d)*\>DDCB d 70\\
BIT 6,(IX+d)*\>DDCB d 71\\
BIT 6,(IX+d)*\>DDCB d 72\\
BIT 6,(IX+d)*\>DDCB d 73\\
BIT 6,(IX+d)*\>DDCB d 74\\
BIT 6,(IX+d)*\>DDCB d 75\\
BIT 6,(IX+d)*\>DDCB d 77\\
BIT 6,(IX+d)\>DDCB d 76\\
BIT 6,(IY+d)*\>FDCB d 70\\
BIT 6,(IY+d)*\>FDCB d 71\\
BIT 6,(IY+d)*\>FDCB d 72\\
BIT 6,(IY+d)*\>FDCB d 73\\
BIT 6,(IY+d)*\>FDCB d 74\\
BIT 6,(IY+d)*\>FDCB d 75\\
BIT 6,(IY+d)*\>FDCB d 77\\
BIT 6,(IY+d)\>FDCB d 76\\
BIT 6,A\>CB77\\
BIT 6,B\>CB70\\
BIT 6,C\>CB71\\
BIT 6,D\>CB72\\
BIT 6,E\>CB73\\
BIT 6,H\>CB74\\
BIT 6,L\>CB75\\
BIT 7,(HL)\>CB7E\\
BIT 7,(IX+d)*\>DDCB d 78\\
BIT 7,(IX+d)*\>DDCB d 79\\
BIT 7,(IX+d)*\>DDCB d 7A\\
BIT 7,(IX+d)*\>DDCB d 7B\\
BIT 7,(IX+d)*\>DDCB d 7C\\
BIT 7,(IX+d)*\>DDCB d 7D\\
BIT 7,(IX+d)*\>DDCB d 7F\\
BIT 7,(IX+d)\>DDCB d 7E\\
BIT 7,(IY+d)*\>FDCB d 78\\
BIT 7,(IY+d)*\>FDCB d 79\\
BIT 7,(IY+d)*\>FDCB d 7A\\
BIT 7,(IY+d)*\>FDCB d 7B\\
BIT 7,(IY+d)*\>FDCB d 7C\\
BIT 7,(IY+d)*\>FDCB d 7D\\
BIT 7,(IY+d)*\>FDCB d 7F\\
BIT 7,(IY+d)\>FDCB d 7E\\
BIT 7,A\>CB7F\\
BIT 7,B\>CB78\\
BIT 7,C\>CB79\\
BIT 7,D\>CB7A\\
BIT 7,E\>CB7B\\
BIT 7,H\>CB7C\\
BIT 7,L\>CB7D\\
CALL nn\>CD n n\\
CALL C,nn\>DC n n\\
CALL M,nn\>FC n n\\
CALL NC,nn\>D4 n n\\
CALL NZ,nn\>C4 n n\\
CALL P,nn\>F4 n n\\
CALL PE,nn\>EC n n\\
CALL PO,nn\>E4 n n\\
CALL Z,nn\>CC n n\\
CCF\>3F\\
CP (HL)\>BE\\
CP (IX+d)\>DDBE d\\
CP (IY+d)\>FDBE d\\
CP A\>BF\\
CP B\>B8\\
CP C\>B9\\
CP D\>BA\\
CP E\>BB\\
CP H\>BC\\
CP IXh*\>DDBC\\
CP IXl*\>DDBD\\
CP IYh*\>FDBC\\
CP IYl*\>FDBD\\
CP L\>BD\\
CP n\>FE n\\
CPDR\>EDB9\\
CPD\>EDA9\\
CPIR\>EDB1\\
CPI\>EDA1\\
CPL\>2F\\
DAA\>27\\
DEC (HL)\>35\\
DEC (IX+d)\>DD35 d\\
DEC (IY+d)\>FD35 d\\
DEC A\>3D\\
DEC BC\>0B\\
DEC B\>05\\
DEC C\>0D\\
DEC DE\>1B\\
DEC D\>15\\
DEC E\>1D\\
DEC HL\>2B\\
DEC H\>25\\
DEC IX\>DD2B\\
DEC IXh*\>DD25\\
DEC IXl*\>DD2D\\
DEC IY\>FD2B\\
DEC IYh*\>FD25\\
DEC IYl*\>FD2D\\
DEC L\>2D\\
DEC SP\>3B\\
DI\>F3\\
DJNZ (PC+e)\>10 e\\
EI\>FB\\
EX (SP),HL\>E3\\
EX (SP),IX\>DDE3\\
EX (SP),IY\>FDE3\\
EX AF,AF'\>08\\
EX DE,HL\>EB\\
EXX\>D9\\
HALT\>76\\
IM 0*\>ED4E\\
IM 0*\>ED66\\
IM 0*\>ED6E\\
IM 0\>ED46\\
IM 1*\>ED76\\
IM 1\>ED56\\
IM 2*\>ED7E\\
IM 2\>ED5E\\
IN A,(C)\>ED78\\
IN A,(n)\>DB n\\
IN B,(C)\>ED40\\
IN C,(C)\>ED48\\
IN D,(C)\>ED50\\
IN E,(C)\>ED58\\
IN F,(C)* / IN (C)*\>ED70\\
IN H,(C)\>ED60\\
IN L,(C)\>ED68\\
INC (HL)\>34\\
INC (IX+d)\>DD34 d\\
INC (IY+d)\>FD34 d\\
INC A\>3C\\
INC BC\>03\\
INC B\>04\\
INC C\>0C\\
INC DE\>13\\
INC D\>14\\
INC E\>1C\\
INC HL\>23\\
INC H\>24\\
INC IX\>DD23\\
INC IXh*\>DD24\\
INC IXl*\>DD2C\\
INC IY\>FD23\\
INC IYh*\>FD24\\
INC IYl*\>FD2C\\
INC L\>2C\\
INC SP\>33\\
INDR\>EDBA\\
IND\>EDAA\\
INIR\>EDB2\\
INI\>EDA2\\
JP (HL)\>E9\\
JP (IX)\>DDE9\\
JP (IY)\>FDE9\\
JP nn\>C3 n n\\
JP C,nn\>DA n n\\
JP M,nn\>FA n n\\
JP NC,nn\>D2 n n\\
JP NZ,nn\>C2 n n\\
JP P,nn\>F2 n n\\
JP PE,nn\>EA n n\\
JP PO,nn\>E2 n n\\
JP Z,nn\>CA n n\\
JR e\>18 e\\
JR C,e\>38 e\\
JR NC,e\>30 e\\
JR NZ,e\>20 e\\
JR Z,e\>28 e\\
LD (BC),A\>02\\
LD (DE),A\>12\\
LD (HL),A\>77\\
LD (HL),B\>70\\
LD (HL),C\>71\\
LD (HL),D\>72\\
LD (HL),E\>73\\
LD (HL),H\>74\\
LD (HL),L\>75\\
LD (HL),n\>36 n\\
LD (IX+d),A\>DD77 d\\
LD (IX+d),B\>DD70 d\\
LD (IX+d),C\>DD71 d\\
LD (IX+d),D\>DD72 d\\
LD (IX+d),E\>DD73 d\\
LD (IX+d),H\>DD74 d\\
LD (IX+d),L\>DD75 d\\
LD (IX+d),n\>DD36 d n\\
LD (IY+d),A\>FD77 d\\
LD (IY+d),B\>FD70 d\\
LD (IY+d),C\>FD71 d\\
LD (IY+d),D\>FD72 d\\
LD (IY+d),E\>FD73 d\\
LD (IY+d),H\>FD74 d\\
LD (IY+d),L\>FD75 d\\
LD (IY+d),n\>FD36 d n\\
LD (nn),A\>32 n n\\
LD (nn),BC\>ED43 n n\\
LD (nn),DE\>ED53 n n\\
LD (nn),HL\>22 n n\\
LD (nn),HL\>ED63 n n\\
LD (nn),IX\>DD22 n n\\
LD (nn),IY\>FD22 n n\\
LD (nn),SP\>ED73 n n\\
LD A,(BC)\>0A\\
LD A,(DE)\>1A\\
LD A,(HL)\>7E\\
LD A,(IX+d)\>DD7E d\\
LD A,(IY+d)\>FD7E d\\
LD A,(nn)\>3A n n\\
LD A,A\>7F\\
LD A,B\>78\\
LD A,C\>79\\
LD A,D\>7A\\
LD A,E\>7B\\
LD A,H\>7C\\
LD A,IXh*\>DD7C\\
LD A,IXl*\>DD7D\\
LD A,IYh*\>FD7C\\
LD A,IYl*\>FD7D\\
LD A,I\>ED57\\
LD A,L\>7D\\
LD A,R\>ED5F\\
LD A,n\>3E n\\
LD B,(HL)\>46\\
LD B,(IX+d)\>DD46 d\\
LD B,(IY+d)\>FD46 d\\
LD B,A\>47\\
LD B,B\>40\\
LD B,C\>41\\
LD B,D\>42\\
LD B,E\>43\\
LD B,H\>44\\
LD B,IXh*\>DD44\\
LD B,IXl*\>DD45\\
LD B,IYh*\>FD44\\
LD B,IYl*\>FD45\\
LD B,L\>45\\
LD B,n\>06 n\\
LD BC,(nn)\>ED4B n n\\
LD BC,nn\>01 n n\\
LD C,(HL)\>4E\\
LD C,(IX+d)\>DD4E d\\
LD C,(IY+d)\>FD4E d\\
LD C,A\>4F\\
LD C,B\>48\\
LD C,C\>49\\
LD C,D\>4A\\
LD C,E\>4B\\
LD C,H\>4C\\
LD C,IXh*\>DD4C\\
LD C,IXl*\>DD4D\\
LD C,IYh*\>FD4C\\
LD C,IYl*\>FD4D\\
LD C,L\>4D\\
LD C,n\>0E n\\
LD D,(HL)\>56\\
LD D,(IX+d)\>DD56 d\\
LD D,(IY+d)\>FD56 d\\
LD D,A\>57\\
LD D,B\>50\\
LD D,C\>51\\
LD D,D\>52\\
LD D,E\>53\\
LD D,H\>54\\
LD D,IXh*\>DD54\\
LD D,IXl*\>DD55\\
LD D,IYh*\>FD54\\
LD D,IYl*\>FD55\\
LD D,L\>55\\
LD D,n\>16 n\\
LD DE,(nn)\>ED5B n n\\
LD DE,nn\>11 n n\\
LD E,(HL)\>5E\\
LD E,(IX+d)\>DD5E d\\
LD E,(IY+d)\>FD5E d\\
LD E,A\>5F\\
LD E,B\>58\\
LD E,C\>59\\
LD E,D\>5A\\
LD E,E\>5B\\
LD E,H\>5C\\
LD E,IXh*\>DD5C\\
LD E,IXl*\>DD5D\\
LD E,IYh*\>FD5C\\
LD E,IYl*\>FD5D\\
LD E,L\>5D\\
LD E,n\>1E n\\
LD H,(HL)\>66\\
LD H,(IX+d)\>DD66 d\\
LD H,(IY+d)\>FD66 d\\
LD H,A\>67\\
LD H,B\>60\\
LD H,C\>61\\
LD H,D\>62\\
LD H,E\>63\\
LD H,H\>64\\
LD H,L\>65\\
LD H,n\>26 n\\
LD HL,(nn)\>2A n n\\
LD HL,(nn)\>ED6B n n\\
LD HL,nn\>21 n n\\
LD I,A\>ED47\\
LD IX,(nn)\>DD2A n n\\
LD IX,nn\>DD21 n n\\
LD IXh,A*\>DD67\\
LD IXh,B*\>DD60\\
LD IXh,C*\>DD61\\
LD IXh,D*\>DD62\\
LD IXh,E*\>DD63\\
LD IXh,IXh*\>DD64\\
LD IXh,IXl*\>DD65\\
LD IXh,n*\>DD26 n \\
LD IXl,A*\>DD6F\\
LD IXl,B*\>DD68\\
LD IXl,C*\>DD69\\
LD IXl,D*\>DD6A\\
LD IXl,E*\>DD6B\\
LD IXl,IXh*\>DD6C\\
LD IXl,IXl*\>DD6D\\
LD IXl,n*\>DD2E n\\
LD IY,(nn)\>FD2A n n\\
LD IY,nn\>FD21 n n\\
LD IYh,A*\>FD67\\
LD IYh,B*\>FD60\\
LD IYh,C*\>FD61\\
LD IYh,D*\>FD62\\
LD IYh,E*\>FD63\\
LD IYh,IYh*\>FD64\\
LD IYh,IYl*\>FD65\\
LD IYh,n*\>FD26 n\\
LD IYl,A*\>FD6F\\
LD IYl,B*\>FD68\\
LD IYl,C*\>FD69\\
LD IYl,D*\>FD6A\\
LD IYl,E*\>FD6B\\
LD IYl,IYh*\>FD6C\\
LD IYl,IYl*\>FD6D\\
LD IYl,n*\>FD2E n\\
LD L,(HL)\>6E\\
LD L,(IX+d)\>DD6E d\\
LD L,(IY+d)\>FD6E d\\
LD L,A\>6F\\
LD L,B\>68\\
LD L,C\>69\\
LD L,D\>6A\\
LD L,E\>6B\\
LD L,H\>6C\\
LD L,L\>6D\\
LD L,n\>2E n\\
LD R,A\>ED4F\\
LD SP,(nn)\>ED7B n n\\
LD SP,HL\>F9\\
LD SP,IX\>DDF9\\
LD SP,IY\>FDF9\\
LD SP,nn\>31 n n\\
LDDR\>EDB8\\
LDD\>EDA8\\
LDIR\>EDB0\\
LDI\>EDA0\\
NEG*\>ED4C\\
NEG*\>ED54\\
NEG*\>ED5C\\
NEG*\>ED64\\
NEG*\>ED6C\\
NEG*\>ED74\\
NEG*\>ED7C\\
NEG\>ED44\\
NOP\>00\\
OR (HL)\>B6\\
OR (IX+d)\>DDB6 d\\
OR (IY+d)\>FDB6 d\\
OR A\>B7\\
OR B\>B0\\
OR C\>B1\\
OR D\>B2\\
OR E\>B3\\
OR H\>B4\\
OR IXh*\>DDB4\\
OR IXl*\>DDB5\\
OR IYh*\>FDB4\\
OR IYl*\>FDB5\\
OR L\>B5\\
OR n\>F6 n\\
OTDR\>EDBB\\
OTIR\>EDB3\\
OUT (C),0*\>ED71\\
OUT (C),A\>ED79\\
OUT (C),B\>ED41\\
OUT (C),C\>ED49\\
OUT (C),D\>ED51\\
OUT (C),E\>ED59\\
OUT (C),H\>ED61\\
OUT (C),L\>ED69\\
OUT (n),A\>D3 n\\
OUTD\>EDAB\\
OUTI\>EDA3\\
POP AF\>F1\\
POP BC\>C1\\
POP DE\>D1\\
POP HL\>E1\\
POP IX\>DDE1\\
POP IY\>FDE1\\
PUSH AF\>F5\\
PUSH BC\>C5\\
PUSH DE\>D5\\
PUSH HL\>E5\\
PUSH IX\>DDE5\\
PUSH IY\>FDE5\\
RES 0,(HL)\>CB86\\
RES 0,(IX+d),A*\>DDCB d 87\\
RES 0,(IX+d),B*\>DDCB d 80\\
RES 0,(IX+d),C*\>DDCB d 81\\
RES 0,(IX+d),D*\>DDCB d 82\\
RES 0,(IX+d),E*\>DDCB d 83\\
RES 0,(IX+d),H*\>DDCB d 84\\
RES 0,(IX+d),L*\>DDCB d 85\\
RES 0,(IX+d)\>DDCB d 86\\
RES 0,(IY+d),A*\>FDCB d 87\\
RES 0,(IY+d),B*\>FDCB d 80\\
RES 0,(IY+d),C*\>FDCB d 81\\
RES 0,(IY+d),D*\>FDCB d 82\\
RES 0,(IY+d),E*\>FDCB d 83\\
RES 0,(IY+d),H*\>FDCB d 84\\
RES 0,(IY+d),L*\>FDCB d 85\\
RES 0,(IY+d)\>FDCB d 86\\
RES 0,A\>CB87\\
RES 0,B\>CB80\\
RES 0,C\>CB81\\
RES 0,D\>CB82\\
RES 0,E\>CB83\\
RES 0,H\>CB84\\
RES 0,L\>CB85\\
RES 1,(HL)\>CB8E\\
RES 1,(IX+d),A*\>DDCB d 8F\\
RES 1,(IX+d),B*\>DDCB d 88\\
RES 1,(IX+d),C*\>DDCB d 89\\
RES 1,(IX+d),D*\>DDCB d 8A\\
RES 1,(IX+d),E*\>DDCB d 8B\\
RES 1,(IX+d),H*\>DDCB d 8C\\
RES 1,(IX+d),L*\>DDCB d 8D\\
RES 1,(IX+d)\>DDCB d 8E\\
RES 1,(IY+d),A*\>FDCB d 8F\\
RES 1,(IY+d),B*\>FDCB d 88\\
RES 1,(IY+d),C*\>FDCB d 89\\
RES 1,(IY+d),D*\>FDCB d 8A\\
RES 1,(IY+d),E*\>FDCB d 8B\\
RES 1,(IY+d),H*\>FDCB d 8C\\
RES 1,(IY+d),L*\>FDCB d 8D\\
RES 1,(IY+d)\>FDCB d 8E\\
RES 1,A\>CB8F\\
RES 1,B\>CB88\\
RES 1,C\>CB89\\
RES 1,D\>CB8A\\
RES 1,E\>CB8B\\
RES 1,H\>CB8C\\
RES 1,L\>CB8D\\
RES 2,(HL)\>CB96\\
RES 2,(IX+d),A*\>DDCB d 97\\
RES 2,(IX+d),B*\>DDCB d 90\\
RES 2,(IX+d),C*\>DDCB d 91\\
RES 2,(IX+d),D*\>DDCB d 92\\
RES 2,(IX+d),E*\>DDCB d 93\\
RES 2,(IX+d),H*\>DDCB d 94\\
RES 2,(IX+d),L*\>DDCB d 95\\
RES 2,(IX+d)\>DDCB d 96\\
RES 2,(IY+d),A*\>FDCB d 97\\
RES 2,(IY+d),B*\>FDCB d 90\\
RES 2,(IY+d),C*\>FDCB d 91\\
RES 2,(IY+d),D*\>FDCB d 92\\
RES 2,(IY+d),E*\>FDCB d 93\\
RES 2,(IY+d),H*\>FDCB d 94\\
RES 2,(IY+d),L*\>FDCB d 95\\
RES 2,(IY+d)\>FDCB d 96\\
RES 2,A\>CB97\\
RES 2,B\>CB90\\
RES 2,C\>CB91\\
RES 2,D\>CB92\\
RES 2,E\>CB93\\
RES 2,H\>CB94\\
RES 2,L\>CB95\\
RES 3,(HL)\>CB9E\\
RES 3,(IX+d),A*\>DDCB d 9F\\
RES 3,(IX+d),B*\>DDCB d 98\\
RES 3,(IX+d),C*\>DDCB d 99\\
RES 3,(IX+d),D*\>DDCB d 9A\\
RES 3,(IX+d),E*\>DDCB d 9B\\
RES 3,(IX+d),H*\>DDCB d 9C\\
RES 3,(IX+d),L*\>DDCB d 9D\\
RES 3,(IX+d)\>DDCB d 9E\\
RES 3,(IY+d),A*\>FDCB d 9F\\
RES 3,(IY+d),B*\>FDCB d 98\\
RES 3,(IY+d),C*\>FDCB d 99\\
RES 3,(IY+d),D*\>FDCB d 9A\\
RES 3,(IY+d),E*\>FDCB d 9B\\
RES 3,(IY+d),H*\>FDCB d 9C\\
RES 3,(IY+d),L*\>FDCB d 9D\\
RES 3,(IY+d)\>FDCB d 9E\\
RES 3,A\>CB9F\\
RES 3,B\>CB98\\
RES 3,C\>CB99\\
RES 3,D\>CB9A\\
RES 3,E\>CB9B\\
RES 3,H\>CB9C\\
RES 3,L\>CB9D\\
RES 4,(HL)\>CBA6\\
RES 4,(IX+d),A*\>DDCB d A7\\
RES 4,(IX+d),B*\>DDCB d A0\\
RES 4,(IX+d),C*\>DDCB d A1\\
RES 4,(IX+d),D*\>DDCB d A2\\
RES 4,(IX+d),E*\>DDCB d A3\\
RES 4,(IX+d),H*\>DDCB d A4\\
RES 4,(IX+d),L*\>DDCB d A5\\
RES 4,(IX+d)\>DDCB d A6\\
RES 4,(IY+d),A*\>FDCB d A7\\
RES 4,(IY+d),B*\>FDCB d A0\\
RES 4,(IY+d),C*\>FDCB d A1\\
RES 4,(IY+d),D*\>FDCB d A2\\
RES 4,(IY+d),E*\>FDCB d A3\\
RES 4,(IY+d),H*\>FDCB d A4\\
RES 4,(IY+d),L*\>FDCB d A5\\
RES 4,(IY+d)\>FDCB d A6\\
RES 4,A\>CBA7\\
RES 4,B\>CBA0\\
RES 4,C\>CBA1\\
RES 4,D\>CBA2\\
RES 4,E\>CBA3\\
RES 4,H\>CBA4\\
RES 4,L\>CBA5\\
RES 5,(HL)\>CBAE\\
RES 5,(IX+d),A*\>DDCB d AF\\
RES 5,(IX+d),B*\>DDCB d A8\\
RES 5,(IX+d),C*\>DDCB d A9\\
RES 5,(IX+d),D*\>DDCB d AA\\
RES 5,(IX+d),E*\>DDCB d AB\\
RES 5,(IX+d),H*\>DDCB d AC\\
RES 5,(IX+d),L*\>DDCB d AD\\
RES 5,(IX+d)\>DDCB d AE\\
RES 5,(IY+d),A*\>FDCB d AF\\
RES 5,(IY+d),B*\>FDCB d A8\\
RES 5,(IY+d),C*\>FDCB d A9\\
RES 5,(IY+d),D*\>FDCB d AA\\
RES 5,(IY+d),E*\>FDCB d AB\\
RES 5,(IY+d),H*\>FDCB d AC\\
RES 5,(IY+d),L*\>FDCB d AD\\
RES 5,(IY+d)\>FDCB d AE\\
RES 5,A\>CBAF\\
RES 5,B\>CBA8\\
RES 5,C\>CBA9\\
RES 5,D\>CBAA\\
RES 5,E\>CBAB\\
RES 5,H\>CBAC\\
RES 5,L\>CBAD\\
RES 6,(HL)\>CBB6\\
RES 6,(IX+d),A*\>DDCB d B7\\
RES 6,(IX+d),B*\>DDCB d B0\\
RES 6,(IX+d),C*\>DDCB d B1\\
RES 6,(IX+d),D*\>DDCB d B2\\
RES 6,(IX+d),E*\>DDCB d B3\\
RES 6,(IX+d),H*\>DDCB d B4\\
RES 6,(IX+d),L*\>DDCB d B5\\
RES 6,(IX+d)\>DDCB d B6\\
RES 6,(IY+d),A*\>FDCB d B7\\
RES 6,(IY+d),B*\>FDCB d B0\\
RES 6,(IY+d),C*\>FDCB d B1\\
RES 6,(IY+d),D*\>FDCB d B2\\
RES 6,(IY+d),E*\>FDCB d B3\\
RES 6,(IY+d),H*\>FDCB d B4\\
RES 6,(IY+d),L*\>FDCB d B5\\
RES 6,(IY+d)\>FDCB d B6\\
RES 6,A\>CBB7\\
RES 6,B\>CBB0\\
RES 6,C\>CBB1\\
RES 6,D\>CBB2\\
RES 6,E\>CBB3\\
RES 6,H\>CBB4\\
RES 6,L\>CBB5\\
RES 7,(HL)\>CBBE\\
RES 7,(IX+d),A*\>DDCB d BF\\
RES 7,(IX+d),B*\>DDCB d B8\\
RES 7,(IX+d),C*\>DDCB d B9\\
RES 7,(IX+d),D*\>DDCB d BA\\
RES 7,(IX+d),E*\>DDCB d BB\\
RES 7,(IX+d),H*\>DDCB d BC\\
RES 7,(IX+d),L*\>DDCB d BD\\
RES 7,(IX+d)\>DDCB d BE\\
RES 7,(IY+d),A*\>FDCB d BF\\
RES 7,(IY+d),B*\>FDCB d B8\\
RES 7,(IY+d),C*\>FDCB d B9\\
RES 7,(IY+d),D*\>FDCB d BA\\
RES 7,(IY+d),E*\>FDCB d BB\\
RES 7,(IY+d),H*\>FDCB d BC\\
RES 7,(IY+d),L*\>FDCB d BD\\
RES 7,(IY+d)\>FDCB d BE\\
RES 7,A\>CBBF\\
RES 7,B\>CBB8\\
RES 7,C\>CBB9\\
RES 7,D\>CBBA\\
RES 7,E\>CBBB\\
RES 7,H\>CBBC\\
RES 7,L\>CBBD\\
RET C\>D8\\
RET M\>F8\\
RET NC\>D0\\
RET NZ\>C0\\
RET PE\>E8\\
RET PO\>E0\\
RET P\>F0\\
RET Z\>C8\\
RETI\>ED4D\\
RETN*\>ED55\\
RETN*\>ED5D\\
RETN*\>ED65\\
RETN*\>ED6D\\
RETN*\>ED75\\
RETN*\>ED7D\\
RETN\>ED45\\
RET\>C9\\
RL (HL)\>CB16\\
RL (IX+d),A*\>DDCB d 17\\
RL (IX+d),B*\>DDCB d 10\\
RL (IX+d),C*\>DDCB d 11\\
RL (IX+d),D*\>DDCB d 12\\
RL (IX+d),E*\>DDCB d 13\\
RL (IX+d),H*\>DDCB d 14\\
RL (IX+d),L*\>DDCB d 15\\
RL (IX+d)\>DDCB d 16\\
RL (IY+d),A*\>FDCB d 17\\
RL (IY+d),B*\>FDCB d 10\\
RL (IY+d),C*\>FDCB d 11\\
RL (IY+d),D*\>FDCB d 12\\
RL (IY+d),E*\>FDCB d 13\\
RL (IY+d),H*\>FDCB d 14\\
RL (IY+d),L*\>FDCB d 15\\
RL (IY+d)\>FDCB d 16\\
RL A\>CB17\\
RL B\>CB10\\
RL C\>CB11\\
RL D\>CB12\\
RL E\>CB13\\
RL H\>CB14\\
RL L\>CB15\\
RLA\>17\\
RLC (HL)\>CB06\\
RLC (IX+d),A*\>DDCB d 07\\
RLC (IX+d),B*\>DDCB d 00\\
RLC (IX+d),C*\>DDCB d 01\\
RLC (IX+d),D*\>DDCB d 02\\
RLC (IX+d),E*\>DDCB d 03\\
RLC (IX+d),H*\>DDCB d 04\\
RLC (IX+d),L*\>DDCB d 05\\
RLC (IX+d)\>DDCB d 06\\
RLC (IY+d),A*\>FDCB d 07\\
RLC (IY+d),B*\>FDCB d 00\\
RLC (IY+d),C*\>FDCB d 01\\
RLC (IY+d),D*\>FDCB d 02\\
RLC (IY+d),E*\>FDCB d 03\\
RLC (IY+d),H*\>FDCB d 04\\
RLC (IY+d),L*\>FDCB d 05\\
RLC (IY+d)\>FDCB d 06\\
RLC A\>CB07\\
RLC B\>CB00\\
RLC C\>CB01\\
RLC D\>CB02\\
RLC E\>CB03\\
RLC H\>CB04\\
RLC L\>CB05\\
RLCA\>07\\
RLD\>ED6F\\
RR (HL)\>CB1E\\
RR (IX+d),A*\>DDCB d 1F\\
RR (IX+d),B*\>DDCB d 18\\
RR (IX+d),C*\>DDCB d 19\\
RR (IX+d),D*\>DDCB d 1A\\
RR (IX+d),E*\>DDCB d 1B\\
RR (IX+d),H*\>DDCB d 1C\\
RR (IX+d),L*\>DDCB d 1D\\
RR (IX+d)\>DDCB d 1E\\
RR (IY+d),A*\>FDCB d 1F\\
RR (IY+d),B*\>FDCB d 18\\
RR (IY+d),C*\>FDCB d 19\\
RR (IY+d),D*\>FDCB d 1A\\
RR (IY+d),E*\>FDCB d 1B\\
RR (IY+d),H*\>FDCB d 1C\\
RR (IY+d),L*\>FDCB d 1D\\
RR (IY+d)\>FDCB d 1E\\
RR A\>CB1F\\
RR B\>CB18\\
RR C\>CB19\\
RR D\>CB1A\\
RR E\>CB1B\\
RR H\>CB1C\\
RR L\>CB1D\\
RRA\>1F\\
RRC (HL)\>CB0E\\
RRC (IX+d),A*\>DDCB d 0F\\
RRC (IX+d),B*\>DDCB d 08\\
RRC (IX+d),C*\>DDCB d 09\\
RRC (IX+d),D*\>DDCB d 0A\\
RRC (IX+d),E*\>DDCB d 0B\\
RRC (IX+d),H*\>DDCB d 0C\\
RRC (IX+d),L*\>DDCB d 0D\\
RRC (IX+d)\>DDCB d 0E\\
RRC (IY+d),A*\>FDCB d 0F\\
RRC (IY+d),B*\>FDCB d 08\\
RRC (IY+d),C*\>FDCB d 09\\
RRC (IY+d),D*\>FDCB d 0A\\
RRC (IY+d),E*\>FDCB d 0B\\
RRC (IY+d),H*\>FDCB d 0C\\
RRC (IY+d),L*\>FDCB d 0D\\
RRC (IY+d)\>FDCB d 0E\\
RRC A\>CB0F\\
RRC B\>CB08\\
RRC C\>CB09\\
RRC D\>CB0A\\
RRC E\>CB0B\\
RRC H\>CB0C\\
RRC L\>CB0D\\
RRCA\>0F\\
RRD\>ED67\\
RST 0H\>C7\\
RST 10H\>D7\\
RST 18H\>DF\\
RST 20H\>E7\\
RST 28H\>EF\\
RST 30H\>F7\\
RST 38H\>FF\\
RST 8H\>CF\\
SBC A,(HL)\>9E\\
SBC A,(IX+d)\>DD9E d\\
SBC A,(IY+d)\>FD9E d\\
SBC A,A\>9F\\
SBC A,B\>98\\
SBC A,C\>99\\
SBC A,D\>9A\\
SBC A,E\>9B\\
SBC A,H\>9C\\
SBC A,IXh*\>DD9C\\
SBC A,IXl*\>DD9D\\
SBC A,IYh*\>FD9C\\
SBC A,IYl*\>FD9D\\
SBC A,L\>9D\\
SBC A,n\>DE n\\
SBC HL,BC\>ED42\\
SBC HL,DE\>ED52\\
SBC HL,HL\>ED62\\
SBC HL,SP\>ED72\\
SCF\>37\\
SET 0,(HL)\>CBC6\\
SET 0,(IX+d),A*\>DDCB d C7\\
SET 0,(IX+d),B*\>DDCB d C0\\
SET 0,(IX+d),C*\>DDCB d C1\\
SET 0,(IX+d),D*\>DDCB d C2\\
SET 0,(IX+d),E*\>DDCB d C3\\
SET 0,(IX+d),H*\>DDCB d C4\\
SET 0,(IX+d),L*\>DDCB d C5\\
SET 0,(IX+d)\>DDCB d C6\\
SET 0,(IY+d),A*\>FDCB d C7\\
SET 0,(IY+d),B*\>FDCB d C0\\
SET 0,(IY+d),C*\>FDCB d C1\\
SET 0,(IY+d),D*\>FDCB d C2\\
SET 0,(IY+d),E*\>FDCB d C3\\
SET 0,(IY+d),H*\>FDCB d C4\\
SET 0,(IY+d),L*\>FDCB d C5\\
SET 0,(IY+d)\>FDCB d C6\\
SET 0,A\>CBC7\\
SET 0,B\>CBC0\\
SET 0,C\>CBC1\\
SET 0,D\>CBC2\\
SET 0,E\>CBC3\\
SET 0,H\>CBC4\\
SET 0,L\>CBC5\\
SET 1,(HL)\>CBCE\\
SET 1,(IX+d),A*\>DDCB d CF\\
SET 1,(IX+d),B*\>DDCB d C8\\
SET 1,(IX+d),C*\>DDCB d C9\\
SET 1,(IX+d),D*\>DDCB d CA\\
SET 1,(IX+d),E*\>DDCB d CB\\
SET 1,(IX+d),H*\>DDCB d CC\\
SET 1,(IX+d),L*\>DDCB d CD\\
SET 1,(IX+d)\>DDCB d CE\\
SET 1,(IY+d),A*\>FDCB d CF\\
SET 1,(IY+d),B*\>FDCB d C8\\
SET 1,(IY+d),C*\>FDCB d C9\\
SET 1,(IY+d),D*\>FDCB d CA\\
SET 1,(IY+d),E*\>FDCB d CB\\
SET 1,(IY+d),H*\>FDCB d CC\\
SET 1,(IY+d),L*\>FDCB d CD\\
SET 1,(IY+d)\>FDCB d CE\\
SET 1,A\>CBCF\\
SET 1,B\>CBC8\\
SET 1,C\>CBC9\\
SET 1,D\>CBCA\\
SET 1,E\>CBCB\\
SET 1,H\>CBCC\\
SET 1,L\>CBCD\\
SET 2,(HL)\>CBD6\\
SET 2,(IX+d),A*\>DDCB d D7\\
SET 2,(IX+d),B*\>DDCB d D0\\
SET 2,(IX+d),C*\>DDCB d D1\\
SET 2,(IX+d),D*\>DDCB d D2\\
SET 2,(IX+d),E*\>DDCB d D3\\
SET 2,(IX+d),H*\>DDCB d D4\\
SET 2,(IX+d),L*\>DDCB d D5\\
SET 2,(IX+d)\>DDCB d D6\\
SET 2,(IY+d),A*\>FDCB d D7\\
SET 2,(IY+d),B*\>FDCB d D0\\
SET 2,(IY+d),C*\>FDCB d D1\\
SET 2,(IY+d),D*\>FDCB d D2\\
SET 2,(IY+d),E*\>FDCB d D3\\
SET 2,(IY+d),H*\>FDCB d D4\\
SET 2,(IY+d),L*\>FDCB d D5\\
SET 2,(IY+d)\>FDCB d D6\\
SET 2,A\>CBD7\\
SET 2,B\>CBD0\\
SET 2,C\>CBD1\\
SET 2,D\>CBD2\\
SET 2,E\>CBD3\\
SET 2,H\>CBD4\\
SET 2,L\>CBD5\\
SET 3,(HL)\>CBDE\\
SET 3,(IX+d),A*\>DDCB d DF\\
SET 3,(IX+d),B*\>DDCB d D8\\
SET 3,(IX+d),C*\>DDCB d D9\\
SET 3,(IX+d),D*\>DDCB d DA\\
SET 3,(IX+d),E*\>DDCB d DB\\
SET 3,(IX+d),H*\>DDCB d DC\\
SET 3,(IX+d),L*\>DDCB d DD\\
SET 3,(IX+d)\>DDCB d DE\\
SET 3,(IY+d),A*\>FDCB d DF\\
SET 3,(IY+d),B*\>FDCB d D8\\
SET 3,(IY+d),C*\>FDCB d D9\\
SET 3,(IY+d),D*\>FDCB d DA\\
SET 3,(IY+d),E*\>FDCB d DB\\
SET 3,(IY+d),H*\>FDCB d DC\\
SET 3,(IY+d),L*\>FDCB d DD\\
SET 3,(IY+d)\>FDCB d DE\\
SET 3,A\>CBDF\\
SET 3,B\>CBD8\\
SET 3,C\>CBD9\\
SET 3,D\>CBDA\\
SET 3,E\>CBDB\\
SET 3,H\>CBDC\\
SET 3,L\>CBDD\\
SET 4,(HL)\>CBE6\\
SET 4,(IX+d),A*\>DDCB d E7\\
SET 4,(IX+d),B*\>DDCB d E0\\
SET 4,(IX+d),C*\>DDCB d E1\\
SET 4,(IX+d),D*\>DDCB d E2\\
SET 4,(IX+d),E*\>DDCB d E3\\
SET 4,(IX+d),H*\>DDCB d E4\\
SET 4,(IX+d),L*\>DDCB d E5\\
SET 4,(IX+d)\>DDCB d E6\\
SET 4,(IY+d),A*\>FDCB d E7\\
SET 4,(IY+d),B*\>FDCB d E0\\
SET 4,(IY+d),C*\>FDCB d E1\\
SET 4,(IY+d),D*\>FDCB d E2\\
SET 4,(IY+d),E*\>FDCB d E3\\
SET 4,(IY+d),H*\>FDCB d E4\\
SET 4,(IY+d),L*\>FDCB d E5\\
SET 4,(IY+d)\>FDCB d E6\\
SET 4,A\>CBE7\\
SET 4,B\>CBE0\\
SET 4,C\>CBE1\\
SET 4,D\>CBE2\\
SET 4,E\>CBE3\\
SET 4,H\>CBE4\\
SET 4,L\>CBE5\\
SET 5,(HL)\>CBEE\\
SET 5,(IX+d),A*\>DDCB d EF\\
SET 5,(IX+d),B*\>DDCB d E8\\
SET 5,(IX+d),C*\>DDCB d E9\\
SET 5,(IX+d),D*\>DDCB d EA\\
SET 5,(IX+d),E*\>DDCB d EB\\
SET 5,(IX+d),H*\>DDCB d EC\\
SET 5,(IX+d),L*\>DDCB d ED\\
SET 5,(IX+d)\>DDCB d EE\\
SET 5,(IY+d),A*\>FDCB d EF\\
SET 5,(IY+d),B*\>FDCB d E8\\
SET 5,(IY+d),C*\>FDCB d E9\\
SET 5,(IY+d),D*\>FDCB d EA\\
SET 5,(IY+d),E*\>FDCB d EB\\
SET 5,(IY+d),H*\>FDCB d EC\\
SET 5,(IY+d),L*\>FDCB d ED\\
SET 5,(IY+d)\>FDCB d EE\\
SET 5,A\>CBEF\\
SET 5,B\>CBE8\\
SET 5,C\>CBE9\\
SET 5,D\>CBEA\\
SET 5,E\>CBEB\\
SET 5,H\>CBEC\\
SET 5,L\>CBED\\
SET 6,(HL)\>CBF6\\
SET 6,(IX+d),A*\>DDCB d F7\\
SET 6,(IX+d),B*\>DDCB d F0\\
SET 6,(IX+d),C*\>DDCB d F1\\
SET 6,(IX+d),D*\>DDCB d F2\\
SET 6,(IX+d),E*\>DDCB d F3\\
SET 6,(IX+d),H*\>DDCB d F4\\
SET 6,(IX+d),L*\>DDCB d F5\\
SET 6,(IX+d)\>DDCB d F6\\
SET 6,(IY+d),A*\>FDCB d F7\\
SET 6,(IY+d),B*\>FDCB d F0\\
SET 6,(IY+d),C*\>FDCB d F1\\
SET 6,(IY+d),D*\>FDCB d F2\\
SET 6,(IY+d),E*\>FDCB d F3\\
SET 6,(IY+d),H*\>FDCB d F4\\
SET 6,(IY+d),L*\>FDCB d F5\\
SET 6,(IY+d)\>FDCB d F6\\
SET 6,A\>CBF7\\
SET 6,B\>CBF0\\
SET 6,C\>CBF1\\
SET 6,D\>CBF2\\
SET 6,E\>CBF3\\
SET 6,H\>CBF4\\
SET 6,L\>CBF5\\
SET 7,(HL)\>CBFE\\
SET 7,(IX+d),A*\>DDCB d FF\\
SET 7,(IX+d),B*\>DDCB d F8\\
SET 7,(IX+d),C*\>DDCB d F9\\
SET 7,(IX+d),D*\>DDCB d FA\\
SET 7,(IX+d),E*\>DDCB d FB\\
SET 7,(IX+d),H*\>DDCB d FC\\
SET 7,(IX+d),L*\>DDCB d FD\\
SET 7,(IX+d)\>DDCB d FE\\
SET 7,(IY+d),A*\>FDCB d FF\\
SET 7,(IY+d),B*\>FDCB d F8\\
SET 7,(IY+d),C*\>FDCB d F9\\
SET 7,(IY+d),D*\>FDCB d FA\\
SET 7,(IY+d),E*\>FDCB d FB\\
SET 7,(IY+d),H*\>FDCB d FC\\
SET 7,(IY+d),L*\>FDCB d FD\\
SET 7,(IY+d)\>FDCB d FE\\
SET 7,A\>CBFF\\
SET 7,B\>CBF8\\
SET 7,C\>CBF9\\
SET 7,D\>CBFA\\
SET 7,E\>CBFB\\
SET 7,H\>CBFC\\
SET 7,L\>CBFD\\
SLA (HL)\>CB26\\
SLA (IX+d),A*\>DDCB d 27\\
SLA (IX+d),B*\>DDCB d 20\\
SLA (IX+d),C*\>DDCB d 21\\
SLA (IX+d),D*\>DDCB d 22\\
SLA (IX+d),E*\>DDCB d 23\\
SLA (IX+d),H*\>DDCB d 24\\
SLA (IX+d),L*\>DDCB d 25\\
SLA (IX+d)\>DDCB d 26\\
SLA (IY+d),A*\>FDCB d 27\\
SLA (IY+d),B*\>FDCB d 20\\
SLA (IY+d),C*\>FDCB d 21\\
SLA (IY+d),D*\>FDCB d 22\\
SLA (IY+d),E*\>FDCB d 23\\
SLA (IY+d),H*\>FDCB d 24\\
SLA (IY+d),L*\>FDCB d 25\\
SLA (IY+d)\>FDCB d 26\\
SLA A\>CB27\\
SLA B\>CB20\\
SLA C\>CB21\\
SLA D\>CB22\\
SLA E\>CB23\\
SLA H\>CB24\\
SLA L\>CB25\\
SLL (HL)*\>CB36\\
SLL (IX+d)*\>DDCB d 36\\
SLL (IX+d),A*\>DDCB d 37\\
SLL (IX+d),B*\>DDCB d 30\\
SLL (IX+d),C*\>DDCB d 31\\
SLL (IX+d),D*\>DDCB d 32\\
SLL (IX+d),E*\>DDCB d 33\\
SLL (IX+d),H*\>DDCB d 34\\
SLL (IX+d),L*\>DDCB d 35\\
SLL (IY+d)*\>FDCB d 36\\
SLL (IY+d),A*\>FDCB d 37\\
SLL (IY+d),B*\>FDCB d 30\\
SLL (IY+d),C*\>FDCB d 31\\
SLL (IY+d),D*\>FDCB d 32\\
SLL (IY+d),E*\>FDCB d 33\\
SLL (IY+d),H*\>FDCB d 34\\
SLL (IY+d),L*\>FDCB d 35\\
SLL A*\>CB37\\
SLL B*\>CB30\\
SLL C*\>CB31\\
SLL D*\>CB32\\
SLL E*\>CB33\\
SLL H*\>CB34\\
SLL L*\>CB35\\
SRA (HL)\>CB2E\\
SRA (IX+d),A*\>DDCB d 2F\\
SRA (IX+d),B*\>DDCB d 28\\
SRA (IX+d),C*\>DDCB d 29\\
SRA (IX+d),D*\>DDCB d 2A\\
SRA (IX+d),E*\>DDCB d 2B\\
SRA (IX+d),H*\>DDCB d 2C\\
SRA (IX+d),L*\>DDCB d 2D\\
SRA (IX+d)\>DDCB d 2E\\
SRA (IY+d),A*\>FDCB d 2F\\
SRA (IY+d),B*\>FDCB d 28\\
SRA (IY+d),C*\>FDCB d 29\\
SRA (IY+d),D*\>FDCB d 2A\\
SRA (IY+d),E*\>FDCB d 2B\\
SRA (IY+d),H*\>FDCB d 2C\\
SRA (IY+d),L*\>FDCB d 2D\\
SRA (IY+d)\>FDCB d 2E\\
SRA A\>CB2F\\
SRA B\>CB28\\
SRA C\>CB29\\
SRA D\>CB2A\\
SRA E\>CB2B\\
SRA H\>CB2C\\
SRA L\>CB2D\\
SRL (HL)\>CB3E\\
SRL (IX+d),A*\>DDCB d 3F\\
SRL (IX+d),B*\>DDCB d 38\\
SRL (IX+d),C*\>DDCB d 39\\
SRL (IX+d),D*\>DDCB d 3A\\
SRL (IX+d),E*\>DDCB d 3B\\
SRL (IX+d),H*\>DDCB d 3C\\
SRL (IX+d),L*\>DDCB d 3D\\
SRL (IX+d)\>DDCB d 3E\\
SRL (IY+d),A*\>FDCB d 3F\\
SRL (IY+d),B*\>FDCB d 38\\
SRL (IY+d),C*\>FDCB d 39\\
SRL (IY+d),D*\>FDCB d 3A\\
SRL (IY+d),E*\>FDCB d 3B\\
SRL (IY+d),H*\>FDCB d 3C\\
SRL (IY+d),L*\>FDCB d 3D\\
SRL (IY+d)\>FDCB d 3E\\
SRL A\>CB3F\\
SRL B\>CB38\\
SRL C\>CB39\\
SRL D\>CB3A\\
SRL E\>CB3B\\
SRL H\>CB3C\\
SRL L\>CB3D\\
SUB (HL)\>96\\
SUB (IX+d)\>DD96 d\\
SUB (IY+d)\>FD96 d\\
SUB A\>97\\
SUB B\>90\\
SUB C\>91\\
SUB D\>92\\
SUB E\>93\\
SUB H\>94\\
SUB IXh*\>DD94\\
SUB IXl*\>DD95\\
SUB IYh*\>FD94\\
SUB IYl*\>FD95\\
SUB L\>95\\
SUB n\>D6 n\\
XOR (HL)\>AE\\
XOR (IX+d)\>DDAE d\\
XOR (IY+d)\>FDAE d\\
XOR A\>AF\\
XOR B\>A8\\
XOR C\>A9\\
XOR D\>AA\\
XOR E\>AB\\
XOR H\>AC\\
XOR IXh*\>DDAC\\
XOR IXl*\>DDAD\\
XOR IYh*\>FDAC\\
XOR IYl*\>FDAD\\
XOR L\>AD\\
XOR n\>EE n

\end{tabbing}
}
\end{multicols}
\normalsize

\chapter{GNU Free Documentation License}
\label{gfdl}

Version 1.1, March 2000\\

 Copyright \copyright\ 2000  Free Software Foundation, Inc.\\
     59 Temple Place, Suite 330, Boston, MA  02111-1307  USA\\
 Everyone is permitted to copy and distribute verbatim copies
 of this license document, but changing it is not allowed.

\section*{Preamble}

The purpose of this License is to make a manual, textbook, or other
written document ``free'' in the sense of freedom: to assure everyone
the effective freedom to copy and redistribute it, with or without
modifying it, either commercially or non-commercially.  Secondarily,
this License preserves for the author and publisher a way to get
credit for their work, while not being considered responsible for
modifications made by others.

This License is a kind of ``copyleft'', which means that derivative
works of the document must themselves be free in the same sense.  It
complements the GNU General Public License, which is a copyleft
license designed for free software.

We have designed this License in order to use it for manuals for free
software, because free software needs free documentation: a free
program should come with manuals providing the same freedoms that the
software does.  But this License is not limited to software manuals;
it can be used for any textual work, regardless of subject matter or
whether it is published as a printed book.  We recommend this License
principally for works whose purpose is instruction or reference.

\section{Applicability and Definitions}

This License applies to any manual or other work that contains a
notice placed by the copyright holder saying it can be distributed
under the terms of this License.  The ``Document'', below, refers to any
such manual or work.  Any member of the public is a licensee, and is
addressed as ``you''.

A ``Modified Version'' of the Document means any work containing the
Document or a portion of it, either copied verbatim, or with
modifications and/or translated into another language.

A ``Secondary Section'' is a named appendix or a front-matter section of
the Document that deals exclusively with the relationship of the
publishers or authors of the Document to the Document's overall subject
(or to related matters) and contains nothing that could fall directly
within that overall subject.  (For example, if the Document is in part a
textbook of mathematics, a Secondary Section may not explain any
mathematics.)  The relationship could be a matter of historical
connection with the subject or with related matters, or of legal,
commercial, philosophical, ethical or political position regarding
them.

The ``Invariant Sections'' are certain Secondary Sections whose titles
are designated, as being those of Invariant Sections, in the notice
that says that the Document is released under this License.

The ``Cover Texts'' are certain short passages of text that are listed,
as Front-Cover Texts or Back-Cover Texts, in the notice that says that
the Document is released under this License.

A ``Transparent'' copy of the Document means a machine-readable copy,
represented in a format whose specification is available to the
general public, whose contents can be viewed and edited directly and
straightforwardly with generic text editors or (for images composed of
pixels) generic paint programs or (for drawings) some widely available
drawing editor, and that is suitable for input to text formatters or
for automatic translation to a variety of formats suitable for input
to text formatters.  A copy made in an otherwise Transparent file
format whose mark-up has been designed to thwart or discourage
subsequent modification by readers is not Transparent.  A copy that is
not ``Transparent'' is called ``Opaque''.

Examples of suitable formats for Transparent copies include plain
ASCII without mark-up, Texinfo input format, \LaTeX~input format, SGML
or XML using a publicly available DTD, and standard-conforming simple
HTML designed for human modification.  Opaque formats include
PostScript, PDF, proprietary formats that can be read and edited only
by proprietary word processors, SGML or XML for which the DTD and/or
processing tools are not generally available, and the
machine-generated HTML produced by some word processors for output
purposes only.

The ``Title Page'' means, for a printed book, the title page itself,
plus such following pages as are needed to hold, legibly, the material
this License requires to appear in the title page.  For works in
formats which do not have any title page as such, ``Title Page'' means
the text near the most prominent appearance of the work's title,
preceding the beginning of the body of the text.


\section{Verbatim Copying}

You may copy and distribute the Document in any medium, either
commercially or non-commercially, provided that this License, the
copyright notices, and the license notice saying this License applies
to the Document are reproduced in all copies, and that you add no other
conditions whatsoever to those of this License.  You may not use
technical measures to obstruct or control the reading or further
copying of the copies you make or distribute.  However, you may accept
compensation in exchange for copies.  If you distribute a large enough
number of copies you must also follow the conditions in section 3.

You may also lend copies, under the same conditions stated above, and
you may publicly display copies.


\section{Copying in Quantity}

If you publish printed copies of the Document numbering more than 100,
and the Document's license notice requires Cover Texts, you must enclose
the copies in covers that carry, clearly and legibly, all these Cover
Texts: Front-Cover Texts on the front cover, and Back-Cover Texts on
the back cover.  Both covers must also clearly and legibly identify
you as the publisher of these copies.  The front cover must present
the full title with all words of the title equally prominent and
visible.  You may add other material on the covers in addition.
Copying with changes limited to the covers, as long as they preserve
the title of the Document and satisfy these conditions, can be treated
as verbatim copying in other respects.

If the required texts for either cover are too voluminous to fit
legibly, you should put the first ones listed (as many as fit
reasonably) on the actual cover, and continue the rest onto adjacent
pages.

If you publish or distribute Opaque copies of the Document numbering
more than 100, you must either include a machine-readable Transparent
copy along with each Opaque copy, or state in or with each Opaque copy
a publicly-accessible computer-network location containing a complete
Transparent copy of the Document, free of added material, which the
general network-using public has access to download anonymously at no
charge using public-standard network protocols.  If you use the latter
option, you must take reasonably prudent steps, when you begin
distribution of Opaque copies in quantity, to ensure that this
Transparent copy will remain thus accessible at the stated location
until at least one year after the last time you distribute an Opaque
copy (directly or through your agents or retailers) of that edition to
the public.

It is requested, but not required, that you contact the authors of the
Document well before redistributing any large number of copies, to give
them a chance to provide you with an updated version of the Document.


\section{Modifications}

You may copy and distribute a Modified Version of the Document under
the conditions of sections 2 and 3 above, provided that you release
the Modified Version under precisely this License, with the Modified
Version filling the role of the Document, thus licensing distribution
and modification of the Modified Version to whoever possesses a copy
of it.  In addition, you must do these things in the Modified Version:

\begin{itemize}

\item Use in the Title Page (and on the covers, if any) a title distinct
   from that of the Document, and from those of previous versions
   (which should, if there were any, be listed in the History section
   of the Document).  You may use the same title as a previous version
   if the original publisher of that version gives permission.
\item List on the Title Page, as authors, one or more persons or entities
   responsible for authorship of the modifications in the Modified
   Version, together with at least five of the principal authors of the
   Document (all of its principal authors, if it has less than five).
\item State on the Title page the name of the publisher of the
   Modified Version, as the publisher.
\item Preserve all the copyright notices of the Document.
\item Add an appropriate copyright notice for your modifications
   adjacent to the other copyright notices.
\item Include, immediately after the copyright notices, a license notice
   giving the public permission to use the Modified Version under the
   terms of this License, in the form shown in the Addendum below.
\item Preserve in that license notice the full lists of Invariant Sections
   and required Cover Texts given in the Document's license notice.
\item Include an unaltered copy of this License.
\item Preserve the section entitled ``History'', and its title, and add to
   it an item stating at least the title, year, new authors, and
   publisher of the Modified Version as given on the Title Page.  If
   there is no section entitled ``History'' in the Document, create one
   stating the title, year, authors, and publisher of the Document as
   given on its Title Page, then add an item describing the Modified
   Version as stated in the previous sentence.
\item Preserve the network location, if any, given in the Document for
   public access to a Transparent copy of the Document, and likewise
   the network locations given in the Document for previous versions
   it was based on.  These may be placed in the ``History'' section.
   You may omit a network location for a work that was published at
   least four years before the Document itself, or if the original
   publisher of the version it refers to gives permission.
\item In any section entitled ``Acknowledgements'' or ``Dedications'',
   preserve the section's title, and preserve in the section all the
   substance and tone of each of the contributor acknowledgements
   and/or dedications given therein.
\item Preserve all the Invariant Sections of the Document,
   unaltered in their text and in their titles.  Section numbers
   or the equivalent are not considered part of the section titles.
\item Delete any section entitled ``Endorsements''.  Such a section
   may not be included in the Modified Version.
\item Do not retitle any existing section as ``Endorsements''
   or to conflict in title with any Invariant Section.

\end{itemize}

If the Modified Version includes new front-matter sections or
appendices that qualify as Secondary Sections and contain no material
copied from the Document, you may at your option designate some or all
of these sections as invariant.  To do this, add their titles to the
list of Invariant Sections in the Modified Version's license notice.
These titles must be distinct from any other section titles.

You may add a section entitled ``Endorsements'', provided it contains
nothing but endorsements of your Modified Version by various
parties -- for example, statements of peer review or that the text has
been approved by an organization as the authoritative definition of a
standard.

You may add a passage of up to five words as a Front-Cover Text, and a
passage of up to 25 words as a Back-Cover Text, to the end of the list
of Cover Texts in the Modified Version.  Only one passage of
Front-Cover Text and one of Back-Cover Text may be added by (or
through arrangements made by) any one entity.  If the Document already
includes a cover text for the same cover, previously added by you or
by arrangement made by the same entity you are acting on behalf of,
you may not add another; but you may replace the old one, on explicit
permission from the previous publisher that added the old one.

The author(s) and publisher(s) of the Document do not by this License
give permission to use their names for publicity for or to assert or
imply endorsement of any Modified Version.


\section{Combining Documents}

You may combine the Document with other documents released under this
License, under the terms defined in section 4 above for modified
versions, provided that you include in the combination all of the
Invariant Sections of all of the original documents, unmodified, and
list them all as Invariant Sections of your combined work in its
license notice.

The combined work need only contain one copy of this License, and
multiple identical Invariant Sections may be replaced with a single
copy.  If there are multiple Invariant Sections with the same name but
different contents, make the title of each such section unique by
adding at the end of it, in parentheses, the name of the original
author or publisher of that section if known, or else a unique number.
Make the same adjustment to the section titles in the list of
Invariant Sections in the license notice of the combined work.

In the combination, you must combine any sections entitled ``History''
in the various original documents, forming one section entitled
``History''; likewise combine any sections entitled ``Acknowledgements'',
and any sections entitled ``Dedications''.  You must delete all sections
entitled ``Endorsements.''


\section{Collections of Documents}

You may make a collection consisting of the Document and other documents
released under this License, and replace the individual copies of this
License in the various documents with a single copy that is included in
the collection, provided that you follow the rules of this License for
verbatim copying of each of the documents in all other respects.

You may extract a single document from such a collection, and distribute
it individually under this License, provided you insert a copy of this
License into the extracted document, and follow this License in all
other respects regarding verbatim copying of that document.



\section{Aggregation With Independent Works}

A compilation of the Document or its derivatives with other separate
and independent documents or works, in or on a volume of a storage or
distribution medium, does not as a whole count as a Modified Version
of the Document, provided no compilation copyright is claimed for the
compilation.  Such a compilation is called an ``aggregate'', and this
License does not apply to the other self-contained works thus compiled
with the Document, on account of their being thus compiled, if they
are not themselves derivative works of the Document.

If the Cover Text requirement of section 3 is applicable to these
copies of the Document, then if the Document is less than one quarter
of the entire aggregate, the Document's Cover Texts may be placed on
covers that surround only the Document within the aggregate.
Otherwise they must appear on covers around the whole aggregate.


\section{Translation}

Translation is considered a kind of modification, so you may
distribute translations of the Document under the terms of section 4.
Replacing Invariant Sections with translations requires special
permission from their copyright holders, but you may include
translations of some or all Invariant Sections in addition to the
original versions of these Invariant Sections.  You may include a
translation of this License provided that you also include the
original English version of this License.  In case of a disagreement
between the translation and the original English version of this
License, the original English version will prevail.


\section{Termination}

You may not copy, modify, sublicense, or distribute the Document except
as expressly provided for under this License.  Any other attempt to
copy, modify, sublicense or distribute the Document is void, and will
automatically terminate your rights under this License.  However,
parties who have received copies, or rights, from you under this
License will not have their licenses terminated so long as such
parties remain in full compliance.


\section{Future Revisions of This License}

The Free Software Foundation may publish new, revised versions
of the GNU Free Documentation License from time to time.  Such new
versions will be similar in spirit to the present version, but may
differ in detail to address new problems or concerns. See
http://www.gnu.org/copyleft/.

Each version of the License is given a distinguishing version number.
If the Document specifies that a particular numbered version of this
License "or any later version" applies to it, you have the option of
following the terms and conditions either of that specified version or
of any later version that has been published (not as a draft) by the
Free Software Foundation.  If the Document does not specify a version
number of this License, you may choose any version ever published (not
as a draft) by the Free Software Foundation.

\end{document}
