\section{Tilemap}
\label{zx_next_tilemap}

% ────────────────────────────────────────────────────────────────────────
% ─██████████████─██████████─██████─────────██████████████─██████████████─
% ─██░░░░░░░░░░██─██░░░░░░██─██░░██─────────██░░░░░░░░░░██─██░░░░░░░░░░██─
% ─██████░░██████─████░░████─██░░██─────────██░░██████████─██░░██████████─
% ─────██░░██───────██░░██───██░░██─────────██░░██─────────██░░██─────────
% ─────██░░██───────██░░██───██░░██─────────██░░██████████─██░░██████████─
% ─────██░░██───────██░░██───██░░██─────────██░░░░░░░░░░██─██░░░░░░░░░░██─
% ─────██░░██───────██░░██───██░░██─────────██░░██████████─██████████░░██─
% ─────██░░██───────██░░██───██░░██─────────██░░██─────────────────██░░██─
% ─────██░░██─────████░░████─██░░██████████─██░░██████████─██████████░░██─
% ─────██░░██─────██░░░░░░██─██░░░░░░░░░░██─██░░░░░░░░░░██─██░░░░░░░░░░██─
% ─────██████─────██████████─██████████████─██████████████─██████████████─
% ────────────────────────────────────────────────────────────────────────

Tilemap is fast and effective way of displaying 8x8 pixel blocks on the screen. There are two possible resolutions available: 40x32 or 80x32 tiles. Tilemap layer overlaps ULA by 32 pixels on each side. Or in other words, similar to 320x256 and 640x256 modes of Layer 2, tilemap also covers the whole of the screen, including the border.

Tilemap is defined by 2 data structures: tile definitions and tilemap data itself.


\subsection{Tile Definitions}

Tiles are 8x8 pixels with each pixel representing an index of the colour from the currently selected tilemap palette.

Each pixel occupies 4-bits, meaning tiles can use 16 colours. However, as we'll see in the next section, it's possible to specify a 4-bit palette offset for each tile which allows us to reach all 256 colours from the palette.

A maximum of 256 tile definitions are possible, but this can be extended to 512 if needed using \PortLink{Tilemap Control Register}{6B}.

All tiles definitions are specified in a contiguous memory block. The offset of tile definitions memory address relative to the start of bank 5 needs to be specified with \PortLink{Tile Definitions Base Address Register}{6F}.


\subsection{Tilemap Data}

Tilemap data requires 2 bytes per tile:

\begin{BitTableWord}
	\BitStartMulti{4}{\rotatebox[origin=c]{90}{Palette Offset}} &
	\rotatebox[origin=c]{90}{X Mirror} &
	\rotatebox[origin=c]{90}{Y Mirror} &
	\rotatebox[origin=c]{90}{Rotate} &
	\rotatebox[origin=c]{90}{ULA Mode} &
	\BitMulti{8}{\rotatebox[origin=c]{90}{Tile Index}} \\
\end{BitTableWord}

\begin{basedescript}{
	\desclabelstyle{\multilinelabel}
	\desclabelwidth{3cm}}
	\setlength\itemsep{0pt}

	\newcommand{\RightItem}[1]{\item[#1]}

	\RightItem{Palette Offset} 4-bit palette offset for this tile. This allows shifting colours to other 16-colour ``banks'' thus allowing us to reach the whole 256 colours from the palette.
	
	\RightItem{X Mirror} If {\tt 1}, this tile will be mirrored in X direction.

	\RightItem{Y Mirror} If {\tt 1}, this tile will be mirrored in Y direction.
	
	\RightItem{Rotate} If {\tt 1}, this tile will be rotated 90\Deg clockwise.
	
	\RightItem{ULA Mode} If {\tt 1}, this tile will be rendered on top, if {\tt 0} below ULA display. However in 512 tile mode, this is the 8th bit of tile index.
	
	\RightItem{Tile Index} 8-bit tile index within the tile definitions.
\end{basedescript}

However, it's possible to eliminate attributes byte by setting bit 5 in \PortLink{Tilemap Control Register}{6B}. This only leaves an 8-bit tile index. Tileset then only occupies half the memory. But we lose the option to specify attributes for each tile separately. Instead attributes for all tiles are taken from \PortLink{Default Tilemap Attribute Register}{6C}.

The offset of the tilemap data memory address relative to the start of bank 5 needs to be specified with \PortLink{Tilemap Base Address Register}{6E}.


\subsection{Memory Organization}

The Tilemap layer is closely tied with ULA. Memory wise, it always exists in 16K slot 5. By default, this page is loaded into 16K slot 1 \MemAddr{4000}-\MemAddr{7FFF} (examples here will assume this configuration, if you load into a different slot, you will have to adjust addresses accordingly).

If both ULA and tilemap are used, memory should be arranged to avoid overlap. Given ULA pixel and attributes memory occupied memory addresses \MemAddr{4000}-\MemAddr{5AFF}, this leaves \MemAddr{5B00}-\MemAddr{7FFF} for tilemap. If we also take into account various system variables that reside on top of ULA attributes, \MemAddr{6000} should be used for starting address. This leaves us:

\begin{ElegantTable}{|l|l|l|l|l|}
	\ElegantHeader{& \multicolumn{2}{c|}{\EH{40x32}} & \multicolumn{2}{c|}{\EH{80x32}}}

	Bytes per tile & 1 & 2 & 1 & 2 \\
	\hline
	Bytes per tileset & 1280 & 2560 & 2560 & 5120 \\
	\hline
	Max Tile Definitions & 215 & 175 & 175 & 95 \\
\end{ElegantTable}

We as programmers need to tell hardware where in the memory tilemap and tile definitions are stored. \PortLink{Tilemap Base Address Register}{6E} and \PortLink{Tile Definitions Base Address Register}{6F} are used for that.

Both addresses are provided as most significant byte of the offset into memory slot 5 (which starts at \MemAddr{4000}). This means we can only store data at multiples of 256 bytes. For example, if data is stored at \MemAddr{6000}, the MSB offset value would be {\tt \$20} ({\tt \$6000 - \$4000 = \$\underline{20}00}).

Generic formula to calculate MSB of the offset is: {\tt (Address - \$4000) >> 8}.


\subsection{Combining ULA and Tilemap}

ULA and Tilemap can be combined in two ways:

\begin{itemize}[topsep=1pt,itemsep=1pt]
	\item Standard mode: uses bit {\tt 0} from tile's attribute byte to determine if a tile is above or below ULA. If tilemap uses 2 bytes per tile, we can specify the priority for each tile separately, otherwise we specify it for all tiles. Transparent pixels are taken into account - if the top layer is transparent, the bottom one is visible through.
	
	\item Stencil mode: only used if both, ULA and tileset are enabled. The final pixel is transparent if both, ULA and tilemap pixels are transparent. Otherwise final pixel is {\tt AND} of both colour bits. This mode allows one layer to act as a cut-out for the other.
\end{itemize}


\subsection{Examples}

Using tilemaps is very simple. The most challenging part in my experience was finding a drawing program that would export to required formats in full. The best results I have achieved were with Remy's Sprite, Tile and Palette editor website\footnote{\url{https://zx.remysharp.com/sprites/}}. Even then, I had to manually tweak binary files to achieve desired results (single byte per tile).

Regardless of the editor, we need 3 pieces of data: palette, tile definitions and tileset itself. In this example, they are included as binary files:

\begin{tcblisting}{}
tilemap:
	INCBIN "tiles.map"
tilemapLength = &-tilemap

tiles:
	INCBIN "tiles.spr"
tilesLength = &-tiles

palette:
	INCBIN "tiles.pal"
paletteLength = &-palette
\end{tcblisting}

With all data in place, we can start setting up tilemap:

\begin{tcblisting}{}
START_OF_BANK_5     = &4000
START_OF_TILEMAP    = &6000     ; Just after ULA attributes and system vars
START_OF_TILES      = &6600     ; Just after 40x32 tilemap

OFFSET_OF_MAP       = (START_OF_TILEMAP - START_OF_BANK_5) >> 8
OFFSET_OF_TILES     = (START_OF_TILES - START_OF_BANK_5) >> 8

	; Enable tilemap mode
	NEXTREG &6B, %10100001       ; 40x32, 8-bit entries
	NEXTREG &6C, %00000000       ; palette offset, visuals

	; Tell hardware where to find tiles
	NEXTREG &6E, OFFSET_OF_MAP   ; MSB of tilemap in bank 5
	NEXTREG &6F, OFFSET_OF_TILES ; MSB of tilemap definitions
\end{tcblisting}

Above code uses couple neat preprocessing tricks to automatically calculate MSB for tilemap and tile definitions offsets. The rest is simply setting up desired behaviour using Next registers.

\pagebreak
The only remaining piece is to actually copy all the data to expected memory locations:

\begin{tcblisting}{}
	; Setup tilemap palette
	NEXTREG &43, %00110000       ; Auto increment, select first tilemap palette

	; Copy palette
	LD HL, palette               ; Address of palette data in memory
	LD B, 16                     ; Copy 16 colours
	CALL Copy8BitPalette         ; Call routine for copying

	; Copy tile definitions to expected memory
	LD HL, tiles                 ; Address of tiles in memory
	LD BC, tilesLength           ; Number of bytes to copy
	CALL CopyTileDefinitions     ; Copy all tiles data

	; Copy tilemap to expected memory
	LD HL, tilemap               ; Addreess of tilemap in memory
	CALL CopyTileMap40x32        ; Copy 40x32 tilemaps
\end{tcblisting}

We already know {\tt Copy8BitPalette} routine from Layer 2 chapter, the other two are straightforward {\tt LDIR} loops:

\begin{tcblisting}{}
CopyTileDefinitions:
	LD DE, START_OF_TILES
	LDIR
	RET

CopyTileMap40x32:
	LD BC, 40*32		; This variant always loads 40x32
	JR copyTileMap

CopyTileMap80x32:
	LD BC, 80*32		; This variant always loads 80x32

CopyTileMap:
	LD DE, START_OF_TILEMAP
	LDIR
	RET
\end{tcblisting}

You can find fully working example in companion code on GitHub in folder {\tt tilemap}.


\pagebreak
\subsection{Tilemap Registers}
\label{zx_next_tilemap_registers}

\subsubsection{Sprite and Layers System Register \MemAddr{15}}

\begin{NextPort}
	\PortBits{7}
		\PortDesc{{\tt 1} to enable lo-res layer, {\tt 0} disable it}
	\PortBits{6}
		\PortDesc{{\tt 1} to flip sprite rendering priority, i.e. sprite 0 is on top ({\tt 0} after reset)}
	\PortBits{5}
		\PortDesc{{\tt 1} to change clipping to ``over border'' mode (doubling X-axis coordinates of clip window, {\tt 0} after reset)}
	\PortBits{4-2}
		\PortDesc{Layers priority and mixing}
		\PortDescOnly{
			\begin{PortBitConfig}
				\PortBitLine{000}{{\tt S L U} (Sprites are at top, Layer 2 under, Enhanced ULA at bottom)}
				\PortBitLine{001}{{\tt L S U}}
				\PortBitLine{010}{{\tt S U L}}
				\PortBitLine{011}{{\tt L U S}}
				\PortBitLine{100}{{\tt U S L}}
				\PortBitLine{101}{{\tt U L S}}
				\PortBitLine{110}{Core 3.1.1+: {\tt (U|T)S(T|U)(B+L)} blending layer and Layer 2 combined}
				\PortBitLine{   }{Older cores: {\tt S(U+L)} colours from ULA and L2 added per R/G/B channel}
				\PortBitLine{111}{Core 3.1.1+: {\tt (U|T)S(T|U)(B+L-5)} blending layer and Layer 2 combined}
				\PortBitLine{   }{Older cores: {\tt S(U+L-5)} similar as {\tt 110}, but per R/G/B channel {\tt (U+L-5)}}
				\PortBitLine{   }{{\tt 110} and {\tt 111} modes: colours are clamped to {\tt [0,7]}}
			\end{PortBitConfig}
		}
	\PortBits{1}
		\PortDesc{{\tt 1} to enable sprites over border ({\tt 0} after reset)}
	\PortBits{0}
		\PortDesc{{\tt 1} to enable sprite visibility ({\tt 0} after reset)}
\end{NextPort}


\subsubsection{Clip Window Tilemap Register \MemAddr{1B}}

\begin{NextPort}
	\PortBits{7-0}
		\PortDesc{Reads and writes clip-window coordinates for Tilemap}
\end{NextPort}

4 coordinates need to be set: X1, X2, Y1 and Y2. Tilemap will only be visible within these coordinates. X coordinates are internally doubled for 40x32 or quadrupled for 80x32 mode. Positions are inclusive. Default values are {\tt 0}, {\tt 159}, {\tt 0}, {\tt 255}. Origin (0,0) is located 32 pixels to the top-left of ULA top-left coordinate.

Which coordinate gets set, is determined by index. As each write to this register will also increment index, the usual flow is to reset the index to 0 in \PortLink{Clip Window Control Register}{1C}, then write all 4 coordinates in succession.


\subsubsection{Clip Window Control Register \MemAddr{1C}}

\PortReference{zx_next_layer2_registers}{Layer 2}


\subsubsection{Tilemap Offset X MSB Register \MemAddr{2F}}

\begin{NextPort}
	\PortBits{7-2}
		\PortDesc{Reserved, use {\tt 0}}
	\PortBits{1-0}
		\PortDesc{Most significant bit(s) of X offset}
\end{NextPort}

In 40x32 mode, meaningful range is {\tt 0-319}, for 80x32 {\tt 0-639}. Low 8-bits are stored in \PortLink{Tilemap Offset X LSB Register}{30}.


\subsubsection{Tilemap Offset X LSB Register \MemAddr{30}}

\begin{NextPort}
	\PortBits{7-0}
		\PortDesc{X offset for drawing tilemap in pixels}
\end{NextPort}

Tilemap X offset in pixels. Meaningful range is {\tt 0-319} for 40x32 and {\tt 0-639} for 80x32 mode. To write values larger than 255, \PortLink{Tilemap Offset X MSB Register}{2F} is used to store MSB.


\subsubsection{Tilemap Offset Y Register \MemAddr{31}}

\begin{NextPort}
	\PortBits{7-0}
		\PortDesc{Y offset for drawing tilemap in pixels}
\end{NextPort}

Y offset is {\tt 0-255}.


\subsubsection{Palette Index Register \MemAddr{40}}
\vspace*{-2ex}
\subsubsection{Palette Value Register \MemAddr{41}}
\vspace*{-2ex}
\subsubsection{Enhanced ULA Control Register \MemAddr{43}}
\vspace*{-2ex}
\subsubsection{Enhanced ULA Palette Extension \MemAddr{44}}
\PortReference{zx_next_palette_registers}{Palette}


\subsubsection{Tilemap Transparency Index Register \MemAddr{4C}}

\begin{NextPort}
	\PortBits{7-5}
		\PortDesc{Reserved, must be {\tt 0}}
	\PortBits{4-0}
		\PortDesc{Index of transparent colour into tilemap palette}
\end{NextPort}

The pixel index from tile definitions is compared before palette offset is applied to the upper 4 bits, so there's always one index between {\tt 0} and {\tt 15} that works as transparent colour.


\subsubsection{ULA Control Register \MemAddr{68}}

\begin{NextPort}
	\PortBits{7}
		\PortDesc{{\tt 1} to disable ULA output ({\tt 0} after soft reset)}
	\PortBits{6-5}
		\PortDesc{(Core 3.1.1+) Blending in SLU modes 6 \& 7}
		\PortDescOnly{
			\begin{PortBitConfig}
				\PortBitLine{00}{ULA as blend colour}
				\PortBitLine{01}{No blending}
				\PortBitLine{10}{ULA/tilemap as blend colour}
				\PortBitLine{11}{Tilemap as blend colour}
			\end{PortBitConfig}
		}
	\PortBits{4}
		\PortDesc{(Core 3.1.4+) Cancel entries in 8x5 matrix for extended keys}
	\PortBits{3}
		\PortDesc{{\tt 1} to enable ULA+ ({\tt 0} after soft reset)}
	\PortBits{2}
		\PortDesc{{\tt 1} to enable ULA half pixel scroll ({\tt 0} after soft reset)}
	\PortBits{1}
		\PortDesc{Reserved, set to {\tt 0}}
	\PortBits{0}
		\PortDesc{{\tt 1} to enable stencil mode when both the ULA and tilemap are enabled.}
\end{NextPort}

See \PortLink{Sprite and Layers System Register}{15} for different priorities and mixing of ULA, Layer 2 and Sprites.


\subsubsection{Tilemap Control Register \MemAddr{6B}}

\begin{NextPort}
	\PortBits{7}
		\PortDesc{{\tt 1} to enable tilemap, {\tt 0} disable tilemap}
	\PortBits{6}
		\PortDesc{{\tt 1} for 80x32, {\tt 0} 40x32 mode}
	\PortBits{5}
		\PortDesc{{\tt 1} to eliminate attribute byte in tilemap}
	\PortBits{4}
		\PortDesc{{\tt 1} for second, {\tt 0} for first tilemap palette}
	\PortBits{3}
		\PortDesc{{\tt 1} to activate ``text mode''\See{1}}
	\PortBits{2}
		\PortDesc{Reserved, set to {\tt 0}}
	\PortBits{1}
		\PortDesc{{\tt 1} to activate 512, {\tt 0} for 256 tile mode}
	\PortBits{0}
		\PortDesc{{\tt 1} to force tilemap on top of ULA}
\end{NextPort}

\See{1}In the text mode, tiles are defined as 1-bit B\&W bitmaps, same as original Spectrum UDGs. Each tile only requires 8 bytes. In this mode, the tilemap attribute byte is also interpreted differently: bit 0 is still ULA over Tilemap (or 9th bit of tile data index) but the top 7 bits are extended palette offset (the least significant bit is the value of the pixel itself). In this mode, transparency is checked against \PortLink{Global Transparency Register}{14} colour, not against the four-bit tilemap colour index.


\pagebreak
\subsubsection{Default Tilemap Attribute Register \MemAddr{6C}}

If single byte tilemap mode is selected (bit 5 of \PortLink{Tilemap Control Register}{6B} set), this register defines attributes for all tiles.

\begin{NextPort}
	\PortBits{7-4}
		\PortDesc{Palette offset}
	\PortBits{3}
		\PortDesc{{\tt 1} to mirror tiles in X direction}
	\PortBits{2}
		\PortDesc{{\tt 1} to mirror tiles in Y direction}
	\PortBits{1}
		\PortDesc{{\tt 1} rotate tiles 90\Deg clockwise}
	\PortBits{0}
		\PortDesc{In 512 tile mode, bit 8 of tile index}
		\PortDescOnly{{\tt 1} for ULA over tilemap, {\tt 0} for tilemap over ULA}
\end{NextPort}


\subsubsection{Tilemap Base Address Register \MemAddr{6E}}

\begin{NextPort}
	\PortBits{7-6}
		\PortDesc{Ignored, set to {\tt 0}}
	\PortBits{5-0}
		\PortDesc{Most significant byte of tilemap data offset in bank 5}
\end{NextPort}


\subsubsection{Tile Definitions Base Address Register \MemAddr{6F}}

\begin{NextPort}
	\PortBits{7-6}
		\PortDesc{Ignored, set to {\tt 0}}
	\PortBits{5-0}
		\PortDesc{Most significant byte of tile definitions offset in bank 5}
\end{NextPort}


\pagebreak