% ▒█▀▀█ ▒█▀▀▀ ▒█▄░▒█ ▒█▀▀▀ ▒█▀▀█ ░█▀▀█ ▒█░░░ 
% ▒█░▄▄ ▒█▀▀▀ ▒█▒█▒█ ▒█▀▀▀ ▒█▄▄▀ ▒█▄▄█ ▒█░░░ 
% ▒█▄▄█ ▒█▄▄▄ ▒█░░▀█ ▒█▄▄▄ ▒█░▒█ ▒█░▒█ ▒█▄▄█

% Writes register name (WR0...) in the proper style.
\newcommand{\DMARegName}[1]{{\tt #1}}


% ▒█▀▀█ ▒█░░▒█ ▀▀█▀▀ ▒█▀▀▀   ▒█▀▀▄ ▒█▀▀▀ ▒█▀▀▀█ ▒█▀▀█ ▒█▀▀█ ▀█▀ ▒█▀▀█ ▀▀█▀▀ ▀█▀ ▒█▀▀▀█ ▒█▄░▒█ 
% ▒█▀▀▄ ▒█▄▄▄█ ░▒█░░ ▒█▀▀▀   ▒█░▒█ ▒█▀▀▀ ░▀▀▀▄▄ ▒█░░░ ▒█▄▄▀ ▒█░ ▒█▄▄█ ░▒█░░ ▒█░ ▒█░░▒█ ▒█▒█▒█ 
% ▒█▄▄█ ░░▒█░░ ░▒█░░ ▒█▄▄▄   ▒█▄▄▀ ▒█▄▄▄ ▒█▄▄▄█ ▒█▄▄█ ▒█░▒█ ▄█▄ ▒█░░░ ░▒█░░ ▄█▄ ▒█▄▄▄█ ▒█░░▀█

% Writes an environment for describing individual components of a DMA byte
\newenvironment{DMADescription}{
	\begin{basedescript}{
		% setup basedescript styling for labels
		\desclabelstyle{\multilinelabel}
		\desclabelwidth{2.5cm}
	}

	% setup spacing between items
	\setlength\itemsep{1ex}
}{
	\end{basedescript}
}

% Writes an itemize within the DMADescription environment; it uses some less vertical spacing
\newenvironment{DMAList}{
	\begin{itemize}[topsep=-3pt,itemsep=-1pt]
}{
	\end{itemize}
}

% Use for individual items within DMADescription environment. Parameters:
% - #1 Single bit (0) or range (2-0)
% - #2 Title
% - #3 Description
\newcommand{\DMADescriptionItem}[3]{
	\item[\footnotesize \IfInteger{#1}{Bit}{Bits} {\tt #1}:\\#2]

	#3
}

% Use for description items that are not bits - typically what remains are bytes. Parameters:
% - #1 Title
% - #2 Description
\newcommand{\DMADescriptionByte}[2]{
	\item[\footnotesize #1]
	
	#2
}


% ▒█▀▀▄ ▒█▀▀█ ░█▀▀█ ▒█░░▒█ ▀█▀ ▒█▄░▒█ ▒█▀▀█ ▒█▀▀▀█   ░░   ▒█▀▀█ ░█▀▀█ ▒█▀▀▀█ ▒█▀▀▀ 
% ▒█░▒█ ▒█▄▄▀ ▒█▄▄█ ▒█▒█▒█ ▒█░ ▒█▒█▒█ ▒█░▄▄ ░▀▀▀▄▄   ▀▀   ▒█▀▀▄ ▒█▄▄█ ░▀▀▀▄▄ ▒█▀▀▀ 
% ▒█▄▄▀ ▒█░▒█ ▒█░▒█ ▒█▄▀▄█ ▄█▄ ▒█░░▀█ ▒█▄▄█ ▒█▄▄▄█   ░░   ▒█▄▄█ ▒█░▒█ ▒█▄▄▄█ ▒█▄▄▄

\newcommand{\DMAScaledReg}[2]{
	\textbf{#1}
	
	\scalebox{0.75}{#2}
}

% Note: counters and macros below depends on specific order; but this makes drawings much more straighforward. Just be very careful if changing anything as it may have unexpected effects on other nodes!

\newcounter{DMAHeaderIndex}         % Current bit header index; used with \DMABitParHeader
\newcounter{DMABitParameterIndex}		% Current parameter index
\newcounter{DMABitDescIndex}		% Current bit description index
\newcounter{DMABitDescItemIndex}	% Current bit description item index

% This declares all styles used within tikz drawings for DMA commands so we don't have to repeat
\tikzset{
	dmatitle/.style={font=\small\bf},
	dmacode/.style={font=\ttfamily},
	dmalegend/.style={font=\scriptsize\bf, text=PrintableLightGray},
	dmabit/.style={minimum width=1.3em, minimum height=1.3em},
	dmabyte/.style={
		dmabit, 
		draw=PrintableLightGray, 
		rectangle split,
		rectangle split parts=8,
		rectangle split horizontal,
		rectangle split empty part width=0.2em},
	dmabyteouter/.style={dmabit, draw, minimum width=10.68em},
	dmamono/.style={dmabit, dmacode, draw, inner sep=0pt, outer sep=0pt},
	dmaheader/.style={dmabit, dmatitle, draw, fill=PrintableLightGray},
	dmaconnector/.style={Circle-stealth', line width=0.6pt},
	dmavalue/.style={dmacode},
	dmabyteexpl/.style={font=\small\it, text=PrintableDarkGray},
	dmaexpl/.style={},
}

% Returns various IDs for the given bit, taking into account current DMAHeaderIndex
\newcommand{\DMALegendID}{LegendHor\arabic{DMAHeaderIndex}}					% LegendHor0...
\newcommand{\DMALegendRegID}{LegendReg\arabic{DMAHeaderIndex}}				% LegendReg0...
\newcommand{\DMALegendParID}{LegendPar\arabic{DMAHeaderIndex}}				% LegendPar0...
\newcommand{\DMABitDefAnyID}[2]{BitDef#1_#2}								% BitDef0_0, BitDef0_1...
\newcommand{\DMABitDefID}[1]{\DMABitDefAnyID{\arabic{DMAHeaderIndex}}{#1}}	% BitDef0_0, BitDef0_1...
\newcommand{\DMABitValID}[1]{BitVal\arabic{DMAHeaderIndex}_#1}				% BitVal0_0, BitVal0_1...
\newcommand{\DMABitDescTitleID}[1]{Bit\arabic{DMAHeaderIndex}_#1Title}		% Bit0_0Title, Bit0_1Title...
\newcommand{\DMABitDescItemAnyID}[2]{Bit\arabic{DMAHeaderIndex}_#1Item#2}	% Bit0_0Item0...
\newcommand{\DMABitDescItemID}[1]{\DMABitDescItemAnyID{#1}{\arabic{DMABitDescItemIndex}}}			% Bit0_0Item0...
\newcommand{\DMABitDescExplID}[1]{Bit\arabic{DMAHeaderIndex}_#1Expl\arabic{DMABitDescItemIndex}}	% Bit0_0Expl0...
\newcommand{\DMABitParID}[1]{Bit\arabic{DMAHeaderIndex}_#1ParByte}									% Bit0_0ParByte
\newcommand{\DMABitParExplID}[1]{Bit\arabic{DMAHeaderIndex}_\arabic{DMABitDescIndex}_#1ParExpl}		% Bit0_0_0ParExpl


% ▒█▀▀▄ ▒█▀▀█ ░█▀▀█ ▒█░░▒█ ▀█▀ ▒█▄░▒█ ▒█▀▀█ ▒█▀▀▀█   ░░   ▒█▀▀█ ▒█▀▀▀█ ▒█▀▄▀█ ▒█▀▀█ ▒█▀▀▀█ ▒█▄░▒█ ▒█▀▀▀ ▒█▄░▒█ ▀▀█▀▀ ▒█▀▀▀█ 
% ▒█░▒█ ▒█▄▄▀ ▒█▄▄█ ▒█▒█▒█ ▒█░ ▒█▒█▒█ ▒█░▄▄ ░▀▀▀▄▄   ▀▀   ▒█░░░ ▒█░░▒█ ▒█▒█▒█ ▒█▄▄█ ▒█░░▒█ ▒█▒█▒█ ▒█▀▀▀ ▒█▒█▒█ ░▒█░░ ░▀▀▀▄▄ 
% ▒█▄▄▀ ▒█░▒█ ▒█░▒█ ▒█▄▀▄█ ▄█▄ ▒█░░▀█ ▒█▄▄█ ▒█▄▄▄█   ░░   ▒█▄▄█ ▒█▄▄▄█ ▒█░░▒█ ▒█░░░ ▒█▄▄▄█ ▒█░░▀█ ▒█▄▄▄ ▒█░░▀█ ░▒█░░ ▒█▄▄▄█

% Inserts enough spacing to properly render double bit configuration (0 0, 0 1, 1 0, 1 1). Parameters:
% - #1 First bit value
% - #2 Second bit value
\newcommand{\DMATwoBits}[2]{#1\hspace*{1.8ex}#2}

% Renders header with all 8 bits. Parameters:
% - (optional) #1 empty (default) for main header, otherwise it will be treated as parameter positioning wise
\newcommand{\DMABitHeader}{
	\begin{scope}
		\node[dmaheader] (\DMABitDefID{7}) {7};
		\node[dmaheader, right=0pt of \DMABitDefID{7}] (\DMABitDefID{6}) {6};
		\node[dmaheader, right=0pt of \DMABitDefID{6}] (\DMABitDefID{5}) {5};
		\node[dmaheader, right=0pt of \DMABitDefID{5}] (\DMABitDefID{4}) {4};
		\node[dmaheader, right=0pt of \DMABitDefID{4}] (\DMABitDefID{3}) {3};
		\node[dmaheader, right=0pt of \DMABitDefID{3}] (\DMABitDefID{2}) {2};
		\node[dmaheader, right=0pt of \DMABitDefID{2}] (\DMABitDefID{1}) {1};
		\node[dmaheader, right=0pt of \DMABitDefID{1}] (\DMABitDefID{0}) {0};
		
		% description at the right side
		\node[dmabyteexpl, anchor=west]
			at(\DMABitDefID{0}.south east)
			{Base Register Byte};

		% Reset all command related counters
		\setcounter{DMAHeaderIndex}{0}
		\setcounter{DMABitDescIndex}{-1}
		\setcounter{DMABitParameterIndex}{-1}
	\end{scope}
}

% Renders individual bit below header. Parameters:
% - (optional) #1 number of bits to span (default = 1)
% - #2 bit number (0-7) - used to generate node ID
% - #3 bit value (empty for no value)
\newcommand{\DMABitValue}[3][1]{
	\begin{scope}
		\IfEq{#1}{1}{
			% single bit
			\node[dmamono, 
				fit=(\DMABitDefID{#2}.west)(\DMABitDefID{#2}.east),
				anchor=north west,
				text height=1.35ex] 
				at(\DMABitDefID{#2}.south west)
				(\DMABitValID{#2}) 
				{#3};
		}{
			% span multiple bits
			\pgfmathtruncatemacro{\DMATo}{#2 - #1 + 1}
			\node[dmamono, 
				fit=(\DMABitDefID{#2}.west)(\DMABitDefID{\DMATo}.east),
				anchor=north west,
				text height=1.35ex]
				at(\DMABitDefID{#2}.south west)
				(\DMABitValID{#2})
				{#3};
		}
	\end{scope}
}

% Renders legend on the left side. Parameters:
% - #1 bit number for the top parameter (0-7)
% - (optional) #2 title for paramters part (default = PARAMETERS)
% - (optional) #3 title for command part (default = REGISTER)
\NewDocumentCommand{\DMALegend}{ m O{PARAMETERS} O{REGISTER} }{
	\begin{scope}[
		every path/.style={PrintableLightGray, line width=0.6pt},
		every node/.style={rotate=90}]

		% Horizontal delimiter
		\draw
			([xshift=1ex,yshift=1ex]#1.north west) -- 
			([xshift=-1.5em,yshift=1ex]#1.north west) 
			node (\DMALegendID) {};

		% Command text above delimiter
		\node[dmalegend, above=0pt of \DMALegendID, anchor=north west, yshift=-1.5pt]
			(\DMALegendRegID)
			{#3};
		
		% Parameters text below delimiter
		\node[dmalegend, below=0pt of \DMALegendID, anchor=east]
			(\DMALegendParID)
			{#2};
	\end{scope}
}

% Renders title for bit description group. Parameters:
% - (optional) #1 left offset (default = -1.6ex)
% - #2 Bit number (0-7)
% - #3 Description text
% - (optional) #4 top offset when first parameter (default = 2ex)
\NewDocumentCommand{\DMABitDescTitle}{ O{-1.6ex} m m O{2ex} }{
	\begin{scope}
		\IfEq{\arabic{DMABitDescIndex}}{-1}{
			% If this is first description group, position it below bit "table"
			\node[
				dmatitle, 
				xshift=#1, 
				below=#4 of \DMABitValID{#2}, 
				anchor=north west]
				(\DMABitDescTitleID{#2})
				{#3};
		}{
			% Subsequent description groups should be positioned below last item of the previous group
			\node[
				dmatitle, 
				anchor=north west, 
				xshift=#1, 
				yshift=-1em]
				at(\DMABitValID{#2} |- \DMABitDescItemID{\arabic{DMABitDescIndex}})
				(\DMABitDescTitleID{#2})
				{#3};
		}
		
		% Draw vertical connecting line from bit to title
		\draw[dmaconnector] 
			([yshift=2pt]\DMABitValID{#2}.center) -- 
			([yshift=-1pt]\DMABitValID{#2}.center |- \DMABitDescTitleID{#2}.north);
		
		% Change bit description counter to given bit number and reset item counter
		\setcounter{DMABitDescIndex}{#2}
		\setcounter{DMABitDescItemIndex}{-1}
	\end{scope}
}

% Renders individual bit description item. Parameters:
% - #1 Bit number (0-7)
% - #2 Bit value (0/1)
% - #3 Explanation text
\newcommand{\DMABitDescItem}[3]{
	\begin{scope}
		% Update item counter for next item
		\stepcounter{DMABitDescItemIndex}

		\IfEq{\arabic{DMABitDescItemIndex}}{0}{
			% if this is first item for this group, position it below previously generated title
			\node[dmavalue, below=1ex of \DMABitDescTitleID{#1}.west, anchor=north west]
				(\DMABitDescItemID{#1})
				{#2};
		}{
			% Otherwise position it below previous item
			\pgfmathtruncatemacro{\prev}{\arabic{DMABitDescItemIndex}-1}
			\node[dmavalue]
				at([yshift=-5pt]\DMABitDescItemAnyID{#1}{\prev}.south) 
				(\DMABitDescItemID{#1}) 
				{#2};
		}

		% render explanation text
		\node[dmaexpl, anchor=west] 
			at(\DMABitDescItemID{#1}.east) 
			(\DMABitDescExplID{#1}) 
			{#3};
	\end{scope}
}

% Renders parameter for given bit. Parameters:
% - #1 Bit number (0-7)
% - #2 Parameter description text
% - (optional) #3 vertical spacing from parent when this is first parameter (default = 1ex)
% - (optional) #4 positioning for "bit=1" text on connector line (default = near start)
% - (optional) #5 text to use in front of parameter description (default = +1 byte) (note: if empty, then connector node also doesn't display text on it)
\NewDocumentCommand{\DMABitPar}{ m m O{1ex} O{near start} O{+1 byte} }{
	\begin{scope}
		% increment parameter counter
		\stepcounter{DMABitParameterIndex}

		\IfEq{\arabic{DMABitParameterIndex}}{0}{
			% if this is first parameter, position it below bit table
			\node[dmabyte, 
				below=#3 of \DMABitValID{7}.west |- \DMABitDescExplID{\arabic{DMABitDescIndex}}.south, 
				anchor=north west]
				(\DMABitParID{#1})
				{};
		}{
			\IfEq{\arabic{DMABitDescIndex}}{-1}{
				\IfEq{#1}{0}{
					% if this is the first parameter for current header and it's for bit 0, position it below previous parameter or header. But we can't use -1 as it will end up in negative index...
						\node[dmabyte, 
							below=#3 of \DMABitValID{7}.south west,
							anchor=north west]
							(\DMABitParID{#1})
							{};
					}{
					% if this is the first parameter for current header, position it below previous parameter or header
					\pgfmathtruncatemacro{\prev}{#1-1}
					\node[dmabyte, 
						below=2pt of \DMABitParID{\prev}]
						(\DMABitParID{#1})
						{};
				}
			}{
				% if this is the first parameter for current header, position it below last item of the previous group
				\pgfmathtruncatemacro{\prev}{\arabic{DMABitParameterIndex}-1}
				\node[dmabyte, 
					below=2.7ex of \DMABitValID{7}.west |- \DMABitDescItemID{\arabic{DMABitDescIndex}},
					anchor=north west]
					(\DMABitParID{#1})
					{};
					
				% afterwards reset desc index; once parameters are listed, this no longer applies
				\setcounter{DMABitDescIndex}{-1}
			}
		}
		
		% above node renders all 8 bits with lighter color, this one renders black frame around all bits
		\node[dmabyteouter, left=0pt of \DMABitParID{#1}, anchor=west] {};
		
		% render description and explanation text
		\IfStrEq{#5}{}{
			% if empty string, render only explanation text
			\node[dmaexpl, anchor=west]
				at(\DMABitParID{#1}.east)
				(\DMABitParExplID{#1})
				{#2};
		}{
			% otherwise render both
			\node[dmabyteexpl, anchor=west]
				at(\DMABitParID{#1}.east)
				(\DMABitParExplID{#1}ByteText)
				{#5};
			\node[dmaexpl, anchor=base west, xshift=-3pt]
				at(\DMABitParExplID{#1}ByteText.base east)
				(\DMABitParExplID{#1})
				{#2};
		}
		
		% render connection arrow from bit to parameter
		\draw[dmaconnector]
			([yshift=2pt]\DMABitValID{#1}.center) -- 
			node[dmabyteexpl, above, #4, xshift=3pt, font=\tiny, rotate=90] {\IfStrEq{#5}{}{}{\fbox{#1}={\tt 1}}}
			([yshift=-3pt]\DMABitValID{#1}.center |- \DMABitParID{#1}.north);

		% Reset all parameter related counters
		\setcounter{DMABitDescIndex}{-1}
	\end{scope}
}

% Same as \DMABitHeader but when used as parameter for given bit. Parameters:
% - #1 Bit number (0-7), leave empty for no bit
% - #2 parameter description
% - (optional) #3 text to use in front of parameter description (default = +1 byte) (note: if this is empty, then connector line also doesn't display text on it)
\NewDocumentCommand{\DMABitParHeader}{ m m O{+1 byte} }{
	\begin{scope}
		% increment parameter counter
		\stepcounter{DMABitParameterIndex}

		\pgfmathtruncatemacro{\DMAHeadNextID}{\arabic{DMAHeaderIndex}+1}

		\IfEq{\arabic{DMABitParameterIndex}}{0}{
			% if this is first parameter, position it below bit table
			\node[dmaheader, 
				below=1ex of \DMABitValID{7}.west |- \DMABitDescExplID{\arabic{DMABitDescIndex}}.south, 
				anchor=north west]
				(\DMABitDefAnyID{\DMAHeadNextID}{7})
				{7};
		}{
			\IfEq{\arabic{DMABitDescIndex}}{-1}{
				% if this is the first parameter for current header, position it below previous parameter
				\pgfmathtruncatemacro{\prev}{\arabic{DMABitParameterIndex}-1}
				\node[dmaheader, 
					below=2pt of \DMABitParID{\prev}]
					(\DMABitDefAnyID{\DMAHeadNextID}{7})
					{7};
			}{
				% if this is the first parameter for current header, position it below last item of the previous group
				\node[dmaheader, 
					below=2.7ex of \DMABitValID{7}.west |- \DMABitDescItemID{\arabic{DMABitDescIndex}},
					anchor=north west] 
					(\DMABitDefAnyID{\DMAHeadNextID}{7})
					{7};
			}
		}
		
		\node[dmaheader, right=0pt of \DMABitDefAnyID{\DMAHeadNextID}{7}] (\DMABitDefAnyID{\DMAHeadNextID}{6}) {6};
		\node[dmaheader, right=0pt of \DMABitDefAnyID{\DMAHeadNextID}{6}] (\DMABitDefAnyID{\DMAHeadNextID}{5}) {5};
		\node[dmaheader, right=0pt of \DMABitDefAnyID{\DMAHeadNextID}{5}] (\DMABitDefAnyID{\DMAHeadNextID}{4}) {4};
		\node[dmaheader, right=0pt of \DMABitDefAnyID{\DMAHeadNextID}{4}] (\DMABitDefAnyID{\DMAHeadNextID}{3}) {3};
		\node[dmaheader, right=0pt of \DMABitDefAnyID{\DMAHeadNextID}{3}] (\DMABitDefAnyID{\DMAHeadNextID}{2}) {2};
		\node[dmaheader, right=0pt of \DMABitDefAnyID{\DMAHeadNextID}{2}] (\DMABitDefAnyID{\DMAHeadNextID}{1}) {1};
		\node[dmaheader, right=0pt of \DMABitDefAnyID{\DMAHeadNextID}{1}] (\DMABitDefAnyID{\DMAHeadNextID}{0}) {0};
		
		% if bit is provided, render connector and explanation
		\IfEq{#1}{}{}{
			% render connection arrow from bit to parameter
			\draw[dmaconnector]
				([yshift=2pt]\DMABitValID{#1}.center) -- 
				node[dmabyteexpl, above, near start, xshift=3pt, font=\tiny, rotate=90] {\IfStrEq{#3}{}{}{\fbox{#1}={\tt 1}}}
				([yshift=-2pt]\DMABitValID{#1}.center |- \DMABitDefAnyID{\DMAHeadNextID}{#1}.north);
		}
		
		% increment header index
		\stepcounter{DMAHeaderIndex}
   
		% Reset all parameter related counters
		\setcounter{DMABitDescIndex}{-1}

		% render description and explanation text
		\IfStrEq{#3}{}{
			% if empty string, render only explanation text
			\node[dmaexpl, anchor=west]
				at(\DMABitDefAnyID{\DMAHeadNextID}{0}.south east)
				(\DMABitParExplID{#1})
				{#2};
		}{
			% otherwise render both
			\node[dmabyteexpl, anchor=west]
				at(\DMABitDefAnyID{\DMAHeadNextID}{0}.south east)
				(\DMABitParExplID{#1}ByteText)
				{#3};
			\node[dmaexpl, anchor=base west, xshift=-3pt]
				at(\DMABitParExplID{#1}ByteText.base east)
				(\DMABitParExplID{#1})
				{#2};
		}

	\end{scope}
}
