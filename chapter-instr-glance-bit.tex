\section{Bit Set, Reset and Test}

\begin{minipage}{\textwidth}

\begin{instrtable}

	\begin{instruction}{BIT b,r} 
		\Symbol{\SymBIT{r}}
			\FlagsBITr
			\OpCode{11}{001}{011}
			\Hex{CB}{2}
			\Cycles{2}{8}
			\Comment{
				\multirow{6}{*}{
					\tt
					\begin{tabular}{ll}
						r & \OCT{r} \\
						\hline
						B & 000 \\
						C & 001 \\
						D & 010 \\
						E & 011 \\
						H & 100 \\
						L & 101 \\
						A & 111 \\
					\end{tabular}
				}
			}
		\SkipToOpCode 
			\OpCode{01}{\OCT{b}}{\OCT{r}}
			\Hex{..}{}
	\end{instruction}

	\begin{instruction}{BIT b,(HL)} 
		\Symbol{\SymBIT{(HL)}}
			\FlagsBITr
			\OpCode{11}{001}{011}
			\Hex{CB}{2}
			\Cycles{3}{12}
		\SkipToOpCode 
			\OpCode{01}{\OCT{b}}{110}
			\Hex{..}{}
	\end{instruction}

	\begin{instruction}{BIT b,(IX+d)\See{2}} 
		\Symbol{\SymBIT{(IX+d)}}
			\FlagsBITr
			\OpCode{11}{011}{101}
			\Hex{DD}{4}
			\Cycles{5}{20}
		\SkipToOpCode 
			\OpCode{11}{001}{011}
			\Hex{CB}{}
		\SkipToOpCode
			\OpRange{d}
			\Hex{..}{}
		\SkipToOpCode 
			\OpCode{01}{\OCT{b}}{110}
			\Hex{..}{}
	\end{instruction}

	\begin{instruction}{BIT b,(IY+d)\See{2}} 
		\Symbol{\SymBIT{(IY+d)}}
			\FlagsBITr
			\OpCode{11}{111}{101}
			\Hex{FD}{4}
			\Cycles{5}{20}
			\Comment{
				\multirow{7}{*}{
					\tt
					\begin{tabular}{ll}
						b & \OCT{b} \\
						\hline
						0 & 000 \\
						1 & 001 \\
						2 & 010 \\
						3 & 011 \\
						4 & 100 \\
						5 & 101 \\
						6 & 110 \\
						7 & 111 \\
					\end{tabular}
				}
			}
		\SkipToOpCode 
			\OpCode{11}{001}{011}
			\Hex{CB}{}
		\SkipToOpCode 
			\OpRange{d} 
			\Hex{..}{}
		\SkipToOpCode 
			\OpCode{01}{\OCT{b}}{110}
			\Hex{..}{}
	\end{instruction}

	\begin{instruction}{SET b,r} 
		\Symbol{\SymSET{r}}
			\FlagsSETr
			\OpCode{11}{001}{011}
			\Hex{CB}{2}
			\Cycles{2}{8}
		\SkipToOpCode 
			\OpCode{\fbox{11}}{\OCT{b}}{\OCT{r}}
			\Hex{..}{}
	\end{instruction}

	\begin{instruction}{SET b,(HL)} 
		\Symbol{\SymSET{(HL)}}
			\FlagsSETr
			\OpCode{11}{001}{011}
			\Hex{CB}{2}
			\Cycles{4}{15}
		\SkipToOpCode 
			\OpCode{\fbox{11}}{\OCT{b}}{110}
			\Hex{..}{}
	\end{instruction}

	\begin{instruction}{SET b,(IX+d)} 
		\Symbol{\SymSET{(IX+d)}}
			\FlagsSETr
			\OpCode{11}{011}{101}
			\Hex{DD}{4}
			\Cycles{6}{23}
		\SkipToOpCode 
			\OpCode{11}{001}{011}
			\Hex{CB}{}
		\SkipToOpCode 
			\OpRange{d} 
			\Hex{..}{}
		\SkipToOpCode 
			\OpCode{\fbox{11}}{\OCT{b}}{110}
			\Hex{..}{}
	\end{instruction}

	\begin{instruction}{SET b,(IY+d)} 
		\Symbol{\SymSET{(IY+d)}}
			\FlagsSETr
			\OpCode{11}{111}{101}
			\Hex{FD}{4}
			\Cycles{6}{23}
		\SkipToOpCode 
			\OpCode{11}{001}{011}
			\Hex{CB}{}
		\SkipToOpCode 
			\OpRange{d} 
			\Hex{..}{}
		\SkipToOpCode 
			\OpCode{\fbox{11}}{\OCT{b}}{110}
			\Hex{..}{}
	\end{instruction}

	\begin{instruction}{SET b,(IX+d),r} 
		\Symbol{\SymSETu[0]{r}{IX}}
			\FlagsSETr
			\OpCode{11}{011}{101}
			\Hex{DD}{4}
			\Cycles{6}{23}
		\SkipToSymbol
			\Symbol{\SymSETu[1]{r}{IX}}
			\FromSymbolToOpCode
			\OpCode{11}{001}{011}
			\Hex{CB}{}
		\SkipToSymbol
			\Symbol{\SymSETu[2]{r}{IX}}
			\FromSymbolToOpCode
			\OpRange{d} 
			\Hex{..}{}
		\SkipToOpCode 
			\OpCode{\fbox{11}}{\OCT{b}}{\OCT{r}}
			\Hex{..}{}
	\end{instruction}

	\begin{instruction}{SET b,(IY+d),r} 
		\Symbol{\SymSETu[0]{r}{IY}}
			\FlagsSETr
			\OpCode{11}{111}{101}
			\Hex{FD}{4}
			\Cycles{6}{23}
		\SkipToSymbol 
			\Symbol{\SymSETu[1]{r}{IY}}
			\FromSymbolToOpCode
			\OpCode{11}{001}{011}
			\Hex{CB}{}
		\SkipToSymbol
			\Symbol{\SymSETu[2]{r}{IY}}
			\FromSymbolToOpCode
			\OpRange{d} 
			\Hex{..}{}
		\SkipToOpCode 
			\OpCode{\fbox{11}}{\OCT{b}}{\OCT{r}}
			\Hex{..}{}
	\end{instruction}

	\StartWithOpCode\OpCode{$\uparrow$}{}{} \\

	\begin{lastinstruction}{RES b,m\See{3}} 
		\Symbol{\SymRES{m}}
			\FlagsRESr
			\OpCode{\fbox{10}}{...}{...}
	\end{lastinstruction} 
						
\end{instrtable}

\begin{notestable}
	\NoteItem{\See{1}See section \ref{z80_undocumented_flags_bit} for complete description\noteb}
	\NoteItem{\See{2}Instruction has other undocumented opcodes\noteb}
	\NoteItem{\See{3}{\tt m} is one of {\tt r}, {\tt (HL)}, {\tt (IX+d)}, {\tt (IY+d)}. To form RES instruction, replace \fbox{{\tt 11}} with \fbox{{\tt 10}}. Ts also the same}
\end{notestable}

\end{minipage}